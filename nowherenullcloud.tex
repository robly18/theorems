\documentclass{article}

\usepackage{amsmath}
\usepackage{amsthm}

\title{On the existance of a nowhere null measure cloud in the unit interval}
\author{}
\date{}

\DeclareMathOperator{\m}{m}

\begin{document}
	\maketitle
	
	\textbf{Exercise 33:} There exists a Borel set $A \subseteq [0,1]$ such that for all open subintervals $I$ of $[0,1]$ we have $0 < \m(A \cap I) < \m(I)$.
	
	\bigskip
	
	\section{Demonstration}
	
	We begin by establishing the existance and certain properties about generalized Cantor sets.
	
	Fix a sequence $\beta_i$ of real numbers in the interval $]0,1[$. We define \emph{the generalized Cantor set induced by $\beta_i$} as the set obtained from $[0,1]$ by the following procedure:
	
	First, remove the open interval of length $\beta_1$ from the middle of this interval, leaving us with two closed intervals. Then, remove from each an interval of length corresponding to $\beta_2$ times their length from their middle, leaving us with four closed intervals. Then, remove $\beta_3$ times their length from their middle, and so on.
	
	The limit set is clearly closed (its complementary is a union of opens) and thus a Borel set, and therefore has some Lebesgue measure. It is easy to see that this measure is equal to the infinite product $\prod (1 - \beta_i)$. This product can be made to take any desired value in $[0, 1[$, by the following procedure:
	
	Let $c \in [0, 1[$ be the desired measure. Fix any sequence $c_i$ descending strictly towards $c$, and such that $c_1 = 1$. Define $\gamma_i = \frac{c_{i+1}}{c_i} \in ]0, 1[$, and $\beta_i = 1 - \gamma_i \in ]0,1[$. Then, we can consider the generalized Cantor set induced by $\beta_i$, and its measure shall be:
	
	\[ \prod (1 - \beta_i) = \prod \gamma_i = \prod \frac{c_{i+1}}{c_i} = c \]
	
	A detail that will be useful in the sequence is that, in this construction, we can assert that $\beta_1$ take any value in $]0, 1-c[$: simply set $c_2$ to be $1 - \beta_1$ (for the desired value of $\beta_1$), and since this value will be in $]c, 1[$, the sequence $c_i$ necessary for the construction does exist.
	
	As such, we will use the symbol $C^c_{\beta_1}$ to denote a generalized Cantor set of measure $c$ and of first interval length $\beta_1$. If the length of the first interval is not relevant, we simply use the symbol $C^c$.
	
	To build the desired cloud, we begin considering the following procedure: let $k_i$ be a sequence contained in $]0,1[$. Consider $C^{k_1}$. Its complementary is a countable union of open intervals. Consider inserting a (scaled and translated) copy of $C^{k_2}$ on (the closure of) each of these. Again, the complimentary of this set will be a countable disjoint union of opens, and we could insert a copy of $C^{k_3}$ on each of these, and so on. Call the resulting set $D_{k_1, k_2, \ldots}$
	
	We want this cloud to be nowhere null, and in particular we want it not to be null. As such, we proceed to look at the measure of this.
	
	Notice the measure of the complementary. It is not hard to see it is given by $\prod (1 - k_i)$. By a procedure similar to above, we can choose the $k_i$ in order to make this product any value in $[0,1]$ we desire. For example, $\frac 1 2$. Furthermore, consider the complementary of some finite iteration, let's say $n$, and pick any open interval $I$ composing it. In the full iteration, this open interval will have a copy of $D_{k_{n+1}, k_{n+2}, \ldots}$, scaled down and translated. As such, $\m(I \cap D_{k_1, k_2, \ldots}) = \m(I) (1 - \prod_{n+1}^\infty (1 - k_i))$, which is the same as
	
	\[\m(I) (1 - (1 - k_1)(1-k_2)\cdots(1-k_n) \m(D_{k_1, k_2, \ldots}))\]
	
	Which, if all $k_i$ are in $]0,1[$ and the measure of $D_{k_1, k_2, \ldots}$ is nonzero, amounts to a number strictly between 0 and $\m(I)$.
	
	More generally, any interval $I$ containing such an interval; that is, any interval $I$ that contains within it one of the open sets of the complementary of some finite iteration; will have a measure strictly between 0 and $\m(I)$. As such, if we show that, for some sequence $k_i$ any open interval contains one of the open sets of the complementary of some finite iteration, then we prove the existance of a set with the desired properties.
	
	This is where the notion that we may choose $\beta_1$ more or less as desired comes into play. Notice that, by construction, the complementary of $C^c$ is dense in $[0,1]$. To see this, notice that the $n$th iteration of the construction is a disjoint union of closed intervals, each of length less than $\frac 1 {2^n}$. As such, suppose $x < y$ are two numbers in $[0,1]$. After some iteration, these intervals will have length less than their distance, and so $x$ and $y$ cannot be contained in the same interval. But then there must be a point between them in the complementary of that iteration, and thus in the complementary of $C^c$. Thus, any open $]x,y[$ contains an element of the complementary, which means it (the complementary) is dense.
	
	Therefore, $C^c$ splits the interval $[0,1]$ into countably many disjoint open intervals, such that any open subset of $[0,1]$ intersects one of these. The next step in the iteration (of $D_{k_1, k_2, \ldots}$) will split each of these intervals once again, and notice the following: any open subinterval $I_0$ of $[0,1]$ will have non-null intersection with one of the open intervals in the first iteration. Let's say this intersection (which will be open) is $I_1 = I_0 \cap J_0$. Then, $I_1$ will have non-null intersection with one of the open intervals the second iteration split $[0,1]$, and in particular $J_0$, into, and therefore so will $I_0$. This argument can be repeated indefinitely, leading to the conclusion that, for any finite number of iterations, any open subinterval $I_0$ of $[0,1]$ will intersect one of the open intervals composing the complementary.
	
	We want something stronger than this: we wish for $I_0$ to \emph{contain} one of these open intervals, as we have already shown that if this is the case we have solved the problem. To do this, notice that it will suffice to show that $I_0$ intersects \emph{three} open intervals of the complementary of some iteration. To do this, on the other hand, it suffices to show that after some number of iterations, the length of each open interval will be less than half the length of $I_0$. (See addendum) Therefore, we construct our $D_{k_1, k_2, \ldots}$ such that the maximum length of the intervals gets arbitrarily small as we add more generalized Cantor sets.
	
	Let $M$ be the maximum length of the intervals after some number of iterations. Suppose we iterate once again, filling each of these opens with (scaled copies of) $C^k$. Then, the maximum length of the intervals in this new iteration will be $M$ times the maximum length of the intervals in $C^k$. This could potentially be hard to control, but if $k$ happens to be less than $2/3$, we can make the length of the first interval be $1/3$, and it's not hard to see that all other intervals will be less than this. As such, under appropriate conditions, we can make the size of the maximum interval decrease by a factor of $1/3$ per iteration.
	
	To do this, we need to make sure that the $k_i$ are, at least eventually, less than $2/3$. But since, in order for the infinite product $\prod (1 - k_i)$ to be nonzero, the $k_i$ must converge to 0, this is eventually possible, and we can thus ensure the maximum interval length goes to zero as we iterate.
	
	This concludes our demonstration.
	
	\section{Addendum}
	
	All that is left is to show the following: if we have a collection of open disjoint intervals such that any open interval in $[0,1]$ intersects one of these, and if we have an interval $I$ of length greater than twice the maximum length of these intervals, then $I$ contains at least one of these.
	
	To see this, notice that $I = ]a,b[$ intersects at least one, $I_1 = ]a_1, b_1[$. If it is contained within $I$, we are done. If not, either $b_1 > b$ or $a_1 < a$. Suppose, without loss of generality, that it is the second case.
	
	Then, since the length of it is less than $\frac{b-a}2$, we have $b_1 < \frac{a+b}2$. Consider, then, the interval $I^+$ given by $]\frac{a+b}2, b[$. It is open, and thus intersects another interval $I_2 = ]a_2, b_2[$. If $I_2$ is contained within $I^+$ it is also contained within $I$ and we are done. Therefore, either $b_2 > b$ or $a_2 < \frac{a+b}2$. If not the former, we are also done, as $a_2$ must be greater than $b_1$ and thus than $a$, implying $I_2$ is contained in $I$. Therefore, suppose $b_2 > b$.
	
	Finally, consider the interval $]b_1, a_2[$. It is nondegenerate, as $b_1 < \frac{a+b}2 < a_2$. Therefore, it must intersect at least one interval $I_3$. But $I_3$ must be between $I_2$ and $I_1$, which, as can easily be seen, implies it is contained in $I$, concluding our demonstration.

\end{document}