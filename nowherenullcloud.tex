\documentclass{article}

\usepackage{amsmath}
\usepackage{amsthm}

\title{On the existance of a nowhere null measure cloud in the unit interval}
\author{}
\date{}

\DeclareMathOperator{\m}{m}

\begin{document}
	\maketitle
	
	\textbf{Exercise 33:} There exists a borel set $A \subseteq [0,1]$ such that for all open subintervals $I$ of $[0,1]$ we have $0 < \m(A \cap I) < \m(I)$.
	
	\bigskip
	
	We begin by establishing the existance and certain properties about generalized Cantor sets.
	
	Fix a sequence $\beta_i$ of real numbers in the interval $]0,1[$. We define \emph{the generalized Cantor set induced by $\beta_i$} as the set obtained from $[0,1]$ by the following procedure:
	
	First, remove the open interval of length $\beta_1$ from the middle of this interval, leaving us with two closed intervals. Then, remove from each an interval of length corresponding to $\beta_2$ times their length from their middle, leaving us with four closed intervals. Then, remove $\beta_3$ times their length from their middle, and so on.
	
	The limit set is clearly closed (its complementary is a union of opens) and thus a Borel set, and therefore has some Lebesgue measure. It is easy to see that this measure is equal to the infinite product $\prod (1 - \beta_i)$. This product can be made to take any desired value in $[0, 1[$, by the following procedure:
	
	Let $c \in [0, 1[$ be the desired measure. Fix any sequence $c_i$ descending strictly towards $c$, and such that $c_1 = 1$. Define $\gamma_i = \frac{c_{i+1}}{c_i} \in ]0, 1[$, and $\beta_i = 1 - \gamma_i \in ]0,1[$. Then, we can consider the generalized Cantor set induced by $\beta_i$, and its measure shall be:
	
	\[ \prod (1 - \beta_i) = \prod \gamma_i = \prod \frac{c_{i+1}}{c_i} = c \]
	
	A detail that will be useful in the sequence is that, in this construction, we can assert that $\beta_1$ take any value in $]0, 1-c[$: simply set $c_2$ to be $1 - \beta_1$ (for the desired value of $\beta_1$), and since this value will be in $]c, 1[$, the sequence $c_i$ necessary for the construction does exist.
	
	As such, we will use the symbol $C^c_{\beta_1}$ to denote a generalized Cantor set of measure $c$ and of first interval length $\beta_1$. If the length of the first interval is not relevant, we simply use the symbol $C^c$.
	
	To build the desired cloud, we begin considering the following procedure: let $k_i$ be a sequence contained in $]0,1[$. Consider $C^{k_1}$. Its complementary is a countable union of open intervals. Consider inserting a (scaled and translated) copy of $C^{k_2}$ on (the closure of) each of these. Again, the complimentary of this set will be a countable disjoint union of opens, and we could insert a copy of $C^{k_3}$ on each of these, and so on. Call the resulting set $D_{k_1, k_2, \ldots}$
	
	We want this cloud to be nowhere null, and in particular we want it not to be null. As such, we proceed to look at the measure of this.
	
	Notice the measure of the complementary. It is not hard to see it is given by $\prod (1 - k_i)$. By a procedure similar to above, we can choose the $k_i$ in order to make this product any value in $[0,1]$ we desire. For example, $\frac 1 2$. Furthermore, consider the complementary of some finite iteration, let's say $n$, and pick any open interval $I$ composing it. In the full iteration, this open interval will have a copy of $D_{k_{n+1}, k_{n+2}, \ldots}$, scaled down and translated. As such, $\m(I \cap D_{k_1, k_2, \ldots}) = \m(I) (1 - \prod_{n+1}^\infty (1 - k_i))$, which is the same as
	
	\[\m(I) (1 - (1 - k_1)(1-k_2)\cdots(1-k_n) \m(D_{k_1, k_2, \ldots}))\]
	
	Which, if all $k_i$ are in $]0,1[$ and the measure of $D_{k_1, k_2, \ldots}$ is nonzero, amounts to a number strictly between 0 and $\m(I)$.
	
	More generally, any interval $I$ containing such an interval; that is, any interval $I$ that contains within it one of the open sets of the complementary of some finite iteration; will have a measure strictly between 0 and $\m(I)$. As such, if we show that, for some sequence $k_i$ any open interval contains one of the open sets of the complementary of some finite iteration, then we prove the existance of a set with the desired properties.
	
	TODO

\end{document}