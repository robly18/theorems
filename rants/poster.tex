\documentclass{beamer}
  \usepackage{times}
  \usepackage{amsmath,amsthm, amssymb}
  \boldmath
  \usetheme{Copenhagen}
  \setbeamertemplate{navigation symbols}{}
  \usepackage[orientation=portrait,size=a4,scale=1.6]{beamerposter}
  \geometry{
	  hmargin=2cm,vmargin=1cm % little modification of margins
	}
\usepackage{changepage}
\addtobeamertemplate{block begin}
  {}
  {\vspace{0.2cm} % Pads top of block
   \setlength{\parskip}{5pt plus 1pt minus 1pt}%
   \begin{adjustwidth}{1cm}{1cm}
}
\addtobeamertemplate{block end}
  {\end{adjustwidth}%
   \vspace{0.4cm}}

\setbeamertemplate{frametitle}{%
    \nointerlineskip%
    \begin{beamercolorbox}[wd=1.1\linewidth,ht=2.4cm,dp=1.5ex,left]{frametitle}
        \hspace*{2.5ex}\insertframetitle\\[1ex]
        \hspace*{2.5ex}\insertframesubtitle
    \end{beamercolorbox}%
}


\begin{document}
\begin{frame}[t]{}
\frametitle{\VeryHuge Seminar:}
\framesubtitle{\VeryHuge Visualizing Differential Forms}
\begin{block}{}
\centering
\huge Duarte Maia, MMA

\LARGE May 3, 14:30--16:00, Room 3.10
\end{block}
\begin{block}{\Large Abstract}
\large
In this talk, I will introduce and motivate a visual way to represent differential forms and explain how some usual operations are interpreted in this framework, and with it visually `prove' the Stokes theorem. As an example of application, I will relate it to deRham cohomology and show how it can be used to furnish a rigorous proof of the Poincaré lemma.
%The mathematician's main tools are theorems and proofs, but also heuristics, analogies and geometric diagrams. These last methods are not rigorous demonstrations, but they are often very valuable in suggesting legitimate proofs, by proposing angles of attack and suggesting what may or may not be reasonable to conjecture.

%The field of geometry is often explained hand-to-hand with diagrams, which is not surprising given its nature. However, within this field, the algebra of tensors and differential forms is usually treated purely algebraically.

%When learning the subject, I sought out a geometric way to visualize these notions. After some research, I found a way to depict forms pictorially in a way that was satisfactory to me, and as I have learned more geometry I have found it a very useful analogy.

%In this talk, I will introduce and motivate this visualization, explain how some usual operations on differential forms are interpreted in this framework, and with it `prove' the Stokes theorem. As an example of application, I will relate it to deRham cohomology and show how it can be used to furnish a rigorous proof of the Poincaré lemma.
\end{block}
\begin{block}{Pre-Requisites}
\large This seminar should be mostly accessible to anyone with basic knowledge of manifolds and differential forms.
\end{block}
\begin{columns}[t]
	\begin{column}{.4 \linewidth}
		\begin{block}{Example: A Couple of Forms}
		\includegraphics[width=\linewidth]{posterdiff2}
		\end{block}
	\end{column}
	\begin{column}{.58 \linewidth}
		\begin{block}{Example: Visual Proof of Cartan's Magic Formula}
		\includegraphics[width=\linewidth]{posterdiff21}
		\end{block}
	\end{column}
\end{columns}
\end{frame}
\end{document}

