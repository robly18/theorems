\documentclass{article}

\usepackage[utf8]{inputenc}

\usepackage{amsmath}
\usepackage{amssymb}
\usepackage{amsfonts}
\usepackage{amsthm}

\usepackage{cite}


\title{On measurement spaces}
\author{Duarte Maia}

\theoremstyle{definition}
\newtheorem{problem}{Problem}

\DeclareMathOperator{\Hom}{Hom}

\newcommand{\R}{\mathbb{R}}

\begin{document}
\maketitle

\section{Introduction}

This document is meant as a repository for my notes on measurement spaces. It stores intuition I have developed and attempts at motivation behind definitions.

\section{Struggles with definition}

The phenomenon we are trying to model is as follows: there exists a physical system, and we are in possession of some measurement device. This device can take measurements in some set $M$, but some measurements might not be finitely recordable. For example, we may be in possession of an infinitely sharp needle to measure the position of a particle with, but it is impossible for the device to output with infinite precision the result it has measured: the best thing it can do is give us intervals inside of which we can be sure the particle lies. This motivates the definition of the collection of physical properties that can be finitely recorded, which we will call $\Omega$.

A standard argument (as can be found in \cite{topologyvialogic}) is enough to convince me that $\Omega$ should form a frame, where the order represents implication. (To be more precise, $\Omega$ should rather represent the collection of finitely recordable properties modulo equivalence, as to make $\leq$ a partial order.) Furthermore, given a measurement $m \in M$ and a finitely recordable property (henceforth frp) $U \in \Omega$, it might or might not be that the measurement $m$ is compatible with the property $U$. In other words, we may imagine our measuring device as something which, upon taking a measurement, requests frps and outputs whether or not it is possible that the frp is true; in other words, it says `no' to properties it can disprove. If $m$ does not disprove the frp $U$, we represent so by $m \vDash U$ (as per the notation in \cite{topologyvialogic}).

This brings us close to the notion of what Vickers calls a topological system: a triple $(M, \Omega, \vDash)$, where $M$ is a set, $\Omega$ is a frame and $\vDash$ is a relation in $M \times \Omega$, satisfying the property that for all $m \in M$ the function $(m\vDash) \colon \Omega \to \{\bot, \top\}$ is a frame homomorphism. Therefore, it would be desirable to have an intuitive argument to show this.

In natural language, we wish to show that the following three properties are true:
\begin{enumerate}
\item ($m\vDash$ is monotonous.) If upon measuring $m$ our machine cannot refute $U$, and $U \leq V$, then our machine cannot refute $V$. This is intuitively true, as the counterreciprocal makes clear: if our machine could refute $V$ and $U$ implies $V$, then the machine can also refute $U$.

\item (If $m\vDash \bigcup U_\alpha$ then $\exists_\alpha m \vDash U_\alpha$.) If upon measuring $m$ our machine cannot disprove `at least one of the properties $U_\alpha$ is true', then there exists one of the properties that the machine can't disprove. This is intuitively plausible, as, as per the counter reciprocal, if the machine can disprove all $U_\alpha$ then it should also be able to disprove the idea that some $U_\alpha$ is true. This `disproof' might not be able to be exhibited in finite time, but that isn't too problematic because the measurements themselves are also possibly not representable in finite time.

\item (If $m\vDash U_1, \dots, U_n$ then $m \vDash \bigcap U_k$.) If our machine cannot, upon measuring $m$, disprove any of the finite number of properties $U_1, \dots, U_n$, then it cannot also disprove that their conjunction is true.
\end{enumerate}

\begin{problem}\label{probhom}
The first property is intuitively obvious, and the second one is plausible. However, it is not obvious that the last property should be true.

A quantum interpretation might be as follows. Using the example of Schrödinger's electron, the property $Z^\uparrow \cap Z^\downarrow$ might represent a superposition of the states $\lvert z^\uparrow \rangle$ and $\lvert z^\downarrow \rangle$. However, this interpretation is incompatible with the interpretation of $\leq$ as `implies', for even though $Z^\uparrow \cap Z^\downarrow \leq Z^\uparrow$, it can hardly be said that `the property of an electron that has been measured to have spin up or spin down (which of these, it is unknown)' implies, for example, `the electron has spin up'. 

Intuitively it seems to make sense that $Z^\uparrow \cap Z^\downarrow$ means `it has spin up and spin down at the same time', but the meaning of this is unclear and interpreting it this way requires a revision of the interpretation of the order.
\end{problem}

\begin{problem}
The paper I'm basing myself in (\cite{measurement}) uses not the notion of topological system, but that of topological space. Vickers defines this as a topological system isomorphic to one of the kind $(X, \mathcal{T}, \in)$, where $\mathcal{T}$ is a topology on $X$. It is necessary and sufficient in order for a topological system to be spatial that its `opens' are uniquely defined by their elements. In the context of measurements, to require spatiality is to identify two physical properties $U$ and $V$ if they are compatible with the same set of measurements. In other words, we're identifying physical properties that are measurably identical.

This can be interpreted as some statement of `realism', in the sense that there are no hidden variables: for any two different states of the system there exists a measurement that distinguishes them. However, the story doesn't quite end here, because it might be that our apparatus is insufficient for measuring some properties. For example, if my apparatus is capable of measuring an electron's position but not its spin, requiring our formalism to be spatial is to identify the properties `the electron is between 0 and 1 with spin up' and `the electron is between 0 and 1 with spin down', because we do not have access to a measurement that distinguishes them. Therefore, the requirement of spatiality is also, in a sense, an identification: we are identifying properties that are measurably identical, even if they might, in some sense, be distinct. Is this desirable?
\end{problem}

\begin{problem}
Another requirement that is made in \cite{measurement} is sobriety of the space. Under Vickers' notion of topological system, this corresponds to saying that the system is localic, in the sense that it is isomorphic to a topological system of the form
\[(\Hom(L, \{\bot, \top\}), L, \Vdash),\]
where the relation $\Vdash$ is such that $h \Vdash U$ iff $h(U) = \top$. It is a theorem that a topological system $(M, \Omega, \vDash)$ is localic if and only if for all homomorphisms $h : \Omega \to \{\bot, \top\}$ there exists a point $m$ such that $h = (m \Vdash)$.

This is, in a sense, a statement that we have `enough measurements'. Indeed, (modulo problem \ref{probhom}) any $m$ induces a homomorphism $\Omega \to \{\bot,\top\}$, and to say that the system is localic is to say that to any such homomorphism we can correspond a measurement.

This can be interpreted as follows. One such homomorphism can be thought of as a `unambiguous consistent set of properties'. That is, from the point of view of a homomorphism $h$ some properties are `true' and some properties are `false'. Furthermore, $h$ is consistent, in the sense that if $h$ `sees' the property $U$ as true and $U$ implies $V$ then $h$ also sees $V$ as true, and furthermore if $h$ sees $U_1, U_2, \dots, U_n$ all as true then it also sees their conjunction as true. Finally $h$ allows no ambiguity, in the sense that if $h$ sees `some $U_\alpha$ is true' as a true statement (i.e. $\bigcup U_\alpha$) then there exists no ambiguity as to which of the $U_\alpha$ is true, in the sense that there does exist some $\alpha$ such that $h(U_\alpha) = \top$.

Under this point of view, the sobriety of the space corresponds to the idea that any unambiguous consistent set of properties corresponds to a measurement. I have problems with this postulate. Unlike the concept of spatiality, which can be guaranteed by identifying properties, sobriety requires that there be a plethora of measurements. The reference paper \cite{measurement} justifies this principle as follows: `if from the logical structure of properties it is derived (albeit possibly transfinitely) that a certain measuring process exists, then such a process should really exist.` However, this does not feel like a justification.

Here is a (more or less physical) `counterexample'. Suppose a particle lies in the real line, and we are attempting to measure its location using two infinitely sharp pointers: one which seeks the particle, and another one which is fixed at 0, detecting if the particle lies there. In this scenario, the finitely recordable properties are the usual opens in $\R$, together with the set $\{0\}$ and the other opens that must be added to make this a topology (namely, sets of the form $U \cup \{0\}$). This space is SOBER DAMMIT PAST ME you checked this wrong
\end{problem}

\bibliography{bibliography}{}
\bibliographystyle{plain}

\end{document}
