\documentclass{article}

\usepackage[utf8]{inputenc}

\usepackage{amsmath}
\usepackage{amssymb}
\usepackage{amsfonts}
\usepackage{amsthm}

\usepackage{braket}

\usepackage{cite}

\usepackage{tikz}


\title{An attempt to represent measurement spaces with quantum mechanical concepts}
\author{Duarte Maia}

\theoremstyle{definition}
\newtheorem{problem}{Problem}
\newtheorem{definition}{Definition}

\theoremstyle{plain}
\newtheorem*{prop}{Prop}
\newtheorem*{idea}{Idea}

\DeclareMathOperator{\spann}{span}
\DeclareMathOperator{\ext}{ext}
\DeclareMathOperator{\Max}{Max}

\newcommand{\R}{\mathbb{R}}
\newcommand{\C}{\mathbb{C}}


\newcommand{\HH}{\mathcal{H}}

\newcommand{\ps}{\mathcal{P}}
\newcommand{\pr}{\mathbb{P}}


\newcommand{\id}{\mathrm{id}}


\newcommand{\two}{\mathbf{2}}

\begin{document}
\maketitle

\section{Preliminary Definitions}

Context: there exists a quantum system which is being measured. We have at our disposal a device capable of measuring this system and outputting its results to us. The purpose is to formalize this concept in a way that is, hopefully, equivalent to \cite{measurement}.

\subsection{Microstates and Macrostates}

Consider a quantum mechanical system. It has an internal state, a wavefunction, but we don't know what that wavefunction is. Therefore, there is a distinction between the (single) state the system is in and the `metastate' we are aware of: the set of states the system \emph{could} be in from our perspective. I believe in thermodynamics this is the distinction between the so-called microstates and macrostates.

As such, let us begin by defining the set of microstates. As per the postulates of quantum mechanics, the set of possible microstates is a projective Hilbert space, let's say $S = \pr \HH$. At every moment, the system is in one and only one of these states.

However, from the outside, we may yet be unable to distinguish between a set of states, leaving us with a \emph{macrostate}: a collection of microstates, i.e. an element of $\ps(S)$. Furthermore, note that the set of macrostates has a natural operation of `superposition', which I will call $\sigma$. It is simply the operation that takes an element $A$ of $\ps(S)$ and returns the (projective) subspace of $S$ given by either $\spann A$ or $\overline\spann A$; I still haven't thought about infinite-dimensional cases so the distinction hasn't been relevant yet. This operation can probably be axiomatized as a monotone idempotent operator or something of the likes, but for now I'll be dealing with the very particular case of projective spaces.

So, in summary: I have a Hilbert space $\HH$; the set of microstates $S = \pr \HH$; the set of macrostates $\ps(S)$, and the operation $\sigma : \ps(S) \to \ps(S)$ corresponding to `subspace generated by'.

\subsection{Measurements}

Now, we are assuming that, in order to investigate our quantum system, we have at our disposal a machine which can do some kind of measurements. In abstract, we may think of a measurement as something that only happens under certain circumstances, and when it does it might disturb the state of the system. For example, a destructive measurement might allow us to conclude that previously a certain electron was in its $n$th excited state, but the measurement might have been done by waiting until the electron decayed to its ground state and inspecting the emitted radiation.

Therefore, let us think of each measurement $m$ as a binary relation, where $(s_1, s_2) \in m$ if it is compatible with the measurement $m$ that we were previously in the state $s_1$ and are now in the state $s_2$. Of course, this induces a very natural composition of measurements, which coincides with composition of binary relations, which in turn corresponds to the intuitive notion of `doing one measurement after the other'.

\subsection{Recording Protocols}

Even though we might allow things like an infinitely sharp needle, the outcomes of measurements need to be recorded in a finite medium, for example, a screen, or a Turing-machine-like tape. In \cite{measurement}, it is assumed that the recording depends on the quality of the machine, and better machines yield sharper results. I propose, instead, a query-response method. The user inserts a query, and the machine responds with `plausible' or `implausible'.

With this in mind, we consider fixed a set of queries $Q$ (which we might want to assume countable later on), and we will consider the frame $F = \braket{Q}$. It might also be reasonable to allow for a set of physical assumptions $A$ which relate the queries. For example, it is plausible that we could allow for two queries $q_1$ and $q_2$ which imply one another, e.g. `the particle is in the first excited state or above' vs `the particle is in the second excited state or above', and so our frame $F$ would not be freely generated, but we would add the relation $q_1 \land q_2 = q_2$.

Note that in this case the meet does not quite correspond to an `intersection'. Indeed, it might very well that $q_1$ and $q_2$ are two disjoint queries (they can't both be true at the same time), but $q_1 \land q_2$ is not the impossible query. Indeed, if the state is ill-known enough, it is very well possible that both $q_1$ and $q_2$ are plausible, if we haven't enough information to eliminate either of the possibilities.

Each query is supposed to reflect reality in some way. To be more specific about this, it is important to first look at the general physical meaning of a query. We could, for example, do a measurement and ask `is the electron's spin up?', but there is a distinction between finding out that the spin is up \emph{now}, after the measurement, versus that it was up before. The first case corresponds to something like using a Stern-Gerlach apparatus, while the second case corresponds to some kind of measurement that measures, say, the polarity of a photon, though in a destructive way.

As such, let us attribute to each query $q$ a relation $r(q) \subseteq \ps(S) \times \ps(S)$. The interpretation is that $(s_1, s_2) \in r(q)$ if the query $q$ is compatible with the situation where the system was in state $s_1$ prior to measurement, and is now in state $s_2$. It would hypothetically be plausible to add queries regarding intermediate states, but let us not go down that rabbit hole yet.

Now, given the function $r$, it is possible to define when our machine should reply `yes' or `inconclusive' to a query. If we have just made a measurement $m$, we will say $m \vDash q$ if there exist microstates $s_1, s_2$ such that $(s_1, s_2) \in m$ and $(s_1, s_2) \in r(q)$. In other words,
\begin{equation}\label{defvdash}
m \vDash q \text{ iff } m \cap r(q) \neq \emptyset.
\end{equation}

Note that for the resulting homomorphism $(m\vDash) : F \to \two$ to be well defined there has to be some kind of compatibility between the `query interpretation function' $r$ and the physical assumptions $A$. It is possible (and it is true in all examples we will see) that the necessary physical assumptions are all deducible from the interpretation function $r$. I don't know if this should be true in general.

\subsection{Summary}

\begin{itemize}
\item We are assuming given a quantum system $S$ (with a `superposition function' $\sigma$ which hasn't proven useful yet but represents the underlying vector space structure so I feel like it shouldn't be ignored);

\item We will define a \emph{measuring device on $S$} to be composed of
\begin{itemize}
\item A set of measurements $M$,
\item A set of queries $Q$ and a set of physical assumptions $A$ in order to form the frame $F = \braket{Q|A}$,
\item A function $r$ which, given a query $q \in Q$, returns a relation in $\ps(S) \times \ps(S)$, which must be in some sense compatible with the physical assumptions,
\item A compatibility relation $(\vDash) \subseteq M \times F$ defined on the generators by
\begin{equation}
m \vDash q \text{ iff } m \cap r(q) \neq \emptyset. \tag{\ref{defvdash}}
\end{equation}
\end{itemize}
\end{itemize}

Note: the triple $(M,F,\vDash)$ forms what Vickers \cite{topologyvialogic} calls a \emph{topological system}, which allows us to ask if it is spatial, i.e. if it corresponds to a topological space, or if it is localic, i.e. if it corresponds to a locale.

\section{Examples}

\subsection{Schrödinger's Electron}

Consider the Schrödinger's electron experiment. We will now enumerate the ingredients mentioned in the previous section.

Our set of states, in principle, should be $S = \pr \C^2$, since the underlying Hilbert space in this case is $\C^2$.

The set of measurements is given by the following four relations. In what follows, we will use the following notation: $xx$ represents the $x$ axis, i.e. the subspace generated by $(1,0)$; $yy$ represents the $y$ axis; $\text{Else}$ is a stand-in for any 1d subspace that is neither the $x$ axis or the $y$ axis, and $\text{Any}$ is a stand-in for any 1d subspace at all.
\begin{itemize}
\item $0 = \emptyset$,
\item $z^\uparrow = \{(xx,xx), (\text{Else}, xx)\}$,
\item $z^\downarrow = \{(yy,yy), (\text{Else},yy)\}$,
\item $z = \{(xx,xx), (yy, yy), (\text{Else}, xx), (\text{Else}, yy)\}$.
\end{itemize}

There are two fundamental queries: `is the electron's spin up [now]?' and `is the electron's spin down [now]?'. We call these $Z^\uparrow$ and $Z^\downarrow$ respectively, and
\begin{equation}
r(Z^\uparrow) = \{(\text{Any}, xx)\}, \quad r(Z^\downarrow) = \{(\text{Any}, yy)\}.
\end{equation}

We do not need to add any assumptions, though we could try modifying the experiment, adding the assumption that the electron starts (and therefore is always) in a state of spin up or spin down.

Therefore, for the moment, the frame $F$ is given by the diagram
\begin{center}
\begin{tikzpicture}
\node (top) at (0,2) {$\top$};
\node (join) at (0,1) {$Z^\uparrow \lor Z^\downarrow$};
\node (up) at (-1,0) {$Z^\uparrow$};
\node (down) at (1,0) {$Z^\downarrow$};
\node (meet) at (0,-1) {$Z^\uparrow \land Z^\downarrow$};
\node (bot) at (0,-2) {$\bot$};

\draw (top)--(join)--(up)--(meet)--(bot);
\draw (join)--(down)--(meet);
\end{tikzpicture}
\end{center}

Now, to calculate $\vDash$, we need only know how it behaves on the fundamental queries $Z^\uparrow$ and $Z^\downarrow$, because they generate the frame, and it is very easy to conclude:
\begin{itemize}
\item $0$ is not compatible with any element of $F$ except $\top$,
\item $z^\uparrow \vDash Z^\uparrow$ but $z^\uparrow \nvDash Z^\downarrow$,
\item Likewise for $z^\downarrow$,
\item $z \vDash Z^\uparrow$ and $z \vDash Z^\downarrow$, so $z$ is compatible with all elements of $F$ except $\bot$.
\end{itemize}

Note that there are a few small variations of this construction that yield effectively the same result. For example, suppose we replaced $r(Z^\uparrow)$ by
\begin{equation}
r'(Z^\uparrow) = \{(xx,xx), (\text{Else}, xx)\}.
\end{equation}

Then, since there is no measurement containing the pair $(yy,xx)$, the frame F and compatibility relation $\vDash$ would be the same, and so our measurement space would be effectively the same.

Note that (it is trivial to see) the topological system $(M,F,\vDash)$ is a spatial locale, i.e. it corresponds to a sober topological space.

\subsection{Schrödinger's Electron With a `Scramble' Button}

Even though we have been calling the elements of $M$ measurements, the framework allows them to do slightly more: they can be used to modify, or set up, the system. To exemplify this, let us consider Schrödinger's electron, except we add a button (measurement) that scrambles the electron inside the box, giving it a completely new spin.

The set of states $S$ is the same as in the previous experiment, and the set of measurements is nearly the same, with the addition of a `scramble'
\begin{equation}
\xi = \{(\text{Any}, \text{Any})\}.
\end{equation}
(Note: the two `Any's in this example are independent; this is the maximal binary relation, not the diagonal one.)

The set $\{0,z^\uparrow, z^\downarrow, z, \xi\}$ is not closed under composition, so we need to consider all possible compositions, which yields the set $M$.

Note that compositions are independent of process. For example $\xi z^\uparrow \xi = \xi$. This embodies the assumption that the only thing that matters about a measurement is the possibilities it allows or disallows.

Now, we could use the same basic queries $Z^\uparrow$ and $Z^\downarrow$, but the resulting system would have a lot of points and very few opens, and thus would fail to be sober. There are two ways to find the frame $F$ appropriate for this situation.

The first, most obvious way, is to brute force it. There is a finite number of measurements, so we can just add a query for each measurement and add relations regarding how those measurements relate to each other, e.g. $z^\uparrow \cup z^\downarrow = z$, so it would make sense that $Z^\uparrow \lor Z^\downarrow = Z$. This is probably doable, but time-consuming.

The second way is to look at the meaning of the measurements we have. As well as the five basic measurements enumerated above, we have new, weird compositions. Any such composition must have $\xi$ in it, as we've already seen that the other four measurements are closed under composition. This allows us to do things like:
\begin{itemize}
\item The measurement $\xi z^\uparrow$, which means `I scrambled the electron and afterwards measured its spin $z$ to be up'. This corresponds to the relation $\{(\text{Any}, xx)\}$, which is distinct from just $z^\uparrow$, as the latter does not contain $(yy,xx)$.

\item The measurement $z^\uparrow \xi$, which means `I measured the electron's spin $z$ to be up, but scrambled it afterwards'. From this measurement we cannot conclude anything about the current state of the electron, but we can conclude that the state prior to the measurement was definitely not $yy$.

\item The measurement $z^\uparrow \xi z^\downarrow$, from which we can conclude that previously the electron's state was anything but $yy$, but that it is $yy$ now.

\item These three possibilities (along with their versions with $z^\uparrow$ and $z^\downarrow$ replaced by each other or by $z$) are the only possible new measurements, because any expression with more than one $\xi$ is of the form
\begin{equation}
z_1 \xi A \xi z_2,
\end{equation}
where $z_i \in \{0,z^\uparrow,z^\downarrow,z, \id\}$ and $A$ is some measurement. Now, either $A$ is zero (in which case $\xi A \xi = 0$), or it is not (in which case $\xi A \xi = \xi$). As such, the three possibilities mentioned above really do describe all possible (new) measurements.
\end{itemize}

With this in mind, we can conclude that a set of queries that allows us to discern all possible measurements (which is a necessary condition to make this into a locale) would allow us to discern statements about the system prior to and after measurement. Therefore, let us construct four basic queries:
\begin{equation}
Q = \{Z^\uparrow_0, Z^\downarrow_0, Z^\uparrow_1, Z^\downarrow_1\},
\end{equation}
where the subscript $0$ or $1$ represents the possibility of the electron's spin being up/down before or after the measurement, respectively. So, for example, $Z^\uparrow_0$ is compatible with $\xi z^\downarrow$, but $Z^\uparrow_1$ is not. This allows us to define the function $r$, which I will not do explicitly, but will exemplify with $Z^\uparrow_0$ and $Z^\downarrow_1$:
\begin{equation}
r(Z^\uparrow_0) = \{(xx, \text{Any})\}, \quad r(Z^\uparrow_1) = \{(\text{Any}, xx)\}
\end{equation}

Now, these four queries are independent, and so we conclude that $F = \braket{Q}$.

\section{Niceness of These Constructions (or Lack Thereof)}

In what I've done so far, I've mostly set up a language to talk about the possible representations, and done almost no work in restricting them. For example, at the moment, there is no guarantee that the operation of composition is continuous, or for that matter the resulting topological system has no obligation to be spatial or localic. There is also a lot of redundancy. For example, since we're looking for sobriety, we should be able to construct $M$ as the set of homomorphisms $F \to \two$, or equivalently the completely prime filters of $F$. Of course, it isn't enough to say that, because elements of $M$ are supposed to be sets of relations, not filters, so I need to find a way to construct a relation from a filter.

\section{The emergence of the Scott topology}

I will now propose a notion of `biggest possible set of queries', from which the Scott topology on the set of binary relations will naturally arise.

Let us consider, for now, a set of states $S$, and let $M$ be the set of \emph{all} binary relations on $S$. We will find the set of all possible subsets of $M$ that could hypothetically correspond to a query.

For our purposes, a query $q$ is a set of pairs $(s_1, s_2)$, and a measurement $m$ is compatible with $q$ if it is possible that, upon measuring $m$, some element of $q$ was verified, in the sense that there exists $(s_1, s_2) \in q$ such that before measurement the state was $s_1$ and after measurement it was $s_2$.

The set of possible combinations of queries is the frame $F$ freely generated by $\ps(S \times S)$. However, some of these combinations are distinct elements in the frame $F$, but correspond to the same physical property in the sense that the measurements they distinguish are the same. The topology we obtain by looking at the opens corresponding to the frame elements turns out to coincide precisely with the Scott topology, which is perhaps an indication that these ideas aren't entirely far-fetched.

\begin{prop}
Let $T$ be a set. In the remainder of this document, we will mostly use the particular case $T = S \times S$, but it is unnecessary for this proposition. Let $M = \ps(T)$.

Let $F$ be the frame freely generated by $\ps(T)$. Then, we may define a relation $\vDash$ on $M \times F$. by defining it on $M \times \ps(T)$ and extending homomorphically in the right-hand side. Given $m \in M$ and $q \in \ps(T)$, define $m \vDash q$ iff $m \cap q \neq \emptyset$.

Given $a \in F$, define $\ext(a)$ as the set of $m \in M$ such that $m \vDash a$. It is easy to check that $\ext$ is a frame homomorphism $F \to \ps(M)$, and so the image of $\ext$ forms a topology on $M$, which we will call $\Omega$.

Then, the following three topologies on $M$ are equal:
\begin{itemize}
\item The topology $\Omega$,
\item The topology generated by the subbasis $U_{t} = \{\, m \in M \mid t \in m \,\}$, $t \in T$
\item The Scott topology on $\ps(M)$.
\end{itemize}

Furthermore, the subbasis $U_{t}$ freely generates this topology as a frame.
\end{prop}

\begin{proof}\leavevmode
\begin{itemize}
\item (Any open in $\Omega$ is in the topology generated by $U_{t}$.) Since $\ext$ is a frame homomorphism, it is enough to show that $\ext([\mu])$ is open for all $\mu \in \ps(T)$, where the brackets represent the insertion of $\ps(T)$ in $F$. Furthermore, it is easy to see that $[\mu] = \bigvee_{t \in \mu} [\{t\}]$, and so it suffices to show that $\ext([\{t\}])$ is always open. Of course, a trivial check shows that this is precisely equal to $U_{t}$ and we're done.

\item (Any open in the topology generated by $U_{t}$ is open in $\Omega$.) We have already seen that the subbasis elements correspond to opens in $\Omega$, and the proof is complete.

\item (Every open in $\Omega$ is Scott-open.) We need only show that the $U_{t}$ are all Scott-open. They are obviously upper-closed, and given a directed join (or any join) $\bigcup_{a \in A} a$ in  $U_{t}$, we get that some $a$ contains $t$ and therefore must be in $U_{t}$.

\item (Every Scott-open is in $\Omega$.) Let $U$ be Scott-open. We will show that, for all $m \in U$, there exists and $\Omega$-neighbourhood containing $m$ and contained in $U$.

Consider the collection $A$ of finite subsets of $m$. Then, $m = \bigcup_{a \in A} a$. Since $U$ is inaccessible by directed joins and $A$ is directed, there exists a finite subset $a$ of $M$ such that $a \in U$. Then, the neighbourhood we seek is given by $\ext([a])$.

To see why, notice that $n \in \ext([a])$ iff $a \subseteq n$. Therefore, it is certainly a neighbourhood of $m$. Furthermore, it is contained in $U$, because $a \in U$ and $U$ is upper closed. The proof is complete.
\end{itemize}

The only thing left to check is that the topology is freely generated by $U_{t}$. To do so, let us assume given a function $f : T \to F'$ where $F'$ is an arbitrary frame, and we will show there exists exactly one frame homomorphism $\tilde f : \Omega \to F'$ such that $\tilde f(U_t) = f(t)$.

Uniqueness is obvious, as any open in $\Omega$ can be written as
\[V = \bigcup_\alpha \bigcap_{t \in A_\alpha} U_t,\]
for some collection of finite sets $A_\alpha \subseteq T$. Therefore, we necessarily have
\begin{equation}\label{freehomdef}
\tilde f(V) = \tilde f\left(\bigcup_\alpha \bigcap_{t \in A_\alpha} U_t\right) = \bigvee_{\alpha} \bigwedge_{t \in A_\alpha} f(t).
\end{equation}

The only thing we need to check is that this expression is uniquely defined, for if this is the case all other properties (it is a frame homomorphism and $\tilde f(t) = f(U_t)$) are immediately deduced.

To this effect, consider two different ways to write $V$ in terms of the subbasis
\[V = \bigcup_\alpha \bigcap_{t \in A_\alpha} U_t = \bigcup_\beta \bigcap_{r \in B_\beta} U_r.\]

Now, consider $\alpha$ fixed. We have $\bigcap_{t \in A_\alpha} U_t \subseteq \bigcup_\beta \bigcap_{r \in B_\beta} U_r$. Now, the left-hand side is the set of $m \in M$ such that $A_\alpha \subseteq m$. In particular, the left-hand side contains $A_\alpha$ (as an element of $M$), and so $A_\alpha$ is in the right-hand side. Consequently, there exists $\beta$ such that $A_\alpha \in \bigcap_{r \in B_\beta} U_r$, and thus $B_\beta \subseteq A_\alpha$. For this particular $\beta$ we have
\[\bigwedge_{t \in A_\alpha} f(t) \leq \bigwedge_{r \in B_\beta} f(t),\]
and so, taking suprema in the right-hand side followed by the left-hand side we obtain
\[\bigvee_\alpha \bigwedge_{t \in A_\alpha} f(t) \leq \bigvee_\beta \bigwedge_{r \in B_\beta} f(t).\]

Obviously the argument can be reversed, and so we conclude that the result of expression \eqref{freehomdef} does not depend on how $V$ is represented in terms of subbasis elements, and the proof is complete.
\end{proof}

Now, obviously this is a very particular case, but any other measurement space (under the definition given above) corresponds to a subset $M \subseteq \ps(S \times S)$, so given $M$ we can, using the subspace topology, construct the `biggest possible set of queries on $M$', and so for our purposes the topology we give $M$ must necessarily be coarser than this one. In general, if we lack context about the opens we want $M$ to have, we probably want to be most general and give $M$ all possible opens it could have.

\section{Application to Schrödinger's Electron}

We will now use the tools we just constructed to describe how we could rediscover the topology on the set of measurements $M = \{0,z^\uparrow,z^\downarrow,z\}$ corresponding to Schrödinger's Electron.

Since we're considering the subspace topology of $M$ as a subset of $\ps(S \times S)$, with $S = \pr \C^2$, the sub-basic opens are of the form $V_{(s_1,s_2)} = U_{(s_1,s_2)} \cap M$. These sub-basic opens can be easily categorized:
\begin{itemize}
\item If $s_2 = xx$ and $s_1 \neq yy$, then $V_{(s_1,s_2)} = Z^\uparrow = \{z^\uparrow, z\}$,
\item If $s_2 = yy$ and $s_1 \neq xx$, then $V_{(s_1,s_2)} = Z^\downarrow = \{z^\downarrow, z\}$,
\item In any other case, $V_{(s_1,s_2)}$ is empty.
\end{itemize}

Therefore, we obtain the topology we already had.

\section{Next steps}

A possible generalization of this procedure would be the case where $T$ already has a pre-existing topology, in which case we would have considered the frame $F$ generated by the open sets in $T$. The case we investigated was the particular case of the discrete topology. Of course, in our particular cases, $T$ is of the form $S \times S$, and $S$ is a projective space, so $T$ will have a topology which is not the discrete. However, in some sense, Schrödinger's electron is a `discrete' example, so it makes no difference, but if we were to consider measurements on all axes the scenario would be rather different.

A question this raises is whether, given a set of measurements $M$, it would be best to consider the subspace topology by the method we constructed here, or the Scott topology on $M$. There is no reason to expect that the two would always coincide. Is there some reasonable set of conditions we can give $M$ to ensure they do coincide? If not, if we imagine $M$ as an abstract space with operations satisfying certain axioms, would we be able to define the `subspace topology of the Scott topology' in a way that doesn't depend on ambient space?

\section{The Emergence of the Lower Vietoris\\Topology}

\subsection{Identification of measurements}

With some small changes to the current framework, we can reconstruct the $\Max A$ example and the topology associated with it.

First of all, we will need to identify some of the measurements. I'm unsure of why this identification makes physical sense, but it is simple enough and makes the math work out.

Given the projective space $S = \pr V$, define $\sigma : \ps(S) \to \ps(S)$ as the closed span operator, i.e., for $A \subseteq S$,
\[\sigma(A) = \overline \spann A.\]

The above expression isn't exactly correct: to be rigorous we should be taking the (subset of $V$) union of all elements of $A$, taking the closed span (in $V$), and then considering the set of 1-dimensional subspaces contained in this subspace of $V$.

In what follows, we will identify the mesurements $m$ and $n$ if, for all $s \in S$,
\[\sigma(m^\sharp(s)) = \sigma(n^\sharp(s)),\]
where $m^\sharp(s)$ is the set of $t \in S$ such that $(s,t) \in m$.

A possible motivation for this identification is as follows. First of all, define the orthogonal complement operator $A \mapsto A^\perp$ similarly to how we defined the span operator. Then, it is known that in a Hilbert space we can define the closed span operator as
\[\sigma(A) = A^{\perp\perp},\]
and furthermore it is known that
\[A^{\perp\perp\perp} = A^\perp,\]
so we conclude the equivalence
\[\sigma(A) = \sigma(B) \text{ iff } A^\perp = B^\perp.\]

Now, we can give our identification a physical interpretation. Given $m \in M$, define $m^\perp$ as the relation defined by
\[(s_1, s_2) \in m^\perp \text{ iff } s_2 \in (m^\sharp(s_1))^\perp.\]

Intuitively, we say $(s_1, s_2) \in m^\perp$ if the measurement $m$ is incompatible with the scenario where the state was previously $s_1$ and we now make a measurement concluding the system is in state $s_2$. For example, if $m = z^\uparrow$ in the Schrödinger's Electron example, $m^\perp$ contains exclusively the pairs of the form $(\text{Any}, z^\downarrow)$. This is because, for example, after taking measurement $m$, it would still be possible to take a measurement concluding that the spin, say, $x$ of the electron is in the down direction. The only possibility that is truly eliminated is that we measure the electron's $z$ spin to be down.

This could probably be slightly more fleshed out, perhaps by adding a notion of a `pure measurement' (which corresponds to an orthogonal projection in the Hilbert space) and identifying measurements $m$ and $n$ if $mp = np$ for all pure measurements. Furthermore, this identification opens up some potential for ill-definedness of composition, as we need to ensure that if $m$ and $n$ are identified then all compositions involving them are well-defined. However, for the following example that will not be problematic.

\subsection{Measurements corresponding to spaces of matrices}

The most `pure' possible mesurements are given by attempting to specify the system's state as precisely as possible. Consider a state $p \in S$. Then, it is possible to measure that the system is in state $p$. For example, if $p = z^\uparrow$ in Schrödinger's Electron, we can use a Stern-Gerlarch apparatus to measure the electron's spin, and once we measure it to be upwards in the $z$ direction, the electron's state is collapsed and equal to $p$.

In terms of relations, this can be realized as a relation whose elements are all of the form $(q, p)$, for $q$ in some set. Which set should this be? Well, the (generalized) probability of measuring $p$ given that the system was previously in state $q$ is given by $\braket{q|p}$, so we should include precisely the $q$ such that this probability is not null. In other words, our relation will be
\[m_p := \{\, (q,p) \mid q \not \in \{p\}^\perp\,\}.\]

It is natural to consider the orthogonal projection $\pi$ on $p$, and $m_p$ can be written in these terms as
\[m_p = \{\, (q, \pi q) \mid \pi q \neq 0\,\},\]
where the condition that $q$ is not orthogonal to $p$ now arises naturally by demanding that $\pi q$ is an element of the projective space (i.e. a 1-dimensional subspace).

A natural generalization of this concept is to allow for arbitrary linear transformations, yielding the definition, for $T : \HH \to \HH$ linear,
\[m_T := \{\, (q, Tq) \mid Tq \neq 0\,\}.\]

Finally, if we wish to make the set of these measurements closed under joins, we might define, for arbitrary sets $U$ of linear transformations,
\[m_U := \{\, (q, Tq) \mid T \in U, Tq \neq 0 \,\},\]
but some elementary point set topology shows that, under our identification, this is identified with $m_{\braket U}$, where $\braket U$ is the closed linear subspace (of the space of endomorphisms of $V$) generated by $U$.

In this way, we naturally get a space corresponding to the set $\Max \HH$\footnote{In truth, it could correspond to a quotient of it, as we have not shown that if $U$ and $V$ are two elements of $\Max \HH$ then $m_U = m_V$ iff $U = V$.}. We have already described how to compose measurements, and it is trivial to see that, under our identifications, we get the relation
\[m_U m_V = m_{UV} = m_{\braket{UV}},\]
as it should be.

So, in conclusion: we have a function $m : \Max \HH \to M$, which is a homomorphism in the sense that $m_U m_V = m_{\braket{U V}}$. One last thing we might want to show is that this is a one-to-one identification, that is, if $m_U = m_V$ then $U = V$.

%Let $\HH = \C^2$, and consider the following two elements of $\Max \HH$


\bibliography{bibliography}{}
\bibliographystyle{plain}

\end{document}
