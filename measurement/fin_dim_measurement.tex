\documentclass{article}

\usepackage[utf8]{inputenc}

\usepackage{amsmath}
\usepackage{amssymb}
\usepackage{amsfonts}

\usepackage[thmmarks]{ntheorem}
\usepackage{enumitem}

\usepackage{braket}

\usepackage{cite}

\usepackage{tikz}


\title{Measurement Spaces in Finite Dimension}
\author{Duarte Maia}

\theoremstyle{plain}
\theoremseparator{.}
\theorembodyfont{\rm}
\newtheorem{problem}{Problem}
\newtheorem{definition}{Definition}

\newtheorem{prop}{Proposition}
\renewtheorem*{prop*}{Proposition}
\newtheorem*{idea}{Idea}

\theoremstyle{nonumberplain}
\theorembodyfont{\upshape}
\theoremheaderfont{\scshape}
\theoremsymbol{\ensuremath{\blacksquare}}
\newtheorem{proof}{Proof}

\DeclareMathOperator{\spann}{span}
\DeclareMathOperator{\ext}{ext}
\DeclareMathOperator{\Max}{Max}
\DeclareMathOperator{\Aut}{Aut}

\newcommand{\R}{\mathbb{R}}
\newcommand{\C}{\mathbb{C}}


\newcommand{\HH}{\mathcal{H}}

\newcommand{\ps}{\mathcal{P}}
\newcommand{\pr}{\mathbb{P}}


\newcommand{\id}{\mathrm{id}}


\newcommand{\two}{\mathbf{2}}

\begin{document}
\maketitle

\section{Showing that $\Max A$ is a Measurement Space, when $A$ is an Involutive Algebra of Matrices}

First, we recall the definitions at play. Some of these are not done with the greatest generality for the sake of simplicity.

\begin{definition}
Let $V$ be a finite dimensional complex vector space. We denote by $\Max V$ the topological space defined as follows:

\begin{itemize}
\item The points of $\Max V$ are the subspaces of $V$,
\item The topology on $\Max V$ is the one generated by the following subbasis: For each open $U$ of $V$, construct $\tilde U \subseteq \Max V$ as
\[\tilde U = \{\, \text{$m$ subspace of $V$} \mid m \cap U \neq \emptyset\,\}.\]
\end{itemize}
\end{definition}

\begin{definition}\label{defmaxa}
Let $A$ be a finite dimensional involutive complex algebra. In this case, $\Max A$ inherits the following operations from $A$:
\begin{itemize}
\item The multiplication of elements of $A$ induces a multiplication of elements of $\Max A$, by taking pairwise products and then considering the span;
\item The involution of $A$ induces involution on elements of $\Max A$, by taking pointwise involutions.
\end{itemize}
\end{definition}

\begin{definition}
Let $X$ be a topological space. A nonempty closed set $C \subseteq X$ is said to be irreducible if it cannot be written as the union of two closed sets, neither of which is equal to $C$. To each point $x \in X$ corresponds the irreducible closed set $\overline{\{x\}}$. We say that $X$ is sober if this correspondence is a bijection.
\end{definition}

\begin{definition}
Let $X$ be a topological space. We order its points by the specialization order: $x \leq y$ iff every open which contains $x$ also contains $y$. This is a partial order so long as $X$ is $T_0$.
\end{definition}

\begin{definition}
A measurement space is a topological space $M$ such that:
\begin{enumerate}[label=\roman*)]
\item \label{msober} $M$ is sober,
\item \label{leastelem} $M$ contains a least element in the specialization order, called $0$,
\item \label{binsup} The specialization order on $M$ is closed under binary suprema,
\item \label{contsup} The supremum operation $\lor$ is a continuous function $M \times M \to M$,
\item \label{binop} $M$ is equipped with a continuous binary operation called `composition', satisfying the axioms:
\begin{enumerate}
\item \label{assoc} $(mn)p = m(np)$, \quad \textit{(associativity)}
\item \label{dist} $(m \lor n)p = mp \lor np$, \quad \textit{(distributivity)}
\item \label{absorp} $0p = 0$,
\end{enumerate}
\item \label{inv1} $M$ is equipped with a continuous unary operation $(\cdot)^*$ which is an involution,
\item \label{inv2} $(mn)* = n^* m^*$, and
\item \label{symmetry} If $m m^* m \leq m$ then $m m^* m = m$. \quad \textit{(symmetry)}
\end{enumerate}
\end{definition}

The goal of this chapter is to show the following proposition:

\begin{prop*}
Let $A$ be an involutive algebra of matrices, that is, a linear subspace of $M_{n \times n}(\C)$ which is closed under products and conjugate transposition. Then, $\Max A$, equipped with the operations of definition \ref{defmaxa}, is a measurement space.
\end{prop*}

\subsection{The Specialization Order}

First, we study the specialization order on $\Max A$, in order to prove properties \ref{leastelem}, \ref{binsup} and \ref{contsup}. The following results are valid in the case where $A$ is a finite dimensional vector space; the algebra operations are not relevant.

Indeed, the specialization order is very simply the subset order.

\begin{prop}
Let $m$ and $n$ be elements of $\Max A$, that is, subspaces of $A$. Then, $m \leq n$ iff $m \subseteq n$.
\end{prop}

\begin{proof}
One of the implications is obvious. Indeed, if $m \subseteq n$, any subbasic open $\tilde U$ which contains $m$ must also contain $n$, for if $U$ intersects $m$ it also intersects $n$. This result can be extended to arbitrary opens, as if $W$ is an open in $\Max A$, and $m \in W$, then there exists an intersection of subbasic opens $B = \tilde U_1 \cup \dots \tilde U_k$ with $m \in B \subseteq W$, and trivially $n \in B \subseteq W$ as well.

Now, suppose that $m$ is \emph{not} a subset of $n$. We will construct an open which contains $m$ but not $n$.

Since $m \nsubseteq n$, there is a point $x \in m \setminus n$. Now, since we are in a finite dimensional space, the complement of $n$ is an open set, let us call it $U$. Then, $\tilde U$ is an open set in $\Max A$, which obviously contains $m$ but not $n$.
\end{proof}

With this in mind, properties \ref{leastelem} and \ref{binsup} become trivial. The least element is simply the zero subspace, and binary suprema are given by subspace sums. Slightly less trivial is \ref{contsup}:

\begin{prop}
Let $A$ be a finite dimensional complex vector space. Then, the operation of subspace addition
\[ + \colon \Max A \times \Max A \to \Max A\]
is continuous.
\end{prop}

\begin{proof}
To show that subspace addition is continuous, by unfurling the definition of product topology and subbasis it suffices to prove the following. Let $m,n \in \Max A$, and let $\tilde U$ be a subbasic neighborhood of $m+n$. Then, we wish to show that there exist neighborhoods $V$ and $W$ of $m$ and $n$ such that, for all $m' \in V$ and $n' \in W$, $m'+n' \in \tilde U$.

To this effect, let us introduce an auxiliary tool into the problem. Since $A$ is a finite dimensional complex vector space, we may introduce on it a norm (simply pick an arbitrary basis and induce the Euclidean norm based on it), and it is known that all norms are equivalent on finite dimensional vector spaces.

Now, suppose that $\tilde U$ is a subbasic neighborhood of $m+n$. Then, there exists some point $x+y \in U$, with $x \in m$ and $y \in n$. Consequently, some ball around $x+y$ is contained in $U$, with a positive radius $\delta$. That said, we may now define $V_0$ and $W_0$ as balls, around $x$ and $y$ respectively, of radius $\delta/2$. Then, it is trivial to check that $V = \tilde{V_0}$ and $W = \tilde{W_0}$ are the desired neighborhoods of $m$ and $n$.
\end{proof}



\nocite{measurement}

\bibliography{bibliography}{}
\bibliographystyle{plain}

\end{document}
