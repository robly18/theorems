\documentclass{article}

\usepackage{amsmath}
\usepackage{amssymb}
\usepackage{amsfonts}
\usepackage{mathtools}
\usepackage[utf8]{inputenc}

\usepackage{graphicx}

\usepackage{diffcoeff}
\diffdef{}{op-symbol=\mathrm{d},op-order-sep=0mu}

\usepackage{cancel}

\usepackage{enumitem}

\setlist[enumerate,1]{label=\alph*)}

\newcommand{\Z}{\mathbb{Z}}
\newcommand{\R}{\mathbb{R}}
\newcommand{\C}{\mathbb{C}}
\renewcommand{\H}{\mathbb{H}}
\renewcommand{\O}{\mathbb{O}}
\renewcommand{\S}{\mathbb{S}}

\newcommand{\PP}{\mathbb{P}}
\newcommand{\FF}{\mathcal{F}}

\newcommand{\I}{\mathrm{i}}
\newcommand{\e}{\mathrm{e}}

\title{Reais, Complexos, Quaterniões, e mais: A construção de Cayley-Dickinson}
\author{Duarte Maia}

\begin{document}

\maketitle

Ao longo da história, os matemáticos criaram e estudaram diversos géneros de números. Os mais úteis no dia-a-dia são os números inteiros $\Z$, reais $\R$ e complexos $\C$, mas existem outros conjuntos, estes menos conhecidos mas ainda com utilidade, como os quaterniões $\H$ e os octoniões $\O$.

O objetivo deste artigo é falar destes conjuntos mais complicados, não sob a perspetiva de aplicações, mas sim sobre a perspetiva de como os construir e como essa construção pode ser generalizada, dando azo a conjuntos ainda maiores, como os sedeniões $\S$ e mais.

Começando com os números reais, estes conjuntos podem ser vistos como construídos uns em cima dos outros. Os complexos podem ser construídos a partir dos reais adicionando um elemento $i$ que satisfaz $i^2 = -1$. Os quaterniões podem ser vistos como adicionando mais dois elementos, $j$ e $k$, também satisfazendo $j^2 = k^2 = -1$. Os octoniões adicionam ainda mais quatro raízes quadradas de $-1$, os sedeniões adicionam oito, etc.

Para perceber melhor estas construções, é conveniente começar pelo caso mais simples: Como podemos construír $\C$ a partir de $\R$? Esta questão poderá parecer supérflua, dado que todos sabemos o que são os números complexos, mas é uma questão importante porque se soubermos construir os complexos a partir dos reais poderemos tentar iterar essa construção, fazendo surgir os quaterniões e assim por diante. Assim sendo, apresentam-se algumas definições do conjunto dos números complexos $\C$:
\begin{enumerate}[label=\roman*)]
\item Definição `à engenheiro': Os números complexos são coisas da forma $a+bi$, com $a, b \in \R$ e $i^2 = -1$,
\item Definição rigorosa simples: Identificar os complexos com $\R^2$, ou seja, ver o número complexo $a+bi$ como o par $(a,b)$. Definir a soma e o produto de complexos como intuido pela regra $i^2 = -1$, i.e.
\begin{align*}
(a,b)+(c,d) &= (a+c,b+d),\\
(a,b) \cdot (c,d) &= (ac-bd, bc+ad).
\end{align*}
\item Definição simbólica usando construções da álgebra. Resume-se a uma versão mais rigorosa da definição à engenheiro. A sua explicação requer falar de anéis de polinómios e quocientes, por isso regista-se apenas a definição simbólica:
\[\C \cong \frac{\R[x]}{\langle x^2 - 1 \rangle},\]
\item Finalmente, a minha definição preferida: A definição matricial. É sobre esta definição que o resto do artigo incide, por isso explicá-la-ei em detalhe.
\end{enumerate}

Existem várias formas de intuir à definição matricial. A mais elementar é reparar que a matriz
\[J = \begin{bmatrix} 0 & 1 \\ -1 & 0 \end{bmatrix}\]
satisfaz a identidade $J^2 = -I$, que é semelhante a $i^2 = -1$. Assim sendo, podemos identificar os números complexos com um conjunto de matrizes (reais) $2 \times 2$:
\[a + bi \approx a I + b J.\]

Por outras palavras, definimos $\C$ como o conjunto das matrizes $2 \times 2$ da forma
\[\C = \left\{ \begin{bmatrix} a & b \\ -b & a \end{bmatrix} \;\middle|\; a, b \in \R \right\} \subseteq M(2 \times 2, \R).\]

As operações neste conjunto são as operações induzidas pela estrutura de matriz, sendo necessário verificar que somas e produtos de matrizes desta forma continuam a ser desta forma. Pelo lado positivo, recebemos algumas propriedades de graça, por exemplo a associatividade e distributividade do produto (mas não a comutatividade, que tem de ser verificada manualmente).

Um detalhe agradável é que várias operações nos números complexos podem ser vistas como induzida a partir de operações de matrizes. Por exemplo:
\begin{enumerate}
\item A operação de conjugado $z \mapsto \overline{z}$ coincide com a transposição de matrizes, sendo imediata a propriedade $\overline{zw} = \overline{z} \cdot \overline{w}$ da regra $(AB)^T = B^T A^T$,
\item A norma pode ser calculada como $\lvert z \rvert^2 = \det z$, vindo de graça propriedades como $\lvert z w \rvert = \lvert z \rvert \lvert w \rvert$.
\end{enumerate}

\medskip

O conjunto dos quaterniões é uma coleção de números `acima dos complexos', da mesma forma que os complexos estão `acima dos reais'. Há também várias formas de os definir. Historicamente, eles foram concebidos de forma semelhante à definição `à engenheiro' dos complexos: Os quaterniões são coisas da forma $a+bi+cj+dk$, em que $i^2 = j^2 = k^2 = -1$ e, crucialmente, a multiplicação entre estes novos `números' satisfaz as regras
\[ ij = k = -ji , jk = i = -kj, ki = j = -ik.\]

Repare-se no facto que os quaterniões \emph{não são comutativos para a multiplicação}. Devido a isto, seria difícil adivinhar a definição de quaterniões sem saber da sua existência. Isto também dificulta a adaptação das definições dadas acima. Por exemplo, não é imediato adaptar a definição ii), pois a mesma demonstração usada para justificar que os complexos são comutativos poderia ser usada para justificar que os quaterniões o são.

Entra então a definição matricial. Por analogia, vemos os quaterniões como uma coleção de matrizes $2 \times 2$ de entradas complexas da forma
\[\begin{bmatrix} a & b \\ -b & a \end{bmatrix}, a, b \in \C,\]
mas não é assim tão simples...

A primeira complicação surge porque, tal como com a definição ii),  a mesma demonstração usada em $\C$ serve para justificar que estas matrizes são todas comutativas. Outra complicação surge porque algumas das regras que tão bem funcionavam anteriormente deixam agora de funcionar. Por exemplo, antes tinhamos que o determinante de uma matriz nos dava a sua norma $a^2 + b^2$, e por arrasto qualquer matriz não-nula tinha inverso (que era preciso verificar estar sempre em $\C$). No entanto agora o determinante, que continua a ser $a^2 + b^2$, nem sempre é diferente de zero, por exemplo para a matriz $\begin{bmatrix} 1 & i \\ -i & 1 \end{bmatrix}$. Consequentemente, os `quaterniões' resultantes nem sempre permitem divisão.

Isto sugere tentar encontrar alguma modificação que faça com que o determinante seja $\lvert a \rvert^2 + \lvert b \rvert^2$. Para esse efeito, podemos tentar conjugar uma das linhas da matriz de modo ao determinante ser $a \overline{a} + b \overline{b}$, o que leva, por exemplo, à definição
\[\H = \left\{ \begin{bmatrix} a & b \\ -\overline{b} & \overline{a} \end{bmatrix} \;\middle|\; a, b \in \C \right\} \subseteq M(2 \times 2, \C).\]
(Nota: Os quaterniões chamam-se $\H$ em honra de Sir William Rowan Hamilton, que os descobriu em 1843.)

Agora sim, tudo o que foi feito para os complexos se transpõe para os quaterniões:
\begin{enumerate}[label=\roman*)]
\item A associatividade do produto vem da associatividade da multiplicação de matrizes,
\item Idem para distributividade,
\item O conjugado quaterniónico é induzido pelo transposto conjugado, isto é, $\overline{w} = w^*$,
\item O determinante dá o quadrado da norma, etc.
\end{enumerate}

Tudo funciona como esperado, exceto a demonstração de comutatividade, visto que os conjugados na segunda linha estragam alguma da simetria necessária para essa prova.

O processo não precisa de parar aqui. Podemos iterar e construir os chamados octoniões:
\begin{equation}\label{oct1}
\begin{bmatrix} a & b \\ -\overline{b} & \overline{a} \end{bmatrix}, a, b \in \H,
\end{equation}
mas mais uma vez encontramos um problema.

É desejado que o produto de dois octoniões seja também um octonião. No entanto, se calcularmos o produto de duas matrizes da forma \eqref{oct1}, obtemos
\[\begin{bmatrix} a & b \\ -\overline{b} & \overline{a} \end{bmatrix} \begin{bmatrix} c & d \\ -\overline{d} & \overline{c} \end{bmatrix} = \begin{bmatrix} ac - b \overline{d} & ac + b \overline{c} \\ -\overline{b}c - \overline{a} \overline{d} & -\overline{b} d + \overline{a} \, \overline{c} \end{bmatrix}\]
e repare-se que \emph{em geral, a matriz obtida não é da forma} \eqref{oct1}, dado que
\[\overline{ac - b\overline{d}} = \overline{c} \, \overline{a} - d \overline{b},\]
que não é igual a $\overline{a}\,\overline{c} - \overline{b} d$ \emph{a não ser que a multiplicação das entradas seja comutativa}. Isto explica porque é que isto não foi um problema até os octoniões, mas passa a ser a partir destes.

A solução parece excessiva, mas consiste em modificar ligeiramente a noção de produto de matrizes. A noção usual de produto de matrizes $2 \times 2$ é dada por
\[\begin{bmatrix} a_1 & b_1 \\ c_1 & d_1 \end{bmatrix} \begin{bmatrix} a_2 & b_2 \\ c_2 & d_2 \end{bmatrix} = \begin{bmatrix} a_1 a_2 + b_1 c_2 & a_1 b_2 + b_1 d_2  \\ c_1 a_2 + d_1 c_2 & c_1 b_2 + d_1 d_2 \end{bmatrix}\]
e para prosseguir com o nosso processo substituimo-la pela operação subtilmente diferente
\[\begin{bmatrix} a_1 & b_1 \\ c_1 & d_1 \end{bmatrix} \circ \begin{bmatrix} a_2 & b_2 \\ c_2 & d_2 \end{bmatrix} = \begin{bmatrix} a_1 a_2 + c_2 b_1 & b_2 a_1 + b_1 d_2  \\ a_2 c_1 + d_1 c_2 & c_1 b_2 + d_2 d_1 \end{bmatrix}\]
em que a única mudança é a ordem de alguns produtos em algumas casas. Claro que esta mudança não faz diferença se a multiplicação dos elementos das matrizes for comutativa, mas passa a fazer diferença precisamente quando os escalares são os quaterniões.

Felizmente, esta é a última mudança necessária para definir a construção de Cayley-Dickinson: Se $\mathbb{A}_n$ é um conjunto no qual está definido uma soma, um produto e uma operação de conjugação, definimos $\mathbb{A}_{n+1}$ como sendo o conjunto das matrizes da forma
\[\mathbb{A}_{n+1} = \left\{ \begin{bmatrix} a & b \\ -\overline{b} & \overline{a} \end{bmatrix} \;\middle|\; a, b \in \mathbb{A}_n \right\} \subseteq M(2 \times 2, \mathbb{A}_n),\]
em que definimos a soma em $\mathbb{A}_{n+1}$ como a soma usual de matrizes, o produto como o produto modificado de matrizes $A \circ B$, e o conjugado como o transposto conjugado (transpôr a matriz e conjugar todos os seus elementos).

Começando este processo com $\mathbb{A}_0 = \R$ obtemos $\mathbb{A}_1 = \C$, $\mathbb{A}_2 = \H$, $\mathbb{A}_3 = \O$, $\mathbb{A}_4 = \mathbb{S}$ (um conjunto de números chamados sedeniões), e assim por diante.

Uma propriedade interessante desta sequência de conjuntos é que a regularidade da multiplicação vai ficando cada vez pior. Em todos os casos, a soma é comutativa e associativa, e verifica-se a propriedade distributiva $z(w_1 + w_2) = z w_1 + z w_2$. No entanto, verificar que $\mathbb{A}_{n+1}$ tem uma certa propriedade requer que $\mathbb{A}$ tenha uma propriedade mais forte, por exemplo:
\begin{enumerate}[label=\roman*)]
\item Para justificar que $\C$ é comutativo, é necessário usar o facto que em $\R$, $\overline{x} = x$,
\item Para justificar que $\H$ é associativo (i.e. $(xy)z = x(yz)$) é necessário usar o facto que $\C$ é comutativo,
\item Consequentemente, não é possível provar (porque não é verdade) que os octoniões $\O$ são associativos... Mas é possível justificar uma propriedade mais fraca, chamada alternância, que diz que se $z$ e $w$ são octoniões temos $z(zw) = (zz)w$ e $z(ww) = (zw)w$. Para justificar esta propriedade, é necessária a associatividade de $\H$,
\item Pelo que alternância não se verifica nos sedeniões $\mathbb{S}$...
\end{enumerate}

A construção de Cayley-Dickinson não é normalmente exposta desta forma, assemelhando-se mais à definição dos complexos como pares. Normalmente define-se $\mathbb{A}_{n+1}$ como sendo o conjunto dos pares $(a,b)$, $a,b \in \mathbb{A}_n$, com multiplicação definida de certa forma. No entanto, a definição matricial tem o benefício de nos permitir usar certas ferramentas de álgebra linear, como determinantes, que nos podem dar informação extra sobre elementos destes objetos novos. O maior inconveniente é a necessidade de adaptar algumas das operações de modo aos $\mathbb{A}_n$ terem comportamentos razoáveis, mas felizmente este é um processo que converge após três passos.

\end{document}