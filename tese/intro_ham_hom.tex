\documentclass{article}

\usepackage{amsmath}
\usepackage{amssymb}
\usepackage{amsfonts}
\usepackage{mathtools}

\usepackage[thmmarks, amsmath]{ntheorem}

\usepackage{graphicx}

\usepackage{tikz-cd}

\usepackage{diffcoeff}
\diffdef{}{op-symbol=\mathrm{d},op-order-sep=0mu}

\usepackage{cancel}
\usepackage{interval}
\usepackage{mleftright}

\usepackage[inline]{enumitem}
\SetEnumitemKey{algorithm}{label={Step \Roman*.}, labelindent=0pt, labelsep=!, leftmargin=*, align=left,itemindent=0pt}
\setlist[enumerate,1]{label=\roman*)}

\usepackage{cite}

\usepackage{showlabels}

\title{A Short Introduction to Hamiltonian Persistence Homology}
\author{Duarte Maia}
%\date{}

\theorembodyfont{\upshape}
\theoremseparator{.}
\newtheorem{theorem}{Theorem}
\newtheorem{prop}{Proposition}
\newtheorem{corollary}{Corollary}
\renewtheorem*{prop*}{Prop}
\newtheorem{lemma}{Lemma}
\newtheorem{definition}{Definition}
\newtheorem{remark}{Remark}

\theoremstyle{nonumberplain}
\theoremheaderfont{\itshape}
\theorembodyfont{\upshape}
\theoremseparator{:}
\theoremsymbol{\ensuremath{\blacksquare}}
\newtheorem{proof}{Proof}

\theoremsymbol{\ensuremath{\text{\textit{(End proof of lemma)}}}}
%\theoremsymbol{\ensuremath{\square}}
\newtheorem{lemmaproof}{Proof of lemma}

%Sets of numbers
\newcommand{\R}{\mathbb{R}}
\newcommand{\C}{\mathbb{C}}
\newcommand{\Z}{\mathbb{Z}}

\newcommand{\FF}{\mathbb{F}} %Arbitrary field

%Contractible orbits
\newcommand{\LL}{\mathrm{L}}
%Action functional
\renewcommand{\AA}{\mathrm{A}}
%Generic Barcode
\newcommand{\BB}{\mathrm{B}}

%Floer Complex and Homology
\newcommand{\CF}{\mathrm{CF}}
\newcommand{\HF}{\mathrm{HF}}

%Morse Complex
\newcommand{\MC}{\mathrm{C}}

\newcommand{\I}{\mathrm{i}}
\newcommand{\e}{\mathrm{e}}


\DeclareMathOperator{\sign}{sign}
\let\Im\relax
\DeclareMathOperator{\Im}{Im}
\let\Re\relax
\DeclareMathOperator{\Re}{Re}

\newcommand{\Ix}{\mathop{\mathrm{I\mkern0.7mu x}}}
%\newcommand{\Ix}{\mathop{\mathrm{ix}}}


\newcommand{\id}{\mathrm{id}}


\DeclareMathOperator{\Ind}{Ind}


\DeclareMathOperator{\inte}{int}
\DeclareMathOperator{\codim}{codim}
\newcommand{\grad}{\nabla}
\newcommand{\into}{\mathbin{\lrcorner}}

\DeclarePairedDelimiter{\norm}{\lvert}{\rvert}
\DeclarePairedDelimiter{\Norm}{\lVert}{\rVert}
\DeclarePairedDelimiter{\abs}{\lvert}{\rvert}
\DeclarePairedDelimiter{\braket}{\langle}{\rangle}

\newcommand{\conj}[1]{\overline{#1}}
\newcommand{\transposed}{\top}
\DeclareMathOperator{\trace}{Tr}


\begin{document}
\maketitle

\pagebreak

\tableofcontents

\pagebreak

\section{Introduction}

This is a first draft and not intended to be quality work. It is heavily based on \cite{polterovich}.

To do: Add a motivating example.

\section{Persistence Homology 101}

\begin{definition}
A \emph{persistence module} is a collection of finite-dimensional vector spaces $V_t$ over a field $\FF$, for $t \in \R$, as well as a collection of linear maps
\begin{equation}
\pi_{ts} \colon V_t \to V_s, \quad t\leq s,
\end{equation}
satisfying the rules $\pi_{sr} \circ \pi_{ts} = \pi_{tr}$ and $\pi_{tt} = \id_{V_t}$.

The examples that will be discussed in the sequence also satisfy the following assumptions.

\begin{enumerate}[label=\roman*)]
\item\label{pm1} For all but finitely many $t \in \R$, there exists a neighborhood $U$ of $t$ such that $\pi$ is an isomorphism for all pairs of indices in $U$,
\item\label{pm2} \textit{(Semicontinuity)} For all $t \in \R$, there exists $\varepsilon > 0$ such that $\pi_{tr}$ is an isomorphism for $t-\varepsilon < r \leq t$.
\end{enumerate}

It is easy to show as a consequence of \ref{pm1} that $V_t$ takes finitely many values up to isomorphism. In particular, for $t$ close enough to $+\infty$, all $V_t$ can be simultaneously identified with the same vector space, which we call $V_+$. The same could be done for $t$ close to $-\infty$ to identify all such $V_t$ with a vector space $V_-$. However, we add the following hypothesis, which trivializes $V_-$:

\begin{enumerate}[resume*]
\item\label{pm3} For $t$ close enough to $-\infty$, $V_t = 0$.
\end{enumerate}

A persistence module which satisfies \ref{pm1}, \ref{pm2} and \ref{pm3} is said to be \emph{of finite type}.
\end{definition}

It is easy to tell from the definition of persistence module of finite type  that such modules have very simple structure.

Let $\sigma_1 < \dots < \sigma_N$ be the finite collection of real numbers that property \ref{pm1} refers to, i.e. those around which there is no neighborhood on which $\pi$ is an isomorphism. A simple topological argument shows that $\pi_{ts}$ is an isomorphism for $\sigma_i < t \leq s < \sigma_{i+1}$, $i = 1, \dots, N-1$. In other words, all such $V_t$, $t \in \ointerval{\sigma_i}{\sigma_{i+1}}$ can be canonically identified. The same argument holds for $i = 0$ and $i = N$, if we add the conventions that $\sigma_0 = -\infty$ and $\sigma_{N+1} = \infty$. In other words, property \ref{pm1} is ensuring that, in some sense, $V_t$ `only takes finitely many values'.

In turn, property \ref{pm2}, semicontinuity, is telling us what the behavior at the border points looks like. It guarantees that each $V_{\sigma_i}$ is the same as those for $t$ slightly below it. In other words, while property \ref{pm1} guarantees that $V_t$ is essentially the same for $t \in \ointerval{\sigma_i}{\sigma_{i+1}}$, semicontinuity adds the upper extremum, ensuring that $V_t$ is canonically isomorphic for $t \in \linterval{\sigma_i}{\sigma_{i+1}}$.

Note the edge case $i = N$, in which we obtain an interval of the form $\linterval{\sigma_N}{+\infty}$. This suggests that closing the real line on the right, considering $\R \cup \{+\infty\}$ instead of $\R$, might be a natural choice, but instead we choose to follow \cite{polterovich} and simply adopt the convention that an interval of the form $\linterval{a}{+\infty}$ should be taken to mean the same as $\ointerval{a}{+\infty}$.

Finally, property \ref{pm3} guarantees that the very first of these vector spaces, $V_t$ for $t \leq \sigma_1$, is the trivial space.

It is possible to continue this study, obtaining ever simpler forms of representing finite-type persistence modules, culminating in the so-called Normal Form Theorem, which states that these persistence modules can be represented by particularly simple objects called barcodes.

\begin{definition}
A \emph{barcode} is a finite multiset of intervals in $\R$, of the form $\linterval a b$, where $-\infty < a < b \leq +\infty$. Keep in mind the convention: $\linterval a {+\infty}$ means the same as $\ointerval a {+\infty}$.
\end{definition}

\begin{definition}
Let $B$ be a barcode. We define $\FF(B)$ to be the persistence module defined as follows.

Index the intervals in $B$ as $I_1, I_2, \dots, I_N$. For each $t \in \R$, let $\Ix(t)$ be the collection of indices $i$ such that $t \in I_i$. Define $\FF(B)_t$ as the vector space generated by $\Ix(t)$, i.e. which has $\Ix(t)$ as a basis.

We now define $\pi_{ts}$, for $t \leq s$. To this effect, note that, since $\Ix(t)$ is a basis for $\FF(B)_t$, it suffices to define $\pi_{ts}(i)$ for $i \in \Ix(t)$. We define it as
\begin{equation}
\pi_{ts}(i) = \begin{cases} i & i \in \Ix(s),\\ 0 & \text{otherwise}. \end{cases}
\end{equation}
\end{definition}

It is easy to check that $\FF(B)$ is a finite-type persistence module. Properties of intervals guarantee that the $\pi$ functions compose as they should. The fact that there are finitely many intervals ensures property \ref{pm1}. The fact that the intervals are open on the left and closed on the right ensures property \ref{pm2}, semicontinuity, and the fact that the lower extremum cannot be $-\infty$ ensures property \ref{pm3}.

The normal form theorem ensures that any finite-type persistence module is isomorphic to $\FF(B)$ for some unique barcode $B$.

\begin{definition}
A homorphism $h \colon V \to W$ of persistence modules is a collection of linear maps $h_t \colon V_t \to W_t$ such that all diagrams of the following form commute.
\begin{equation}
\begin{tikzcd}
V_t \arrow[d, "h_t"] \arrow[r, "\pi_{ts}"] & V_s \arrow[d, "h_s"] \\
W_t \arrow[r, "\pi_{ts}"]                  & W_s                 
\end{tikzcd}
\end{equation}

An isomorphism of persistence modules is a diffeomorphism $h$ of persistence modules such that each $h_t$ is an isomorphism. In this case, the inverse $h^{-1}$ is also an isomorphism of persistence modules.
\end{definition}

\begin{theorem}[Normal form theorem]
Let $V$ be a persistence module of finite type. Then, there exists a unique barcode $B$ such that $V$ is isomorphic to $\FF(B)$.
\end{theorem}

The normal form theorem allows us to represent certain types of data (time-dependent vector fields) as a simple combinatorial object (barcodes). Therefore, if we have some invariant which is a vector field (like, say, homology over a field) and some space which can be `grown continuously from the empty set' (like a filtration of a manifold using a Morse function), we can construct a persistence module and therefore (with finiteness assumptions) a corresponding barcode.

\section{Barcodes from Filtered Floer Homology}

\subsection{Introduction to Floer Homology}

Let $(M, \omega)$ be a compact aspherical symplectic manifold. Let $\LL M$ be the space of smooth contractible loops on $M$, i.e. smooth maps $S^1 \to M$ which can be extended to the disk. Note that by the asphericity condition any such extensions are homotopic.

Let $H$ be a smooth Hamiltonian on $M$. Through it, define the so-called action functional on $\LL M$. For $x \in \LL M$, set
\begin{equation}
\AA(x) = - \int_D \omega + \int_{S^1} H \circ x \dl3 t,
\end{equation}
where $D$ is any extension of $x$ to the disk. The value of $\AA$ does not depend on the extension by the Stokes theorem and the fact that $\omega$ is closed.

It is possible to visualize $\LL M$ as a manifold (it's not) and the action functional as a (maybe) Morse function on this (not) manifold, and hence do Morse theory on it. With this in mind, it is useful to first intuit what the derivative of $\AA$ should be.

In this metaphor, a `tangent vector to $x$' is a vector field $\xi$ on $x$. By this what is meant is a smooth collection of vectors $\xi(t) \in T_{x(t)} M$. With that, it is possible to compute
\begin{equation}
 (\dl A)_x(\xi) = - \int_{S^1} \xi \into \omega + \int_{S^1} (\dl H)_{x(t)} (\xi(t)) \dl3t. 
\end{equation}

It is easy to verify that this is null for all $\xi$ if and only if the orbit $x$ satisfies the equation
\begin{equation}
(\dl H)_{x(t)} = \dot x(t) \into \omega.
\end{equation}

Consequently, the periodic orbits correspond to critical points of the action functional. It is possible to define a notion of non-degenerate critical points, and therefore of `Morse action functional'. Since the action functional is obtained from the Hamiltonian, this non-degeneracy condition is attributed to the Hamiltonian. A Hamiltonian which induces a `Morse action functional' is said to be non-degenerate.

For such a Hamiltonian, the only thing left to be able to do Morse theory on the space of loops is to establish an equivalent to Riemannian structure so that we may take (negative) gradients of functionals. A standard way to establish a Riemannian metric on a symplectic manifold is to establish a compatibly almost complex structure $J$ and set $\braket{v,w} := \omega(v, Jw)$. With this metric, one can also establish a `Riemannian metric' on the space of loops:
\begin{equation}
\braket{\xi, \zeta} = \int_{S^1} \braket{\xi(t), \zeta(t)} \dl3t.
\end{equation}

With this metric, the gradient of $A$ can be written as
\begin{equation}
\grad A(x) = J(\dot x - X_H),
\end{equation}
where $X_H$ satisfies $X_H \into \omega = - \dl H$. As a consequence, we may write the `ODE' necessary for a `path of orbits'
\begin{equation}
u(t,s) \colon S^1 \times \R \to M
\end{equation}
to be following the flow of the negative gradient of $A$. That is, the (metaphorical) $\diffp u s = - \grad A(u(\cdot, s))$. It can be rendered (more formally) as the PDE
\begin{equation}
\diffp u s + J \diffp u t = J X_H.
\end{equation}

A function $u$ which satisfies this PDE is said to be a pseudo-holomorphic curve.

With this in mind, the only ingredient left to do Morse theory on $\LL M$ is a notion of (Maslov) index of a periodic orbit of $H$. The construction is not trivial, and is based on the so-called Conley-Zehnder index of a path of symplectic matrices. With it, it is possible to build the Floer chain complex (with coefficients in $\Z_2$)
\begin{equation}
\CF_k(M, H) := \braket{\text{periodic orbits of $H$ of degree $k$}},
\end{equation}
where the angled brackets denote the free vector space whose basis is the set inside the brackets.

The differential of this chain complex is analogous to the one considered for Morse theory, where the differential of an orbit $x$ is given as the sum of all the orbits $y$ of degree one less than $x$, counted with multiplicity, such that there exists some pseudo-holomorphic curve $u$ whose limit as $s \to \infty$ is $y$ and whose limit as $s \to -\infty$ is $x$. (Does the differential depend on $J$?)

It is a highly non-trivial fact that this does form a differential, and hence the complex $\CF_*$ is indeed a complex, from which one may construct the homology $\HF_*(M,H)$. It is possible to show that this homology does not depend on the chosen Hamiltonian, and that it coincides with the Morse homology of $M$.

\subsection{Filtered Floer Homology}

The notion of Floer homology, by itself, is inadequate for persistence homology, because there is no dependency of the homology on a time variable. The notion of filtered Floer homology is a way to remedy that.

Note that the flow of the `gradient of the action functional' always decreases the value of the action. Therefore, for any cycle $x$, $\partial x$ will always be a combination of orbits $y$ with $\AA(y) \leq \AA(x)$. As a consequence, the differential may be restricted to the (space spanned by the) orbits with action less than some fixed $\lambda \in \R$. The resulting chain complex is denoted $\CF_*^\lambda$, and the resulting homology is the so-called \emph{filtered Floer homology} of $M$, which depends on the parameter $\lambda$. Unsurprisingly, unlike plain Floer homology, it also depends on the specific chosen Hamiltonian, as the orbits which are available at a given value of $\lambda$ depend explicitly on the chosen Hamiltonian. Somewhat surprisingly, it only depends on the time-one flow of $H$, $\phi_H$.

\begin{definition}
Let $(M,\omega)$ be a compact aspherical symplectic manifold and $H$ a Hamiltonian on $M$. The \emph{filtered Floer homology} of the Hamiltonian diffeomorphism $\phi_H$, denoted
\begin{equation}
\HF^\lambda_*(M,\phi_H), \quad \lambda \in \R
\end{equation}
is the homology of the chain complex $\CF^\lambda_*(M,H)$.
\end{definition}

This tool is adequate for application of persistence homology, as it gives us a collection of fields which depend on a time variable. To define a persistence module, it remains only to find natural maps $\pi_{ts} \colon \HF^t_* \to \HF^s_*$. Fortunately, there is a natural inclusion of chain complexes $\CF^t_* \subseteq \CF^s_*$ for $t \leq s$, which induces the desired maps in homology. This allows us to define the barcode of a Hamiltonian diffeomorphism (do the $\pi$ not depend on the chosen Hamiltonian?)

\begin{definition}
Let $(M,\omega)$ be a compact aspherical symplectic manifold and $\phi_H$ a Hamiltonian diffeomorphism on $M$. The \emph{barcode of $\phi_H$ in degree $*$}, denoted $\BB_*(\phi_H)$, is defined as the barcode associated to the persistence module given by the filtered Floer homology of $\phi_H$ in degree $*$.

The \emph{total barcode of $\phi_H$}, denoted $\BB(\phi_H)$, is the union $\bigcup_{* \in \Z} \BB_*(\phi_H)$.
\end{definition}

The subject of study now becomes to obtain information about $\phi_H$ from its barcode and vice-versa.

\section{A First Example: Autonomous Hamiltonians}\label{subsecautonomous}

Let $H$ be an autonomous Hamiltonian on a compact symplectic manifold $M$. Assuming that $H$ is a non-degenerate Hamiltonian, what can we say about the barcode of $\phi_H$?

We begin by characterizing the periodic orbits of $H$.

\begin{prop}\label{authamfloermorse}
If $H$ is a non-degenerate autonomous Hamiltonian, its only periodic orbits are fixed points, which coincide with the critical points of $H$. These critical points have non-degenerate Hessian, and therefore $H$ is a Morse function.

Furthermore, the indices of these critical points (seeing $H$ as a Morse function) are related to their Maslov indices by the formula
\begin{equation}
\Ind(x) = \mu(x) + n,
\end{equation}
where $\mu$ is the CZ index and $M$ is $2n$-dimensional.
\end{prop}

\begin{proof}
Suppose that $H$ has a non-constant periodic orbit $\gamma(t)$. Then, intuitively, every point of the form $\gamma(t)$ is a fixed point of $\phi$, where $\phi$ is the time-one flow of $H$, and thus the fixed points of $\phi$ are not isolated.

As a more rigorous proof, we show that $H$ is a degenerate Hamiltonian directly. Since $\gamma$ is not constant, but it is the flow of the autonomous vector field $X^H$, the vector field $X^H$ must not vanish at any point of $\gamma$. In other words, if we set $x = \gamma(0)$, $X^H_x \neq 0$.

Now, note that
\begin{equation}\label{dlphixhxh}
(\dl \phi)(X^H_x) = X^H_x,
\end{equation}
which shows that $(\dl \phi)_x$ has a one-eigenvalue, i.e. it is degenerate. To show \eqref{dlphixhxh}, simply note the equality $\phi(\gamma(t)) = \gamma(t)$, and differentiate both sides.

This completes the proof that if $H$ is a non-degenerate autonomous Hamiltonian, its periodic orbits coincide with the critical points of $H$. Now we show that their Hessians are non-degenerate.

Let $v$ and $w$ be two vectors tangent to $x$. Recall that the Hessian of $H$ at $x$, applied to $v$ and $w$, which we will denote $D^2 H(v,w)$, is defined as
\begin{equation}
D^2 H(v,w) = v \cdot (Y \cdot H) \text{, with $Y$ any extension of $w$.}
\end{equation}

In our case, this can be simplified as
\begin{equation}\label{hessian1}
D^2 H(v,w) = v \cdot (\dl H)(Y) = - v \cdot \omega(X^H, Y).
\end{equation}

To proceed, we apply a sort of Leibniz rule, in the sense that
\begin{equation}\label{fakeleibniz}
\text{``$v \cdot \omega(X^H, Y) = \omega(v \cdot X^H, Y) + \omega(X^H, v \cdot Y)$''.}
\end{equation}

Now, to make sense of \eqref{fakeleibniz}, we need to be able to interpret the expression $v \cdot X^H$. This can be taken as a Lie derivative, or equivalently, a Lie bracket, which requires extending $v$ to a vector field $X$. Furthermore, note that the second term, $\omega(X^H, [X,Y])$, vanishes because $X^H$ is null at $x$. Consequently, we aim to prove
\begin{lemma}\label{leibniz1}
Let $X$, $Y$ and $Z$ be vector fields in a neighborhood of $x$, and suppose that $Y_x = 0$. Then,
\begin{equation}
X \cdot \omega(Y,Z) = \omega([X,Y],Z).
\end{equation}
\end{lemma}

\begin{lemmaproof}
Define the auxilliary form $\eta = Z \into \omega$, and apply the well-known formula for the exterior derivative of a one-form
\begin{equation}
(\dl \eta)(X,Y) = X \cdot \eta(Y) - Y \cdot \eta(X) - \eta([X,Y]).
\end{equation}

Now, note that since $Y$ is null at $x$, two of these terms vanish, giving us the equation
\begin{equation}
X \cdot \eta(Y) = \eta([X,Y]) \equiv X \cdot \omega(Y,Z) = \omega([X,Y],Z),
\end{equation}
as desired.
\end{lemmaproof}

We may now use lemma \ref{leibniz1} to simplify \eqref{hessian1} as
\begin{equation}
D^2 H(v,w) = - v \cdot \omega(X^H, Y) = - \omega([X,X^H], w).
\end{equation}

The nondegeneracy of the Hessian will follow from the nondegeneracy of the symplectic form, from the moment that we show that $[X,X^H]_x \neq 0$ for every $X$ with $X_x \neq 0$.

\begin{lemma}\label{bracketnondegen}
If $[X,X^H] = 0$ then $(\dl \phi) v = v$.
\end{lemma}

\begin{lemmaproof}
To investigate $(\dl \phi) v$, it is useful to recall that it is the time-one flow of $X^H$. Let $\phi_t$ be the flow at time $t$. We use this to set up an ODE for the expression $(\dl \phi_t) v$, but this has slight technical problems, so instead we set up an ODE for the expression
\begin{equation}
((\dl \phi_t) v) \cdot f\text{, for $f \colon M \to \R$.}
\end{equation}

To this effect, let $v(s)$ be a curve whose derivative at $s=0$ equals $v$. Then,
\begin{equation}
\begin{split}
((\dl \phi_0) v) \cdot f &= v \cdot f,\\
\diff*{((\dl \phi_t) v) \cdot f}t &= \diff{}t \left( \diff{}s[0] f(\phi_t(v(s))) \right)\\
&= \diff{}s[0] \left( \diff{}t f(\phi_t(v(s)))\right)\\
&= \diff{}s[0] X^H_{v(s)} \cdot f\\
&= v \cdot (X^H \cdot f),
\end{split}
\end{equation}
and since $X^H$ vanishes at $x$ it is obvious that
\begin{equation}
v \cdot (X^H \cdot f) = [X,X^H]_x \cdot f.
\end{equation}

As a consequence, if $[X,X^H]_x = 0$ then $((\dl \phi_t)v) \cdot f$ is constant equal to $v \cdot f$, and hence $(\dl \phi) v = v$.
\end{lemmaproof}

As a consequence of lemma \ref{bracketnondegen} and the nondegeneracy of $\phi$, $[X,X^H]$ does not vanish at $x$ unless $v = 0$, and consequently the Hessian of $H$ at $x$ is nondegenerate. This completes the proof that $H$ is a Morse function.

The proof of the relation between the Morse and Maslov indices is ommitted for the time being, because I don't know how in depth I will want to go on that subject.
\end{proof}

We have now shown that under these circumstances, the Morse complex coincides with the Floer complex, at least in regards to the vector spaces. As it happens, this is still true if one considers the filtered complexes.

\begin{prop}
Let $\phi$ be a nondegenerate Hamiltonian diffeomorphism generated by the Morse function $H$. Then, the following vector spaces coincide
\begin{equation}
\CF^\lambda_*(M,\phi) = \MC^\lambda_*(M,H).
\end{equation}
\end{prop}

\begin{proof}
A simple calculation shows that if $x$ is a critical point of $H$, then its action (as a periodic orbit) coincides with $H(x)$.
\end{proof}

Unfortunately, the complexes, and therefore the homologies, need not coincide, because the differential is not necessarily the same. Indeed, it is easy to check that if $u(s)$ is an orbit connecting two critical points of $H$ then it induces a $J$-holomorphic curve connecting those two points as periodic orbits (assuming that the choice of Riemannian metric for Morse purposes is compatible with the complex structure chosen for Floer purposes), but it is possible that there are $J$-holomorphic curves which are not of the form $u(t,s) \equiv u(s)$.

It is true that if the Hamiltonian is $C^2$-small enough, all $J$-holomorphic curves are in fact idependent of $t$, but the proof is tecnical. We refer to \cite{audin}, proposition 10.1.9.

\begin{prop}
If $H$ is $C^2$-small, the filtered Morse complex and the filtered Floer complex coincide, and consequently as do the filtered Morse and Floer homologies.
\end{prop}

In the general case, knowing that a Hamiltonian diffeomorphism is generated by an autonomous Hamiltonian gives us little information regarding its complex. However, little is not none, and the following proposition represents an obstruction for a Hamiltonian diffeomorphism to be generated by an autonomous Hamiltonian.

\begin{prop}
Let $\phi$ be a non-degenerate Hamiltonian diffeomorphism. For each $k \in \Z$, let
\begin{equation}
\mu_k = \inf \{\, \lambda \in \R \mid \HF_k^\lambda(M,\phi) \neq 0 \, \}.
\end{equation}

Then, if $\phi$ is generated by an autonomous Hamiltonian, $\mu_0 < \mu_k$ for all $k \neq 0$.

[Note: The indices are iffy because we're using the Morse convention, not Maslov.]
\end{prop}

\begin{proof}
Suppose that $\phi$ is generated by the autonomous Hamiltonian $H$. Let $m = \min H$.

\begin{lemma}\label{mincritval}
There exists $\varepsilon > 0$ such that every critical value of $H$ of index different from zero is at least $m+\varepsilon$.
\end{lemma}

\begin{lemmaproof}
If $y$ is a critical value of index at least one, it cannot be a global minimum, as can easily be seen through a Morse neighborhood. Consequently, $y > m$, and the lemma follows since there are finitely many critical values.
\end{lemmaproof}

Clearly, for $k \neq 0$, $\mu_k > m+\varepsilon$, where $\varepsilon$ is as in lemma \ref{mincritval}. We claim that $\mu_0 < m+\frac\varepsilon2$.

To show this fact, consider $\lambda = m+\frac\varepsilon2$, and look at the filtered Floer complex for this value of $\lambda$. By proposition \ref{authamfloermorse}, $\CF_*^\lambda(M,\phi) = \MC_*^\lambda(M,H)$, and the Morse complex is easily checked to be trivial at all degrees other than zero, and nontrivial at $*=0$. This therefore holds for the Floer complex, and even though we usually have no control over the differential, in this case we know that all differentials must be zero. In particular,
\[\HF_0^\lambda(M,\phi) = \CF_0^\lambda(M,\phi) = \MC_*^\lambda(M,H) \cong \Z_2^n,\]
where $n$ is the number of critical points with value below $\lambda$, which is at least one.
\end{proof}

\begin{corollary}\label{corautonomous}
A necessary condition for a nondegenerate Hamiltonian diffeomorphism $\phi$ on a compact aspherical manifold to be generated by an autonomous Hamiltonian is that
\begin{equation}
\mu(\BB_0(\phi)) < \mu(\BB_*(\phi))\text{, for $* \neq 0$,}
\end{equation}
where we define $\mu$ of a barcode $B$ as the infimum of the lower endpoints of the bars of $B$.
\end{corollary}

\subsection{Application: A non-autonomous Hamiltonian (Part one)}

In this section, we will do an in-depth study of a specific Hamiltonian diffeomorphism on the torus, concluding with an application of the results of \ref{subsecautonomous}, namely of corollary \ref{corautonomous}.

First we introduce the object of study. Let $M = S^1 \times S^1$ be the torus with the usual symplectic form $\dl x \wedge \dl y$. We consider the $x$ and $y$ coordinates to be $2\pi$-periodic.

On this manifold, a class of Hamiltonian diffeomorphisms is given by those diffeomorphisms of the form
\begin{equation}
\phi_f(x,y) = (x, y + f(x)),
\end{equation}
so long as $f$ is a smooth $2\pi$-periodic function with null mean. Indeed, it is easy to check that the Hamiltonian $H(x,y) = \int_0^x f(t) \dl3 t$ has the map $\phi_f$ as time-one flow.

The same can be done in the other coordinate, and two functions of this type can be composed to yield Hamiltonian maps of the form
\begin{equation}
(x,y) \mapsto (x + g(y+f(x)), y+f(x)).
\end{equation}

To conclude, we define our object of study as the particular case when $f = g = a \sin$, where $a$ is a real parameter, yielding the Hamiltonian diffeomorphism
\[\phi(x,y) = ( x + a \sin(y + a \sin(x)), y + a \sin(x)).\]

It is a classical result that composition of two Hamiltonian diffeomorphisms is itself a Hamiltonian diffeomorphism. A Hamiltonian corresponding to $\phi$ can be found by using a bump function to smooth the function
\begin{equation}
H_0(x,y,t) = \begin{cases}
-\cos(x), & t < a,\\
\cos(y), & t > a.
\end{cases}
\end{equation}

More precisely, if we let $\varphi(t)$ be a function with compact support contained in $\ointerval 0 \pi$ and unit integral, the Hamiltonian given by
\begin{equation}
H_1(x,y,t) = \begin{cases}
-\cos(x) \varphi(t), & t \leq a,\\
\cos(y) \varphi(t-\pi), & t \geq a.
\end{cases}
\end{equation}
will have as time $2a$	 flow the Hamiltonian diffeomorphism $\phi$.

We also assume, without loss of generality, that $\varphi$ is symmetrical around $t = \pi/2$. This will be useful to simplify certain calculations, as it endows $H_1$ with several useful symmetries.

\subsubsection{The contractible periodic orbits}

The first step to compute the (filtered) Floer homology is to find the periodic orbits of the Hamiltonian $H_1$, in this case of time two. This is the same as to find the fixed points of $\phi$, which amounts to solving the system
\begin{equation}
\begin{cases}
x \equiv x + a \sin(y + a \sin(x)) &\mod 2\pi,\\
y \equiv y + a \sin(x) &\mod 2\pi.
\end{cases}
\end{equation}

However, the resulting fixed points will not necessarily correspond to contractible orbits. To solve this problem, note that $\R^2$ is the universal covering space of the torus. Consequently, any path in the torus can be lifted to one in $\R^2$, and the contractible paths are precisely those that start and end at the same point. Therefore, to find the contractible orbits it suffices to solve the system
\begin{equation}
\begin{cases}
x = x + a \sin(y + a \sin(x)),\\
y = y + a \sin(x).
\end{cases}
\end{equation}

The following proposition is easy to check.

\begin{prop}
The Hamiltonian $H_1$ has exactly four contractible periodic orbits, all of which are constant at the points: $(0,0)$, $(0,\pi)$, $(\pi,0)$, and $(\pi,\pi)$.
\end{prop}

The computation of the action of each of these orbits is trivial to compute. Since they are constant orbits, the $\int_D \omega$ term in the definition of the action vanishes, and all that is left is to compute the integral of $H_1$ with respect to time in $\interval 0 {2a}$, which is equal to $a(\cos(y) - \cos(x))$.

Now that we know the periodic orbits of $H_1$, all that remains is to compute their Maslov indices and differentials. To compute the Maslov indices it is necessary to take a detour, in order to build the tools necessary.

\section{The Maslov Index}

\subsection{The General Definition}

Let $H$ be a periodic Hamiltonian with flow $\phi_t$. The Maslov index is an integer associated to a nondegenerate contractible $T$-periodic orbit $\gamma$. The process is involved, but the process is roughly the following.
\begin{enumerate}[algorithm]
\item Since $\gamma$ is periodic, it may be seen as a smooth map $S^1 \to M$.
\item Pick an embedding of the disk $D^2 \to M$ whose restriction to the border coincides with $\gamma$.
\item Pick smooth functions $Z_1(x), \dots, Z_{2n}(x) \colon D^2 \to TM$, such that $\{Z_i(x)\}$ forms a basis for $T_x M$.
\item Now that $T_{\gamma(t)} M$ has a chosen basis for each $t$, we may write the derivative of the flow at $x$, i.e. the map $(\dl \phi_t)_x$, as a $2n \times 2n$ symplectic matrix, which we will call $A(t)$. Note that $A(t)$ will not be periodic.
\item Given a path of symplectic matrices, such as $A(t)$, where $A(T)$ has no eigenvalue equal to one, it is possible to associate to it an integer called the Conley-Zehnder index.
\end{enumerate}

In turn, it is now necessary to compute the Conley-Zehnder index of a path of matrices $A(t)$. To do so first requires one to establish the existence of a map from the symplectic matrices to $S^1$.

\begin{theorem}\label{rhodef}
There exists a unique map $\rho \colon Sp(2n) \to S^1$ which satisfies the following properties.
\begin{enumerate}[label=\roman*)]
\item If $A$ and $T$ are symplectic matrices, $\rho(T A T^{-1}) = \rho(A)$,
\item If $A$ and $B$ are symplectic matrices,
\begin{equation}
\rho\left(\begin{bmatrix} A & 0 \\ 0 & B \end{bmatrix}\right) = \rho(A) \rho(B),
\end{equation}
\item\label{rhodef:xy} If $A$ is a symplectic matrix of the form $A = \begin{bmatrix} X & -Y \\ Y & X \end{bmatrix}$ then
\begin{equation}
\rho(A) = \det(X + \I Y).
\end{equation}
\item\label{rhodef:realev} If all eigenvalues of $A$ are real, then
\begin{equation}
\rho(A) = (-1)^{m_0/2},
\end{equation}
where $m_0$ is the number of negative eigenvalues, counted with multiplicity,
\item $\rho(A^\transposed) = \rho(A^{-1}) = \conj{\rho(A)}$.
\end{enumerate}

Furthermore, if $A$ is a symplectic matrix whose eigenvalues are all distinct, $\rho(A)$ may be computed using the following formula
\begin{equation}\label{rhoformula}
\rho(A) = (-1)^{m_0/2} \prod \lambda^{\sign \Im \omega(\conj v, v)},
\end{equation}
where the product is taken over the eigenvalues of $A$ with positive imaginary part and unit norm, and to each such $\lambda$ we have associated  a (in general, complex) eigenvector $v$, and we have implicitly extended the 2-form $\omega$ on $\R^{2n}$ in the obvious way to $\C^{2n}$.
\end{theorem}

\begin{proof}
See \cite{audin}, Theorem 7.1.3 on page 192.
\end{proof}

We are now ready to define the Maslov index of a path of symplectic matrices $A(t)$:

\begin{enumerate}[algorithm]
\item Define $Sp(2n)^*$ as the collection of symplectic matrices that do not have 1 as an eigenvalue. It is a known fact \cite[proposition~7.1.4]{audin} that $Sp(2n)^*$ is composed of two path-connected components, $Sp(2n)^+$ and $Sp(2n)^-$, labeled by the sign of $\det(A-I)$. Note that $A(T) \in Sp(2n)^*$, so it must lie in one of these two components.
\item\label{maslov:step2} Define the matrices $W^+$ and $W^-$ as
\begin{equation}
W^+ = - I, \quad W^- \mleft[\begin{array}{c|c}
\begin{matrix} 2 & 0 \\ 0 & 1/2 \end{matrix} & 0\\
\hline
0 & -I
\end{array}\mright].
\end{equation}
Note that $W^\pm \in Sp(2n)^\pm$, and so any matrix in $Sp(2n)^*$ can be connected via a path (in $Sp(2n)^*$) to either $W^+$ or $W^-$. Consequently, we extend $A(t)$ to the interval $\interval 0 {2T}$, where for $t \in \interval T {2T}$ the path $A(t)$ represents such a path connecting $A(T)$ to $W^\pm$.
\item Define $\alpha(t) = \rho(A(t))$ for $a \in \interval 0 {2T}$. It is trivial to check using property \ref{rhodef:realev} from theorem \ref{rhodef} that $\rho(W^\pm) = (-1)^n$, and also that $\rho(I) = 1$. Therefore, $\rho \circ \alpha$ is a path in $S^1$ starting at $1$ and ending at $1$ or $-1$. Therefore, taking the square (viewing $S^1 \subseteq \C$) one obtains a loop $\rho^2 \circ \alpha$ starting and ending at $1$.
\item Now that we have a loop in $S^1$, we can obtain an integer by considering its winding number: the number of times the loop does a full turn around the circle. It is necessary to establish which direction is positive. Generally, one considers the counterclockwise direction to be the positive direction, but for purposes of the Conley-Zehnder index we count the number of \emph{clockwise turns}.
\end{enumerate}
The resulting number of the above procedure is called the Conley-Zehnder index of $A$, denoted $\mu(A)$, and finally we define the Maslov index of the periodic orbit $\gamma$ above, also denoted with the symbol $\mu(\gamma)$, as the Conley-Zehnder index of the corresponding matrix path $A$.

It is a very nontrivial fact that the corresponding integer does not depend on any of the choices done in the process, only on the path and on the Hamiltonian flow. The details may be found in \cite{audin}, whose whole 7th chapter is dedicated to the definition, and well-definition, of this index.

\subsection{The particular case of $Sp(2)$}

The theory of the Maslov index simplifies greatly in the two-dimensional case. This is partly due to the much greater control over the spectrum of a matrix. In two dimensions, the symplectic matrices coincide with those whose determinant equals one, and so their spectrum is entirely determined by their trace.

\begin{prop}
Let $A = \mleft[\begin{smallmatrix} a & b \\ c & d \end{smallmatrix}\mright]$ be a $2 \times 2$ symplectic matrix, that is, it satisfies the equation $ad - bc = 1$. Then, the spectrum of $A$ can be written in terms of its trace, $\tau = a+d$, as
\begin{equation}\label{lambdafromtau}
\lambda_\pm = \frac{\tau \pm \sqrt{\tau^2 - 4}}2.
\end{equation}

Consequently, the behavior of the spectrum can be qualitatively described as follows
\begin{itemize}
\item If $\tau \geq 2$, both eigenvalues are positive real numbers,
\item If $-2 \leq \tau \leq 2$, both eigenvalues lie on the unit circle as a conjugate pair,
\item If $\tau \leq -2$, both eigenvalues are negative real numbers.
\end{itemize}
\end{prop}

\begin{proof}
Equation \eqref{lambdafromtau} is a trivial consequence of the fact that the characteristic polynomial of a $2 \times 2$ matrix $A$ is $p(\lambda) = \lambda^2 - (\trace A) \lambda + \det A$.

The quantitative behavior is proved in two cases.
\begin{itemize}
\item Suppose that $\abs \tau \geq 2$. Then, it is easy to check that $\lambda_\pm$ is real. Furthermore, $\lambda_{\sign \tau}$ clearly has the same sign as $\tau$, and since $\lambda_+ \lambda_- = 1$, both eigenvalues must have the same sign.
\item Suppose that $\abs \tau \leq 2$. Then, the square root term is purely imaginary and so the absolute value of $\lambda_\pm$ can be easily computed by the pythagorean theorem. Since the norm of $\lambda_\pm$ equals one and $\lambda_+ = 1/\lambda_-$, it is trivial to conclude that $\lambda_+ = \conj{\lambda_-}$.
\end{itemize}
\end{proof}

\begin{corollary}
A matrix $A \in Sp(2)$ is in $Sp(2)^*$ iff $\trace A \neq 2$. In this case, it is in $Sp(2)^{\sign(2 - \trace A)}$.
\end{corollary}

\begin{proof}
Using the notation of proposition \label{lambdafromtau}, since $\lambda_+ \lambda_- = 1$, the only way for either eigenvalue to be equal to 1 is for both of them to be equal to 1. Therefore, if any eigenvalue is one, $\trace A = \lambda_+ + \lambda_- = 2$. On the other hand, if $\trace A = 2$ it is obvious that both eigenvalues are one.

To determine the sign of $\det(A-I)$, we compute it directly as
\begin{equation}
\det(A-I) = p(1) = 1^2 - (\trace A) \times 1 + \det A = 2 - \trace A.
\end{equation}
\end{proof}

\begin{remark}
Note the counter-intuitive signs. If $\trace A > 2$ then $A \in Sp(2)^-$, not $Sp(2)^+$. The counter-intuitiveness extends to the $W^\pm$ matrices, given the regrettable fact that $\rho(W^\pm) = \mp 1$.
\end{remark}

\begin{corollary}\label{sp2pm}
Let $A \in Sp(2)$. Then, $\rho(A) = 1$ iff $A \in Sp(2) \setminus Sp(2)^+$, i.e. iff $\trace A \geq 2$. Furthermore, $\rho(A) = -1$ iff $\trace A \leq -2$.
\end{corollary}

\begin{proof}
We divide the proof in three cases.
\begin{itemize}
\item If $\trace A \geq 2$, both eigenvalues of $A$ are positive, and so using property \ref{rhodef:realev} from theorem \ref{rhodef} one concludes $\rho(A) = 1$.
\item If $\trace A \leq -2$, both eigenvalues of $A$ are negative, and so for the same reason $\rho(A) = -1$.
\item If $-2 < \trace A < 2$, both eigenvalues are in $S^1 \setminus \R$. Furthermore, they are both distinct, and so we may apply equation \eqref{rhoformula} to compute
\begin{equation}\label{sp2pm:3}
\rho(A) = \lambda_+^\sigma,
\end{equation}
where $\sigma$ is either $1$ or $-1$. Therefore, in this case $\rho(A)$ is either $\lambda_+$ or $\lambda_-$ and necessarily not a real number.
\end{itemize}
\end{proof}

The above propositions, notably corollary \ref{sp2pm}, greatly simplify the work of calculating the Maslov index in dimension 2, specifically in step \ref{maslov:step2}, which consists of extending the path $A(t)$ to one with endpoint in $W^\pm$.

\begin{corollary}\label{sp2rhoextension}
Let $A \in Sp(2)^*$, and define $A(t)$ as a path in $Sp(2)^*$ such that $A(0) = A$ and $A(1) = W^\pm$. Then, $\rho(A(t))$ has one of two possible behaviors:
\begin{itemize}
\item If $\rho(A) = 1$ then $\rho(A(t))$ is constant equal to one,
\item If $\rho(A) \neq 1$ then $\rho(A(t))$ is a path in $S^1 \setminus \{1\}$ with $\rho(A(1)) = -1$.
\end{itemize}
\end{corollary}

\begin{proof}
If $\rho(A) = 1$ then $A \in Sp(2)^-$ by corollary \ref{sp2pm}. Since the path connecting $A$ to $W^-$ remains in $Sp(2)^-$, and so $\rho(A(t))$ is constant equal to 1 by the same corollary.

If $\rho(A) \neq 1$ then $A \in Sp(2)^+$, and so $A(t)$ connects $A$ to $W^+$. Since this path does not leave $Sp(2)^+$ it never attains the value $1$, and $\rho(A(1)) = \rho(W^+) = -1$.
\end{proof}

Due to corollary \ref{sp2rhoextension}, the algorithm to calculate the Maslov index of a path of matrices can be simplified as follows.

\begin{prop}
Let $A(t)$ be a path of matrices in $Sp(2)$ with $A(T)$ in $Sp(2)^*$. Then, the Maslov index of the path $A(t)$ can be computed as follows:

\begin{enumerate}[algorithm]
\item Compute the path $\gamma(t) = \rho(A(t))$. 
\item If $\gamma(1) \neq 1$, extend it by connecting $\gamma(1)$ to $-1$, without passing through 1, otherwise leave it as-is. Call the resulting curve $\gamma_0$.
\item Compute the (clockwise) winding number of $\gamma_0^2$.
\end{enumerate}
\end{prop}

To conclude this section, we present a further way to simplify the computations. The essential observations are the following.
\begin{itemize}
\item It is trivial to compute $\rho(A)$ when $\abs{\trace A} \geq 2$, as it coincides with $\sign A$,
\item $\rho(A)$ is never $\pm 1$ when $\abs{\trace A} < 2$,
\item We are only interested in the winding number of $\rho^2(A(t))$,
\item And consequently, if for some $t_0, t_1$ we have $\trace A(t_0) = - \trace A(t_1) = \pm 2$ and $-2 < \trace A(t) < 2$ for $t_0 < t < t_1$, it suffices to compute $\rho(A(t))$ for a sigle value of $t \in \ointerval{t_0}{t_1}$.
\end{itemize}

\begin{prop}
Let $A(t)$, $t \in \interval 0 T$ be a path of matrices with $A(0) = I$, $A(T) =W^\pm$. Then, there exists a partition of $\interval 0 T$ of the form
\begin{equation}\label{calcmaslov:ab1}
0 = a_0 < b_0 < a_1 < b_1 < a_2 < \dots < a_{N-1} < b_{N-1} < a_N = T
\end{equation}
which satisfies \begin{enumerate*}\item\label{calcmaslov:ab2} $\trace A(a_n) = (-1)^n 2$, \item\label{calcmaslov:ab3} $\trace A(b_n) = 0$, \item\label{calcmaslov:ab4} Whenever $\trace A(x) \geq 2$ and $\trace A(y) \leq -2$, there exists some $b_n$ between $x$ and $y$.\end{enumerate*}

For any such partition, the winding number of $\rho^2(A(t))$ is equal to
\begin{equation}\label{calcmaslov:mu}
\mu(A(t)) = \sum (-1)^n \sign(A(b_n)_{12}).
\end{equation}
\end{prop}

\begin{proof}
First, we prove that such a partition exists.

Let $U \subseteq \interval 0 T$ be the preimage under $\trace \circ A$ of $\ointerval{-2}\infty$, and $V$ the preimage of $\ointerval{-\infty}2$. Then, by standard Lebesgue number lemma arguments, there exists a partition 
\begin{equation}
0 = c_0 < c_1 < \dots < c_{N+1} = T,
\end{equation}
such that each interval $\interval{c_n}{c_{n+1}}$ is entirely contained in $U$ or $V$. By removing unnecessary partitions, we may suppose without loss of generality that
\begin{equation}
\interval{c_0}{c_1} \subseteq U, \interval{c_1}{c_2} \subseteq V, \interval{c_2}{c_3} \subseteq U, \dots,
\end{equation}
and furthermore that for each $n$ there exists some $t_n \in \ointerval{c_n}{c_{n+1}}$ such that $t_n \in U \setminus V$ or $t_n \in V \setminus U$. Without loss of generality pick $t_0 = c_0 = 0$ and $t_N = c_{N+1} = T$.

We now have a sequence of numbers
\begin{equation}\label{tcineq}
0 = t_0 < c_1 < t_1 < c_2 < t_2 < \dots < t_{N-1} < c_{N-1} < t_{N} = T.
\end{equation}

We will now perturb this sequence, in order to obtain the $a_n$ from the $t_n$ and the $b_n$ from the $c_n$.

Note that $\trace A(t_0) = 2$, $\trace(A(t_1)) \leq -2$, $\trace(A(t_2)) \geq 2$, and so on. By continuity, we may assume without loss of generality that these inequalities are equalities, and since $\trace(A(c_i)) \in \ointerval{-2}2$ we may assume that the inequalities \eqref{tcineq} still hold.

The resulting sequence $t_0, t_1, \dots$ is renamed to $a_0, a_1, \dots$, and we now turn to constructing the sequence $b_1, b_1, \dots$

By compacity, between each $a_n, a_{n+1}$ there exists a maximal $\alpha_n$ with $\trace A(\alpha_n) = \trace A(a_n)$ and a minimal $\alpha'_{n+1}$ with $\trace A(\alpha'_{n+1}) = \trace A(a_{n+1})$. It is easy to check the inequalities
\begin{equation}
a_n \leq \alpha_n < c_n <\alpha'_{n+1} \leq a_{n+1},
\end{equation}
and therefore by continuity there exists some $b_n$ satisfying
\begin{equation}
\trace A(b_n) = 0, \quad \alpha_n < b_n < \alpha'_{n+1}.
\end{equation}

It is obvious that the inequalities \eqref{calcmaslov:ab1} hold, as well as properties \ref{calcmaslov:ab2} and \ref{calcmaslov:ab3}. It remains to prove property \ref{calcmaslov:ab4}, i.e., that whenever $\trace A(x) \geq 2$ and $\trace A(y) \leq -2$ there exists some $b_n$ between $x$ and $y$.

To prove property \ref{calcmaslov:ab4}, we make the following remark.
\begin{lemma}\label{calcmaslov:lemma1}
The sequence of intervals $\interval 0 {b_0}, \interval {b_0}{b_1}, \interval{b_1}{b_2}, \dots, \interval{b_{N-1}}T$ is alternatively contained in $U$ and in $V$.
\end{lemma}

\begin{lemmaproof}
The proof of lemma \label{calcmaslov:lemma1} is a simple but laborious application of the maximalities and minimalities of the $\alpha_n$ and $\alpha'_n$. As an example, we will prove that $\interval{b_0}{b_1} \subseteq V$.

Suppose that $\trace A(t) \geq 2$ for some $t \in \interval{b_0}{b_1}$. By continuity, we may suppose without loss of generality that $\trace A(t) = 2$. Then, it is the case that either $t \in \interval{a_0}{a_1}$ or $t \in \interval{a_1}{a_2}$. In the first case, by maximality of $\alpha_0$, we have
\begin{equation}
t < \alpha_0 < b_0 \leq t,
\end{equation}
an obvious contradiction. The case where $t \in \interval{a_1}{a_2}$ is done in a similar way, applying the minimality of $\alpha'_2$.
\end{lemmaproof}

The proof of property \ref{calcmaslov:ab4} is a simple consequence of the fact that, under the hypotheses of this property, $x \in U \setminus V$ and $y \in V \setminus U$. Therefore, $x$ and $y$ must be in distinct intervals among those in the statement of lemma \ref{calcmaslov:lemma1} and thus must be separated by at least one value of $b_n$.

\smallskip

We now turn to the second part of the proof: We suppose that we have a partition $(a_n, b_n)$ satisfying \eqref{calcmaslov:ab1} and properties \ref{calcmaslov:ab2}, \ref{calcmaslov:ab3} and \ref{calcmaslov:ab4} and we prove formula \eqref{calcmaslov:mu}.

First, observe that the winding number of $\rho^2(A(t))$ can be calculated in pieces, as the path $\rho^2 \circ A$ can be decomposed into its restriction to $\interval{a_0}{a_1}$, $\interval{a_1}{a_2}$, etc., so it suffices to calculate the winding number of $\rho^2(A(t))$ as $t$ varies from $a_n$ to $a_{n+1}$.

For the sake of argument, suppose that $n$ is even. We apply property \ref{calcmaslov:ab4} and corollary \ref{sp2pm} to observe that between $a_n$ and $b_n$ the trace of $A$ is never $-2$, and consequently $\rho(A(t))$ is never $-1$. Since $S^1 \setminus \{-1\}$ is contractible, there is a homotopy between every path from $\rho(A(a_n)) = 1$ and $\rho(A(b_n)) = \pm\I$.\footnote{It is an easy consequence of equation \eqref{sp2pm:3} from corollary \ref{sp2pm} that whenever a matrix $A$ has null trace, $\rho(A) = \pm \I$.} Therefore, without change to the winding number of $\rho^2$, we may replace the path $\rho(A(t))$ in $\interval{a_n}{b_n}$ by a reparametrization of $\pm \e^{\I \theta}$, $\theta \in \interval 0 {\pi/2}$. By the same argument, the path $\rho(A(t))$ for $t \in \interval{b_n}{a_{n+1}}$ may be replaced by $\pm \e^{\I \theta}$, $\theta \in \interval{\pi/2}\pi$, and so, for $t \in \interval{a_n}{a_{n+1}}$ the path $\rho^2(A(t))$ is homotopic to $\pm \e^{2 \I \theta}$, $\theta \in \interval 0 \pi$, with winding number $\pm 1$, where the sign is the sign of $\Im \rho(A(b_n))$.

If $n$ is odd, the previous argument works the same way, but the expression for the winding number of $\rho^2(A(t))$ has the sign swapped, i.e. it becomes $- \Im \rho(A(b_n))$. Therefore, we conclude the formula
\begin{equation}
\mu(A(t)) = -\sum (-1)^n \sign \Im \rho(A(b_n)).
\end{equation}

Note the sign swap from the definition of Maslov index. All that remains is to compute $\rho(A(b_n))$.

\begin{lemma}
Let $A$ be a $2 \times 2$ symplectic matrix with null trace. Then, $\rho(A) = -\sign(A_{12}) \cdot  \I$.
\end{lemma}

\begin{lemmaproof}
Let us name the entries of $A$. Suppose that
\begin{equation}
A = \begin{bmatrix}
a & b\\
c & d
\end{bmatrix}.
\end{equation}

Using the condition on the trace we obtain $d = -a$. Furthermore, the condition that $A$ is symplectic, i.e. its determinant is 1, yields
\begin{equation}
bc = -(1 + a^2),
\end{equation}
and since $a$ is real the right-hand side can never be null, therefore $b, c \neq 0$ and $c$ can be written in terms of $a$ and $b$, yielding
\begin{equation}
A = \begin{bmatrix}
a & b\\
-\frac{1+a^2}b & -a
\end{bmatrix}.
\end{equation}

To proceed, by equation \eqref{sp2pm:3} from corollary \ref{sp2pm}, $\rho(A) = \pm I$. Furthermore, this is true of any matrix $A$ with null trace, so if we continuously deform $A$ into some matrix $A'$ whose value of $\rho$ is well-known, without ever leaving the matrices with null trace, we $\rho(A) = \rho(A')$.

To begin, we continuously deform $b$ from its current value to $\pm 1$, without passing through zero. Then, we continuously deform $a$ into 0. At the end, we obtain the matrix
\begin{equation}
A' = \begin{bmatrix}
0 & \pm 1\\
\mp 1 & 0
\end{bmatrix},
\end{equation}
whose value of $\rho$ is known to be $\mp 1$ by property \ref{rhodef:xy} from theorem \ref{rhodef}. Since $\sign(b) = \pm 1$, we obtain the expected result
\begin{equation}
\rho(A) = - \sign(A_{12}).
\end{equation}
\end{lemmaproof}

This concludes the proof of the proposition.
\end{proof}

\bibliographystyle{plain}
\bibliography{bibliography}

\end{document}