\documentclass{article}

\usepackage{amsmath}
\usepackage{amssymb}
\usepackage{amsfonts}
\usepackage{mathtools}

\usepackage[thmmarks, amsmath]{ntheorem}

\usepackage{graphicx}

\usepackage{tikz-cd}

\usepackage{diffcoeff}
\diffdef{}{op-symbol=\mathrm{d},op-order-sep=0mu}

\usepackage{cancel}
\usepackage{interval}

\usepackage{enumitem}

\usepackage{cite}

%\setlist[enumerate,1]{label=\alph*)}

\title{A Short Introduction to Hamiltonian Persistence Homology}
\author{Duarte Maia}
%\date{}

\theorembodyfont{\upshape}
\theoremseparator{.}
\newtheorem{theorem}{Theorem}
\newtheorem{prop}{Prop}
\renewtheorem*{prop*}{Prop}
\newtheorem{lemma}{Lemma}
\newtheorem{definition}{Definition}

\theoremstyle{nonumberplain}
\theoremheaderfont{\itshape}
\theorembodyfont{\upshape}
\theoremseparator{:}
\theoremsymbol{\ensuremath{\blacksquare}}
\newtheorem{proof}{Proof}

\newcommand{\R}{\mathbb{R}}
\newcommand{\C}{\mathbb{C}}
\newcommand{\Z}{\mathbb{Z}}
\newcommand{\FF}{\mathbb{F}}

\newcommand{\PP}{\mathbb{P}}
\newcommand{\LL}{\mathrm{L}}
\renewcommand{\AA}{\mathrm{A}}
\newcommand{\BB}{\mathrm{B}}


\newcommand{\CF}{\mathrm{CF}}
\newcommand{\HF}{\mathrm{HF}}

\newcommand{\I}{\mathrm{i}}
\newcommand{\e}{\mathrm{e}}

\newcommand{\Ix}{\mathop{\mathrm{I\mkern0.7mu x}}}
%\newcommand{\Ix}{\mathop{\mathrm{ix}}}


\newcommand{\id}{\mathrm{id}}


\DeclareMathOperator{\inte}{int}
\DeclareMathOperator{\codim}{codim}
\newcommand{\grad}{\nabla}
\newcommand{\into}{\mathbin{\lrcorner}}

\DeclarePairedDelimiter{\norm}{\lvert}{\rvert}
\DeclarePairedDelimiter{\Norm}{\lVert}{\rVert}
\DeclarePairedDelimiter{\abs}{\lvert}{\rvert}
\DeclarePairedDelimiter{\braket}{\langle}{\rangle}


\begin{document}
\maketitle

\section{Introduction}

This is a first draft and not intended to be quality work. It is heavily based on \cite{polterovich}.

To do: Add a motivating example.

\section{Persistence Homology 101}

\begin{definition}
A \emph{persistence module} is a collection of finite-dimensional vector spaces $V_t$ over a field $\FF$, for $t \in \R$, as well as a collection of linear maps
\[\pi_{ts} \colon V_t \to V_s, \quad t\leq s,\]
satisfying the rules $\pi_{sr} \circ \pi_{ts} = \pi_{tr}$ and $\pi_{tt} = \id_{V_t}$.

The examples that will be discussed in the sequence also satisfy the following assumptions.

\begin{enumerate}[label=\roman*)]
\item\label{pm1} For all but finitely many $t \in \R$, there exists a neighborhood $U$ of $t$ such that $\pi$ is an isomorphism for all pairs of indices in $U$,
\item\label{pm2} \textit{(Semicontinuity)} For all $t \in \R$, there exists $\varepsilon > 0$ such that $\pi_{tr}$ is an isomorphism for $t-\varepsilon < r \leq t$.
\end{enumerate}

It is easy to show as a consequence of \ref{pm1} that $V_t$ takes finitely many values up to isomorphism. In particular, for $t$ close enough to $+\infty$, all $V_t$ can be simultaneously identified with the same vector space, which we call $V_+$. The same could be done for $t$ close to $-\infty$ to identify all such $V_t$ with a vector space $V_-$. However, we add the following hypothesis, which trivializes $V_-$:

\begin{enumerate}[resume*]
\item\label{pm3} For $t$ close enough to $-\infty$, $V_t = 0$.
\end{enumerate}

A persistence module which satisfies \ref{pm1}, \ref{pm2} and \ref{pm3} is said to be \emph{of finite type}.
\end{definition}

It is easy to tell from the definition of persistence module of finite type  that such modules have very simple structure.

Let $\sigma_1 < \dots < \sigma_N$ be the finite collection of real numbers that property \ref{pm1} refers to, i.e. those around which there is no neighborhood on which $\pi$ is an isomorphism. A simple topological argument shows that $\pi_{ts}$ is an isomorphism for $\sigma_i < t \leq s < \sigma_{i+1}$, $i = 1, \dots, N-1$. In other words, all such $V_t$, $t \in \ointerval{\sigma_i}{\sigma_{i+1}}$ can be canonically identified. The same argument holds for $i = 0$ and $i = N$, if we add the conventions that $\sigma_0 = -\infty$ and $\sigma_{N+1} = \infty$. In other words, property \ref{pm1} is ensuring that, in some sense, $V_t$ `only takes finitely many values'.

In turn, property \ref{pm2}, semicontinuity, is telling us what the behavior at the border points looks like. It guarantees that each $V_{\sigma_i}$ is the same as those for $t$ slightly below it. In other words, while property \ref{pm1} guarantees that $V_t$ is essentially the same for $t \in \ointerval{\sigma_i}{\sigma_{i+1}}$, semicontinuity adds the upper extremum, ensuring that $V_t$ is canonically isomorphic for $t \in \linterval{\sigma_i}{\sigma_{i+1}}$.

Note the edge case $i = N$, in which we obtain an interval of the form $\linterval{\sigma_N}{+\infty}$. This suggests that closing the real line on the right, considering $\R \cup \{+\infty\}$ instead of $\R$, might be a natural choice, but instead we choose to follow \cite{polterovich} and simply adopt the convention that an interval of the form $\linterval{a}{+\infty}$ should be taken to mean the same as $\ointerval{a}{+\infty}$.

Finally, property \ref{pm3} guarantees that the very first of these vector spaces, $V_t$ for $t \leq \sigma_1$, is the trivial space.

It is possible to continue this study, obtaining ever simpler forms of representing finite-type persistence modules, culminating in the so-called Normal Form Theorem, which states that these persistence modules can be represented by particularly simple objects called barcodes.

\begin{definition}
A \emph{barcode} is a finite multiset of intervals in $\R$, of the form $\linterval a b$, where $-\infty < a < b \leq +\infty$. Keep in mind the convention: $\linterval a {+\infty}$ means the same as $\ointerval a {+\infty}$.
\end{definition}

\begin{definition}
Let $B$ be a barcode. We define $\FF(B)$ to be the persistence module defined as follows.

Index the intervals in $B$ as $I_1, I_2, \dots, I_N$. For each $t \in \R$, let $\Ix(t)$ be the collection of indices $i$ such that $t \in I_i$. Define $\FF(B)_t$ as the vector space generated by $\Ix(t)$, i.e. which has $\Ix(t)$ as a basis.

We now define $\pi_{ts}$, for $t \leq s$. To this effect, note that, since $\Ix(t)$ is a basis for $\FF(B)_t$, it suffices to define $\pi_{ts}(i)$ for $i \in \Ix(t)$. We define it as
\[\pi_{ts}(i) = \begin{cases} i & i \in \Ix(s),\\ 0 & \text{otherwise}. \end{cases}\]
\end{definition}

It is easy to check that $\FF(B)$ is a finite-type persistence module. Properties of intervals guarantee that the $\pi$ functions compose as they should. The fact that there are finitely many intervals ensures property \ref{pm1}. The fact that the intervals are open on the left and closed on the right ensures property \ref{pm2}, semicontinuity, and the fact that the lower extremum cannot be $-\infty$ ensures property \ref{pm3}.

The normal form theorem ensures that any finite-type persistence module is isomorphic to $\FF(B)$ for some unique barcode $B$.

\begin{definition}
A homorphism $h \colon V \to W$ of persistence modules is a collection of linear maps $h_t \colon V_t \to W_t$ such that all diagrams of the following form commute.
\[
\begin{tikzcd}
V_t \arrow[d, "h_t"] \arrow[r, "\pi_{ts}"] & V_s \arrow[d, "h_s"] \\
W_t \arrow[r, "\pi_{ts}"]                  & W_s                 
\end{tikzcd}
\]

An isomorphism of persistence modules is a homomorphism $h$ of persistence modules such that each $h_t$ is an isomorphism. In this case, the inverse $h^{-1}$ is also an isomorphism of persistence modules.
\end{definition}

\begin{theorem}[Normal form theorem]
Let $V$ be a persistence module of finite type. Then, there exists a unique barcode $B$ such that $V$ is isomorphic to $\FF(B)$.
\end{theorem}

The normal form theorem allows us to represent certain types of data (time-dependent vector fields) as a simple combinatorial object (barcodes). Therefore, if we have some invariant which is a vector field (like, say, homology over a field) and some space which can be `grown continuously from the empty set' (like a filtration of a manifold using a Morse function), we can construct a persistence module and therefore (with finiteness assumptions) a corresponding barcode.

\section{Barcodes from Filtered Floer Homology}

\subsection{Introduction to Floer Homology}

Let $(M, \omega)$ be a compact aspherical symplectic manifold. Let $\LL M$ be the space of smooth contractible loops on $M$, i.e. smooth maps $S^1 \to M$ which can be extended to the disk, modulo rotation of the domain. Note that by the asphericity condition any such extensions are homotopic.

Let $H$ be a smooth Hamiltonian on $M$. Through it, define the so-called action functional on $\LL M$. For $x \in \LL M$, set
\[\AA(x) = - \int_D \omega + \int_{S^1} H \circ x \dl3 t,\]
where $D$ is any extension of $x$ to the disk. The value of $\AA$ does not depend on the extension by the Stokes theorem and the fact that $\omega$ is closed.

It is possible to visualize $\LL M$ as a manifold (it's not) and the action functional as a (maybe) Morse function on this (not) manifold, and hence do Morse theory on it. With this in mind, it is useful to first intuit what the derivative of $\AA$ should be.

In this metaphor, a `tangent vector to $x$' is a vector field $\xi$ on $x$. By this what is meant is a smooth collection of vectors $\xi(t) \in T_{x(t)} M$. With that, it is possible to compute
\[ (\dl A)_x(\xi) = - \int_{S^1} \xi \into \omega + \int_{S^1} (\dl H)_{x(t)} (\xi(t)) \dl3t. \]

It is easy to verify that this is null for all $\xi$ if and only if the orbit $x$ satisfies the equation
\[(\dl H)_{x(t)} = \dot x(t) \into \omega.\]

Consequently, the periodic orbits correspond to critical points of the action functional. It is possible to define a notion of non-degenerate critical points, and therefore of `Morse action functional'. Since the action functional is obtained from the Hamiltonian, this non-degeneracy condition is attributed to the Hamiltonian. A Hamiltonian which induces a `Morse action functional' is said to be non-degenerate.

For such a Hamiltonian, the only thing left to be able to do Morse theory on the space of loops is to establish an equivalent to Riemannian structure so that we may take (negative) gradients of functionals. A standard way to establish a Riemannian metric on a symplectic manifold is to establish a compatibly almost complex structure $J$ and set $\braket{v,w} := \omega(v, Jw)$. With this metric, one can also establish a `Riemannian metric' on the space of loops:
\[\braket{\xi, \zeta} = \int_{S^1} \braket{\xi(t), \zeta(t)} \dl3t.\]

With this metric, the gradient of $A$ can be written as
\[\grad A(x) = J(\dot x - X_H),\]
where $X_H$ satisfies $X_H \into \omega = - \dl H$. As a consequence, we may write the `ODE' necessary for a `path of orbits'
\[u(t,s) \colon S^1 \times \R \to M\]
to be following the flow of the negative gradient of $A$. That is, the (metaphorical) $\diffp u s = - \grad A(u(\cdot, s))$. It can be rendered (more formally) as the PDE
\[\diffp u s + J \diffp u t = J X_H.\]

A function $u$ which satisfies this PDE is said to be a pseudo-holomorphic curve.

With this in mind, the only ingredient left to do Morse theory on $\LL M$ is a notion of index of a periodic orbit of $H$. The construction is not trivial, and is based on the so-called Conley-Zehnder index of a path of symplectic matrices. With it, it is possible to build the Floer chain complex (with coefficients in $\Z_2$)
\[\CF_k(M, H) := \braket{\text{periodic orbits of $H$ of degree $k$}},\]
where the angled brackets denote the free vector space whose basis is the set inside the brackets.

The differential of this chain complex is analogous to the one considered for Morse theory, where the differential of an orbit $x$ is given as the sum of all the orbits $y$ of degree one less than $x$, counted with multiplicity, such that there exists some pseudo-holomorphic curve $u$ whose limit as $s \to \infty$ is $y$ and whose limit as $s \to -\infty$ is $x$.

It is a highly non-trivial fact that this does form a differential, and hence the complex $\CF_*$ is indeed a complex, from which one may construct the homology $\HF_*(M,H)$. It is possible to show that this homology does not depend on the chosen Hamiltonian, and that it coincides with the Morse homology of $M$.

\subsection{Filtered Floer Homology}

The notion of Floer homology, by itself, is inadequate for persistence homology, because there is no dependency of the homology on a time variable. The notion of filtered Floer homology is a way to remedy that.

Note that the flow of the `gradient of the action functional' always decreases the value of the action. Therefore, for any cycle $x$, $\partial x$ will always be a combination of orbits $y$ with $\AA(y) \leq \AA(x)$. As a consequence, the differential may be restricted to the (space spanned by the) orbits with action less than some fixed $\lambda \in \R$. The resulting chain complex is denoted $\CF_*^\lambda$, and the resulting homology is the so-called \emph{filtered Floer homology} of $M$, which depends on the parameter $\lambda$. Unsurprisingly, unlike plain Floer homology, it also depends on the specific chosen Hamiltonian, as the orbits which are available at a given value of $\lambda$ depend explicitly on the chosen Hamiltonian. Somewhat surprisingly, it only depends on the time-one flow of $H$, $\phi_H$.

\begin{definition}
Let $(M,\omega)$ be a compact aspherical symplectic manifold and $H$ a Hamiltonian on $M$. The \emph{filtered Floer homology} of the Hamiltonian homomorphism $\phi_H$, denoted
\[\HF^\lambda_*(M,\phi_H), \quad \lambda \in \R\]
is the homology of the chain complex $\CF^\lambda_*(M,H)$.
\end{definition}

This tool (which we will henceforth abbreviate to FFH) is adequate for application of persistence homology, as it gives us a collection of fields which depend on a time variable. To define a persistence module, it remains only to find natural maps $\pi_{ts} \colon \HF^t_* \to \HF^s_*$. Fortunately, there is a natural inclusion of chain complexes $\CF^t_* \subseteq \CF^s_*$ for $t \leq s$, which induces the desired maps in homology. This allows us to define the barcode of a Hamiltonian homomorphism (do the $\pi$ not depend on the chosen Hamiltonian?)

\begin{definition}
Let $(M,\omega)$ be a compact aspherical symplectic manifold and $\varphi_H$ a Hamiltonian homomorphism on $M$. The \emph{barcode of $\varphi_H$}, denoted $\BB(\varphi_H)$, is defined as the barcode associated to the persistence module given by the filtered Floer homology of $\varphi_H$.
\end{definition}

The subject of study now becomes to obtain information about $\varphi_H$ from its barcode and vice-versa.

\subsection{A First Example: Full Squares}



\bibliographystyle{plain}
\bibliography{bibliography}

\end{document}