% !TeX root = Thesis.tex
\chapter{Barcodes from Filtered Floer Homology}\label{chap:bffh}

\section{Introduction to Floer Homology}

Let $(M, \omega)$ be a compact symplectic manifold. We assume that $M$ is aspherical, which means that any continuous map $S^2 \to M$ is nulhomotopic. Let $\LL M$ be the space of smooth contractible loops on $M$, i.e. smooth maps $S^1 \to M$ which can be extended to the disk. By the asphericity condition any such extensions are homotopic.

Let $H$ be a smooth Hamiltonian on $M$. Through it, define the so-called action functional on $\LL M$. For $x \in \LL M$, set
\begin{equation}\label{eq:defaction}
\AA(x) = - \int_D \omega + \int_{S^1} H \circ x \dl3 t,
\end{equation}
where $D$ is any extension of $x$ to the disk. The value of $\AA$ does not depend on the extension by the Stokes theorem and the fact that $\omega$ is closed.

It is possible to visualize $\LL M$ as a manifold and the action functional as a Morse function on this `manifold', and hence do Morse theory on it. With this in mind, it is useful to first intuit what the derivative of $\AA$ should be.

In this metaphor, a `tangent vector to $x$' is a vector field $\xi$ on $x$. By this what is meant is a smooth collection of vectors $\xi(t) \in T_{x(t)} M$. With that, it is possible to compute
\begin{equation}
 (\dl \AA)_x(\xi) = - \int_{S^1} \xi \into \omega + \int_{S^1} (\dl H)_{x(t)} (\xi(t)) \dl3t. 
\end{equation}

Therefore, $x\in \LL M$ is a `critical point of $\AA$' if and only if $(\dl \AA)_x(\xi) = 0$ for all $\xi$, and it is straight-forward to see that this happens iff $x$ satisfies the equation
\begin{equation}
(\dl H)_{x(t)} = \dot x(t) \into \omega.
\end{equation}

Consequently, the periodic orbits correspond to critical points of the action functional. It is possible to define a notion of non-degenerate critical points, and therefore of `Morse action functional'. Since the action functional is obtained from the Hamiltonian, this non-degeneracy condition is attributed to the Hamiltonian. A Hamiltonian which induces a `Morse action functional' is said to be non-degenerate.

For such a Hamiltonian, the only thing left to be able to do Morse theory on the space of loops is to establish an equivalent to Riemannian structure so that we may take (negative) gradients of functionals. A standard way to establish a Riemannian metric on a symplectic manifold is to establish a compatible almost complex structure $J$ and set $\braket{v,w} := \omega(v, Jw)$. With this metric, one can also establish a `Riemannian metric' on the space of loops:
\begin{equation}
\braket{\xi, \zeta} = \int_{S^1} \braket{\xi(t), \zeta(t)} \dl3t,
\end{equation}
with which the gradient of $A$ can be written as
\begin{equation}
\grad \AA(x) = J(\dot x - X_H),
\end{equation}
where $X_H$ satisfies $X_H \into \omega = - \dl H$. As a consequence, we may write the `ODE' necessary for a `path of orbits'
\begin{equation}
u(t,s) \colon S^1 \times \R \to M
\end{equation}
to be following the flow of the negative gradient of $\AA$. This results in the PDE
\begin{equation}
\diffp u s + J \diffp u t = J X_H.
\end{equation}
A function $u$ which satisfies this PDE is said to be a pseudo-holomorphic curve.

With this in mind, the only ingredient left to do Morse theory on $\LL M$ is a notion of index of a periodic orbit of $H$. The relevant notion is called the Maslov index of a periodic orbit, and it is a complex enough subject that we have chosen to dedicate all of chapter \ref{chap:maslov} to it. Until then, let it be known that to each periodic orbit $x$ there is an associated integer, the Maslov index, denoted $\mu(x)$. With it, it is possible to build the Floer chain complex
\begin{equation}
\CF_k(M, H) := \braket{\text{periodic orbits of $H$ of degree $k$}},
\end{equation}
where the angled brackets denote the free vector space whose basis is the set inside the brackets. In principle, we could consider vector spaces over any field, but in order to avoid issues of orientation and signs, \emph{over the rest of this chapter, it should be assumed that vector fields generated by periodic orbits are taken over $\Z_2$}.

The differential of this chain complex is analogous to the one considered for Morse theory, where the differential of an orbit $x$ is of the form
\begin{equation}
\sum n(x,y) y,
\end{equation}
where the sum is taken over the periodic orbits $y$ satisfying $\mu(y) = \mu(x) - 1$, and $n(x,y)$ is the number of pseudo-holomorphic curves $u(t,s)$ whose limit as $s \to \infty$ is $y$ and whose limit as $s \to -\infty$ is $x$.

It is a highly non-trivial fact that this does form a differential, and hence the complex $\CF_*(M,H,J)$ is indeed a complex, from which one may construct the homology $\HF_*(M,H,J)$. It is possible to show that this homology does not depend on the chosen Hamiltonian or almost complex structure, and that it coincides with the Morse homology of $M$. For details, see chapter 11 of \cite{audin}. See also proposition 3.11 \cite{schwarz}, which generalizes this to filtered Floer homology; see below.

\section{Filtered Floer Homology}

The notion of Floer homology, by itself, is inadequate for persistence homology, because there is no dependency of the homology on a time variable. The notion of filtered Floer homology is a way to remedy that.

Note that the flow of the `gradient of the action functional' always decreases the value of the action. Therefore, for any cycle $x$, $\partial x$ will always be a combination of orbits $y$ with $\AA(y) \leq \AA(x)$. As a consequence, the differential may be restricted to the (space spanned by the) orbits with action less than some fixed $\lambda \in \R$. The resulting chain complex is denoted $\CF_*^\lambda$, and the resulting homology is the so-called \emph{filtered Floer homology} of $M$, which depends on the parameter $\lambda$. Unlike plain Floer homology, different Hamiltonians yield distinct filtered Floer homologies, as the orbits whose actions are below a certain value of $\lambda$ depend explicitly on the chosen Hamiltonian. Somewhat surprisingly, it only depends on the time-one flow of $H$, $\phi_H$.

\begin{definition}
Let $(M,\omega)$ be a compact aspherical symplectic manifold, $H$ a Hamiltonian on $M$ and $J$ a generic almost complex structure. The \emph{filtered Floer homology} of $H$ and $J$, denoted
\begin{equation}
\HF^\lambda_*(M,H,J), \quad \lambda \in \R
\end{equation}
is the homology of the chain complex $\CF^\lambda_*(M,H,J)$, whose vector spaces are
\begin{equation}
\CF^\lambda_k(M,H) = \{\, x \in \CF_k(M,H) \mid \AA(x) < \lambda\,\}
\end{equation}
and whose differential is the restriction of the differential of the unfiltered Floer homology.
\end{definition}

\begin{prop}
The filtered Floer homology does not depend on the choice of generic almost complex structure, and depends only on the time-one flow of $H$.
\end{prop}

\begin{proof}
The proof of this fact is far beyond the scope of this work, but can be found in \cite{schwarz}. For convenience, we sketch Schwarz's approach.

The kernel of the proof lies in proposition 3.11 in \cite{schwarz}. In it, he constructs a map, which he calls $\Phi_{HK}$, which maps the Floer complex of the Hamiltonian $H$ to the Floer complex of the Hamiltonian $K$, where $H$ and $K$ generate the same Hamiltonian diffeomorphism \emph{in the universal covering of $\Ham(M)$}. We will return to this point later.

It is not explicit in the notation, but $H$ and $K$ come bundled with almost complex structures, say $J_H$ and $J_K$. The almost-complex structures $\bar J_H$ and $\bar J_K$, upon which the map $\Phi_{HK}$ depends, are built from these (see the beginning of page 439). As a consequence, his proof also shows independence on the almost complex structure.

It is then shown that $\Phi_{HK}$ is a chain map, so that it commutes with the differential and hence can be made into a map in homology, and also that it preserves the action, so it may be restricted to the filtered Floer complex and hence the filtered Floer homology. Finally, it is shown that the resulting map in homology is independent of the choices made in the construction of $\Phi_{HK}$, namely, the choice of an almost complex structure to interpolate between $\bar J_H$ and $\bar J_K$. As a consequence, $\Phi_{HK}$ and $\Phi_{KH}$ are inverses of each other, establishing that the filtered Floer homologies of $H$ and $K$ are isomorphic.

Let us go back to the following detail: proposition 3.11 requires not only that the two Hamiltonians $H$ and $K$ have the same time-one flow, but also that they do so in the universal covering of $\Ham(M)$. As a consequence, we have not ruled out the possibility that there exist two Hamiltonians which have the same time-one flow but distinct filtered Floer homologies. 

[[New wall]]
\end{proof}

[[Work in progress]]

\begin{definition}
Let $(M,\omega)$ be a compact aspherical symplectic manifold and $H$ a Hamiltonian on $M$. The \emph{filtered Floer homology} of the Hamiltonian diffeomorphism $\phi_H$, denoted
\begin{equation}
\HF^\lambda_*(M,\phi_H), \quad \lambda \in \R
\end{equation}
is the homology of the chain complex $\CF^\lambda_*(M,H)$.
\end{definition}

This tool is adequate for application of persistence homology, as it gives us a collection of fields indexed on a time variable. To define a persistence module, it remains only to find natural maps $\pi_{ts} \colon \HF^t_* \to \HF^s_*$. Fortunately, there is a natural inclusion of chain complexes $\CF^t_* \hookrightarrow \CF^s_*$ for $t \leq s$, which induces the desired maps in homology. This allows us to define the barcode of a Hamiltonian diffeomorphism. [[do the $\pi$ not depend on the chosen Hamiltonian?]]

\begin{definition}
Let $(M,\omega)$ be a compact aspherical symplectic manifold and $\phi_H$ a Hamiltonian diffeomorphism on $M$. The \emph{barcode of $\phi_H$ in degree $*$}, denoted $\BB_*(\phi_H)$, is defined as the barcode associated to the persistence module given by the filtered Floer homology of $\phi_H$ in degree $*$.

The \emph{total barcode of $\phi_H$}, denoted $\BB(\phi_H)$, is the union $\bigcup_{* \in \Z} \BB_*(\phi_H)$.
\end{definition}

The study of what properties of a Hamiltonian diffeomorphism can be deduced from its barcode is an active area of research: see for example \cite{polterovich}, \cite{kislev2022bounds}, \cite{roux2018barcodes}, and \cite{polterovich2016autonomous}. In this thesis, we will instead compute the barcodes of a few such diffeomorphisms, namely of non-autonomous Hamiltonians, as the barcodes computed in the literature are mostly those of small autonomous Hamiltonians, in which case they can be computed instead via filtered Morse homology.