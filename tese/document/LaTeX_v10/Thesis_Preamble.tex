%%%%%%%%%%%%%%%%%%%%%%%%%%%%%%%%%%%%%%%%%%%%%%%%%%%%%%%%%%%%%%%%%%%%%%%%
%                                                                      %
%     File: Thesis_Preamble.tex                                        %
%     Tex Master: Thesis.tex                                           %
%                                                                      %
%     Author: Andre C. Marta                                           %
%     Last modified : 3 Mar 2020                                       %
%                                                                      %
%%%%%%%%%%%%%%%%%%%%%%%%%%%%%%%%%%%%%%%%%%%%%%%%%%%%%%%%%%%%%%%%%%%%%%%%

% 'natbib' package
%
% Flexible bibliography support.
% http://www.ctan.org/tex-archive/macros/latex/contrib/natbib/
%
% > produce author-year style citations
%
% \citet  and \citep  for textual and parenthetical citations, respectively
% \citet* and \citep* that print the full author list, and not just the abbreviated one
% \citealt is the same as \citet but without parentheses. Similarly, \citealp is \citep without parentheses
% \citeauthor
% \citeyear
% \citeyearpar
%
%% natbib options can be provided when package is loaded \usepackage[options]{natbib}
%%
%% Following options are valid:
%%
%%   round  -  round parentheses are used (default)
%%   square -  square brackets are used   [option]
%%   curly  -  curly braces are used      {option}
%%   angle  -  angle brackets are used    <option>
%%   semicolon  -  multiple citations separated by semi-colon (default)
%%   colon  - same as semicolon, an earlier confusion
%%   comma  -  separated by comma
%%   authoryear - for author–year citations (default)
%%   numbers-  selects numerical citations
%%   super  -  numerical citations as superscripts, as in Nature
%%   sort   -  sorts multiple citations according to order in ref. list
%%   sort&compress   -  like sort, but also compresses numerical citations
%%   compress - compresses without sorting
%%
% ******************************* SELECT *******************************
%\usepackage{natbib}          % <<<<< References in alphabetical list Correia, Silva, ...
\usepackage[numbers,sort&compress]{natbib} % <<<<< References in numbered list [1],[2],...
% ******************************* SELECT *******************************


% 'notoccite' package
%
% Prevent trouble from citations in table of contents, etc.
% http://ctan.org/pkg/notoccite
%
% > If you have \cite com­mands in \sec­tion-like com­mands, or in \cap­tion,
%   the ci­ta­tion will also ap­pear in the ta­ble of con­tents, or list of what­ever.
%   If you are also us­ing an un­srt-like bib­li­og­ra­phy style, these ci­ta­tions will
%   come at the very start of the bib­li­og­ra­phy, which is con­fus­ing. This pack­age
%   sup­presses the ef­fect.
%
\usepackage{notoccite}


% ----------------------------------------------------------------------
% Define document language.
% ----------------------------------------------------------------------

% 'inputenc' package
%
% Accept different input encodings.
% http://www.ctan.org/tex-archive/macros/latex/base/
%
% > allows typing non-english text in LaTeX sources.
%
% ******************************* SELECT *******************************
%\usepackage[latin1]{inputenc} % <<<<< Windows
\usepackage[utf8]{inputenc}   % <<<<< Linux
% ******************************* SELECT *******************************


% 'babel' package
%
% Multilingual support for Plain TeX or LaTeX.
% http://www.ctan.org/tex-archive/macros/latex/required/babel/
%
% > sets the variable names according to the language selected
%
% ******************************* SELECT *******************************
%\usepackage[portuguese]{babel} % <<<<< Portuguese
\usepackage[english]{babel} % <<<<< English
%\usepackage[francais]{babel} % <<<<< French (requires package texlive-lang-french)
% ******************************* SELECT *******************************


% List of LaTeX variable names: \abstractname, \appendixname, \bibname,
%   \chaptername, \contentsname, \listfigurename, \listtablename, ...)
% http://www.tex.ac.uk/cgi-bin/texfaq2html?label=fixnam
%
% Changing the words babel uses (uncomment and redefine as necessary...)
%
\newcommand{\acknowledgments}{@undefined} % new LaTeX variable name
%
% > English
%
\addto\captionsenglish{\renewcommand{\acknowledgments}{Acknowledgments}}
%\addto\captionsenglish{\renewcommand{\contentsname}{Contents}}
%\addto\captionsenglish{\renewcommand{\listtablename}{List of Tables}}
%\addto\captionsenglish{\renewcommand{\listfigurename}{List of Figures}}
%\addto\captionsenglish{\renewcommand{\nomname}{Nomenclature}}
%\addto\captionsenglish{\renewcommand{\glossaryname}{Glossary}}
%\addto\captionsenglish{\renewcommand{\acronymname}{List of Acronyms}}
%\addto\captionsenglish{\renewcommand{\bibname}{References}} % Bibliography
%\addto\captionsenglish{\renewcommand{\appendixname}{Appendix}}

% > French
%
\addto\captionsfrench{\renewcommand{\acknowledgments}{Remerciements}}
%\addto\captionsfrench{\renewcommand{\contentsname}{Table des matières}}
%\addto\captionsfrench{\renewcommand{\listtablename}{Liste des tableaux}}
\addto\captionsfrench{\renewcommand{\listfigurename}{Liste des figures}} % Table des figures
%\addto\captionsfrench{\renewcommand{\nomname}{Nomenclature}}
%\addto\captionsfrench{\renewcommand{\glossaryname}{Glossaire}}
%\addto\captionsfrench{\renewcommand{\acronymname}{Liste des acronymes}}
%\addto\captionsfrench{\renewcommand{\bibname}{Bibliographie}}
%\addto\captionsfrench{\renewcommand{\appendixname}{Annexe}}

% > Portuguese
%
\addto\captionsportuguese{\renewcommand{\acknowledgments}{Agradecimentos}}
%\addto\captionsportuguese{\renewcommand{\contentsname}{Conte\'{u}do}}
%\addto\captionsportuguese{\renewcommand{\listtablename}{Lista de Figuras}}
%\addto\captionsportuguese{\renewcommand{\listfigurename}{Lista de Tabelas}}
\addto\captionsportuguese{\renewcommand{\nomname}{Lista de S\'{i}mbolos}} % Nomenclatura
%\addto\captionsportuguese{\renewcommand{\glossary}{Gloss\'{a}rio}}
%\addto\captionsportuguese{\renewcommand{\acronymname}{Lista de Abrevia\c{c}\~{o}es}}
%\addto\captionsportuguese{\renewcommand{\bibname}{Refer\^{e}ncias}} % Bibliografia
%\addto\captionsportuguese{\renewcommand{\appendixname}{Anexo}} % Apendice


% ----------------------------------------------------------------------
% Define cover fields in both english and portuguese.
% ----------------------------------------------------------------------
%
\newcommand{\coverThesis}{@undefined} % new LaTeX variable name
\newcommand{\coverSupervisors}{@undefined} % new LaTeX variable name
\newcommand{\coverExaminationCommittee}{@undefined} % new LaTeX variable name
\newcommand{\coverChairperson}{@undefined} % new LaTeX variable name
\newcommand{\coverSupervisor}{@undefined} % new LaTeX variable name
\newcommand{\coverMemberCommittee}{@undefined} % new LaTeX variable name
% > English
\addto\captionsenglish{\renewcommand{\coverThesis}{Thesis to obtain the Master of Science Degree in}}
\addto\captionsenglish{\renewcommand{\coverSupervisors}{Supervisor(s)}}
\addto\captionsenglish{\renewcommand{\coverExaminationCommittee}{Examination Committee}}
\addto\captionsenglish{\renewcommand{\coverChairperson}{Chairperson}}
\addto\captionsenglish{\renewcommand{\coverSupervisor}{Supervisor}}
\addto\captionsenglish{\renewcommand{\coverMemberCommittee}{Member of the Committee}}
% > French
\addto\captionsfrench{\renewcommand{\coverThesis}{Th\`ese pour l'obtention du Maîtrise des Sciences en}}
\addto\captionsfrench{\renewcommand{\coverSupervisors}{Directeur(s) de th\`ese}}
\addto\captionsfrench{\renewcommand{\coverExaminationCommittee}{Jury}}
\addto\captionsfrench{\renewcommand{\coverChairperson}{Pr\'esident}}
\addto\captionsfrench{\renewcommand{\coverSupervisor}{Directeur de th\`ese}}
\addto\captionsfrench{\renewcommand{\coverMemberCommittee}{Rapporteur}}
% > Portuguese
\addto\captionsportuguese{\renewcommand{\coverThesis}{Disserta\c{c}\~{a}o para obten\c{c}\~{a}o do Grau de Mestre em}}
\addto\captionsportuguese{\renewcommand{\coverSupervisors}{Orientador(es)}}
\addto\captionsportuguese{\renewcommand{\coverExaminationCommittee}{J\'{u}ri}}
\addto\captionsportuguese{\renewcommand{\coverChairperson}{Presidente}}
\addto\captionsportuguese{\renewcommand{\coverSupervisor}{Orientador}}
\addto\captionsportuguese{\renewcommand{\coverMemberCommittee}{Vogal}}


% ----------------------------------------------------------------------
% Define default and cover page fonts.
% ----------------------------------------------------------------------

% Use Arial font as default
%
\renewcommand{\rmdefault}{phv}
\renewcommand{\sfdefault}{phv}

% Define cover page fonts
%
%         encoding     family       series      shape
%  \usefont{T1}     {phv}=helvetica  {b}=bold    {n}=normal
%                   {ptm}=times      {m}=normal  {sl}=slanted
%                                                {it}=italic
% see more examples at
% http://julien.coron.free.fr/languages/latex/fonts/
%
\def\FontLn{% 16 pt normal
  \usefont{T1}{phv}{m}{n}\fontsize{16pt}{16pt}\selectfont}
\def\FontLb{% 16 pt bold
  \usefont{T1}{phv}{b}{n}\fontsize{16pt}{16pt}\selectfont}
\def\FontMn{% 14 pt normal
  \usefont{T1}{phv}{m}{n}\fontsize{14pt}{14pt}\selectfont}
\def\FontMb{% 14 pt bold
  \usefont{T1}{phv}{b}{n}\fontsize{14pt}{14pt}\selectfont}
\def\FontSn{% 12 pt normal
  \usefont{T1}{phv}{m}{n}\fontsize{12pt}{12pt}\selectfont}


% ----------------------------------------------------------------------
% Define page margins and line spacing.
% ----------------------------------------------------------------------

% 'setspace' package
%
% Set space between lines.
% http://www.ctan.org/tex-archive/macros/latex/contrib/setspace/
%
% > allow setting line spacing (line spacing of 1.5, as per IST rules)
%
\usepackage{setspace}
%gross
%\renewcommand{\baselinestretch}{1.5}
%added this to make matrices look better
\setstretch{1.5}
%\everydisplay{\setstretch{1.1}}
%\everydisplay=\expandafter{\the\everydisplay\setstretch{1.1}}
\everydisplay\expandafter{\the\everydisplay\def\baselinestretch{1.1}\selectfont}

% 'geometry' package
%
% Flexible and complete interface to document dimensions.
% http://www.ctan.org/tex-archive/macros/latex/contrib/geometry/
%
% > set the page margins (2.5cm minimum in every side, as per IST rules)
%
\usepackage{geometry}	
\geometry{verbose,tmargin=2.5cm,bmargin=2.5cm,lmargin=2.5cm,rmargin=2.5cm}




% ----------------------------------------------------------------------
% Define paragraph formating.
% ----------------------------------------------------------------------

% 'indentfirst' package
%
% Indent first paragraph after section header.
% https://ctan.org/pkg/indentfirst
%
% > indent all paragraphs (as per IST rules)
%
\usepackage{indentfirst}	


% ----------------------------------------------------------------------
% Include external packages.
% Note that not all of these packages may be available on all system
% installations. If necessary, include the .sty files locally in
% the <jobname>.tex file directory.
% ----------------------------------------------------------------------


% 'nomencl' package
%
% Produce lists of symbols as in nomenclature.
% http://www.ctan.org/tex-archive/macros/latex/contrib/nomencl/
%
% The nomencl package makes use of the MakeIndex program
% in order to produce the nomenclature list.
%
% Nomenclature
% 1) On running the file through LATEX, the command \makenomenclature
%    in the preamble instructs it to create/open the nomenclature file
%    <jobname>.nlo corresponding to the LATEX file <jobname>.tex and
%    writes the information from the \nomenclature commands to this file.
% 2) The next step is to invoke MakeIndex in order to produce the
%    <jobname>.nls file. This can be achieved by making use of the
%    command: makeindex <jobname>.nlo -s nomencl.ist -o <jobname>.nls
% 3) The last step is to invoke LATEX on the <jobname>.tex file once
%    more. There, the \printnomenclature in the document will input the
%    <jobname>.nls file and process it according to the given options.
%
% http://www-h.eng.cam.ac.uk/help/tpl/textprocessing/nomencl.pdf
%
% Nomenclature (produces *.nlo *.nls files)
\usepackage{nomencl}
\makenomenclature
%
% Group variables according to their symbol type
%
\RequirePackage{ifthen} 
\ifthenelse{\equal{\languagename}{english}}%
    { % English
    \renewcommand{\nomgroup}[1]{%
      \ifthenelse{\equal{#1}{R}}{%
        \item[\textbf{Roman symbols}]}{%
        \ifthenelse{\equal{#1}{G}}{%
          \item[\textbf{Greek symbols}]}{%
          \ifthenelse{\equal{#1}{S}}{%
            \item[\textbf{Subscripts}]}{%
            \ifthenelse{\equal{#1}{T}}{%
              \item[\textbf{Superscripts}]}{}}}}}%
    }{%
    \ifthenelse{\equal{\languagename}{french}}%
    { % French
    \renewcommand{\nomgroup}[1]{%
      \ifthenelse{\equal{#1}{R}}{%
        \item[\textbf{Symbole romains}]}{%
        \ifthenelse{\equal{#1}{G}}{%
          \item[\textbf{Symboles grecs}]}{%
          \ifthenelse{\equal{#1}{S}}{%
            \item[\textbf{Indices}]}{% lettre inférieure
            \ifthenelse{\equal{#1}{T}}{%
              \item[\textbf{Exposants}]}{}}}}}% lettre supérieure
    }{ % Portuguese
    \renewcommand{\nomgroup}[1]{%
      \ifthenelse{\equal{#1}{R}}{%
        \item[\textbf{Simbolos romanos}]}{%
        \ifthenelse{\equal{#1}{G}}{%
          \item[\textbf{Simbolos gregos}]}{%
          \ifthenelse{\equal{#1}{S}}{%
            \item[\textbf{Subscritos}]}{%
            \ifthenelse{\equal{#1}{T}}{%
              \item[\textbf{Sobrescritos}]}{}}}}}%
    }}%


% 'glossary' package
%
% Create a glossary.
% http://www.ctan.org/tex-archive/macros/latex/contrib/glossary/
%
% Glossary (produces *.glo *.ist files)
\usepackage[number=none]{glossary}
% (remove blank line between groups)
\setglossary{gloskip={}}
% (redefine glossary style file)
%\renewcommand{\istfilename}{myGlossaryStyle.ist}
\makeglossary



% 'hyperref' package
%
% Extensive support for hypertext in LaTeX.
% http://www.ctan.org/tex-archive/macros/latex/contrib/hyperref/
%
% > Extends the functionality of all the LATEX cross-referencing
%   commands (including the table of contents, bibliographies etc) to
%   produce \special commands which a driver can turn into hypertext
%   links; Also provides new commands to allow the user to write adhoc
%   hypertext links, including those to external documents and URLs.
%



% 'pdfpages' package
%
% Include PDF documents in LaTeX
% http://www.ctan.org/pkg/pdfpages
%
% > in­clu­sion of ex­ter­nal multi-page PDF doc­u­ments in LaTeX doc­u­ments.
%   Pages may be freely se­lected and sim­i­lar to psnup it is pos­si­ble to put
%   sev­eral log­i­cal pages onto each sheet of pa­per.
%
% \includepdf{filename.pdf}
% \includepdf[pages={4-9},nup=2x3,landscape=true]{filename.pdf}
%
\usepackage{pdfpages}



\usepackage{amsmath}
\usepackage{amssymb}
\usepackage{amsfonts}
\usepackage{mathtools}

\usepackage[thmmarks, amsmath]{ntheorem}

\usepackage{graphicx}
\usepackage{xcolor}
\usepackage{float}

\usepackage{tikz}
\usetikzlibrary{arrows, positioning, intersections}
\usepackage{tikz-cd}

\usepackage{diffcoeff}
\diffdef{}{op-symbol=\mathrm{d},op-order-sep=0mu,}
\diffdef{p}{left-delim=\left.,right-delim=\right|,subscr-nudge=0mu}

\usepackage{cancel}
\usepackage{interval}
\usepackage{mleftright}

\usepackage[inline]{enumitem}
\SetEnumitemKey{algorithm}{label={Step \Roman*.}, ref=\Roman*, labelindent=0pt, labelsep=!, leftmargin=*, align=left,itemindent=0pt}
\setlist[enumerate,1]{label=\roman*)}

%\usepackage{showlabels}


\usepackage[pdftex]{hyperref} % enhance documents that are to be
                              % output as HTML and PDF
\hypersetup{colorlinks,       % color text of links and anchors,
                              % eliminates borders around links
%            linkcolor=red,    % color for normal internal links
            linkcolor=black,  % color for normal internal links
            anchorcolor=black,% color for anchor text
%            citecolor=green,  % color for bibliographical citations
            citecolor=black,  % color for bibliographical citations
%            filecolor=magenta,% color for URLs which open local files
            filecolor=black,  % color for URLs which open local files
%            menucolor=red,    % color for Acrobat menu items
            menucolor=black,  % color for Acrobat menu items
%            urlcolor=cyan,    % color for linked URLs
            urlcolor=black,   % color for linked URLs
	          bookmarks=true,         % create PDF bookmarks
	          bookmarksopen=false,    % don't expand bookmarks
	          bookmarksnumbered=true, % number bookmarks
	          pdftitle={Thesis},
            pdfauthor={Duarte Maia},
            pdfsubject={Thesis Title},
            pdfkeywords={Thesis Keywords},
            pdfstartview=FitV,
            pdfdisplaydoctitle=true}

% 'hypcap' package
%
% Adjusting the anchors of captions.
% http://www.ctan.org/tex-archive/macros/latex/contrib/oberdiek/
%
% > fixes the problem with hyperref, that links to floats points
%   below the caption and not at the beginning of the float.
%
\usepackage[figure,table]{hypcap}

% ----------------------------------------------------------------------
% Define new commands to assure consistent treatment throughout document
% ----------------------------------------------------------------------



\theorembodyfont{\upshape}
\theoremseparator{.}
\newtheorem{theorem}{Theorem}
\renewtheorem*{theorem*}{Theorem}
\newtheorem{prop}{Proposition}
\renewtheorem*{prop*}{Proposition}
\newtheorem{corollary}{Corollary}
\newtheorem{lemma}{Lemma}
\newtheorem{definition}{Definition}
\newtheorem{remark}{Remark}

\theoremstyle{break}
\theorembodyfont{\upshape}
\theoremseparator{.}
\theoremindent0.5cm
\newtheorem{algorithm}{Algorithm}

\theoremstyle{nonumberplain}
\theoremheaderfont{\itshape}
\theorembodyfont{\upshape}
\theoremseparator{:}
\theoremsymbol{\ensuremath{\blacksquare}}
\theoremindent0cm
\newtheorem{proof}{Proof}

\theoremsymbol{\ensuremath{\text{\textit{(End proof of lemma)}}}}
%\theoremsymbol{\ensuremath{\square}}
\newtheorem{lemmaproof}{Proof of lemma}

%Sets of numbers
\newcommand{\R}{\mathbb{R}}
\newcommand{\C}{\mathbb{C}}
\newcommand{\Z}{\mathbb{Z}}

\newcommand{\FF}{\mathbb{F}} %Arbitrary field

%Contractible orbits
\newcommand{\LL}{\mathrm{L}}
%Action functional
\renewcommand{\AA}{\mathrm{A}}
%Generic Barcode
\newcommand{\BB}{\mathrm{B}}

%Floer Complex and Homology
\newcommand{\CF}{\mathrm{CF}}
\newcommand{\HF}{\mathrm{HF}}

\newcommand{\moduli}{M}


\DeclareMathOperator{\Ham}{Ham}

%Morse Complex
\newcommand{\MC}{\mathrm{C}}

\newcommand{\HH}{\mathrm{H}}

\DeclareMathOperator{\Ind}{Ind}
\DeclareMathOperator{\hessian}{Hess}
\DeclareMathOperator{\jacobian}{J}
\newcommand{\Lie}{L}

\newcommand{\I}{\mathrm{i}}
\newcommand{\e}{\mathrm{e}}


\DeclareMathOperator{\sign}{sign}
\let\Im\relax
\DeclareMathOperator{\Im}{Im}
\let\Re\relax
\DeclareMathOperator{\Re}{Re}

\newcommand{\Ix}{\mathop{\mathrm{I\mkern0.7mu x}}}
%\newcommand{\Ix}{\mathop{\mathrm{ix}}}


\newcommand{\id}{\mathrm{id}}

\DeclareMathOperator{\image}{im}


\DeclareMathOperator{\inte}{int}
\DeclareMathOperator{\codim}{codim}
\newcommand{\grad}{\nabla}
\newcommand{\into}{\mathbin{\lrcorner}}

\DeclarePairedDelimiter{\norm}{\lvert}{\rvert}
\DeclarePairedDelimiter{\Norm}{\lVert}{\rVert}
\DeclarePairedDelimiter{\abs}{\lvert}{\rvert}
\DeclarePairedDelimiter{\braket}{\langle}{\rangle}

\newcommand{\conj}[1]{\overline{#1}}
\newcommand{\transposed}{\top}
\DeclareMathOperator{\trace}{Tr}

%conley zehnder index a la hofer and kriener
\newcommand{\muhk}{\mu_{\mathrm{HK}}}
\newcommand{\murs}{\mu_{\mathrm{RS}}}


%\includeonly{maslov}