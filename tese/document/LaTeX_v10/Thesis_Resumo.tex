% !TeX root = Thesis.tex
%%%%%%%%%%%%%%%%%%%%%%%%%%%%%%%%%%%%%%%%%%%%%%%%%%%%%%%%%%%%%%%%%%%%%%%%
%                                                                      %
%     File: Thesis_Resumo.tex                                          %
%     Tex Master: Thesis.tex                                           %
%                                                                      %
%     Author: Andre C. Marta                                           %
%     Last modified :  2 Jul 2015                                      %
%                                                                      %
%%%%%%%%%%%%%%%%%%%%%%%%%%%%%%%%%%%%%%%%%%%%%%%%%%%%%%%%%%%%%%%%%%%%%%%%

\section*{Resumo}

% Add entry in the table of contents as section
\addcontentsline{toc}{section}{Resumo}

A homologia de Floer filtrada de uma variedade simplética $M$ (que satisfaça certas propriedades técnicas) é essencialmente uma versão da homologia de Floer parametrizada pela ação das órbitas sob consideração. Isto faz da homologia de Floer filtrada um invariante mais fino do que a homologia de Floer usual, ao custo de dependência no Hamiltoniano escolhido para a calcular. No entanto, é possível recuperar alguma independência, e portanto a homologia de Floer filtrada é uma invariante de difeomorfismos Hamiltonianos.

Este assunto é uma área ativa de investigação, havendo já alguma literatura sobre o assunto, mas há muito poucos exemplos concretos de computação de homologia de Floer filtrada para difeomorfismos Hamiltonianos particulares. Nesta tese, calcula-se a homologia de Floer filtrada para um par de classes de difeomorfismos Hamiltonianos no toros. No processo, é exposto um método de computação de índices de Maslov/Conley-Zehnder, que é adequado para cálculos práticos.

\vfill

\textbf{\Large Palavras-chave:} geometria simplética, homologia persistente, homologia de floer, índice de maslov, índice de conley-zehnder
