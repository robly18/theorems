% !TeX root = Thesis.tex

\chapter{Introduction}

\paragraph{Persistence Homology}
Persistence Homology is a relatively recent area of mathematics. It was borne from the need to approximate homology of spaces in data science, but flourished into a tool applicable to many areas of mathematics, including differential and symplectic geometry. For a historical overview see \cite{historypersistence}. For a recent exposition of persistence homology applied to geometry and topological function theory see \cite{polterovich}.

To understand persistence homology, it is instructive to look at a concrete example. Consider a torus embedded vertically in $\R^3$ as in figure \ref{fig:torus1}, with the critical values of the height function at 
$z_1 < z_2 < z_3 < z_4$. Classical homology (which does not care how the torus is embedded in $\R^3$) tells us that the torus has one connected component, two holes in dimension 1, and one hole in dimension 2. Persistence homology goes further, by telling us roughly where these holes are located.

\begin{figure}
\centering
\begin{tikzpicture}
\draw[->] (-2,-3) -- (-2,3.5) node[right] {$z$};

\draw (0,0) ellipse (1.5 and 2.7);
  \begin{scope}
    \clip (1.4,0) ellipse (1.8 and 3);
    \draw[name path global=p1] (-1,0) ellipse (1.2 and 2);
  \end{scope}
  \begin{scope}
    \clip (-1,0) ellipse (1.2 and 2);
    \draw[name path global=p2] (1,0) ellipse (1.2 and 2);
  \end{scope}
  
  
\draw[dashed] (0, 2.7) -- (-1.9, 2.7);
\draw (-1.9,2.7) -- (-2.1,2.7) node[left] {$z_4$};

\draw[dashed] (0, -2.7) -- (-1.9, -2.7);
\draw (-1.9,-2.7) -- (-2.1,-2.7) node[left] {$z_1$};

\node (origin) at (0,0) {};
\node (tr) at (-1.9,0) {};
\node (tl) at (-2.1,0) {};


\path [name intersections={of=p1 and p2}];


\draw[dashed] (origin |- intersection-1) -- (tr |- intersection-1);
\draw (tr |- intersection-1) -- (tl |- intersection-1) node[left] {$z_3$};

\draw[dashed] (origin |- intersection-2) -- (tr |- intersection-2);
\draw (tr |- intersection-2) -- (tl |- intersection-2) node[left] {$z_2$};


\end{tikzpicture}
\caption{An embedding of a torus in $\R^3$}\label{fig:torus1}
\end{figure}

Define $X_t$, for $t \in \R$, as the subset of the torus $T$ given by
\begin{equation}
X_t = \{ (x,y,z) \in T \mid z < t \}.
\end{equation}

This is an example of a \emph{filtration of the torus}: a decomposition of $T$ as an increasing union of suitable spaces. If we compute, say, the first homology of the spaces $X_t$, we get a family of modules
\begin{equation}
V_t = H_1(X_t), t \in \R,
\end{equation}
which give us far more information that $H_1(T)$ alone.

From this point onward, we restrict our attention to the case of homology with coefficients in a field $\FF$, as the theory on persistence homology over other rings is much sparser.

The inclusion maps $i_{ts} \colon X_t \hookrightarrow X_s$ for $t \leq s$ induce linear maps $\pi_{ts} \colon V_t \to V_s$, and these maps satisfy the equation
\begin{equation}
\pi_{sr} \circ \pi_{ts} = \pi_{tr}, \quad t < s < r,
\end{equation}
as well as $\pi_{tt} = \id$. Such a collection of data (a real-indexed family of vector spaces and maps $\pi_{ts}$ as above) is called a \emph{persistence module} and besides giving us information on the topological features of the space, it gives us an idea of where they are located.

Let us look at the example of the torus from figure \ref{fig:torus1}. Its persistence homology is (up to isomorphism) given by the following description. For $t \leq z_1$, $X_t$ is empty, and for $z_1 < t \leq z_2$, $X_t$ is homeomorphic to a disk. Therefore, for $t \leq z_2$, $X_t = \{0\}$. At $t = z_2$, the disk suddenly becomes closed, so $V_t$ becomes isomorphic to $\FF$. Finally, at $t = z_3$ the second handle is closed, so $V_t$ becomes $\FF^2$.

This concludes the computation of the spaces $V_t$, but it is also necessary to specify the linear maps $\pi_{ts} \colon V_t \to V_s$. In this case, it is easy to check that they are the inclusions $\FF^n \hookrightarrow \FF^m$, where $m$ and $n$ are as appropriate. Thus, we have the persistence module of first homology the torus $T$ (associated to the particular embedding from figure \ref{fig:torus1}), but we do not yet have a good representation for it.

Enter barcodes and the Normal Form Theorem. A barcode is a simple combinatorial object, given by a finite multiset\footnote{This is a set whose elements are counted with multiplicity} of intervals in $\R$. To each barcode $B$ is associated a persistence module denoted $\FF(B)$. It is a surprising fact that, under reasonable finiteness assumptions (a module which satisfies them is said to be of \emph{finite type}), to each persistence module is associated a unique barcode:
\begin{theorem*}[Normal form theorem]
Let $V = (\{V_t\}_{t \in \R}, \{\pi_{ts}\}_{t<s})$ be a persistence module of finite type. Then, there exists a unique barcode $B$ such that $V$ is isomorphic to $\FF(B)$.
\end{theorem*}

This theorem may be applied to the persistence module of the torus computed before, yielding a barcode with two bars, which is represented in figure \ref{fig:bctorus}.

\begin{figure}
\centering
\begin{tikzpicture}[xscale=3]
\draw[->,thick] (-0.300,0.000)--(3.300,0.000);
\draw[] (0.000,-0.200)--(0.000,0.200) node[above] {$z_1$};
\draw[] (1.000,-0.200)--(1.000,0.200) node[above] {$z_2$};
\draw[] (2.000,-0.200)--(2.000,0.200) node[above] {$z_3$};
\draw[] (3.000,-0.200)--(3.000,0.200) node[above] {$z_4$};
\draw[(-,thick] (1.000,-0.500)--(3.255,-0.500) node[right] {};
\draw[(-,thick] (2.000,-0.800)--(3.255,-0.800) node[right] {};
\end{tikzpicture}
\caption{Visual representation of the barcode associated to the first homology of the torus in \ref{fig:torus1}.}\label{fig:bctorus}
\end{figure}

This correspondence is akin to a very familiar fact from linear algebra: every finite-dimensional vector space is uniquely identified by a single natural number, its dimension. Persistence modules are a bit more complicated than vector spaces, so they are represented by a more complex object, but it is still surprising that all the information in a (finite type) persistence module can be summarized in such a simple object.

\paragraph{Symplectic Geometry and Floer Homology}

Symplectic Geometry is technically not a recent area, with the first recorded instance of something akin to a symplectic structure being found in a 1809 paper by Lagrange; see \cite{marle2009inception} for an account of this and other contemporary papers using modern-day notation. However, the field had a large explosion following a 1985 article by Mikhail Gromov, after which symplectic geometry started absorbing methods from algebraic geometry, differential topology, and PDEs, especially with the introduction of a new type of homology by Andreas Floer.

The basic object of study of symplectic geometry is a symplectic manifold, which consists of a smooth manifold $M$ equipped with a rank-2 differential form $\omega$ which is non-degenerate.\footnote{For every vector $v$ tangent to $M$, there exists some $w$ tangent to the same point such that $\omega(v,w) \neq 0$.} This innocuous definition hides a large amount of complexity. For a brief introduction to the subject, see chapter 7 of \cite{polterovich}. For a slightly less brief introduction see chapter 5 of \cite{audin}, and for a standard textbook see \cite{mcduff}. For the remainder of the introduction, we will assume that the reader is familiar with basic concepts from symplectic geometry, such as almost-complex structures and Hamiltonian systems.

An important (now solved) problem in symplectic geometry, the Arnol'd conjecture, regards the number of periodic orbits of a Hamiltonian system on a symplectic manifold. The major breakthrough in proving the so-called Arnol'd conjecture was the introduction of Floer homology in 1988, but since then Floer homology has grown a life of its own.

Floer homology is a bit like a symplectic version of Morse theory: one begins with a (possibly time-dependent) Hamiltonian $H$ on a compact symplectic manifold\footnote{We also require a technical condition called \emph{asphericity}: the manifold $M$ must satisfy $\pi_2(M) = 0$, i.e. any continuous map $S^2 \to M$ must be homotopic to a constant map.} $(M,\omega)$ and counts the number of orbits of $H$ which are time-one periodic and contractible. These coincide with the critical points of a certain functional, called the \emph{action functional}, which is the start of the analogy between Floer and Morse homology: this functional $\AA$ takes the place that a Morse function would in Morse theory. Then to each such orbit one associates an integer, called the Maslov index, which takes the role of the Morse index and to which we dedicate the entirety of chapter \ref{chap:maslov}. Finally, with the help of an almost-complex structure $J$ on $M$ (which takes the role that a Riemannian metric would in Morse theory) one may define a boundary map $\partial \colon \CF_k(M,H) \to \CF_{k-1}(M,H)$, where $\CF_k(M,H)$ denotes the vector space (in a chosen field $\FF$) freely spanned by the time-one contractible periodic orbits of $H$ whose index is $k$. Like in Morse theory, this map is computed by counting the number of solutions to a certain differential equation which connect an orbit $\gamma$ of index $k$ and another orbit $\eta$ of index $k-1$. The main difference, which makes the theory much more intricate and heavy on analysis, is that while in Morse theory these connecting paths are solutions of a certain ODE, in Floer theory the paths connecting $\gamma$ and $\eta$ must satisfy a \emph{PDE}; these curves are said to be \emph{pseudo-holomorphic}. Interestingly, this PDE can also be interpreted as an ODE in the space of contractible loops, given by the flow of the negative gradient of the action functional $\AA$.

It is a surprising but well-known theorem that the Floer homology $\HF_k(M,H,J)$ (which is the homology associated to the complex $(\CF_k(M,H), \partial(M,H,J))$) does not depend on the chosen Hamiltonian $H$ or almost-complex structure $J$, and in fact agrees with the homology of $M$ in the usual sense (modulo a shift in index, depending on the conventions chosen for the Maslov index).

For an excellent introduction to the subject, including both Floer and Morse homology, see the outstanding book by Michèle Audin and Mihai Damian \cite{audin}.

\paragraph{Filtered Floer Homology}

We are now ready to unite Floer homology and persistence. The idea is to introduce a filtration, akin to the $X_t$ from the torus example above, but instead of filtering the space $M$, we filter the space of orbits.

Recall that Floer homology is the homology associated to a certain chain complex,
\begin{equation}
(\CF_k(M,H), \partial(M,H,J)),
\end{equation}
where the vector spaces $\CF_k(M,H)$ are spanned by periodic orbits of $H$. The action functional is defined on these orbits, and it is an elementary fact that
\begin{equation}\label{eq:actiondecrease}
\text{If $a \in \CF_k$, $b_1, \dots b_N \in \CF_{k-1}$, and } \partial(a) = \sum n_i b_i \text{ then } \forall_i \, \AA(b_i) < \AA(a).
\end{equation}

Therefore, if we filter the space of orbits by their action, the differential restricts nicely to the filtered spaces. In other words, if we define
\begin{equation}
\begin{gathered}
\CF_k^\lambda(M,H) = \braket{a \in \CF_k(M,H) \mid \AA(a) < \lambda},\\
\partial^\lambda = \partial|_{\CF_k^\lambda} \colon \CF_k^\lambda(M,H) \to \CF_{k-1}^\lambda(M,H),
\end{gathered}
\end{equation}
the differentials $\partial^\lambda$ are well-defined because of \eqref{eq:actiondecrease}.

This construction yields a real-indexed family of chain complexes, with an obvious chain map given by the inclusion $i_{st} \colon \CF_k^s \to \CF_k^t$ for $s \leq t$. This chain map passes to homology, which yields the so-called \emph{filtered Floer homology}, which is denoted by $\HF_k^\lambda(M,H,J)$, with $k \in \Z$ and $\lambda \in \R$, and is equipped with maps $\pi_{st} \colon \HF_k^s \to \HF_k^t$ for $s \leq t$. Unlike usual Floer homology, filtered Floer homology depends in general on the chosen Hamiltonian $H$, in the same way that the filtered homology of the torus shown in the beginning of the introduction depends on the chosen embedding of the torus in $\R^3$. However, it depends only on the time-one flow of $H$, so we can actually define the \emph{filtered Floer homology of a Hamiltonian diffeomorphism} $\phi$, denoted $\HF_k^\lambda(M,\phi)$, with associated maps $\pi_{ts} \colon \HF_k^t \to \HF_k^s$, $t<s$.

Applying the normal form theorem to the filtered Floer homology of a Hamiltonian diffeomorphism, we obtain a barcode (in each degree $k$) associated to a Hamiltonian diffeomorphism $\phi$, denoted $\BB_k(\phi)$. We may also construct the so-called \emph{total barcode of $\phi$},
\begin{equation}
\BB(\phi) = \bigcup_k \BB_k(\phi),
\end{equation}
and from here we may ask the question: `What properties of $\phi$ can we tell from its barcode?' 

This is an active area of research. Polterovich et al. \cite{polterovich} show that one can find estimates for Hofer's metric using the so-called \emph{bottleneck metric} on the space of barcodes, a subject upon which we will not touch here, and with it prove that Hofer's metric is nondegenerate for aspherical closed manifolds. Albers and Frauenfelder \cite{albers2014square} find a barcode-based necessary condition for a Hamiltonian diffeomorphism to be of the form $\psi \circ \psi$, with $\psi$ another Hamiltonian diffeomorphism. Usher \cite{usher1} defines and studies an invariant of Hamiltonian diffeomorphisms that he calls \emph{boundary depth}, which he uses in \cite{usher2} to prove that the space of Hamiltonian diffeomorphisms on a certain class of symplectic manifolds has infinite diameter wrt. Hofer's metric. These and more examples can be found in \cite{polterovich}, which provides an overview of persistence homology and its applications to symplectic geometry and a few other areas, such as topological function theory.

\paragraph{This Work}

Despite the abundance of material on properties of barcodes of Hamiltonian diffeomorphisms, the author of this thesis found next to no literature on computing barcodes of specific Hamiltonian diffeomorphisms. Indeed, the only relevant literature we have been able to find is example 8.2.4 in \cite{polterovich}, in which it is shown that the barcode of a $C^\infty$-small autonomous Hamiltonian may be computed via Morse theory; we go over this subject in chapter \ref{subsecautonomous}.

In an effort to fill this gap in the literature, we compute in detail the barcodes of two specific (non-autonomous) Hamiltonian diffeomorphisms on the torus.

We have already briefly gone over the process for computing the barcode of the time-one flow of $H$: First, compute the contractible time-one periodic orbits of $H$, then compute their actions and Maslov indices, and finally compute the differential maps. Of these steps, computation of the differential maps is the one which is most difficult, as it requires counting solutions of certain PDEs with boundary conditions at infinity, which is no easy task. Fortunately, in simple enough cases such as the ones we work with in this thesis, the barcode is actually fully determined by the fact that Floer homology (and hence filtered Floer homology for high values of $\lambda$) agrees with the usual homology of the space.

The most difficult part of computing barcodes then becomes, in our case, computing the Maslov indices of our orbits. We did not find many approaches to computing Maslov indices which proved suitable to calculation: the best we found is a formula based on signed crossings between a certain path of symplectic matrices and a certain space $\Sigma$ (see def. 8.1.1 in \cite{polterovich} or page 7 of \cite{robbin1993maslov}), but that approach proved unsuitable for the paths that we will be dealing with, as they cross $\Sigma$ an infinite number of times and hence the formula is not applicable. Therefore, we devised a new (to the best of our knowledge) way to compute Maslov indices, specialized to dimension two, which is based on computing the trace and entries of a certain path of matrices, all of which are readily accessible when computing particular examples.

Let us be a bit more specific. Let $\gamma(t)$ be a periodic orbit of a Hamiltonian $H$, whose time-$t$ flow we call $\phi_t$, and we suppose that it satisfies a technical condition called nondegeneracy, the details of which are not relevant at the moment. To define the Maslov index of $\gamma$, we begin by picking an appropriate symplectic basis $Z_1(t), \dots, Z_{2n}(t)$ for the tangent space at each $\gamma(t)$, and compute the path of symplectic matrices $A(t)$ satisfying
\begin{equation}
(\dl \phi_t)(Z_j(0)) = \sum_i A(t)_{ij} Z_i(t).
\end{equation}

Then, we compute the so-called Conley-Zehnder index of the path $A(t)$, also known as its Maslov index, denoted by $\mu(A(t))$. We use the name Maslov index in this thesis. Our main result on the Maslov index is the following formula to compute $\mu(A(t))$ in the two-dimensional case.
\begin{prop*}
Let $A(t)$, $t \in \interval 0 T$ be a path of $2 \times 2$ symplectic matrices with $A(0) = I$ and $A(T) \in Sp(2)^*$. If $\trace A(t) > -2$ for all time and $\trace A(T) > 2$, the Maslov index of $A$ is zero. Otherwise, there exists a partition of the form
\begin{equation*}
0 = a_0 < b_0 < a_1 < b_1 < a_2 < \dots < a_N < b_N \leq T
\end{equation*}
which satisfies the properties
\begin{enumerate}
\item $(-1)^n \trace A(a_n) \geq 2$,
\item $\trace A(b_n) \in \ointerval{-2}2$,
\item Whenever $\trace A(x) \geq 2$ and $\trace A(y) \leq -2$, there exists some $b_n$ between $x$ and $y$,
\item Exactly one of $\trace A(a_N)$ and $\trace A(T)$ is in $\rinterval 2 \infty$.
\end{enumerate}

For any such partition, we have the formula
\begin{equation*}
\mu(A(t)) = \sum (-1)^n \sign(A(b_n)_{12})
\end{equation*}
\end{prop*}
This proposition goes by the name of corollary \ref{calcmaslov1} below, and a significant portion of the work following it consists of applying it to different scenarios in order to expediently compute certain Maslov indices.  In \ref{sec:twodimensionalcase} we use it to compute the Maslov index of constant periodic orbits of autonomous Hamiltonians, in \ref{sec:hoferkriener} we relate our construction to an alternative definition of the Maslov index, also specialized to the two-dimensional case, due to Hofer and Kriener \cite{hoferkriener}. Finally, in chapters \ref{chap:firstexample} and \ref{chap:secondexample} we apply this formula to compute the barcodes of two specific Hamiltonians on the torus.
