% !TeX root = Thesis.tex
\chapter{The Barcode of a Small Autonomous Hamiltonian and Morse Homology}
\label{subsecautonomous}

Floer theory can be seen as a complicated symplectic equivalent to Morse theory: in both cases we compute an invariant of a space using critical points of a function or functional, and calculate differentials using the number of orbits which connect such critical points. Therefore, it is not very surprising that in the simplest cases Floer theory would reduce to Morse theory.

A standard reference for Morse theory is \cite{milnor}; see also chapters 1--4 of \cite{audin}. For the link between Morse homology and Floer homology, see chapter 10 of \cite{audin}, and for the filtered version see \cite{polterovich}. For convenience, in this chapter we summarize the main results, which trivialize the computation of barcodes of small autonomous (Morse) Hamiltonians.

To begin, we give a quick recap of filtered Morse homology.

\begin{definition}[Morse Homology]
Let $M$ be a compact smooth manifold, and let $f \colon M \to \R$ be a Morse function. For each critical point $x$ of $f$, define the index of $x$, denoted $\Ind(x)$, as the number of negative eigenvalues of the Hessian of $f$ at $x$.

Define the vector spaces of the Morse complex (over a field $\FF$), denoted $\MC_k(f)$, $k \in \Z$, as
\begin{equation}
\MC_k(f) = \braket{x \mid \text{$x$ is a critical point of $f$ with $\Ind(x) = k$}}_\FF.
\end{equation}

Suppose now that $M$ is endowed with a Riemannian metric which satisfies a technical condition called the Smale condition; details can be found in section 2.2 of \cite{audin}. Then, we may define the differential $\partial \colon \MC_k(f) \to \MC_{k-1}(f)$ by
\begin{equation}
\partial(x) = \sum n(x,y) y
\end{equation}
where $x$ is a critical point of $f$ and the sum is taken over the set of critical points $y$ with $\Ind(y) = \Ind(x) - 1$. Like in Floer homology, the numbers $n(x,y) \in \FF$ are given by counting the number of curves that connect $x$ and $y$, in the following sense. A curve $\gamma \colon \R \to M$ is said to connect the critical points $x$ and $y$ if
\begin{equation}\label{eq:curveconnect}
\begin{cases}
\lim_{t \to -\infty} \gamma(t) = x,\\
\lim_{t \to \infty} \gamma(t) = y,\\
\dot \gamma(t) = - \grad f(\gamma(t)).
\end{cases}
\end{equation}

Each such curve $\gamma$ may count positively or negatively, depending on certain orientations. To sidestep this issue, we consider $\FF = \Z_2$.

The Morse homology of $f$, denoted $\HH_*(f)$, is defined as the homology of the Morse complex $\MC_*(f)$.
\end{definition}

\begin{theorem}
The Morse homology of a compact manifold $M$ (computed using any Morse function and suitable Riemannian metric) coincides with the simplicial homology of $M$.
\end{theorem}

\begin{proof}
See chapter 4.9 of \cite{audin} for a proof that the Morse homology agrees with cellular homology, and see theorem 2.35 of \cite{hatcher} for a proof that cellular homology agrees with simplicial homology.
\end{proof}

We begin by relating the Morse homology with the Floer homology.

\begin{prop}\label{prop:mceqcf}
Let $H$ be an autonomous Morse function on a $2n$-dimensional compact symplectic manifold $M$. Then, for small enough $\varepsilon > 0$, the Hamiltonian $\varepsilon H$ is nondegenerate and the vector spaces of the filtered Floer homology of $\varepsilon H$ coincide with the vector spaces of its filtered Morse homology up to a shift of $n$. More concretely,
\begin{equation}\label{eq:mceqcf}
\MC_k(\varepsilon H) = \CF_{k-n}(\varepsilon H). 
\end{equation}
\end{prop}

\begin{proof}
By proposition \ref{prop:nonconstantorbits}, small enough multiples of autonomous Morse Hamiltonians are nondegenerate, and their time-one periodic orbits are precisely their critical points. Theorem \ref{thm:dphimatrix} and proposition \ref{prop:maslovexpjs} show that the indices match up as per equation \eqref{eq:mceqcf}.
\end{proof}

\begin{prop}\label{prop:mdeqfd}
Using the notation of of proposition \ref{prop:mceqcf}, there exists a dense set of almost-complex structures $J$ on $M$ which are compatible with $H$ both as a Morse function\footnote{In the sense that the Riemannian metric given by $\braket{u,v} = \omega(u, Jv)$ satisfies the Smale condition.} and as a Hamiltonian, so that the differential $\partial$ can be defined in the Morse complex and in the Floer complex using the same $J$.

In this event, possibly by decreasing $\varepsilon$, the Morse differential agrees with the Floer differential, and hence the Morse homology agrees with Floer homology.
\end{prop}

\begin{proof}
For the first part, see theorem 10.1.2 of \cite{audin}. For the second part, see proposition 10.1.9 of \cite{audin}.
\end{proof}

We now extend these results to the filtered case. We begin by defining filtered Morse homology.

\begin{definition}
Define the filtered Morse complex of $f$ by
\begin{equation}
\MC_k^\lambda(f) = \braket{x \mid \text{$x \in \MC_k(f)$ and $f(x) < \lambda$}}.
\end{equation}

Whenever $\gamma$ is a curve that connects the critical points $x$ and $y$ in the sense of \eqref{eq:curveconnect}, $f(y) < f(x)$. Therefore, if $n(x,y) \neq 0$ we have $f(y) < f(x)$ and so the differential restricts to the filtered Morse complex, yielding the filtered Morse homology $\HH_k^\lambda(f)$.

Associated to this filtered homology are obvious maps $\pi_{\lambda \tau} \colon \HH_k^\lambda(f) \to \HH_k^\tau(f)$ for $\lambda \leq \tau$, and it is straight-forward to check that the collection of data $(\{\HH_k^\lambda(f)\}_{\lambda \in \R}, \{\pi_{\lambda\tau}\}_{\lambda \leq \tau})$ forms a persistence module of finite type, and so by the Normal Form theorem we may associate a barcode to any Morse function $f$ via the Morse homology.
\end{definition}

\begin{theorem}\label{thm:barcodemorse}
Let $H$ be an autonomous Morse function on a compact symplectic manifold $M$ of dimension $2n$. Let $\varepsilon > 0$ be small enough that propositions \ref{prop:mceqcf} and \ref{prop:mdeqfd} apply. Then, the persistence module induced by Morse homology in degree $k$ coincides with the persistence module induced by Floer homology in degree $k-n$. Consequently, the barcode of $H$ as induced by Floer homology can be computed via Morse homology.
\end{theorem}

\begin{proof}
We have already shown that under these hypotheses the periodic orbits of $H$ coincide with its critical points. The computations in the proof of corollary \ref{cor:orbitsepsH} show that the action of a constant orbit $x$ agrees with $H(x)$, so the filtered vector spaces agree, i.e.
\begin{equation}
\MC_k^\lambda(\varepsilon H) = \CF_{k-n}^\lambda(\varepsilon H).
\end{equation}

Since the differential maps in the filtered Floer and Morse complexes are appropriate restrictions of the differential maps in the unfiltered case, which we already know coincide, the filtered Floer and Morse complexes agree, and so the filtered Floer homology coincides with the filtered Morse homology.

Finally, it is easy to show that the maps $\pi_{\lambda\tau}$ are defined in exactly the same way in the Morse and Floer case, so that filtered Floer and Morse homology agree as persistence modules. Consequently, they have the same associated barcode.
\end{proof}

Theorem \ref{thm:barcodemorse} trivializes the computation of (Floer) barcodes of autonomous Hamiltonians, given that solving the ODEs involved in computing Morse homology is much easier than solving the PDEs necessary for Floer homology.

