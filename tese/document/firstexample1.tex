% !TeX root = main.tex
\section{A First Example: Computing a Barcode (Part one)}\label{subsecautonomous}

Let $H$ be an autonomous Hamiltonian on a compact symplectic manifold $M$. Assuming that $H$ is a non-degenerate Hamiltonian, what can we say about the barcode of $\phi_H$?

We begin by characterizing the periodic orbits of $H$.

\begin{prop}\label{authamfloermorse}
If $H$ is a non-degenerate autonomous Hamiltonian, its only periodic orbits are fixed points, which coincide with the critical points of $H$. These critical points have non-degenerate Hessian, and therefore $H$ is a Morse function.

Furthermore, the indices of these critical points (seeing $H$ as a Morse function) are related to their Maslov indices by the formula
\begin{equation}
\Ind(x) = \mu(x) + n,
\end{equation}
where $\mu$ is the CZ index and $M$ is $2n$-dimensional.
\end{prop}

\begin{proof}
Suppose that $H$ has a non-constant periodic orbit $\gamma(t)$. Then, intuitively, every point of the form $\gamma(t)$ is a fixed point of $\phi$, where $\phi$ is the time-one flow of $H$, and thus the fixed points of $\phi$ are not isolated.

As a more rigorous proof, we show that $H$ is a degenerate Hamiltonian directly. Since $\gamma$ is not constant, but it is the flow of the autonomous vector field $X^H$, the vector field $X^H$ must not vanish at any point of $\gamma$. In other words, if we set $x = \gamma(0)$, $X^H_x \neq 0$.

Now, note that
\begin{equation}\label{dlphixhxh}
(\dl \phi)(X^H_x) = X^H_x,
\end{equation}
which shows that $(\dl \phi)_x$ has a one-eigenvalue, i.e. it is degenerate. To show \eqref{dlphixhxh}, simply note the equality $\phi(\gamma(t)) = \gamma(t)$, and differentiate both sides.

This completes the proof that if $H$ is a non-degenerate autonomous Hamiltonian, its periodic orbits coincide with the critical points of $H$. Now we show that their Hessians are non-degenerate.

Let $v$ and $w$ be two vectors tangent to $x$. Recall that the Hessian of $H$ at $x$, applied to $v$ and $w$, which we will denote $D^2 H(v,w)$, is defined as
\begin{equation}
D^2 H(v,w) = v \cdot (Y \cdot H) \text{, with $Y$ any extension of $w$.}
\end{equation}

In our case, this can be simplified as
\begin{equation}\label{hessian1}
D^2 H(v,w) = v \cdot (\dl H)(Y) = - v \cdot \omega(X^H, Y).
\end{equation}

To proceed, we apply a sort of Leibniz rule, in the sense that
\begin{equation}\label{fakeleibniz}
\text{``$v \cdot \omega(X^H, Y) = \omega(v \cdot X^H, Y) + \omega(X^H, v \cdot Y)$''.}
\end{equation}

Now, to make sense of \eqref{fakeleibniz}, we need to be able to interpret the expression $v \cdot X^H$. This can be taken as a Lie derivative, or equivalently, a Lie bracket, which requires extending $v$ to a vector field $X$. Furthermore, note that the second term, $\omega(X^H, [X,Y])$, vanishes because $X^H$ is null at $x$. Consequently, we aim to prove
\begin{lemma}\label{leibniz1}
Let $X$, $Y$ and $Z$ be vector fields in a neighborhood of $x$, and suppose that $Y_x = 0$. Then,
\begin{equation}
X \cdot \omega(Y,Z) = \omega([X,Y],Z).
\end{equation}
\end{lemma}

\begin{lemmaproof}
Define the auxilliary form $\eta = Z \into \omega$, and apply the well-known formula for the exterior derivative of a one-form
\begin{equation}
(\dl \eta)(X,Y) = X \cdot \eta(Y) - Y \cdot \eta(X) - \eta([X,Y]).
\end{equation}

Now, note that since $Y$ is null at $x$, two of these terms vanish, giving us the equation
\begin{equation}
X \cdot \eta(Y) = \eta([X,Y]) \equiv X \cdot \omega(Y,Z) = \omega([X,Y],Z),
\end{equation}
as desired.
\end{lemmaproof}

We may now use lemma \ref{leibniz1} to simplify \eqref{hessian1} as
\begin{equation}
D^2 H(v,w) = - v \cdot \omega(X^H, Y) = - \omega([X,X^H], w).
\end{equation}

The nondegeneracy of the Hessian will follow from the nondegeneracy of the symplectic form, from the moment that we show that $[X,X^H]_x \neq 0$ for every $X$ with $X_x \neq 0$.

\begin{lemma}\label{bracketnondegen}
If $[X,X^H] = 0$ then $(\dl \phi) v = v$.
\end{lemma}

\begin{lemmaproof}
To investigate $(\dl \phi) v$, it is useful to recall that it is the time-one flow of $X^H$. Let $\phi_t$ be the flow at time $t$. We use this to set up an ODE for the expression $(\dl \phi_t) v$, but this has slight technical problems, so instead we set up an ODE for the expression
\begin{equation}
((\dl \phi_t) v) \cdot f\text{, for $f \colon M \to \R$.}
\end{equation}

To this effect, let $v(s)$ be a curve whose derivative at $s=0$ equals $v$. Then,
\begin{equation}
\begin{split}
((\dl \phi_0) v) \cdot f &= v \cdot f,\\
\diff*{((\dl \phi_t) v) \cdot f}t &= \diff{}t \left( \diff{}s[0] f(\phi_t(v(s))) \right)\\
&= \diff{}s[0] \left( \diff{}t f(\phi_t(v(s)))\right)\\
&= \diff{}s[0] X^H_{v(s)} \cdot f\\
&= v \cdot (X^H \cdot f),
\end{split}
\end{equation}
and since $X^H$ vanishes at $x$ it is obvious that
\begin{equation}
v \cdot (X^H \cdot f) = [X,X^H]_x \cdot f.
\end{equation}

As a consequence, if $[X,X^H]_x = 0$ then $((\dl \phi_t)v) \cdot f$ is constant equal to $v \cdot f$, and hence $(\dl \phi) v = v$.
\end{lemmaproof}

As a consequence of lemma \ref{bracketnondegen} and the nondegeneracy of $\phi$, $[X,X^H]$ does not vanish at $x$ unless $v = 0$, and consequently the Hessian of $H$ at $x$ is nondegenerate. This completes the proof that $H$ is a Morse function.

The proof of the relation between the Morse and Maslov indices is ommitted for the time being, because I don't know how in depth I will want to go on that subject.
\end{proof}

We have now shown that under these circumstances, the Morse complex coincides with the Floer complex, at least in regards to the vector spaces. As it happens, this is still true if one considers the filtered complexes.

\begin{prop}
Let $\phi$ be a nondegenerate Hamiltonian diffeomorphism generated by the Morse function $H$. Then, the following vector spaces coincide
\begin{equation}
\CF^\lambda_*(M,\phi) = \MC^\lambda_*(M,H).
\end{equation}
\end{prop}

\begin{proof}
A simple calculation shows that if $x$ is a critical point of $H$, then its action (as a periodic orbit) coincides with $H(x)$.
\end{proof}

Unfortunately, the complexes, and therefore the homologies, need not coincide, because the differential is not necessarily the same. Indeed, it is easy to check that if $u(s)$ is an orbit connecting two critical points of $H$ then it induces a $J$-holomorphic curve connecting those two points as periodic orbits (assuming that the choice of Riemannian metric for Morse purposes is compatible with the complex structure chosen for Floer purposes), but it is possible that there are $J$-holomorphic curves which are not of the form $u(t,s) \equiv u(s)$.

It is true that if the Hamiltonian is $C^2$-small enough, all $J$-holomorphic curves are in fact idependent of $t$, but the proof is tecnical. We refer to \cite{audin}, proposition 10.1.9.

\begin{prop}
If $H$ is $C^2$-small, the filtered Morse complex and the filtered Floer complex coincide, and consequently as do the filtered Morse and Floer homologies.
\end{prop}

In the general case, knowing that a Hamiltonian diffeomorphism is generated by an autonomous Hamiltonian gives us little information regarding its complex. However, little is not none, and the following proposition represents an obstruction for a Hamiltonian diffeomorphism to be generated by an autonomous Hamiltonian.

\begin{prop}
Let $\phi$ be a non-degenerate Hamiltonian diffeomorphism. For each $k \in \Z$, let
\begin{equation}
\mu_k = \inf \{\, \lambda \in \R \mid \HF_k^\lambda(M,\phi) \neq 0 \, \}.
\end{equation}

Then, if $\phi$ is generated by an autonomous Hamiltonian, $\mu_0 < \mu_k$ for all $k \neq 0$.

[Note: The indices are iffy because we're using the Morse convention, not Maslov.]
\end{prop}

\begin{proof}
Suppose that $\phi$ is generated by the autonomous Hamiltonian $H$. Let $m = \min H$.

\begin{lemma}\label{mincritval}
There exists $\varepsilon > 0$ such that every critical value of $H$ of index different from zero is at least $m+\varepsilon$.
\end{lemma}

\begin{lemmaproof}
If $y$ is a critical value of index at least one, it cannot be a global minimum, as can easily be seen through a Morse neighborhood. Consequently, $y > m$, and the lemma follows since there are finitely many critical values.
\end{lemmaproof}

Clearly, for $k \neq 0$, $\mu_k > m+\varepsilon$, where $\varepsilon$ is as in lemma \ref{mincritval}. We claim that $\mu_0 < m+\frac\varepsilon2$.

To show this fact, consider $\lambda = m+\frac\varepsilon2$, and look at the filtered Floer complex for this value of $\lambda$. By proposition \ref{authamfloermorse}, $\CF_*^\lambda(M,\phi) = \MC_*^\lambda(M,H)$, and the Morse complex is easily checked to be trivial at all degrees other than zero, and nontrivial at $*=0$. This therefore holds for the Floer complex, and even though we usually have no control over the differential, in this case we know that all differentials must be zero. In particular,
\[\HF_0^\lambda(M,\phi) = \CF_0^\lambda(M,\phi) = \MC_*^\lambda(M,H) \cong \Z_2^n,\]
where $n$ is the number of critical points with value below $\lambda$, which is at least one.
\end{proof}

\begin{corollary}\label{corautonomous}
A necessary condition for a nondegenerate Hamiltonian diffeomorphism $\phi$ on a compact aspherical manifold to be generated by an autonomous Hamiltonian is that
\begin{equation}
\mu(\BB_0(\phi)) < \mu(\BB_*(\phi))\text{, for $* \neq 0$,}
\end{equation}
where we define $\mu$ of a barcode $B$ as the infimum of the lower endpoints of the bars of $B$.
\end{corollary}


\subsection{Application: A non-autonomous Hamiltonian (Part one)}

In this section, we will do an in-depth study of a specific Hamiltonian diffeomorphism on the torus, concluding with an application of the results of \ref{subsecautonomous}, namely of corollary \ref{corautonomous}.

First we introduce the object of study. Let $M = S^1 \times S^1$ be the torus with the usual symplectic form $\dl x \wedge \dl y$. We consider the $x$ and $y$ coordinates to be $2\pi$-periodic.

On this manifold, a class of Hamiltonian diffeomorphisms is given by those diffeomorphisms of the form
\begin{equation}
\phi_f(x,y) = (x, y + f(x)),
\end{equation}
so long as $f$ is a smooth $2\pi$-periodic function with null mean. Indeed, it is easy to check that the Hamiltonian $H(x,y) = \int_0^x f(t) \dl3 t$ has the map $\phi_f$ as time-one flow.

The same can be done in the other coordinate, and two functions of this type can be composed to yield Hamiltonian maps of the form
\begin{equation}
(x,y) \mapsto (x + g(y+f(x)), y+f(x)).
\end{equation}

To conclude, we define our object of study as the particular case when $f = g = a \sin$, where $a$ is a real parameter, yielding the Hamiltonian diffeomorphism
\begin{equation}
\phi(x,y) = ( x + a \sin(y + a \sin(x)), y + a \sin(x)).
\end{equation}

It is a classical result that composition of two Hamiltonian diffeomorphisms is itself a Hamiltonian diffeomorphism. A Hamiltonian corresponding to $\phi$ can be found by using a bump function to smooth the function
\begin{equation}
H_0(x,y,t) = \begin{cases}
-\cos(x), & t < a,\\
\cos(y), & t > a.
\end{cases}
\end{equation}

More precisely, if we let $\varphi(t)$ be a function with compact support contained in $\ointerval 0 \pi$ and unit integral, the Hamiltonian given by
\begin{equation}
H_1(x,y,t) = \begin{cases}
-\cos(x) \varphi(t), & t \leq a,\\
\cos(y) \varphi(t-\pi), & t \geq a.
\end{cases}
\end{equation}
will have as time $2a$	 flow the Hamiltonian diffeomorphism $\phi$.

We also assume, without loss of generality, that $\varphi$ is symmetrical around $t = \pi/2$. This will be useful to simplify certain calculations, as it endows $H_1$ with several useful symmetries.

\subsubsection{The contractible periodic orbits}

The first step to compute the (filtered) Floer homology is to find the periodic orbits of the Hamiltonian $H_1$, in this case of time $2a$. This is the same as to find the fixed points of $\phi$, which amounts to solving the system
\begin{equation}
\begin{cases}
x \equiv x + a \sin(y + a \sin(x)) &\mod 2\pi,\\
y \equiv y + a \sin(x) &\mod 2\pi.
\end{cases}
\end{equation}

However, the resulting fixed points will not necessarily correspond to contractible orbits. To solve this problem, note that $\R^2$ is the universal covering space of the torus. Consequently, any path in the torus can be lifted to one in $\R^2$, and the contractible paths are precisely those that start and end at the same point. Therefore, to find the contractible orbits it suffices to solve the system
\begin{equation}
\begin{cases}
x = x + a \sin(y + a \sin(x)),\\
y = y + a \sin(x).
\end{cases}
\end{equation}

The following proposition is easy to check.

\begin{prop}
For $a > 0$, the Hamiltonian $H_1$ has exactly four contractible periodic orbits, all of which are constant at the points: $(0,0)$, $(0,\pi)$, $(\pi,0)$, and $(\pi,\pi)$.
\end{prop}

The computation of the action of each of these orbits is trivial to compute. Since they are constant orbits, the $\int_D \omega$ term in the definition of the action vanishes, and all that is left is to compute the integral of $H_1$ with respect to time in $\interval 0 {2a}$, which is equal to $a(\cos(y) - \cos(x))$.

Now that we know the periodic orbits of $H_1$, all that remains is to compute their Maslov indices and differentials. To compute the Maslov indices it is necessary to take a detour, in order to build the tools necessary.