%%%%%%%%%%%%%%%%%%%%%%%%%%%%%%%%%%%%%%%%%%%%%%%%%%%%%%%%%%%%%%%%%%%%%%
%     File: ExtendedAbstract_abstr.tex                               %
%     Tex Master: ExtendedAbstract.tex                               %
%                                                                    %
%     Author: Andre Calado Marta                                     %
%     Last modified : 2 Dez 2011                                     %
%%%%%%%%%%%%%%%%%%%%%%%%%%%%%%%%%%%%%%%%%%%%%%%%%%%%%%%%%%%%%%%%%%%%%%
% The abstract of should have less than 500 words.
% The keywords should be typed here (three to five keywords).
%%%%%%%%%%%%%%%%%%%%%%%%%%%%%%%%%%%%%%%%%%%%%%%%%%%%%%%%%%%%%%%%%%%%%%

%%
%% Abstract
%%
\begin{abstract}


The filtered Floer homology of a symplectic manifold $M$ (satisfying certain technical restrictions) is essentially a version of Floer homology which is parametrized by the actions of the orbits. This makes it a finer invariant than usual Floer homology, at the cost of dependence on the Hamiltonian chosen to calculate it. However, some independence can be recovered, and so filtered Floer homology can be said to be an invariant of Hamiltonian diffeomorphisms. There is already a body of literature on the subject, which is an active area of research, but there are very few concrete examples of filtered Floer homology being computed for particular Hamiltonian diffeomorphisms. In this thesis, we compute the filtered Floer homology of a couple of classes of Hamiltonian diffeomorphisms on the torus. In the process, we give a practical method for computation of Maslov/Conley-Zehnder indices.

\noindent{{\bf Keywords:}} symplectic geometry, persistence homology, floer homology, maslov index, conley-zehnder index
 \\


\end{abstract}

