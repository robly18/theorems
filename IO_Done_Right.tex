\documentclass{article}

\usepackage{amsmath}
\usepackage{amsthm}
\usepackage{amsfonts}
\usepackage[utf8]{inputenc}
\usepackage[portuguese]{babel}

\addto\captionsportuguese{
	\renewcommand*{\proofname}{Dem}
}

\title{IO Done Right}
\author{Duarte Maia}
\date{}

\newcommand{\R}{\mathbb{R}}

\newtheorem{teorema}{Teorema}

\theoremstyle{definition}
\newtheorem{definition}{Definição}
\newtheorem*{notacao}{Notação}

%\newenvironment{notacao}{\smallskip\noindent\textbf{Notação:}}{\smallskip \\}

\begin{document}
	\maketitle
	
	\tableofcontents
	
	\section{Introdução}
	
	TODO: Escrever palavras motivadoras ou algo assim
	
	\section{Prerequisitos}
	
	Vou aqui escrevendo coisas à medida que elas são necessárias.
	
	Obviamente é preciso saber um pouco de cálculo I e AL.
	
	\section{Problemas de Otimização Linear}
	
	\subsection{Introdução}
	
	Um problema de otimização é um tipo específico de problema em que o objetivo é minimizar ou maximizar uma certa função (função-objetivo), dentro de um certo domínio, usualmente (no contexto desta cadeira) um subconjunto de $\R^n$ parametrizado por um conjunto de (in)equações.
	
	Mais concretamente, o tipo de problemas com que se lida nesta cadeira são problemas de otimização linear. Estes são problemas em que a função-objetivo é uma função linear $\R^n \rightarrow \R$ e todas as condições no domínio são da forma $ax \leq b$, $ax = b$ ou $ax \geq b$, onde $a$ é um vetor-linha e $b$ é um escalar.
	
	\begin{definition}
	Um \emph{problema de otimização linear} (normalmente abreviado a \emph{pol}) é um problema da forma:
	
	\textbf{Objetivo: } maximizar/minimizar (em função de $x \in \R^n$) o valor de ${c_1 x_1 + c_2 x_2 + \ldots + c_n x_n}$
	
	\textbf{Restrições: } $x$ tem de obedecer a todas as seguintes igualdades

\[
\begin{cases}
	a_{11} x_1 + a_{12} x_2 + \ldots + a_{1n} x_n = b_1 \\ 
	a_{21} x_1 + a_{22} x_2 + \ldots + a_{2n} x_n = b_2 \\
	\vdots \\
	a_{p1} x_1 + a_{p2} x_2 + \ldots + a_{pn} x_n = b_p
\end{cases}
\]

E todas as seguintes desigualdades

\[
\begin{cases}
	a'_{11} x_1 + a'_{12} x_2 + \ldots + a'_{1n} x_n \leq b'_1 \\ 
	\vdots \\
	a'_{q1} x_1 + a'_{q2} x_2 + \ldots + a'_{qn} x_n \leq b'_q
\end{cases}
\]
\[
\begin{cases}
	a''_{11} x_1 + a''_{12} x_2 + \ldots + a''_{1n} x_n \geq b''_1 \\ 
	\vdots \\
	a''_{r1} x_1 + a''_{r2} x_2 + \ldots + a''_{rn} x_n \geq b''_r
\end{cases}
\]
	\end{definition}
	
	Infelizmente, isto é mesmo nojento de escrever. Como tal, normalmente estas condições são escritas de forma mais compacta usando a linguagem da álgebra linear.
	
	\begin{notacao}
Dados dois vetores $x,y \in \R^n$, dizemos $x \leq y$ se $x_i \leq y_i$ para todo $i$.
	\end{notacao}
	\begin{notacao}
	Considere-se o pol escrito acima. Este é normalmente escrito da seguinte forma:
	
	(Supõe tratar-se de um problema de maximização; caso contrário escreva-se $\min$ no lugar de $\max$)
	
	Seja $c$ o vetor linha $[\,c_1\,c_2\, \cdots \,c_n\,]$,

	$A$ a matriz $p \times n$ cujo $i,j$-ésimo elemento é $a_{ij}$, e considerações análogas para $A'$ a $A''$, com $a'_{ij}$ e $a''_{ij}$ respetivamente
	
	$b$ o vetor coluna $(b_1, b_2, \cdots, b_n)$ e análogamente para $b'$ e $b''$.
	
	\[
	\begin{cases}
	\max\limits_x cx \\
	Ax = b \\
	A'x \leq b' \\
	A''x \geq b''
	\end{cases}
	\]
	\end{notacao}
	
	Normalmente, não é útil considerar um tipo de problema tão geral, pelo que discutímos em baixo formas de passar de um tipo de problemas para outros.
	
	\subsection{Tradução}
	
	Considere-se os seguintes problemas de otimização linear:
	
	\[
	\begin{cases}
	\max\limits_x 2x_1 - 3x_2 \\
	5x_1 + x_2 = 6 \\
	2x_1 + 0x_2 \leq 2
	\end{cases}
	\begin{cases}
	\min\limits_x -2x_1 + 3x_2 \\
	5x_1 + x_2 \leq 6 \\
	5x_1 + x_2 \geq 6 \\
	-2x_1 + 0x_2 \geq -2
	\end{cases}
	\]
	
	Apesar de terem uma aparência diferente, alguma inspeção leva à conclusão que estes são, na realidade, exatamente o mesmo problema. Ou seja, o mesmo problema pode ser representado de mais que uma forma diferente. Isto leva à possíbilidade de considerar `formas canónicas' de expressar os problemas, que sejam mais fáceis de estudar. Se conseguirmos arranjar forma(s) de expressar os pols que sejam bem estudadas, e arranjarmos forma de traduzir qualquer pol para esta forma, estamos numa boa situação.
	
	\begin{notacao}
	Dado um pol $P$,o conjunto de $x \in \R^n$ que satisfazem as suas condições é denominado de \emph{conjunto admissível de $P$}, representado por $X_P$, ou só $X$ se o pol em questão for óbvio de contexto.
	
	Da mesma forma, o \emph{conjunto solução de $P$}, $S_P$, ou $S$ se $P$ é claro de contexto, é o conjunto de $x \in X$ que maximizam, de facto, a função objetivo. Por outras palavras,
	
	\[S_P = \{\,x \in X_p \mid \forall_{y \in X_p} cx \geq cy\,\}\]
	
	Isto no caso de maximização. No caso de minimização, a desigualdade deve estar trocada.
	\end{notacao}
	
	Há várias formas possíveis de transformar um problema noutro.
	
	Por exemplo, a condição de igualdade $a = b$ pode ser expressa como duas desigualdades: $a \leq b$ e $a \geq b$. Assim sendo, sabemos à partida que qualquer pol que nos venha à cabeça pode ser expresso só com condições de desigualdade.
	
	Para mais, a desigualdade $a \geq b$ é equivalente a $-a \leq -b$. Logo, podemos sempre assumir que as condições que restringem um pol são sempre da forma $Ax \leq b$.
	
	Podemos também assumir sem perda de generalidade que um problema de otimização linear é um problema de maximização, pois minimizar $cx$ é o mesmo que maximizar $-cx$.
	
	Logo, dado um pol qualquer, podemos sempre assumir, sem perda de generalidade, que é da forma
	
	\[
	\begin{cases}
	\max\limits_x cx\\
	Ax \leq b
	\end{cases}
	\]
	
	O que foi até agora descrito é o tipo mais simples possível de tradução de um problema a outro: trocar as condições por condições equivalentes, e substituir `maximizar $f$' por `minimizar $-f$'.
	
	\subsection{Canónico vs. Padrão}
	
	

\end{document}