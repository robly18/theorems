\documentclass{article}

\usepackage{amsmath}
\usepackage{amsthm}
\usepackage{amsfonts}
\usepackage{enumerate}

\title{Probability Theory}
\author{}
\date{}

\newtheorem{prop}{Prop}
\newtheorem{theorem}{Theorem}

\theoremstyle{definition}
\newtheorem{definition}{Definition}

\newcommand{\M}{\mathcal{M}}
\newcommand{\CN}{\mathcal{N}}
\DeclareMathOperator{\PP}{\mathcal{P}}

\newcommand{\B}{\mathcal{B}}

\newcommand{\T}{\mathcal{T}}
\newcommand{\A}{\mathcal{A}}

\newcommand{\N}{\mathbb{N}}
\newcommand{\R}{\mathbb{R}}

\newcommand{\cc}{\mathsf{c}}

\newcommand{\dd}{\mathrm{d}}


\begin{document}
	\maketitle
	
	\section{Measure theory 101}
	
	\subsection{Basic definitions}
	
	Fix a set $\Omega$, which will be our so-called \emph{sample space}.
	
	\begin{definition}
	A \emph{sample space} or \emph{$\sigma$-algebra on $\Omega$} is a set $\M \subseteq \PP(\Omega)$ satisfying the following properties:
	
	\begin{enumerate}[i)]
	\item $\emptyset, \Omega \in \M$;
	
	\item If $A \in \M$, then $A^\cc := \Omega \setminus A \in \M$.
	
	\item If $\{A_n\}_{n \in \N}$ is a countable collection of sets in $\M$, then $\bigcup A_n \in \M$.
	\end{enumerate}
	\end{definition}
	
	These properties (closed under complement and countable unions) imply $\M$ is also closed under countable intersection.
	
	\begin{definition}
	Fixed a $\sigma$-algebra $\M$ on $\Omega$, a function $\mu : \M \to \left[0, \infty \right]$ is called a \emph{measure} if $\mu(\emptyset) = 0$ and $\mu$ is countably additive, that is, if $\{A_n\}$ is a countable collection of disjoint elements of $\M$, we have
	
	\[\mu(\bigcup A_n) = \sum \mu A_n.\]
	\end{definition}
	
	A measure $\mu$ is said to be \emph{finite} if $\mu(X) < \infty$. It is said to be \emph{$\sigma$-finite} if there exists a countable collection $\{A_n\}$ of sets of finite measure whose union covers $X$.
	
	\subsection{Topological spaces}
	
	Among the several trivial examples of $\sigma$-algebras, a certain type stands out that turns out to be fundamental when speaking of measures on $\R$.
	
	It is known that there is no measure on $\PP(\R)$ that extends the notion of `length of an interval' reasonably. (Rip Vitali set) Therefore, instead of aiming so high, it is convenient to look for how little work we can do.
	
	In particular, if we want to define a measure on a $\sigma$-algebra that contains all the intervals, we may begin by asking what is the \emph{smallest} $\sigma$-algebra that contains all intervals. Fortunately, arbitrary intersections of $\sigma$-algebras are also $\sigma$-algebras, it makes sense to speak of the smallest $\sigma$-algebra containing a certain class of sets $\A$. We call this \emph{the $\sigma$-algebra generated by $\A$}, sometimes denoted $\M(\A)$.
	
	More generally, fixed a topological space $\T$, we usually consider the open sets to be `nice' in some way. As such, we may wish to look for a measure defined on all the opens. This means it would have to be defined on the $\sigma$-algebra generated by $\T$, which appears so often it is given a name: the \emph{Borel} $\sigma$-algebra, denoted $\B_\T$, or, depending on the context, $\B_\Omega$ or just $\B$.
	
	Back to the particular case of $\R$, the Borel $\sigma$-algebra is precisely the same as the one generated by the intervals, or by the closed intervals, or by the open intervals, or by the half-open intervals, or by the intervals of the form $\left[a, \infty\right[$, or... (The list goes on and on.)
	
	A measure $\mu$ on $\Omega$ is said to be a \emph{Borel measure} if, assuming $\Omega$ is endowed with a topology, the $\sigma$-algebra $\mu$ is defined on contains all the Borel sets. Such a measure is said to be \emph{locally finite} if it is finite on all compacts.
	
	\section{Lebesgue-Stieltjes measures}
	
	This section is dedicated to the construction of Lebesgue-Stieltjes measures on $\R$. In simple terms, there is a correspondence between locally finite measures on the Borels and right-continuous increasing functions on $\R$.
	
	For simplicity, we will henceforth use the term RCIF to mean `right-continuous increasing function'.
	
	It is easy to, given a locally finite Borel measure, find a RCIF $F$ such that
	\begin{equation}\label{lsm}\mu \left]a, b\right] = F(b) - F(a).\end{equation}
	
	Indeed, simply define
	\[
	F(x) = \begin{cases}
	\mu \left]0, x \right] & \text{if $x \geq 0$}\\
	-\mu \left]x, 0\right] & \text{if $x \leq 0$}
	\end{cases}
	\]
	
	On the other hand, to, given a RCIF $F$, construct a measure $\mu$ such that \eqref{lsm} is verified, is significantly harder. In any case, a Borel measure $\mu$ under these conditions is called \emph{the Lebesgue-Stieltjes measure defined by $F$}. Not only is it not obvious that such a thing exists, it is also not clear that it should be unique. And yet.
	
	Todo: maybe do this.
	
	\section{Measurable functions}
	
	Analogously to the notion of continuous function, which is one that in some way preserves the structure of the topology, one can define a \emph{measurable function} as one that, in some sense, preserves the structure of a measure space.
	
	Concretely, if $(X, \M)$ and $(Y, \CN)$ are sample spaces, a function $f : X \to Y$ is said to be measurable if the preimage of any element of $\CN$ is in $\M$. That is, the preimage of a measurable is measurable.
	
	Measurable functions are of utmost importance in the theory of integration, because it is precisely on these functions that we define the Lebesgue integral.
	
	Furthermore, it is measurable functions that capture the concept of random variable. Indeed, a, let's say real-valued, random variable $X$ is no more than a measurable function from the sample space $(\Omega, \M)$ to the real line endowed with the Borel $\sigma$-algebra.
	
	It is semi-obvious that, if $f$ is a continuous function, the codomain is endowed with the Borel $\sigma$-algebra, and the domain with an extension of the Borel $\sigma$-algebra, then $f$ is measurable.
	
	\section{Integrability}
	
	Fix a function $f : \Omega \to \R$. We say $f$ is \emph{Borel measurable} if it is measurable as a function from $(\Omega, \M)$ to $(\R, \B)$.
	
	These are precisely the set of functions we can integrate over. We need these conditions because we need preimages of intervals to be measurable, so that we can quantify `how much of the function is above a certain value'.
	
	For now, we will restrict ourselves to Borel measurable functions $f : \Omega \to \left[0, \infty\right]$. The set of these functions is called $L^+$.
	
	We say $f \in L^+$ is a \emph{simple function} if its image is finite. Fix such a function, with image set $\{a_0, \cdots, a_n\}$.
	
	We define the integral of $f$ with respect to the measure $\mu$ on $\Omega$ as
	\[ \int f \dd \mu := \sum a_i \cdot \mu f^{-1}(a_i). \]
	
	This is always defined, even if equal to infinity.
	
	Note that, as is defined, the integral is kind of linear. That is, $\int f+g = \int f + \int g$ and $\int cf = c \int f$ for $c \in \left[0, \infty\right]$.\footnote{If we adopt the convention that $\infty \times 0 = 0$, which is customary when working with measure theory.} It is also monotone, that is, if $f \leq g$ then $\int f \leq \int g$. Finally, the function $E \mapsto \int_E f$ is a measure on $\Omega$.

	Now, define the integral of \emph{any} measurable function $f$ as
	\[ \int f \dd \mu := \sup \int s \,\dd \mu\]
	where the supremum is taken over all simple functions $s$ such that $s \leq f$.
	
	This is, in fact, an extension of the previous definition. It also satisfies monotony and induces a measure. To help us show kind-of-linearity, we first show the following proposition:
	
	\begin{prop}
	Let $s_n$ be an increasing sequence of simple functions that converges pointwise to $f$. Then, $\int f = \lim \int s_n$.
	
	Furthermore, for any measurable $f$ we can find such a sequence.
	\end{prop}
	
	\begin{proof}
	To find the sequence there is a classical argument which I won't reproduce here because it takes too many words.
	
	The other statement is a simple application of the MCT.
	\end{proof}
	
	\begin{prop}
	(MCT) Suppose $f_n \nearrow f$, all $f_n \in L^+$. Then, $f \in L^+$ and its integral is given by the limits of the integrals.
	\end{prop}
	
	\begin{proof}
	First, to show $f$ is in $L^+$, we show the preimage of a set of the form $\left]a, \infty\right]$ is measurable. To do this, notice that if $f(x) > a$ then there exists some $n$ such that $f_n(x) > a$. Therefore, if we name $E_n$ as the set of $x$ such that $f_n(x) > a$, all of these are measurable, they increase to $f^{-1}(\left]a, \infty\right])$ and therefore this last set is also measurable.
	
	Now that we have shown $f$ measurable, we can reason about $\int f$. 
	
	Since all $f_n$ are $\leq f$, clearly $\lim \int f_n \leq \int f$.
	
	On the other hand, we will now show $\lim \int f_n \geq \int f$. If the latter is 0 this is clear, so we suppose $\int f > 0$.
	
	Fix any $\alpha < 1$. Define $E_n$ as the set of points where $f_n \geq \alpha f$. Clearly $E_n \nearrow \Omega$. Since $E \mapsto \int_E f$ is a measure, $\lim \int_{E_n} \alpha f \rightarrow \alpha \int f$. I think this is trivial now, since $\int f_n \geq \int_{E_n} f_n \geq \int_{E_n} \alpha f$.
	\end{proof}
	
	We can now show that sums of $L^+$ functions are also $L^+$
	
	Fix any two functions $f, g \in L^+$. Let $s_n, t_n$ be increasing sequences of simple functions converging to each. Then, $s_n + t_n$ converges to their sum, its integral converges to the sum of their integrals etc.
	
	I think this also works for the product, which is nice. I mean, not that the product of the integral is the integral of the products obviously. But if $f$ and $g$ are measurable, so is $fg$.
	
	\section{Product spaces}
	
	Given a collection $\{(\Omega_\alpha, \M_\alpha)\}_{\alpha \in I}$ of $\sigma$-algebras, we define the product space $\bigotimes_{\alpha\in I} \M_\alpha$ as the $\sigma$-algebra generated by the sets of the form
	\begin{equation}\label{basesets}
	\pi_\alpha^{-1}(E)\text{, for $\alpha \in I$ and $E \in \M_\alpha$}.
	\end{equation}
	
	This $\sigma$-algebra on $\prod_{\alpha \in I} \Omega_\alpha$ corresponds to the category-theoretic notion of product, which shows, among other things, that it is associative up to canonical isomorphism.
	
	The product of two sample spaces corresponds to running two experiments separately and independently of one another.
	
	If there for each $\alpha$ we have defined a measure on $\Omega_\alpha$, we can define a measure $\mu = \prod \mu_\alpha$ on $\bigotimes \M_\alpha$ as follows:
	
	Let $\A$ be the algebra of sets generated by \eqref{basesets}. That is, finite unions of finite intersections of sets of that form. can i? Assume without loss of generality that these unions are disjoint.

\end{document}