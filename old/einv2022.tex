\documentclass[12pt]{article}

\usepackage{amsmath}
\usepackage{amssymb}
\usepackage{amsfonts}
\usepackage{mathtools}

\usepackage[thmmarks, amsmath]{ntheorem}

\usepackage{graphicx}

\usepackage{diffcoeff}
\diffdef{}{op-symbol=\mathrm{d},op-order-sep=0mu}

\usepackage{cancel}

\usepackage{enumitem}

\setlist[enumerate,1]{label=\alph*)}

\title{Palestra para Escola de Inverno 2022\\
Códigos de Barras de Funções}
\author{Duarte Maia}
%\date{}

\theorembodyfont{\upshape}
\theoremseparator{.}
\newtheorem{theorem}{Theorem}
\newtheorem{prop}{Prop}
\renewtheorem*{prop*}{Prop}
\newtheorem{lemma}{Lemma}

\theoremstyle{nonumberplain}
\theoremheaderfont{\itshape}
\theorembodyfont{\upshape}
\theoremseparator{:}
\theoremsymbol{\ensuremath{\blacksquare}}
\newtheorem{proof}{Proof}

\newcommand{\R}{\mathbb{R}}
\newcommand{\C}{\mathbb{C}}
\newcommand{\Z}{\mathbb{Z}}

\newcommand{\PP}{\mathbb{P}}
\newcommand{\FF}{\mathcal{F}}

\newcommand{\I}{\mathrm{i}}
\newcommand{\e}{\mathrm{e}}


\DeclareMathOperator{\inte}{int}
\DeclareMathOperator{\codim}{codim}
\newcommand{\grad}{\nabla}

\DeclarePairedDelimiter{\abs}{\lvert}{\rvert}
\DeclarePairedDelimiter{\norm}{\lvert}{\rvert}
\DeclarePairedDelimiter{\Norm}{\lVert}{\rVert}

\usepackage[margin=1in]{geometry}


\begin{document}
\maketitle

\section{Introdução}

%begin review
A primeira coisa que têm de saber é que distinguir objetos geométricos é difícil. Considerem os seguintes objetos: Uma esfera. Uma linha. Um círculo. Um nó. Um toro. Será que são todos distintos?

A resposta a essa questão depende da noção de igualdade, mas no que se segue vamos tomar o ponto de vista que dois objetos geométricos são distintos se são existe alguma diferença intrinsínseca, ou seja, que é possível, a partir de dentro do objeto, determinar alguma distinção. Sob esta perspetiva, o nó e o círculo são o mesmo objeto, simplesmente desenhado de forma diferente.

Concluimos então que o mesmo objeto geométrico pode ser apresentado de mais de uma maneira diferente. Assim sendo, questionamo-nos se não será possível mais alguns destes objetos serem iguais uns aos outros. A resposta é que não, e a justificação passa pelos chamados invariantes geométricos.

Exemplo simples: A dimensão. É um facto não-trivial que a dimensão de um objeto geométrico não depende da sua apresentação. Consequentemente, estes dois, que têm dimensão 2, não podem ser iguais a estes, que têm dimensão 1.

Exemplo menos simples: O grupo fundamental. Aqueles que não sabem o que isto é vão dá-lo em Topologia, no terceiro ano. Isto é um objeto que pode ser associado a um espaço, e é possível calcular que o grupo fundamental da esfera é o grupo trivial e o grupo fundamental do toro é $\Z \oplus \Z$. Como estes grupos são distintos (um deles tem apenas um objeto e o outro não), estes dois espaços são necessariamente diferentes.

Uma das utilidades de invariantes geométricos é distinguir espaços, mas alguns invariantes dão-nos informação adicional. Por exemplo, existe uma invariante chamada Característica de Euler, que é calculada decompondo o espaço em triângulos e contando faces e vértices, e nos dá informação sobre campos vetoriais nesse espaço. Para aqueles que já ouviram falar do Hairy Ball Theorem, uma demonstração rápida é: A característica de Euler da esfera é 2, que é diferente de zero.

O objeto que eu vos vou apresentar hoje chama-se um Código de Barras, e é semelhante, mas não exatamente igual, aos códigos de barras que vêm nas mercearias. Códigos de barras podem ser usados para distinguir certos objetos, mas a sua maior utilidade é a informação que eles nos dão. Ou seja, não me vão ver a justificar que esta coisa e aquela são distintas porque os seus códigos de barras são distintos, mas sim a responder à questão: \emph{O que conseguimos ler de um código de barras?}

Definição: Um código de barras é uma coleção finita de intervalos. (Desenhar um exemplo) Existem vários códigos de barras associados ao mesmo objeto geométrico, tal como há vários números (como a dimensão, característica de Euler, etc.). Eu vou falar de uma construção específica, que associa um código de barras a uma função de Morse através da homologia do espaço. Não precisam de saber o que estas palavras significam.

%4 minutes

\section{Exemplo}

O objeto de estudo são funções reais no círculo, ou seja, a cada ponto do círculo estamos a fazer corresponder um número. Isto é inconveniente de representar, por isso vamos ver o círculo como um intervalo e depois colamos as pontas. Podemos agora representar uma função no círculo como um gráfico tipo aqueles aos quais estamos habituados, com o cuidado que a função tem que colar bem nas pontas, portanto não podemos ter descontinuidades tipo uma função linear.

Para ser um pouco mais específico, vamos estar a falar de funções de Morse, cujo significado não precisam de saber, exceto que uma função de Morse genérica parece-se assim. [Desenhar função genérica.]

Existe uma definição geral, e complicada, que associa um código de barras a uma função de Morse, mas felizmente no caso do círculo essa definição pode ser destilada numa receita simples, que vos estou prestes a apresentar.

Passo 1: Identificar o mínimo da função. Agora, vamos desenhar uma barra infinita, que vai desse valor para cima.

Passo 2: Identificar o mínimo a seguir. Agora, vamos desenhar uma barra que vai desse valor para cima, mas em vez de ir para o infinito, vamos parar assim que conseguirmos ver a barra anterior. Aqui em baixo, a nossa vista está bloqueada por estas montanhas, mas quando passamos daqui para cima conseguimos ver a barra inicial, por isso paramos aqui.

Passo 3: Repetir com os próximos mínimos. Uma propriedade essencial de funções de Morse é que existe um número finito destes.

Passo 4: Após percorrermos todos os mínimos, vamos ao máximo global e desenhamos uma barra que vai do máximo para cima. Isto termina o procedimento.

%8 minutes

%end review

\section{Extrair Informação}

Agora que construímos o código de barras de $f$, perguntamos: Que informação podemos extrair dele?

Primeiro que tudo, é óbvio a partir do código de barras de $f$ qual o valor máximo e mínimo de $f$. O código de barras tem exatamente duas barras infinitas, e os seus extremos inferiores correspondem ao mínimo e ao máximo.

Podemos também averiguar quais são os valores críticos de $f$, isto é, os picos e os vales. Os picos são os extremos superiores das barras (com a exceção do máximo global) e os vales são os extremos inferiores.

A partir do código de barras é também possível extrair informação mais analítica, por exemplo a variação total de $f$. A variação total de uma função pode ser definida de várias formas equivalentes, mas resumidamente mede quanto é que a função sobe e desce, sendo que subir um metro e descer um metro conta como variação total de 2 metros. Uma definição possível neste contexto é:
\[\text{variação total de $f$} = \int_0^{2\pi} \abs{f'(t)} \dl t.\]

Isto é uma quantidade que parece um pouco complicada: o integral de um valor absoluto de uma derivada. No entanto, podemos ler a variação total de $f$ a partir do seu código de barras de forma muito simples: Contamos duas vezes o comprimento de cada barra finita, e no final somamos duas vezes o máximo menos o mínimo. (Apontar no desenho a correspondência entre as variações.)

%10 minutes?

\section{Conclusão}

Provavelmente não estão impressionados com o poder dos códigos de barras. Os exemplos que vos mostrei até agora são muito simples, e não dizem nada de novo sobre a função, mas o interessante está nos exemplos que \emph{não} vos mostrei.

O que fizemos foi fazer correspondências entre propriedades importantes de $f$ e propriedades do seu código de barras, mas isso levanta a possibilidade de fazer as coisas ao contrário. O que acontece se eu começar com propriedades do código de barras de $f$ e tentar ver a que corresponde?

Considerem as seguintes quantidades:
\begin{itemize}
\item O número de barras.
\item O tamanho da maior barra finita.
\item O tamanho da menor barra.
\item O maior número de barras sobrepostas.
\item O maior número de barras contidas umas nas outras.
\end{itemize}

Isto são quantidades que eu mandei ao calhas, mas quem sabe se algumas delas não serão alguma propriedade importante da função? Isto já aconteceu nalgumas áreas: Invariantes complicados que já eram conhecidas por outros meios foram reconhecidos como quantidades associados a códigos de barras, ou quantidades associadas a códigos de barras que se tornaram em invariantes importantes.

É isto que estou a estudar para a minha tese de mestrado. Estou a estudar num contexto diferente e mais complicado, mas as ideias são as mesmas. Tenho uns certos objetos, tenho construções complicadas de códigos de barras associados a eles, e estou a tentar perceber: Que informação consigo ler a partir dos códigos de barras?

(Palestra over)

\end{document}