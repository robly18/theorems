\documentclass{article}

\usepackage[utf8]{inputenc}
\usepackage{amsfonts}
\usepackage{amsmath}


\title{Aula 2 de TM}
\author{Duarte Maia}
\date{}

\newcommand{\R}{\mathbb{R}}
\newcommand{\C}{\mathbb{C}}

\newcommand{\e}{\mathrm{e}}

\DeclareMathOperator{\ps}{P}

\begin{document}
\maketitle
\vspace{-1cm}
\begin{itemize}
\item Já temos: definição de $\sigma$-álgebra, definição de medida
\item Dar exemplos simples?
\item Motivar noção de construir medidas a partir de coisas mais simples
\begin{itemize}
\item Construir a medida de Lebesgue começando com comprimentos de intervalos
\item Construir uma medida de probabilidade começando com uma função de distribuição
\item Dados dois espaços $X$ e $Y$ com medidas $\mu$ e $\nu$, construir uma medida em $X \times Y$ a partir de retângulos
\item ...Por exemplo, construir uma medida em $X \times \left[0,\infty\right]$ para depois poder definir integral
\item ...Por exemplo, construir a medida de Lebesgue em $\R^n$
\item ...Por exemplo, construir uma medida de probabilidade produto, que corresponde a fazer duas experiências independentes em paralelo.
\end{itemize}
\item Especificar contexto: seja $X$ um conjunto. Suponha-se definida uma função $\rho : A \to [0,\infty]$ onde $A \subseteq \ps(X)$. Isto simboliza `os conjuntos cuja medida já sabemos'.
\item O nosso objetivo é encontrar uma medida $m$ definida numa $\sigma$-álgebra $M$ que contenha $A$, tal que para todo o $E \in A$ se tenha $m(E) = \rho(E)$.
\item Um primeiro passo razoável é descobrir uma $\sigma$-álgebra na qual o $m$ vai decerto estar definido.
\item Definimos a $\sigma$-álgebra gerada por $A$ como sendo a menor $\sigma$-álgebra que contém $A$. Denotamo-la por $M(A)$. Provar que $M(A)$ existe.
\item Falar de como a $\sigma$-álgebra gerada por um conjunto é um objeto mesmo complicado.
\begin{itemize}
\item \emph{Não} existe um processo contável que gera a $\sigma$-álgebra, em contraste a, e.g. subgrupos gerados por elementos ou espaços topológicos gerados por subbases.
\item Isto é hardcore teoria de conjuntos. É possível usar indução transfinita para provar coisas sobre estes objetos, mas se não sabem nada sobre o assunto, a única coisa que têm em que pegar para provar coisas sobre $\sigma$-álgebras geradas por coisas é a definição.
\end{itemize}
\item Vamos mexer um bocado com estes objetos, por isso vale a pena passar algum tempo em ganhar algum à-vontade com eles. Provar algumas propriedades elementares.
\begin{itemize}
\item $M(M(A)) = M(A)$; Se $A \subseteq B$ então $M(A) \subseteq M(B)$
\item A $\sigma$-álgebra de Borel em $\R$ tem mil definições diferentes possíveis:
\begin{itemize}
\item Intervalos
\item Intervalos meio-abertos
\item Intervalos abertos
\item Abertos
\item Fechados
\item Compactos
\item Intervalos $\left[a, \infty\right[$
\item Intervalos $\left]a, \infty\right[$
\end{itemize}
\item (Talvez) Exercício 1.2.5 do Folland: $M(A) = \bigcup_F M(F)$, onde a união é tomada sob subconjuntos contáveis de $A$.
\end{itemize}
\item Há três construções com $\sigma$-álgebras que são muito importantes. Uma delas é a $\sigma$-álgebra gerada por. Outras são a $\sigma$-álgebra de Borel num espaço topológico, e a $\sigma$-álgebra produto.
\item Definição: Seja $X$ um espaço topológico (os únicos relevantes são $\R^n$). A $\sigma$-álgebra de Borel em $X$, denominada $B(X)$, é a gerada pelos abertos.
\item Definição: Sejam $(X,M)$ e $(Y,N)$ espaços de medida. A $\sigma$-álgebra produto de $M$ e $N$, denotada $M \otimes N$, é a $\sigma$-álgebra gerada por `retângulos' da forma $E \times F$, $E \in M$, $F \in N$.
\item Proposição: Se $E \in M \otimes N$ então qualquer secção de $E$ é mensurável.
\item Proposição: Sejam $X$ e $Y$ espaços topológicos. Então,
\[B(X) \otimes B(Y) \subseteq B(X \times Y).\]
\item Proposição (importante!): Se $X = \R^n$ e $Y = \R^m$ então temos igualdade. (Só é preciso 2nd countable.)

\item (Se houver muito tempo) Introduzir ideia de medida exterior gerada por uma função de conjuntos. Falar de convenções: $\sup \emptyset = 0$, $\inf \emptyset = \infty$.
\item Definir $\mu^*(E) = \inf \sum \rho A_i$, onde o ínfimo é tomado sobre coberturas contáveis $A_i$ de $E$.
\item Provar que $\mu^*$ é uma medida exterior, isto é:
\begin{itemize}
\item $\mu^* \emptyset = 0$
\item Subaditividade contável
\end{itemize}
\end{itemize}

\end{document}
