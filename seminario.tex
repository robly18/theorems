\documentclass{article}

\usepackage{amsmath}
\usepackage{amsthm}
\usepackage{amsfonts}
\usepackage[utf8]{inputenc}
\usepackage[portuguese]{babel}
\usepackage{listings}
\usepackage{pgfplots}
%USAR XELATEX!!

\title{Sobre a facilidade computacional de IPS}
\author{Duarte Maia, ist189623}
\date{}

\newtheorem{prop}{Prop}

\addto\captionsportuguese{
	\renewcommand{\proofname}{Dem}
}

\newcommand{\N}{\mathbb{N}}
\newcommand{\R}{\mathbb{R}}

\renewcommand{\mod}{\mathbin{\mathrm{mod}}}

\lstset{
basicstyle=\ttfamily,
mathescape=true,
breaklines=true,
tabsize=4
}

\begin{document}
	\maketitle
	
	\tableofcontents

	\section{Introdução}

	Este trabalho baseia-se no seminário da professora Patrícia Gonçalves sobre o movimento aleatório de partículas.
	%todo reference
	Na sua apresentação, a professora apresenta um modelo microscópico probabilístico bastante geral para o comportamento de partículas: os chamados Interacting Particle Systems (neste trabalho abreviado a IPS). O âmbito deste trabalho é determinar os limites computacionais destes modelos, e ver quão longe é possível ir com eles. Investigaremos a complexidade assintótica de algumas classes de modelos e comparemos os limites teóricos com dados experimentais.

	\section{Notação}

	Primeiro que tudo, especificamos o que se entende por um IPS neste contexto.
	
	Neste relatório, trabalharemos com o tipo mais básico de IPS: aquele onde existe apenas um tipo de partícula, o número de partículas permanece constante e não pode haver sobreposição de partículas. Chamemos a estes \emph{HCEIPS}, abreviatura para \emph{Homogeneous and Closed Exclusive Interacting Particle Systems}. Por simplicidade, assumimos um modelo discreto.
	
	Para caraterizar um HCEIPS precisamos de:
	
	\begin{itemize}
	\item Um conjunto de posições $X$
	
	\item Um tempo médio de movimento $\lambda$
	
	\item Uma função probabilidade de transição $p : X \times X \rightarrow [0,1]$
	
	\item Uma distribuição inicial (possívelmente aleatória) $\rho_0$
	\end{itemize}
	
	Dado um HCEIPS $H = (X, \lambda, p, \rho_0)$, podemos ver o que acontece após um certo tempo $t \geq 0$. Isto é denotado por $H_t$ (uma variável aleatória representando uma distribuição de partículas), e repare-se nos seguintes factos:
	
	Segundo o modelo dado de IPS, $H$ é um sistema dinâmico. Visto que a distribuição exponencial é amnésica, temos que, se $H = (X, \lambda, p, \rho_0)$, $H_{t+s}$ é igual a $(X, \lambda, p, H_t)_s$.
	
	Uma concretização de $H_t$ será então uma função $\eta : X \rightarrow \N_0$, a chamada função de \emph{número de partículas}.
	
	Claro que em termos práticos não queremos saber da variável aleatória em si, mas sim de uma concretização. Assim sendo, apresentamos o seguinte algoritmo para conseguir uma concretização.
	
	\section{O algoritmo base}
	
	Seja $H = (X, \lambda, p, \rho_0)$. Pretendemos calcular uma concretização de $H_T$
	
	Supomos dada uma função (no sentido computacional) $randomexp : \R^+_0 \rightarrow \R^p_0$ que, dado um valor médio $\lambda$, retorna uma número gerado aleatóriamente com uma distribuição exponencial de valor médio $\lambda$. Da mesma forma, assumimos dada uma função zeroária $unif : \{()\} \rightarrow [0,1]$ que gera um número aleatório uniformemente entre 0 e 1.
	
	Assumimos também a noção de \emph{PriorityQueue} (Ver anexo). Assume-se que uma instância vazia destas pode ser inicializada com $EmptyPriorityQueue()$.
	
	\begin{lstlisting}
	$\eta \leftarrow \rho_0$ //Popular $\eta$ com a distribuição inicial
	$queue \leftarrow EmptyPriorityQueue()$
	For $x \in X$:
		If $\eta(x) = 1$:
			$xt \leftarrow randomexp(\lambda)$
			If $xt \leq T$:
				$queue.push((x,xt))$
	$t \leftarrow 0$
	While $! queue.empty()$:
		$x, t \leftarrow queue.popmin()$ //Próximo evento
		
		$ptotal \leftarrow \sum_{y \in X, \eta(y) = 0} p(x, y)$
		$gen \leftarrow unif() \times ptotal$
		$accp \leftarrow 0$
		For $y \in X, \eta(y) = 0$:
			$accp \leftarrow accp + p(x,y)$
			If $accp \geq gen$:
				$\eta(x) \leftarrow 0$
				$\eta(y) \leftarrow 1$
				$x \leftarrow y$
				Break
		
		$xt \leftarrow t + randomexp(\lambda)$ //Agendar o evento seguinte desta partícula
		If $xt \leq T$:
			$queue.push((x,xt))$
	\end{lstlisting}
	
	No final deste programa, $\eta$ conterá uma concretização de $H_T$.
	
	Estamos interessados em estimar o tempo que este programa demora a correr.
	
	\section{Estimativa de tempo de execução}
	
	Primeiro que tudo, gostaria de esclarecer que a estimativa que se segue não é de todo rigorosa: é uma coisa muito básica, uma \emph{back of the envelope calculation}.
	
	Suporei, primeiro que tudo, que a maior parte do tempo é passada dentro do \emph{While}, pelo que o tempo demorado a popular a distribuição inicial e os relógios é negligível. Assim sendo, a estimativa de tempo demorado será $N \times \Theta$, onde $N$ é o número médio de iterações, e $\Theta$ é o tempo médio demorado a correr o \emph{While} uma vez.
	
	Para estimar $N$, repare-se que os relógios são independentes para cada partícula, pelo que $N = \nu \times M$, onde $M$ é o número médio de iterações do \emph{While}, \emph{por partícula} e $\nu$ é o número de partículas. Ora, sabemos que a simulação corre até um tempo $T$, pelo que o tempo médio entre iterações do loop por cada partícula é $T/M = \lambda$, donde $M = T/\lambda$.
	
	Assim sendo, a nossa estimativa para $N$ é $\frac{T \times \nu}\lambda$.
	
	Prosseguimos a estimar $\Theta$.
	\[\Theta \approx \text{Tempo a dar pop da queue} + \#X + S + \text{Tempo a dar push na queue}\]
	
	Onde $S$ é o tempo médio passado dentro do \emph{For}. É de esperar que o número de iterações deste loop seja da ordem de grandeza de $\#X$, pelo que assumimos $S \approx \#X$.
	
	Finalmente, examinemos o tempo a dar pop e push na queue. Isto depende da forma como a Priority Queue está implementada, mas na maioria das implementações modernas este tempo é assintóticamente da ordem de $\log n$, onde $n$ é o número de elementos da queue. No nosso caso, $n = \nu$, donde estimamos
	
	\[\Theta \approx \log \nu + \#X\]
	
	Juntando tudo, obtemos que o tempo médio de execução do nosso programa é da ordem de:
	
	\[\frac{T \times \nu}\lambda (\log \nu + \#X)\]
	
	Para $\#X$ grande o suficiente, como $\nu \leq \#X$, temos que $\log \nu \leq \log \#X$ é negligível perante $\#X$, donde, para $\#X$ grande,
	
	\[\Theta_{exec} \approx \frac 1 \lambda T \nu \#\!X\]
	
	Procedemos agora a confirmar esta estimativa experimentalmente.
	
	\section{Confirmação experimental}
	
	O primeiro passo é implementar o algoritmo acima descrito para depois o testar sob diversas condições.
	
	O código relevante encontra-se no Notebook Mathematica em anexo, no qual se define a função \texttt{HCEIPS[N\_, lambda\_, p\_, rho0\_, T\_]}.
	
	Por razões computacionais, assumimos que $X$ é da forma $\{1, 2, \cdots, N\}$.
	
	Foi feita uma simulação com os seguintes parâmetros:
	
	\begin{enumerate}
	\item $X = \{1, 2, \cdots, N\}$, onde $N$ é variável
	
	\item $T$ é variável; para manter a exposição mais curta fixamos $\lambda = 1$.
	
	\item $p(x,y) = 1$
	
	\item $\rho_0$ consiste em $\nu$ partículas uniformemente e aleatóriamente distribuidas, onde $\nu$ é variável.
	\end{enumerate}
	
	Temos então três quantidades variáveis em jogo, $N, T$ e $\nu$. Estamos à espera que o tempo de execução seja proporcional a $N$, $T$ e $\nu$. Implementou-se uma função \texttt{TSim[N\_, lambda\_, T\_, nu\_, attemptno\_]} que, dados estes parâmetros e um número de tentativas, devolve o tempo médio necessário de correr a simulação dados esses parâmetros.
	
	Primeiro, investigamos a dependência em $T$.
	
	\subsection{Dependência em $T$}
	
	Fixos $N = 100$ e $\nu = 50$, seguem-se os dados de tempo de execução em função de $T$, calculados como a média de 5 tentativas.
	
	\begin{tabular}{r | l}
	$T$ & \texttt{TSim[100, 1, T, 50, 5]} \\
	\hline
	1 & 0.019 \\
	5 & 0.103\\
	10 & 0.197\\
	30 & 0.594 \\
	50 & 1.01 \\
	75 & 1.49 \\
	100 & 2.12
	\end{tabular}
	
	Graficamente:
	
	\begin{tikzpicture}
	\begin{axis}[axis lines = middle, xmin=0, ymin=0, xlabel = $T$, ylabel = $\Theta_{exec}$]
\addplot [domain=0:110, samples=2, dashed, color = red] {0.02*x};
\addplot [only marks, mark=x] table {
	1 0.019
	5 0.103
	10 0.197
	30 0.594 
	50 1.01 
	75 1.49 
	100 2.12
};
\end{axis}
\end{tikzpicture}

	Representada a vermelho está a reta $y = 0.02x$, donde concluímos que, para estes valores (e para esta implementação específica), $\Theta_{exec} \approx \frac{0.02 \times 1}{100 \times 50} \frac{N \nu T}\lambda = 4 \times 10^{-5} \frac{N \nu T}\lambda$.
	
	\subsection{Dependência em $\nu$}
	
	Fixe-se agora $N = 100$ e $T = 100$. Seguem-se os dados de tempo de execução em função de $\nu$, calculados como a média de 5 tentativas.
	
	\begin{tabular}{r | l}
	$\nu$ & \texttt{TSim[100, 1, nu, 100, 5]} \\
	\hline
	1 & 0.0344 \\
	5 & 0.169 \\
	10 & 0.362 \\
	25 & 0.919 \\
	50 & 1.80 \\
	75 & 2.83 \\
	100 & 4.19
	\end{tabular}
	
	Graficamente:
	
	\begin{tikzpicture}
	\begin{axis}[axis lines = middle, xmin=0, ymin=0, xlabel = $\nu$, ylabel = $\Theta_{exec}$]
\addplot [domain=0:110, samples=2, dashed, color = red] {0.04*x};
\addplot [only marks, mark=x] table {
	1  0.034375 
	5  0.16875 
	10  0.3625 
	25  0.91875 
	50  1.79688 
	75  2.83125 
	100  4.19375
};
\end{axis}
\end{tikzpicture}

	Representada a vermelho está a reta $y = 0.04x$, donde concluímos $\Theta_{exec} \approx \frac{0.04 \times 1}{100 \times 100} \frac{N \nu T}\lambda = 4 \times 10^{-5} \frac{N \nu T}\lambda$, confirmando o resultado anterior.
	
	\subsection{Dependência em $N$}
	
	Fixamos agora $\nu = 25$ e $T = 100$, seguem-se os dados de tempo de execução em função de $N$, calculados como a média de 5 tentativas.
	
	\begin{tabular}{r | l}
	$N$ & \texttt{TSim[N, 1, 100, 25, 5]} \\
	\hline
	25 & 0.440625 \\
	75 & 0.740625 \\
	100 & 0.921875 \\
	200 & 1.525 \\
	300 & 2.47813 \\
	400 & 3.05 \\
	500 & 3.71 \\
	600 & 5.30
	\end{tabular}
	
	Graficamente:
	
	\begin{tikzpicture}
	\begin{axis}[axis lines = middle, xmin=0, ymin=0, xlabel = $N$, ylabel = $\Theta_{exec}$]
\addplot [domain=0:610, samples=2, dashed, color = red] {0.009*x};
\addplot [only marks, mark=x] table {
	25 0.440625
	75 0.740625
	100 0.921875
	200 1.525
	300 2.47813
	400 3.05
	500 3.71
	600 5.30
};
\end{axis}
\end{tikzpicture}

	Representada a vermelho está a reta $y = 0.009x$, obtendo-se então $\Theta_{exec} \approx \frac{0.009 \times 1}{25 \times 100} \frac{N \nu T}\lambda = 0.36  \times 10^{-5} \frac{N \nu T}\lambda \approx 4 \times 10^{-5} \frac{N \nu T}\lambda$, o que confirma mais uma vez o nosso resultado.
	
	\section{Otimização e mais uma classe de IPS}
	
	O nosso trabalho anterior mostra que o tempo de execução de um HCEIPS é da ordem de $10^{-5} \frac{N \nu T}\lambda$. Se assumirmos que $\nu$ é da ordem de $N$ (por exemplo, metade das casas estão ocupadas), isto fica da ordem de $10^{-5} \frac{N^2 T}\lambda$. Isto é quadrático em $N$. É, no entanto, possível ser-se mais eficiente, dadas certas restrições.
	
	Suponhamos $X = \{1,\cdots, N\}$ e que as partículas só se podem mover para espaços a elas adjacentes. Isto é uma restrição bastante comum na prática, que nos permite otimizar fortemente o código. Para maior generalidade e maior facilidade de implementação, supomos que estamos num toro e então 1 conta como adjacente a $N$.
	
	Chamemos a estes sistemas \emph{HCELIPS}, abreviatura para \emph{Homogeneous and Closed Exclusive Locally Interacting Particle Systems}. A otimização principal para estes sistemas é a seguinte: só é preciso calcular $p(x,y)$ para $y = x \pm 1$. Logo, o cálculo de para onde se deve mover a partícula passa de algo feito em $\#X$ passos para algo feito em dois.
	
	O pseudocódigo para este tipo de IPS é exatamente igual ao outro, com a única diferença sendo a seguinte parte:
	
	\begin{lstlisting}
		$ptotal \leftarrow \sum_{y \in X, \eta(y) = 0} p(x, y)$
		$gen \leftarrow unif() \times ptotal$
		$accp \leftarrow 0$
		For $y \in X, \eta(y) = 0$:
			$accp \leftarrow accp + p(x,y)$
			If $accp \geq gen$:
				$\eta(x) \leftarrow 0$
				$\eta(y) \leftarrow 1$
				$x \leftarrow y$
				Break
	\end{lstlisting}
	
	Onde $X$ é substituído por $\{x+1 \mod N, x-1 \mod N\}$, onde $a \mod N$ deve ser entendido como o resíduo modulo $N$ de $a$ contido no conjunto $\{1, \cdots, N\}$.
	
	Isto muda drasticamente a complexidade algorítmica do código. Fazendo as mesmas contas do que há bocado, chegamos a uma expressão da forma
	
	\[\Theta_{exec} \approx \frac T \lambda \nu \log \nu\]
	
	\section{Anexo: Priority Queue}
	
	todo

\end{document}