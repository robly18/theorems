\documentclass[addpoints]{exam}

\usepackage{amsmath}
\usepackage{amssymb}
\usepackage{amsfonts}
\usepackage{mathtools}

\usepackage{float}
\usepackage{tikz}

\DeclarePairedDelimiter{\Norm}\lVert\rVert
\newcommand{\R}{\mathbb{R}}
\newcommand{\cm}{\mathrm{cm}}

\pagestyle{headandfoot}
\runningheader{MATH 152 Section 45}{Quiz 2}{Duarte Maia}
\runningheadrule
\footer{}{\ifincomplete{Exercise continues on next page.}{}}{}


\begin{document}

\begin{center}
\Large Quiz 2

MATH 152 - Section 45

\smallskip

\large Section Instructor: Duarte Maia

\smallskip

October 29, 2025
\end{center}

\noindent
\makebox[\textwidth]{
Name and UCID: \,\hrulefill
}

\begin{center}
\fbox{
\parbox{6in}{
\begin{itemize}
\item You have 20 minutes to answer the following question in the space given below.

\item You may ask for additional paper if necessary.

\item You may ask for another copy of the figure if necessary.
\end{itemize}
}
}
\end{center}

%\noindent Useful Formulas:
%\begin{itemize}
%\item Correlation coefficient: $\frac{\vec{x}\cdot \vec y}{\Norm{\vec x} \Norm{\vec y}}$, where $\vec x$ and $\vec y$ are the deviation vectors.
%\item Normal Equation: $A^\tr A \vec x = A^\tr \vec b$.
%\item Least Squares: $\vec x = (A^\tr A)^{-1} A^\tr \vec b$.
%\end{itemize}

\hrule

\qformat{\textbf{Exercise:} \quad}

\begin{questions}
\question Approximate the area of the figure below with the best error bounds that you can by hand. You may use drawing materials (pen or pencil) and a straightedge\footnote{A straightedge is like a ruler, but with no measurement ticks.} to assist you.

Your answer should be of the form: ``The area of the shape is $\text{[number1]}\cm^2 \pm \text{[number2]} \cm^2$'', and your goal is to make [number2] as small as you can, though your solution doesn't have to be perfect.

\textit{My personal best (by hand) is $51\cm^2\pm 11\cm^2$. Can you do better?}

\begin{figure}[H]
\centering
\begin{tikzpicture}[scale=0.7]
\draw plot[smooth cycle, tension=1] coordinates {(0.2,2.4) (-0.2,5) (3.2,6.1) (7.1,2.6) (4.5,-2.1) (0,-1.4) (0,1.1) (3.1,2.1) (3.3,3.7)};

\foreach \x in {-1,0,...,8} {
\foreach \y in {-3,-2,...,7} {
\draw[fill=black] (\x,\y) circle[radius=1pt];
}
}

\draw (-1,-3) -- (0,-3) node[midway,below]{$1\cm$};
\draw (-1,-3) -- (-1,-2) node[midway,left]{$1\cm$};
\end{tikzpicture}
\qquad
\begin{tikzpicture}[scale=0.7]
\draw plot[smooth cycle, tension=1] coordinates {(0.2,2.4) (-0.2,5) (3.2,6.1) (7.1,2.6) (4.5,-2.1) (0,-1.4) (0,1.1) (3.1,2.1) (3.3,3.7)};

\foreach \x in {-1,0,...,8} {
\foreach \y in {-3,-2,...,7} {
\draw[fill=black] (\x,\y) circle[radius=1pt];
}
}

\draw (-1,-3) -- (0,-3) node[midway,below]{$1\cm$};
\draw (-1,-3) -- (-1,-2) node[midway,left]{$1\cm$};
\end{tikzpicture}
\end{figure}
\end{questions}
\end{document}