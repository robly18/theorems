\documentclass[addpoints]{exam}

\usepackage{amsmath}
\usepackage{amssymb}
\usepackage{amsfonts}
\usepackage{mathtools}

\usepackage{float}
\usepackage{tikz}

\DeclarePairedDelimiter{\Norm}\lVert\rVert
\newcommand{\R}{\mathbb{R}}
\newcommand{\cm}{\mathrm{cm}}
\newcommand{\e}{\mathrm{e}}

\pagestyle{headandfoot}
\runningheader{MATH 152 Section 45}{Quiz 4}{Duarte Maia}
\runningheadrule
\footer{}{\ifincomplete{Exercise continues on next page.}{}}{}


\begin{document}

\begin{center}
\Large Quiz 4

MATH 152 - Section 45

\smallskip

\large Section Instructor: Duarte Maia

\smallskip

December 3, 2025
\end{center}

\noindent
\makebox[\textwidth]{
Name and UCID: \,\hrulefill
}

\begin{center}
\fbox{
\parbox{6in}{
\begin{itemize}
\item You have 20 minutes to answer the following question in the space given below.

\item You may ask for additional paper if necessary.
\end{itemize}
}
}
\end{center}

%\noindent Useful Formulas:
%\begin{itemize}
%\item Correlation coefficient: $\frac{\vec{x}\cdot \vec y}{\Norm{\vec x} \Norm{\vec y}}$, where $\vec x$ and $\vec y$ are the deviation vectors.
%\item Normal Equation: $A^\tr A \vec x = A^\tr \vec b$.
%\item Least Squares: $\vec x = (A^\tr A)^{-1} A^\tr \vec b$.
%\end{itemize}

\hrule

\qformat{\textbf{Exercise:} \quad}

\begin{questions}
\question Let us say that a \emph{polynomial oscillator} is a function $f(t)$ that satisfies a differential equation of the form
\[f''(t) + a_1 f(t) + a_2 f(t)^2 + \dots + a_n f(t)^n = 0,\]
for some positive numbers $k_1$, $k_2$, ..., $k_n$.

A \emph{harmonic oscillator}, on the other hand, is a function $g(t)$ that satisfies the differential equation
\[g''(t) + k g(t) = 0,\]
for some positive number $k$, which is called the \emph{spring constant} of the oscillator.

Show that, if $f(t)$ is a polynomial oscillator such that $f(0)$ is very small, then its behavior can be well-approximated by a harmonic oscillator, at least for some time. What is the spring constant of this harmonic oscillator?

The reasoning you should use is similar to how you compared the logistic and exponential growth in the project. Recall that if a number is small, its square is minuscule.
\end{questions}
\end{document}