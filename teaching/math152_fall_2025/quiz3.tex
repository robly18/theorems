\documentclass[addpoints]{exam}

\usepackage{amsmath}
\usepackage{amssymb}
\usepackage{amsfonts}
\usepackage{mathtools}

\usepackage{float}
\usepackage{tikz}

\DeclarePairedDelimiter{\Norm}\lVert\rVert
\newcommand{\R}{\mathbb{R}}
\newcommand{\cm}{\mathrm{cm}}
\newcommand{\e}{\mathrm{e}}

\pagestyle{headandfoot}
\runningheader{MATH 152 Section 45}{Quiz 3}{Duarte Maia}
\runningheadrule
\footer{}{\ifincomplete{Exercise continues on next page.}{}}{}


\begin{document}

\begin{center}
\Large Quiz 3

MATH 152 - Section 45

\smallskip

\large Section Instructor: Duarte Maia

\smallskip

November 12, 2025
\end{center}

\noindent
\makebox[\textwidth]{
Name and UCID: \,\hrulefill
}

\begin{center}
\fbox{
\parbox{6in}{
\begin{itemize}
\item You have 20 minutes to answer the following question in the space given below.

\item You may ask for additional paper if necessary.
\end{itemize}
}
}
\end{center}

%\noindent Useful Formulas:
%\begin{itemize}
%\item Correlation coefficient: $\frac{\vec{x}\cdot \vec y}{\Norm{\vec x} \Norm{\vec y}}$, where $\vec x$ and $\vec y$ are the deviation vectors.
%\item Normal Equation: $A^\tr A \vec x = A^\tr \vec b$.
%\item Least Squares: $\vec x = (A^\tr A)^{-1} A^\tr \vec b$.
%\end{itemize}

\hrule

\qformat{\textbf{Exercise:} \quad}

\begin{questions}
\question Suppose that Chuck Finley's dome was large enough that the top of the dome absorbs more heat than the bottom. Indeed, assume that a flat shard of material of area $A$, at altitude $h$ (measured from the floor), at an angle with the horizontal $\theta$, will absorb heat at a rate of
\[P = A \cos^3(\theta)\, \e^{h}.\]

Using this information, repeat the procedure you used to solve Project 3 for a dome of radius $R$. Your final answer should be in the form of a \emph{single} integral (which you probably will not be able to solve). \textbf{Draw a picture of how you've decomposed the dome into flat surfaces.} Your final answer should not contain a double integral -- single integrals only.

\begin{figure}[H]
\begin{minipage}[t]{0.5\textwidth}
\centering
\begin{tikzpicture}[z={(-0.5,-0.2)}, scale=0.6]
\draw (0,0,3) -- (-3,0,3) -- (-3,0,-3);
\draw[fill=black!20] (0,0,3) -- (0,0,-3) -- (-3,2,-3) -- (-3,2,3) -- cycle;
\draw (-1,0,3) arc (180:146.31:1) node[midway, left] {$\theta$};

\draw[->] (-1.5,3,1) -- (-1.5,1,1);
\draw[->] (-1.5,3,-1) -- (-1.5,1,-1);

\draw (-1.5,4) circle (0.5cm);
\foreach \i in {45,90,...,360} {
\draw ({-1.5+0.5*cos(\i)},{4+0.5*sin(\i)}) -- ({-1.5+0.7*cos(\i)},{4+0.7*sin(\i)});
}
\end{tikzpicture}
\caption{Diagram of heat absorption.}
\end{minipage}
\begin{minipage}[t]{0.5\textwidth}
\centering
\begin{tikzpicture}[z={(-0.5,-0.2)}, scale=0.6]

\draw[shade, top color=black!10, bottom color=black!30] (2,0) arc (0:180:2cm) arc (180:360:2cm and 0.5cm);
\draw[dashed] (2,0) arc (0:180:2cm and 0.5cm);

\draw[->] (-1,3) -- (-1,1);
\draw[->] (1,3) -- (1,1);

\draw (0,4) circle (0.5cm);
\foreach \i in {45,90,...,360} {
\draw ({0+0.5*cos(\i)},{4+0.5*sin(\i)}) -- ({0+0.7*cos(\i)},{4+0.7*sin(\i)});
}

\draw (0,0) -- (2,0) node[midway, above] {$R$};
\end{tikzpicture}
\caption{Diagram of finished dome.}
\end{minipage}
\end{figure}
\end{questions}
\end{document}