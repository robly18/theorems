\documentclass[addpoints]{exam}

\usepackage{amsmath}
\usepackage{amssymb}
\usepackage{amsfonts}
\usepackage{mathtools}

\usepackage{float}
\usepackage{tikz}

\DeclarePairedDelimiter{\Norm}\lVert\rVert
\newcommand{\R}{\mathbb{R}}
\newcommand{\cm}{\mathrm{cm}}
\newcommand{\e}{\mathrm{e}}
\newcommand{\dl}{\mathrm{d}}
\newcommand{\diff}[1]{\frac{\dl}{\dl #1}}

\pagestyle{headandfoot}
\runningheader{MATH 152 Section 45}{Final Exam Prototype}{Duarte Maia}
\runningheadrule
\footer{}{\ifincomplete{Exercise continues on next page.}{}}{}


\qformat{\textbf{Exercise \thequestion.} (Total points: \totalpoints) \hfill}
\renewcommand{\partlabel}{Part (\thepartno)}


\begin{document}

\begin{center}
\Large Final Exam Prototype

MATH 152 - Section 45

\smallskip

\large Section Instructor: Duarte Maia

\smallskip

No date.
\end{center}

\begin{center}
\fbox{
\parbox{6in}{
\begin{itemize}
\item You have 2 hours to answer the questions below on the provided examination book.

\textbf{Note: This prototype is rated for 2 hours and 30 minutes, not 2 hours.}

\item Write your name and UCID in the cover page of the examination book.

\item You may ask for additional paper. If you do, sign both exam books and number them.

\item Unless otherwise specified, you do not need to justify your answers. However, you must still show your work. You may lose points if I deem you to have performed a significant leap in logic.

\item This exam is graded out of 100 points.

\item You will not be penalized for forgetting to add constants in the calculation of indefinite integrals, unless the constant is of relevance to the problem.
\end{itemize}
}
}
\end{center}

%\gradetable

\noindent Useful Formulas:
\begin{itemize}
\item Half-angle formulas:

\begin{tabular}{c c}
$\cos^2(x) = \frac12 + \frac12 \cos(2x)$, &
$\sin^2(x) = \frac12 - \frac12 \cos(2x)$
\end{tabular}
\item Angle sum formulas:

\begin{tabular}{c c}
$\cos(x+y) = \cos(x)\cos(y) - \sin(x)\sin(y)$,\\
$\sin(x+y) = \sin(x)\cos(y) + \cos(x)\sin(y)$
\end{tabular}
\item Some derivatives:

\begin{tabular}{c c}
$\diff x \arcsin(x) = \frac1{\sqrt{1-x^2}}$, &
$\diff x \arctan(x) = \frac1{1+x^2}$
\end{tabular}

\item Miscellaneous:
\begin{tabular}{c c}
$a^b = \e^{b \ln a}$, for $a > 0$, &
$\log_a(b) = \frac{\ln(b)}{\ln(a)}$, for $a>0$, $b>0$, and $a \neq 1$
\end{tabular}
\end{itemize}

\pagebreak

\begin{questions}
\question
\begin{parts}
\part[6] Calculate the upper, lower, and right Riemann sums of the function pictured below, using the partition of $[0,4]$ into three intervals: $[0,1]$, $[1,2]$, and $[2,4]$.
\begin{center}
\begin{tikzpicture}

\foreach \x in {1,2,3,4} {
\draw[help lines] (\x,-2.5) -- (\x,2.5);
}
\foreach \y in {-2,-1,1,2} {
\draw[help lines] (-0.5,\y) -- (4.5,\y);
}

\draw[->] (-1, 0) -- (5,0) node[below]{$x$};x
\draw[->] (0,-3) -- (0, 3) node[left]{$y$};

\draw[very thick, looseness=0.9] (0,2) to[out=-40, in=180-70] (1,0) to[in=180,out=-70,looseness=0.5] (1.6, -2) to[out=0,in=83-180] (2,-1) to[out=83,in=180] (2.8, 1) to[out=0] (4, -2);

\foreach \x in {1,2,3,4} {
\draw (\x,0.1) -- (\x,-0.1) node[below left]{\x};
}
\foreach \y in {-2,-1,1,2} {
\draw (0.1,\y) -- (-0.1,\y) node[left, fill=white]{\y};
}

\end{tikzpicture}
\end{center}
\part[4] Let $f(x)$ be a function, $[a,b]$ an interval, and $P$ a partition of $[a,b]$.

Determine whether the following statements are true or false.

For those that are false, sketch a counter-example.

Note: In the following, we say that $x$ is between $A$ and $B$ if either $A\leq x\leq B$ or $B\leq x \leq A$.
\begin{subparts}
\subpart The integral $\int_a^b f(x)\,\dl x$ is necessarily between the upper and lower Riemann sums of $f$ relative to the partition $P$.
\subpart The integral $\int_a^b f(x)\,\dl x$ is necessarily between the left and right Riemann sums of $f$ relative to the partition $P$.
\subpart The upper and lower Riemann sums are always distinct.
\subpart The midpoint Riemann sum will give you the exact value of the integral.
\end{subparts}
\end{parts}

\question
\begin{parts}
\part[5] Write the area of the following shape as an integral, or sum of integrals, in two different ways: One, by slicing the figure into many thin vertical rectangles (i.e.\ an integral in $\dl x$) and two, by slicing the figure into many thin horizontal rectangles (i.e.\ an integral in $\dl y$). You do not need to compute the integrals.
\begin{center}
\begin{tikzpicture}

\foreach \x in {1,2,3} {
\draw[help lines] (\x,-0.5) -- (\x,4.5);
}
\foreach \y in {1,2,3,4} {
\draw[help lines] (-0.5,\y) -- (3.5,\y);
}

\draw[->] (-1, 0) -- (4,0) node[below]{$x$};x
\draw[->] (0,-1) -- (0, 5) node[left]{$y$};


\draw plot[domain=0:3] (\x,{(\x-1)*(\x-1)});
\path (2.5,2) node[below right, fill=white]{$y=(x-1)^2$};
\draw plot[domain=0:3] (\x,\x+1);
\path (2,3) node[above left,fill=white]{$y=x+1$};

\foreach \x in {1,2,3} {
\draw (\x,0.1) -- (\x,-0.1) node[below left]{\x};
}
\foreach \y in {1,2,3,4} {
\draw (0.1,\y) -- (-0.1,\y) node[left, fill=white]{\y};
}

\end{tikzpicture}
\end{center}

\part[5] A water tank is being filled at a varying rate. At $t=0$ (measured in minutes), the amount of water is $W(0) = 5$ (measured in gallons). At a given time $t$, the amount of water in the tank is increasing at a rate of $\sin(t)+2$ gallons per minute. Determine a formula for the amount of water in the tank at time $t$. Your formula must not contain any integral signs.
\part[5] Let us define a new function by the formula
\[Q(x) = \int_0^x \frac{\cos(t)}{t^2 + 1} \,\dl t.\]
Complete and justify the following sentence: The function $Q(x)$ is increasing/decreasing (choose one) from $x=0$ to $x=\fillin[xx][0.5cm]$, where it changes direction and decreases/increases (choose one) until $x = \fillin[xx][.5cm]$, where it changes direction again.
\end{parts}

\question
\begin{parts}
\part[5] Consider the function
\[f(x) = x + k \sin(x),\]
where $k$ is a number. For which values of $k$ is $f(x)$ invertible?
\part[5] It turns out that $f(x) = x + \frac{\pi}3 \sin(x)$, restricted to the interval $[-2,2]$, is invertible. Let $g(x)$ be its inverse. Calculate $g'(\frac\pi3)$.
\end{parts}

\question
\begin{parts}
\part[4] A leak in the international space station causes its interior pressure to \textbf{decrease at a rate proportional to the pressure}. When the leak was first detected, the interior pressure was $0.5$ atmospheres. Now, $3$ hours later, the interior pressure is $1/3$ atmospheres. Assuming that a human loses consciousness around the Armstrong limit (about $1/15$ atmospheres), how much time do the astronauts have, starting now, to fix the space station before it becomes unlivable?

Your answer should be in the form of an algebraic expression that you could enter into a calculator, containing only $\ln$ and standard operations (multiplication, division, etc).

\part[6]
\begin{subparts}
\subpart Calculate the derivative of $f(x) = 2^x$.
\subpart Calculate the derivative of $f(x) = \log_5(x)$.
\subpart The number $A = \log_3 28$ is not an integer, but it is close to one. What is this integer? Briefly justify your answer.
\end{subparts}
\end{parts}

\question Solve the following integrals.
\begin{parts}
\part[5] $\displaystyle \int \tan^3 x \,\dl x$
\part[5] $\displaystyle \int_0^{\pi/2} x \sin(x) \,\dl x$
\part[5] $\displaystyle \int_0^{\sqrt\pi} x^3 \sin(x^2) \,\dl x$
\part[5] $\displaystyle \int x^2 \sqrt{x^3 + 1} \,\dl x$
\part[5] $\displaystyle \int \frac{x^3 - 1}{x^2} \,\dl x$
\part[5] $\displaystyle \int \sin(x) \cos(x) \e^{\sin(x)} \,\dl x$
\part[10] $\displaystyle \int \sin(2x) \sin(3x) \,\dl x$

Hint: Calculate $\cos(3x+2x)$ and $\cos(3x-2x)$ by the angle sum formula, and use your result to simplify the integrand.
\end{parts}

\question[5]
Find the function(s) $y(x)$ that satisfy the differential equation
\[y'(x) = x - x y(x),\quad \text{with} \;\;y(1) = 0.\]

\question[10] Consider the sequence of numbers
\[1, \qquad 1+\frac12, \qquad 1+\frac12+\frac13, \qquad \cdots \qquad 1+\frac12+\dots+\frac1n, \qquad \cdots\]
This sequence is very closely related to the $\ln(x)$ function, and can be used to approximate it. Verify this fact by proving the inequality:
\[\frac12+\dots+\frac1n \leq \ln(n) \leq 1+\frac12+\dots+\frac1{n-1}.\]

Hint: Write $\ln(n)$ as an integral, and use an upper and lower Riemann sum.

{\footnotesize Curiosity: As a result, we obtain (and you don't have to prove this) that $1+\frac12+\dots+\frac1n$ and $\ln(n)$ are at most a distance of $1$ away. It turns out that, for very large values of $n$, the distance between these two numbers is approximately $0.5772\dots$ This limiting number is called the \emph{Euler-Mascheroni constant}.}
\end{questions}
\end{document}