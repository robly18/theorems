\documentclass[addpoints]{exam}

\usepackage{amsmath}
\usepackage{amsfonts}
\usepackage{mathtools}

\DeclarePairedDelimiter{\Norm}\lVert\rVert
\newcommand{\R}{\mathbb{R}}

\pagestyle{headandfoot}
\runningheader{MATH 19620 Section 42}{Midterm 1}{Duarte Maia}
\runningheadrule
\footer{}{\ifcontinuation{}{Exercise continues on next page.}}{}


\begin{document}

\begin{center}
\Large Midterm 1

MATH 19620 - Section 42

\smallskip

\large Section Instructor: Duarte Maia

\smallskip

April 17, 2025
\end{center}

\begin{center}
\fbox{
\parbox{6in}{
\begin{itemize}
\item Answer the following questions in the provided examination book.

\item Write the number of the question you are answering at the start of your solution, and ensure that your final answer is clearly delineated.

\item Write your name and UCID in the cover page of the examination book. You may ask for additional paper if necessary.

\item Unless otherwise specified, you do not need to justify your answers.
\end{itemize}
}
}
\end{center}

\noindent Useful Formulas:
\begin{itemize}
\item Norm/Length of a vector in $\mathbb{R}^2$: $\Norm v = \sqrt{v_1^2 + v_2^2}$,
\item Canonical basis in $\mathbb{R}^2$: $\vec{e}_1 = (1,0)$, $\vec{e}_2 = (0,1)$.
\item Matrix of a rotation in $\R^2$ by a \emph{counter-clockwise} rotation by $\theta$:
\[R_\theta = \begin{bmatrix}
\cos\theta & -\sin\theta \\
\sin\theta & \cos\theta
\end{bmatrix}.\]
\end{itemize}

\hrule

\qformat{\textbf{Exercise \thequestion} \quad (Total: \totalpoints \ points)\hfill}

\begin{questions}

\question[30] Consider the linear transformation:
\[T(x,y)=\begin{bmatrix}
-1 & -1 \\
-1 & 1
\end{bmatrix}
\begin{bmatrix} x \\ y \end{bmatrix}.\]

Show the effect of $T$ on the letter $L$, as was done in the project.
 
\question[40] Consider the linear transformation:
\[P(x,y,z,w) = (w,-x,-z,y).\]

Determine whether $P$ is a linear isometry. Justify your answer.

\question[30] Recall that in the last question of the project you found the matrix that represents rotation by $\alpha$ around the line $x=y=z$ in $\R^3$ by composing linear transformations. \textbf{Using this same method}, find a formula for the matrix $Q$ that represents reflection through the line $y = mx$, with $m\geq 0$, using the following facts:
\begin{itemize}
\item The angle $\theta$ between the $x$ axis and the line $y = mx$ satisfies
\[\cos(\theta) = \frac1{\sqrt{m^2 + 1}} \text{ and } \sin(\theta) = \frac m{\sqrt{m^2 + 1}},\text{ and} \]
\item The matrix
\[S = \begin{bmatrix}
1 & 0 \\
0 & -1
\end{bmatrix}\]
represents a reflection about the $x$ axis.
\end{itemize}

Your final answer should be a product of $2\times 2$ matrices, whose entries are written in terms of numbers and possibly the variable $m$. You do not need to calculate the product. Explain (through words or diagrams) the three-part process that your product of matrices represents.
\end{questions}
\end{document}