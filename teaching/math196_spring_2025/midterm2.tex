\documentclass[addpoints]{exam}

\usepackage{amsmath}
\usepackage{amssymb}
\usepackage{amsfonts}
\usepackage{mathtools}

\DeclarePairedDelimiter{\Norm}\lVert\rVert
\newcommand{\R}{\mathbb{R}}
\newcommand{\tr}{\intercal}

\pagestyle{headandfoot}
\runningheader{MATH 19620 Section 42}{Midterm 2}{Duarte Maia}
\runningheadrule
\footer{}{\ifincomplete{Exercise continues on next page.}{}}{}


\begin{document}

\begin{center}
\Large Midterm 2

MATH 19620 - Section 42

\smallskip

\large Section Instructor: Duarte Maia

\smallskip

May 15, 2025
\end{center}

\begin{center}
\fbox{
\parbox{6in}{
\begin{itemize}
\item Answer the following questions in the provided examination book.

\item Write the number of the question you are answering at the start of your solution, and ensure that your final answer is clearly delineated.

\item Write your name and UCID in the cover page of the examination book. You may ask for additional paper if necessary.

\item Unless otherwise specified, you do not need to justify your answers.
\end{itemize}
}
}
\end{center}

\noindent Useful Formulas:
\begin{itemize}
\item Correlation coefficient: $\frac{\vec{x}\cdot \vec y}{\Norm{\vec x} \Norm{\vec y}}$, where $\vec x$ and $\vec y$ are the deviation vectors.
\item Normal Equation: $A^\tr A \vec x = A^\tr \vec b$.
\item Least Squares: $\vec x = (A^\tr A)^{-1} A^\tr \vec b$.
\end{itemize}

\hrule

\qformat{\textbf{Exercise \thequestion} \quad (Total: \totalpoints \ points)\hfill}

\begin{questions}

\question[25] Consider the following table of data about a hypothetical student.
\begin{center}
\begin{tabular}{|l|l|l|}
\hline
Year & Classes taken & Classes passed \\
\hline
2020 & 6 & 5 \\
2021 & 9 & 8 \\
2022 & 6 & 5 \\
\hline
\end{tabular}
\end{center}
Find the correlation coefficient between classes taken and classes passed, for this student.
 
\question[25] I am thinking of eight real numbers, $a_1, a_2, \dots, a_8$. I will give you the following seven linear equations relating my eight numbers:
\[
\begin{cases}
a_1 + a_2 + a_8 = 23,\\
a_1 + a_3 + a_7 = 16,\\
a_2 + a_3 + a_6 = 11,\\
a_4 + a_5 = 8,\\
a_2 + a_5 = 6,\\
a_6 - a_7 = -5,\\
a_3 + a_7 = 15.
\end{cases}
\]

Is it possible for you to figure out what my eight numbers are? Justify.

\question[25] Consider the system of equations
\begin{equation}\label{eq:1}
\begin{cases}
x + 2y = 90,\\
3x + 4y = 200,\\
5x + 6y = 325.
\end{cases}
\end{equation}
Knowing that the least-squares solution to this system is $(x,y)=(30,28.75)$, and that
\[\begin{bmatrix}
1 & 2 \\ 3 & 4 \\ 5 & 6
\end{bmatrix}
\begin{bmatrix}
30 \\ 28.75
\end{bmatrix}
=
\begin{bmatrix}
87.5 \\ 205 \\ 322.5
\end{bmatrix},
\]
how many solutions does the system \eqref{eq:1} have? Briefly justify your answer, based on your understanding of least-squares solutions.
\addquestionobject

\question[25] Write down a question, similar to Exercise 3 of the project\footnote{The exercise with the sine.}, for which the following calculation would be useful as part of the solution. Write down the final answer to your question, using the outcome of the calculation below.
\[\left(
\begin{bmatrix}
1 & 1 & 1 & 1 \\
0 & 2 & 3 & 5 \\
0 & 2^2 & 3^2 & 5^2
\end{bmatrix}
\begin{bmatrix}
1 & 0 & 0\\
1 & 2 & 2^2 \\
1 & 3 & 3^2 \\
1 & 5 & 5^2
\end{bmatrix} \right)^{-1}
\begin{bmatrix}
1 & 1 & 1 & 1 \\
0 & 2 & 3 & 5 \\
0 & 2^2 & 3^2 & 5^2
\end{bmatrix}
\begin{bmatrix}
2\\2\\3\\3
\end{bmatrix} \approx
\begin{bmatrix}
1.9231\\ 0.2308 \\ 0.0000
\end{bmatrix}\]
\end{questions}
\end{document}