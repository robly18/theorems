\documentclass[addpoints]{exam}

\usepackage{amsmath}
\usepackage{amssymb}
\usepackage{amsfonts}
\usepackage{mathtools}

\DeclarePairedDelimiter{\Norm}\lVert\rVert
\newcommand{\N}{\mathbb{N}}
\newcommand{\R}{\mathbb{R}}
\newcommand{\tr}{\intercal}
\newcommand{\I}{\mathrm{i}}
\newcommand{\e}{\mathrm{e}}
\newcommand{\frakB}{\mathfrak{B}}

\pagestyle{headandfoot}
\runningheader{MATH 19620 Section 42}{Final Exam}{Duarte Maia}
\runningheadrule
\footer{}{\ifincomplete{Exercise continues on next page.}{}}{}


\begin{document}

\begin{center}
\Large Final Exam

MATH 19620 - Section 42

\smallskip

\large Section Instructor: Duarte Maia

\smallskip

May 29, 2025
\end{center}

\begin{center}
\fbox{
\parbox{6in}{
\begin{itemize}
\item Answer the following questions in the provided examination book.

\item Write the number of the question you are answering at the start of your solution, and ensure that your final answer is clearly delineated.

\item Write your name and UCID in the cover page of the examination book. You may ask for additional paper if necessary.

\item Unless otherwise specified, you do not need to justify your answers.
\end{itemize}
}
}
\end{center}

\noindent Useful Formulas:
\begin{itemize}
\item Change of basis: $[\vec x]_\frakB = S^{-1} \vec x$, $A = S [A]_\frakB S^{-1}$.
\item $\det(AB) = \det A \det B$, $\det(A^{-1}) = 1/\det A$, $\det A^\tr = \det A$.
\item Laplace Expansion: $\det A = \sum_i (-1)^{i+j} a_{ij} \det A_{ij}$.
\end{itemize}

\hrule

\qformat{\textbf{Exercise \thequestion} \quad (Total: \totalpoints \ points)\hfill}

\begin{questions}

\section*{Part A -- General Linear Algebra}

\question

Consider the matrices
\[A = \begin{bmatrix}
1 & 2 \\
3 & 6 \\
-2 & -4
\end{bmatrix}
\text{ and }
B = \begin{bmatrix}
2 & -4 & 6 \\
-1 & 2 & -3
\end{bmatrix}.\]

\begin{parts}
\part[5] Find a basis for the image of $A$.
\part[5] True or false: The image of $A$ is a plane in $\R^3$. Justify your answer.
\part[15] Sketch the kernel of $A$ and the image of $B$. Are they the same?
\part[15] Consider the linear equation $B \vec x = \vec b$, where $B$ is the matrix above. For what values of $\vec b$ does this equation:
\begin{itemize}
\item have exactly one solution?
\item have infinitely many solutions?
\item have no solutions?
\end{itemize}
\end{parts}
 
\question Consider the following list of vectors:
\[\vec v_1 = (2,1,2), \quad \vec v_2 = (3,1,3) \quad \vec v_3 = (1,2,2), \quad \vec w = (0,1,0).\]

\begin{parts}
\part[10] Justify that $\frakB = \{\vec v_1, \vec v_2, \vec v_3\}$ forms a basis.
\part[10] Justify that $\{\vec v_1, \vec v_2, \vec w\}$ does not form a basis.
\part[10] Let $\vec x = (1,1,2)$. Find the coordinates of $\vec x$ in the basis $\frakB$.
\part[20] Let $T$ be an unknown linear transformation in $\R^3$. Knowing only that $T(\vec v_1) = \vec v_2$, $T(\vec v_2) = \vec v_3$, and $T(\vec v_3) = \vec v_1$, find the matrix corresponding to $T$. You may leave your answer as an unevaluated product of matrices, though you must evaluate any inverses.
\end{parts}

\question
Consider the paralleliped in $\R^4$:
\[P = \{(x,y,z,w) \mid 0\leq x \leq 2, 0 \leq y \leq 3, 0 \leq z \leq 1, 0\leq w \leq \tfrac13\},\]
as well as the matrix
\[A = \begin{bmatrix}
1 & 2 & 12 & 0 \\
0 & 2 & 57 & 7 \\
0 & 0 & 2 & 0 \\
1 & -1 & 99 & 1
\end{bmatrix}.\]

Note that $P$ has 4-dimensional volume equal to $2\times 3 \times 1 \times \frac13 = 2$.
\begin{parts}
\part[20] Let $AP$ be the shape obtained by applying the linear transformation $T(\vec x) = A\vec x$ to the paralleliped $P$. What is the 4-dimensional volume of $AP$?
\part[10] Determine the 4-dimensional volume of $A^{-1} P$.
\part[10] Determine the angle between the first two edges of $AP$, which are defined by the vectors $A(2,0,0,0)$ and $A(0,3,0,0)$. Write your answer in a form that could easily be entered into a calculator.
\end{parts}


\section*{Part B -- Spectral Theory}

\question[25] Diagonalize the matrix
\[A = \begin{bmatrix}
1 & 1 & -1 \\
-1 & 3 & -1 \\
-1 & 1 & 1
\end{bmatrix}.
\]

\question Let $\vec x(t)$, $t \in \N$, be a dynamical system defined by the rules
\[\begin{cases}
\vec x(0) = (8,9,1),\\
\vec x(t+1) = A\vec x(t),
\end{cases}
\]
where $A$ is the $3 \times 3$ matrix
\[
A=\begin{bmatrix}
 2.808 & -1.323 & -0.162 \\
 3.425 & -1.170 & -0.085 \\
 -2.308 & 1.323 & 0.662 \\
\end{bmatrix}.\]

\begin{parts}
\part[10] Knowing that the eigenvalues of $A$ are $1.5$, $0.9 + 0.6\I$, and $0.9 - 0.6\I$, determine the long-term behavior of the dynamical system $\vec x(t)$. Does it converge to $\vec 0$, shoot off to infinity, or something else? Justify.
\part[10] What if $A$ was instead a matrix whose eigenvalues are $0.5$, $-0.5+0.2\I$, and $-0.5-0.2\I$? Does your answer depend on the initial state $\vec x(0)$? Justify.
\part[20] What if $A$ had eigenvalues $0.2$, $1.2$, and $1.5$? Does your answer depend on the initial state $\vec x(0)$? Justify.
\end{parts}

\question Determine whether each of the following statements is true or false. Justify your answer.
\begin{parts}
\part[10] Let $Q$ be an orthogonal matrix, that is: $Q^\tr Q = I$. Then, $\det Q$ is either $1$ or $-1$.
\part[10] The matrix $D = \left[\begin{smallmatrix} 0 & 1 \\ -1 & 2\end{smallmatrix}\right]$ is diagonalizable.
\part[10] If $\vec v$ is an eigenvector of $A$, it must also be an eigenvector of $A^3$.
\part[10] If $A$ is an $n\times n$ matrix with $n$ distinct eigenvalues, then $A$ is invertible.
\end{parts}
\end{questions}
\end{document}