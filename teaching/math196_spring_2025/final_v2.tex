\documentclass[addpoints]{exam}

\usepackage{amsmath}
\usepackage{amssymb}
\usepackage{amsfonts}
\usepackage{mathtools}

\DeclarePairedDelimiter{\Norm}\lVert\rVert
\DeclarePairedDelimiter{\abs}\lvert\rvert
\newcommand{\N}{\mathbb{N}}
\newcommand{\R}{\mathbb{R}}
\newcommand{\tr}{\intercal}
\newcommand{\I}{\mathrm{i}}
\newcommand{\e}{\mathrm{e}}
\newcommand{\frakB}{\mathfrak{B}}

\pagestyle{headandfoot}
\runningheader{MATH 19620 Section 42}{Final Exam}{Duarte Maia}
\runningheadrule
\footer{}{\ifincomplete{Exercise continues on next page.}{}}{}


\begin{document}

\begin{center}
\Large Final Exam

MATH 19620 - Section 42

\smallskip

\large Section Instructor: Duarte Maia

\smallskip

May 29, 2025
\end{center}

\begin{center}
\fbox{
\parbox{6in}{
\begin{itemize}
\item Answer the following questions in the provided examination book.

\item Write the number of the question you are answering at the start of your solution, and ensure that your final answer is clearly delineated.

\item Write your name and UCID in the cover page of the examination book. You may ask for additional paper if necessary.

\item Unless otherwise specified, you do not need to justify your answers.

\item This exam is graded out of \numpoints\ points.
\end{itemize}
}
}
\end{center}

\noindent Useful Formulas:
\begin{itemize}
\item Change of basis: $[\vec x]_\frakB = S^{-1} \vec x$, $A = S [A]_\frakB S^{-1}$.
\item $\det(AB) = \det A \det B$, $\det(A^{-1}) = 1/\det A$, $\det A^\tr = \det A$.
\item Laplace Expansion: $\det A = \sum_i (-1)^{i+j} a_{ij} \det A_{ij}$.
\item Quadratic formula: $ax^2 + bx + c = 0$ solved by $x = \frac{-b\pm\sqrt{b^2 -4ac}}{2a}$.
\item Polar form: $z = r \e^{\I \theta} = r \cos(\theta) + \I r \sin(\theta)$ where $r = \abs z$ and $\theta$ is a real number.
\item Projection on line spanned by unit vector $\vec u$: $\vec u \vec u^\tr$.
\item Projection on plane perpendicular to unit vector $\vec u$: $I - \vec u \vec u^\tr$.
\item Reflection: $2P-I$, where $P$ is projection.
\item 2d dynamical system with complex eigenvalues: If $A$ has eigenvalues $\lambda_\pm = r \e^{\pm \I \theta}$ with eigenvectors $\vec v \pm \I \vec w$ then
\[A^t = r^t S \begin{bmatrix}
\cos(\theta t) & -\sin(\theta t) \\
\sin(\theta t) & \cos(\theta t)
\end{bmatrix}
S^{-1},\text{ where } S = \begin{bmatrix}
\vphantom{\begin{bmatrix}w\\w\end{bmatrix}}\vec w & \vec v
\end{bmatrix}.\]
\item Angle between $\vec v$ and $\vec w$: $\theta = \arccos\left(\frac{\vec v \cdot \vec w}{\Norm{\vec v} \Norm{\vec w}}\right)$.
\end{itemize}

%\hrule

\qformat{\textbf{Exercise \thequestion} \quad (Total: \totalpoints \ points)\hfill}

\begin{questions}

\newpage

\section*{Part A -- General Linear Algebra}

\question

Consider the matrices
\[A = \begin{bmatrix}
1 & 2 & 3 \\
2 & 4 & 6
\end{bmatrix}, \;
B = \begin{bmatrix}
1 & -1 & 1 \\
2 & -1 & -1 \\
3 & -2 & 0
\end{bmatrix}
\text{ and }
C = \begin{bmatrix}
4 & -3 & 1 \\
-1 & 1 & -1
\end{bmatrix}.\]

\begin{parts}
\part[30] Do any of these matrices have the same kernel? Justify your answer.
\part[15] Do any of these matrices have the same image? Justify your answer.
\part[15] Consider the linear equation $A \vec x = \vec b$, where $A$ is the matrix above. For what values of $\vec b$ does this equation:
\begin{itemize}
\item have exactly one solution?
\item have infinitely many solutions?
\item have no solutions?
\end{itemize}
\end{parts}
 
\question Let $\vec v_1 = (1,2,3,1)$ and $\vec v_2 = (1,3,4,2)$ be two vectors in $\R^4$.
\begin{parts}
\part[5] Verify that $\vec v_1$ and $\vec v_2$ forms a basis of $V = \mathrm{span}(\vec v_1, \vec v_2)$.
\part[10] Determine if the following vectors are in $V$. In the affirmative case, write the coordinates in this basis of $V$.
\[\vec x_1 = (0,1,1,2), \quad \vec x_2 = (1,1,2,0).\]
\part[10] Given vectors $\vec w_1, \dots, \vec w_k$ in $\R^n$, the dimension of $\mathrm{span}(\vec w_1,\dots,\vec w_k)$ is (choose one):
\begin{itemize}
\item equal to $n$,
\item equal to $k$,
\item not enough information,
\item other.
\end{itemize}
Justify your answer.
\part[15] Set $\vec v_3 = (0,0,1,0)$ and $\vec v_4 = (0,0,0,1)$. It is given that $\frakB = \{\vec v_1, \vec v_2, \vec v_3, \vec v_4\}$ form a basis of $\R^4$. Let $\vec x = (a,b,c,d)$. Write a fully simplified expression for $[\vec x]_\frakB$.
\end{parts}

\question
\begin{parts}
\part[15] Consider the matrix of rotation by $45^\circ$ around the $z$ axis:
\[Q = \begin{bmatrix}
1 & 0 & 0 \\
0 & \frac 1{\sqrt 2} & -\frac1{\sqrt 2}\\
0 & \frac 1{\sqrt 2} & \frac1{\sqrt 2}
\end{bmatrix}\]
and the vector $\vec v = (1,1,1)$. Compute the angle between $\vec v$ and $Q\vec v$. Write your answer in exact form, which could be entered into a calculator.
\part[15] Let $T(\vec x) = A\vec x$ be an invertible linear transformation. We say that $T$ is \emph{orientation preserving} if $\det A > 0$, and \emph{orientation reversing} if $\det A < 0$. Show that the composition of two orientation reversing transformations is orientation preserving.
\part[20] A certain region $R$ of $\R^3$ has volume equal to 5. A certain $3\times 3$ matrix $B$ satisfies the equation $B^3 = -I$. What is the volume of $BR$, the region obtained by multiplying every element of $R$ by the matrix $B$?
%\part[15] Let $A$ be an $n\times n$ matrix. What is the relation between $\det A$ and $\det(-A)$? \textit{Hint: It may be useful to remember the effect of row operations on the determinant of a matrix.}
\end{parts}

\newpage


\section*{Part B -- Spectral Theory}

\question[50] Diagonalize the matrix
\[A = \begin{bmatrix}
3 & 0 & 0 \\
-2 & 5 & 4 \\
2 & -2 & -1
\end{bmatrix}.
\]
%NB: Eigenvalues are 1,3,3 and eigenvectors are, resp:
%0,1,-1
%1,1,0
%2,0,1

\question Consider a Fibonacci-like sequence defined by
\[\begin{cases}
a_0 = 0,\\
a_1 = 1,\\
a_{n+2}=-a_n+\frac52 a_{n+1}.
\end{cases}\]

\begin{parts}
\part[10] The vectors $\vec x(t) = (a_t, a_{t+1})$, for $t=0,1,2,\dots$ form a linear discrete dynamical system. Find the matrix $A$ associated to this system.
\part[35] The matrix you found may be diagonalized as $A = SDS^{-1}$, with
\[S = \begin{bmatrix}
\frac12 & 2 \\
1 & 1
\end{bmatrix}, \;
D = \begin{bmatrix}
2 & 0 \\
0 & \frac12
\end{bmatrix}, \text{ and }
S^{-1} = \frac13 \begin{bmatrix}
-2 & 4 \\
2 & -1
\end{bmatrix}.\]

Using this information, write an explicit expression for $\vec x(t)$. Use this to write an explicit expression for the sequence $a_n$, and show that $a_n \to \infty$.
\part[20] A friend of yours claims that, by choosing the right values of $a_0$ and $a_1$ (but preserving the rule for $a_{n+2}$), with neither value being zero, it is possible to make it so that $a_n \to 0$ instead. Is this statement true or false? Justify your answer.
\end{parts}

\question Determine whether each of the following statements is true or false. Justify your answer.
\begin{parts}
\part[10] If $A$ is a matrix that admits $\vec e_1, \dots, \vec e_n$ as an eigenbasis, then $A$ is diagonal.
\part[10] If $A$ is a $4\times 4$ matrix that admits the vectors $(1,1,0,0
)$, $(0,1,1,0)$, $(0,0,1,1)$, and $(1,0,0,1)$ as an eigenbasis, then $A$ cannot be a diagonal matrix.
\part[15] There is a value of $x$ for which the matrix
\[B_x = \begin{bmatrix}
x-3 & 0 \\
x^2-5x+4 & 1
\end{bmatrix}\]
is \emph{not} diagonalizable.
%\part[15] A rotation matrix $\left[\begin{smallmatrix} \sin\alpha & -\cos\alpha \\ \cos\alpha & \sin\alpha\end{smallmatrix}\right]$ is always diagonalizable.
\end{parts}
\end{questions}
\end{document}