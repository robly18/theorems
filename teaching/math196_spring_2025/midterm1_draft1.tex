\documentclass[addpoints]{exam}

\usepackage{amsmath}
\usepackage{amsfonts}
\usepackage{mathtools}

\DeclarePairedDelimiter{\Norm}\lVert\rVert
\newcommand{\R}{\mathbb{R}}

\pagestyle{headandfoot}
\runningheader{MATH 19620 Section 42}{Midterm 1}{Duarte Maia}
\runningheadrule
\footer{}{\ifcontinuation{}{Exercise continues on next page.}}{}


\begin{document}

\begin{center}
\Large Midterm 1

MATH 19620 - Section 42

\smallskip

\large Section Instructor: Duarte Maia

\smallskip

April 17, 2025
\end{center}

\begin{center}
\fbox{
\parbox{6in}{
\begin{itemize}
\item Answer the following questions in the provided examination book.

\item The same question may be answered over more than one page, but do not use the same page to answer multiple questions.

\item Write the number of the question you are answering in the top of the page, and ensure that your final answer is clearly delineated. If your answer spans multiple pages, write ``continued in next page'' and ``solution of problem $X$, continued'' when applicable.

\item Write your name and UCID in the cover page of the examination book. You may ask for additional paper if necessary.

\item Unless otherwise specified, you do not need to justify your answers.
\end{itemize}
}
}
\end{center}

\noindent Useful Formulas:
\begin{itemize}
\item Norm/Length of a vector in $\mathbb{R}^2$: $\Norm v = \sqrt{v_1^2 + v_2^2}$,
\item Canonical basis in $\mathbb{R}^2$: $\vec{e}_1 = (1,0)$, $\vec{e}_2 = (0,1)$.
\end{itemize}

\hrule

\qformat{\textbf{Exercise \thequestion} \quad (Total: \totalpoints \ points)\hfill}

\begin{questions}

\question[25] Consider the linear transformation:
\[T(x,y)=\begin{bmatrix}
-1 & -1 \\
-1 & 1
\end{bmatrix}
\begin{bmatrix} x \\ y \end{bmatrix}.\]

Show the effect of $T$ on the letter $L$, as was done in the project.
 
\question[25] Consider the linear transformation:
\[P(x,y,z,w) = (w,-x,-z,y).\]

Determine whether $P$ is a linear isometry. Justify your answer.

\question
\begin{parts}
\part[25] Let $\theta$ be an angle. What is the matrix of the transformation $R_\theta \colon \R^2 \to \R^2$ that rotates the $xy$ plane by the angle $\theta$ in the counter-clockwise direction?

\textit{If you don't know the answer to part (a), you may solve part (b) assuming that the answer to (a) is the matrix $\left[\begin{smallmatrix}\cos\theta & \sin\theta \\ \sin\theta & -\cos\theta\end{smallmatrix}\right]$. Note: This is \emph{not} the solution to part (a).}

\part[25] Recall that in the last question of the project you found the matrix that represents rotation by $\alpha$ around the line $x=y=z$ through an indirect method called ``conjugation''. \textbf{Using this same method}, find a formula for the matrix $Q$ that represents reflection through the line $y = mx$, using the following facts:
\begin{itemize}
\item The angle $\alpha$ between the $x$ axis and the line $y = mx$ satisfies
\[\cos(\theta) = \frac1{\sqrt{m^2 + 1}} \text{ and } \sin(\theta) = \frac m{\sqrt{m^2 + 1}},\text{ and} \]
\item The matrix
\[S = \begin{bmatrix}
1 & 0 \\
0 & -1
\end{bmatrix}\]
represents a reflection about the $x$ axis.
\end{itemize}

Your final answer must be a single $2\times 2$ matrix, whose entries are written in terms of $m$.
\end{parts}
\end{questions}
\end{document}