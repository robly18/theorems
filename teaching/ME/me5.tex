\documentclass{article}
\usepackage[utf8]{inputenc}
\usepackage[portuges]{babel}
\usepackage{ntheorem}
\usepackage{amsfonts}
\usepackage{amsmath}
\usepackage{amssymb}
\usepackage{mathtools}
\usepackage{diffcoeff}

\usepackage{tikz} 
\usetikzlibrary{matrix,decorations.pathreplacing, calc, positioning}

%\usepackage[margin=1.5in]{geometry}
\usepackage{multicol}
\theorembodyfont{\upshape}
\theoremseparator{.}
\newtheorem{ex}{Exercício}

\usepackage{enumitem}
\setlist[enumerate, 1]{label=\alph*)}

\usepackage{listings}
\lstset{basicstyle=\ttfamily,mathescape,keepspaces,tabsize=4,
literate=
  {á}{{\'a}}1
  {à}{{\`a}}1
  {ã}{{\~a}}1
  {é}{{\'e}}1
  {ê}{{\^e}}1
  {í}{{\'i}}1
  {ó}{{\'o}}1
  {ú}{{\'u}}1
  {ç}{{\c{c}}}1}
\usepackage{graphicx}
\usepackage{url}
\usepackage{hyperref}

\title{Exercício: Crivo de Sundaram}
\author{}
\date{}
\setlength{\parindent}{0pt}
\newcommand{\Z}{\mathbb{Z}}
\newcommand{\R}{\mathbb{R}}
\newcommand{\N}{\mathbb{N}}
\newcommand{\Q}{\mathbb{Q}}


\newcommand{\T}{\mathbb{T}}

\begin{document}

Nas aulas teóricas, o aluno terá aprendido sobre o crivo de Erastótenes, que é um algoritmo para determinar todos os números primos até um dado $n$ natural. Neste exercício, vai implementar e estudar outro crivo, chamado o crivo de Sundaram.

O algoritmo é o seguinte:
\begin{lstlisting}[columns=fullflexible]
Seja $L = \mathtt{Range[n]}$,
Seja $i = 1$,
Enquanto $3 i^2 \leq n$,
	Se $L[[i]] \neq \mathtt{"REMOVIDO"}$,
		Seja $j = 3i + 1$,
		Enquanto $j \leq n$:
			$L[[j]] \leftarrow \mathtt{"REMOVIDO"}$,
			$j \leftarrow j + 2i + 1$
	$i \leftarrow i+1$
Para cada elemento $x \in L$,
	Se $x \neq \mathtt{"REMOVIDO"}$,
		Adicionar $x$ a out
Retornar $out$.
\end{lstlisting}

\begin{ex}
Implemente o crivo de Sundaram em Mathematica. Teste-o para vários valores de $n$. Explique, com base nos seus testes, o que crivo retorna.
\end{ex}


\begin{ex}

\end{ex}


\end{document}