\documentclass{article}
\usepackage[utf8]{inputenc}
\usepackage[portuges]{babel}
\usepackage{ntheorem}
\usepackage{amsfonts}
\usepackage{amsmath}
\usepackage{amssymb}
\usepackage{mathtools}
\usepackage{diffcoeff}

\usepackage{tikz} 
\usetikzlibrary{matrix,decorations.pathreplacing, calc, positioning}

%\usepackage[margin=1.5in]{geometry}
\usepackage{multicol}
\theorembodyfont{\upshape}
\theoremseparator{.}
\newtheorem{ex}{Exercício}

\usepackage{enumitem}
\setlist[enumerate, 1]{label=\alph*)}

\usepackage{listings}
\lstset{basicstyle=\ttfamily,mathescape,keepspaces,tabsize=2,
literate=
  {á}{{\'a}}1
  {à}{{\`a}}1
  {ã}{{\~a}}1
  {é}{{\'e}}1
  {ê}{{\^e}}1
  {í}{{\'i}}1
  {ó}{{\'o}}1
  {ú}{{\'u}}1
  {ç}{{\c{c}}}1}
\usepackage{graphicx}
\usepackage{url}
\usepackage{hyperref}

\title{Exercício: Compressão de Imagens}
\author{}
\date{}
\setlength{\parindent}{0pt}
\newcommand{\Z}{\mathbb{Z}}
\newcommand{\R}{\mathbb{R}}
\newcommand{\N}{\mathbb{N}}
\newcommand{\Q}{\mathbb{Q}}


\newcommand{\T}{\mathbb{T}}

\begin{document}

O objetivo deste exercício é implementar e testar um algoritmo simples de compressão de imagens. O objetivo de compressão de imagens (neste caso, compressão \textit{lossy}) é pegar numa imagem que tem uma certa quantidade de informação e guardar apenas parte dela, de modo a reduzir o tamanho dela em disco, mas sem degradar demasiado a sua qualidade visual.

A ideia básica é a seguinte. Uma imagem (a preto e branco, por simplicidade) é representada como uma matriz $M$ de números reais entre 0 e 1, sendo que cada casa da matriz $M$ representa um pixel, e o número correspondente indica a iluminação do pixel (0 significa que o pixel está preto, 1 significa que o pixel está branco). Ora, o espaço das matrizes é um espaço linear, equivalente a $\R^{wh}$, onde $w$ é a largura da imagem e $h$ a sua altura, em pixeis. Assim sendo, temos ao nosso dispor ferramentas de álgebra linear. Neste caso, o que será feito será:
\begin{enumerate}
\item Construir uma base \emph{ortonormada} conveniente do espaço das matrizes,
\item Representar uma imagem nesta base
\item Mandar fora coeficientes pequenos, que não terão muito impacto na imagem final.
\end{enumerate}

O método descrito abaixo tem várias limitações. Por exemplo, só funciona em imagens quadradas, cujo tamanho seja uma potência de dois.

Para começar o exercício, escolha uma imagem para comprimir. Esta imagem deverá ser aproximadamente quadrada e	 de preferência uma imagem `pouco lisa', por exemplo uma fotografia.

\begin{ex}
Importe a imagem para o Mathematica da seguinte forma. Guarde-a na pasta do Notebook, e use o comando:
\begin{verbatim}
img = ImageResize[Import[NotebookDirectory[]<>"nome.png"],{128,128}]
\end{verbatim}

Execute os seguintes comandos e explique o seu output.
\begin{verbatim}
data = ImageData[img, Interleaving -> False];
bw = 0.299 data[[1]] + 0.587 data[[2]] + 0.144 data[[3]];
Image[bw]
\end{verbatim}
\end{ex}

\begin{ex}
Implemente um comando que, dado um $n$, retorne a lista das $n^2$ matrizes $n \times n$ que têm exatamente uma casa igual a 1, e o resto igual a 0.
\end{ex}

\begin{ex}
Considere o espaço das matrizes $n \times n$ como se fosse $\R^{n^2}$, com o produto interno usual. Sob esta perspetiva, verifique computacionalmente para $n=4$ que a base descrita acima é ortonormada.

Calcule a norma da matriz \texttt{bw}. Interprete o significado do número resultante.
\end{ex}

\begin{ex}
Implemente uma função \texttt{basisCoefficients[basis,matrix]} que, dada uma lista \emph{ortonormada} \texttt{basis} de matrizes $n \times n$ e uma matriz calcula a lista $\{c_1, c_2, \dots, c_{n^2}\}$ de coeficientes tais que
\[\mathtt{matrix} = c_1 \mathtt{basis[[1]]} + \dots + c_{n^2} \mathtt{basis[[n^2]]}.\]
\end{ex}

\begin{ex}
Implemente uma função \texttt{compressCoefficients[coefs,epsilon]} que recebe uma lista de coeficientes \texttt{coefs} (como no exercício anterior) e retorna apenas aqueles que sejam maiores em módulo que \texttt{epsilon}.

O formato de retorno desta função é deixado à escolha do utilizador, e deverá ser feito para possibilitar a implementação de outra função (que deverá também implementar), chamada \texttt{decompress[compressedcoefs, basis]} que recebe o output da função anterior e a base usada para calcular os coeficientes, e devolve uma aproximação da matriz original \texttt{matrix}.
\end{ex}

\begin{ex}\label{excomprimir}
Use as ferramentas desenvolvidas acima para tentar comprimir a matriz \texttt{bw}. Escolha um valor de \texttt{epsilon} baixo o suficiente de modo a obter uma imagem que seja parecida com a original, mas alto o suficiente para haver diferenças visíveis.

Aponte as diferenças que ocorreram e explique-as. Diria que este algoritmo de compressão (usando a base canónica) é adequado?
\end{ex}

\begin{ex}\label{exabc}
Seja $n$ uma potência de 2. Considere-se a seguinte construção.

Para $k$ par, definam-se as seguintes matrizes $k \times k$:
\begin{gather*}
A_k = \frac1k \begin{bmatrix}
1 & \dots & 1 & -1 & \dots & -1\\
& \vdots & & &\vdots &\\
1 & \dots & 1 & -1 & \dots & -1
\end{bmatrix}\\
B_k = \frac1k \begin{bmatrix}
1 & & & 1 \\
\vdots & \cdots & \cdots & \vdots \\
1 & & & 1\\
-1 & & & -1 \\
\vdots & \cdots & \cdots & \vdots \\
-1 & & & -1
\end{bmatrix}\\
C_k = \frac1k \begin{bmatrix}
1 & \dots & 1 & -1 & \dots & -1\\
\vdots & \ddots & \vdots & \vdots & \ddots & \vdots\\
1 & \dots & 1 & -1 & \dots & -1\\
-1 & \dots & -1 & 1 & \dots & 1\\
\vdots & \ddots & \vdots & \vdots & \ddots & \vdots\\
-1 & \dots & -1 & 1 & \dots & 1
\end{bmatrix}
\end{gather*}

De seguida, definimos as `matrizes-A' de tipo $n \times n$ como as seguintes (os zeros representam blocos):
\begin{gather*}
A_n,\\
\begin{bmatrix}
A_{n/2} & 0\\
0 & 0
\end{bmatrix},
\begin{bmatrix}
0 & A_{n/2}\\
0 & 0
\end{bmatrix},
\begin{bmatrix}
0 & 0\\
A_{n/2} & 0
\end{bmatrix},
\begin{bmatrix}
0 & 0\\
0 & A_{n/2}
\end{bmatrix},\\
\begin{multlined}
\begin{bmatrix}
A_{n/4} & 0 & 0 & 0\\
0 & 0 & 0 & 0\\
0 & 0 & 0 & 0\\
0 & 0 & 0 & 0
\end{bmatrix},
\begin{bmatrix}
0 & A_{n/4} & 0 & 0\\
0 & 0 & 0 & 0\\
0 & 0 & 0 & 0\\
0 & 0 & 0 & 0
\end{bmatrix},
\begin{bmatrix}
0 & 0 & A_{n/4} & 0\\
0 & 0 & 0 & 0\\
0 & 0 & 0 & 0\\
0 & 0 & 0 & 0
\end{bmatrix},
\begin{bmatrix}
0 & 0 & 0 & A_{n/4}\\
0 & 0 & 0 & 0\\
0 & 0 & 0 & 0\\
0 & 0 & 0 & 0
\end{bmatrix},\\
\begin{bmatrix}
0 & 0 & 0 & 0\\
A_{n/4} & 0 & 0 & 0\\
0 & 0 & 0 & 0\\
0 & 0 & 0 & 0
\end{bmatrix},
\begin{bmatrix}
0 & 0 & 0 & 0\\
0 & A_{n/4} & 0 & 0\\
0 & 0 & 0 & 0\\
0 & 0 & 0 & 0
\end{bmatrix}, \text{ etc.}
\end{multlined}\\
\text{(Repetir para $n/8, n/16 \dots, 4, 2.$)}
\end{gather*}

Definimos as matrizes de tipo-B e tipo-C de forma semelhante. Finalmente, definimos as `matrizes ABC' de tipo $n \times n$ como sendo todas as matrizes de tipo A, B e C, juntamente com a matriz de coeficientes constantes iguais a $1/n$.

Implemente uma função que, dado $n$ uma potência de dois, devolve a lista de todas as matrizes ABC de tipo $n \times n$.

Usando o facto que as matrizes ABC formam uma base ortonormada do espaço das matrizes $n \times n$ (não precisa de o provar), repita a primeira parte do exercício \ref{excomprimir} com esta base, ou seja, comprima a matriz \texttt{bw}. 
\end{ex}

Se tiver feito o exercício \ref{exabc}, faça os exercícios a seguir usando como base as matrizes ABC. Caso contrário, use a base canónica.

\begin{ex}
Seja $X$ uma imagem, e $X'$ uma versão comprimida da imagem. O fator de compressão (de $X'$ relativamente a $X$) consiste no número $\frac{\#X}{\#X'}$, onde $\#X$ é o número de bits que ocupa a imagem $X$ e idem para $X'$.

Defina uma função que calcula o fator de compressão do método acima descrito para um dado \texttt{epsilon}. Teste a sua função para alguns valores de \texttt{epsilon} e compare com a imagem comprimida correspondente. Comente a eficácia do método.

É possível o fator de compressão ser superior a 1? Justifique.

\smallskip

Nota: Em princípio, calcular o fator de compressão é bastante complexo, porque é preciso contabilizar toda a informação que existe. Por exemplo, uma lista com 5 números gasta ligeiramente mais bits do que os 5 números que a compõem, porque precisa por exemplo de guardar o seu próprio comprimento e onde estão as separações entre os seus cinco elementos. Para mais, alguns números gastam mais bits a guardar do que outros.

Para resolver este exercício, ignore todos esses fatores. Assuma que a única informação a guardar são os números, e assuma que todos os números ocupam o mesmo espaço.
\end{ex}

\begin{ex}
Justifique quantitativamente que, sem recorrer a fortes otimizações ou a um supercomputador, o exercício acima não seria possível de executar numa imagem $512 \times 512$.
\end{ex}


\end{document}