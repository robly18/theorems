\documentclass{article}
\usepackage[utf8]{inputenc}
\usepackage[portuges]{babel}
\usepackage{ntheorem}
\usepackage{amsfonts}
\usepackage{amsmath}
\usepackage{amssymb}
\usepackage{mathtools}

\theorembodyfont{\upshape}
\theoremseparator{.}
\newtheorem{ex}{Exercício}[section]


\setlength{\parindent}{0pt}
\newcommand{\Z}{\mathbb{Z}}
\newcommand{\R}{\mathbb{R}}
\usepackage[margin=0.9in]{geometry}
\newcommand{\N}{\mathbb{N}}
\newcommand{\Q}{\mathbb{Q}}

\begin{document}

\noindent {\bf Unidade de Ensino de Matemática Aplicada e Análise Numérica}

\noindent Departamento de  Matemática/Instituto Superior
Técnico \vspace{3mm}

\noindent {{\bf Matem\'atica Experimental }} ({\small LMAC}) -- 2$^{\rm o}$ Período de 2021/2022

\bigskip

\begin{center}
{\bf\Large   Trabalho Computacional}\\ 
\smallskip
{\large Segunda Entrega}\\
\smallskip
Aluno1, Aluno2, Aluno3, etc.
\end{center}

\bigskip
\hrule

\section{O Crivo de Sundaram}

\begin{ex}
Resolução do exercício 1.1
\end{ex}

\begin{ex}
Resolução do exercício 1.2
\end{ex}

\section{Baralhos}

\section{Fracções Contínuas Generalizadas}

\section{O Problema de Frobenius}

\section{Compressão de Imagens}

\end{document}