
\documentclass[11pt]{article}
\headheight0pt \textwidth17.5cm     \textheight24.6cm \topmargin -1.0cm
\oddsidemargin -0.7cm \evensidemargin -0.7cm \baselineskip18pt \headsep0pt
\topskip1.5ex
\parskip0em
% \input option_keysplus
\pagestyle{empty}
\newcommand{\real}{{\rm I\!\!\,R}}
\usepackage{amsfonts,euscript,eufrak,latexsym,amssymb}
\usepackage[english,portuges]{babel}
\usepackage{inputenc}
\usepackage{url}
\usepackage{hyperref}
\usepackage{graphicx}
\usepackage{graphics}
\usepackage{color}
\begin{document}



\noindent {\bf Sec\c{c}\~{a}o de Matem\'{a}tica Aplicada e
An\'alise Num\'erica}

\noindent Departamento de  Matem\'atica/Instituto Superior
T\'ecnico \vspace{3mm}

\noindent {{\bf Matem\'atica Experimental }} ({\small LMAC}) -- 1$^{\rm o}$ Semestre de 2021/2022\\

\section{Como utilizar o sistema  Mathematica  na Sala P13}

\noindent
Para utilizar os computadores do laborat\'orio de computa\c{c}\~ao (sala P13), cada aluno deve previamente, ap\'os entrar na sua \'area pessoal do F\'enix  com as suas credenciais do IST, aceder \`a p\'agina  \texttt{\url{https://ciist.ist.utl.pt/servicos/self\_service/}}
e ativar os servi\c{c}os \emph{Shell} e \emph{AFS}.

\medskip

\noindent
A seguir o aluno deve:

\begin{itemize} 

\item  Iniciar o sistema \textbf{Mathematica}. 
\item Clicar no bot\~ao \textbf{Other ways to activate} na janela \emph{Wolfram Product Activation}.
\item Seleccionar \textbf{Connect to a Network License Server}. 

\item Escrever na janela que aparece:  \textbf{delta.tecnico.ulisboa.pt}

\end{itemize}

\noindent
Se n\~ao consegue entrar no sistema \textbf{Mathematica} com as suas credenciais, pode utilizar o \emph{username} \textbf{guest} e a \emph{password} \textbf{guest}. 


\pagebreak

\begin{center}
{\bf   \Large{Aula Laboratorial I}} \vspace{6mm}
\end{center}

\section{Como utilizar o sistema  Mathematica}

\begin{itemize}


\item Para iniciar uma sess\~ao do \textbf{Mathematica} clique no seu \'\i cone. 



\item Para avaliar um comando prima as  teclas \textsl{shift} e \textsl{return}.  



\item Para interromper um c\'alculo escolha no menu \textbf{Evaluation} a op\c{c}\~ao \textbf{Abort Evaluation}. 



\item Para obter informa\c{c}\~ao acerca de um comando, \textsl{e.g.}, \textbf{Head}, executa o comando \textbf{?Head} no seu \textsl{notebook} ou procure o comando no menu \textbf{Help}$\rightarrow$\textbf{Wolfram Documentation} ou  \textbf{Help}$\rightarrow$\textbf{Find Selected Function}.





\item Para sair do  programa  seleccione a opc\~ao \textbf{Quit Mathematica} do menu \textbf{Mathematica}. 




\item Aconselha-se que grave o seu \textsl{notebook}  com alguma frequ\^encia a fim de n\~ao perder o trabalho se houver um \textsl{crash} do sistema.




\item O \textbf{Mathematica} interpreta letras min\'usculas e mai\'usculas de forma distinta: 

 




\item As fun\c{c}\~oes e constantes do sistema come\c{c}am com letra mai\'uscula.  
 


\item Os nomes que come\c{c}am com letra min\'uscula s\~ao entendidos como vari\'aveis. Recomenda-se pois que inicie os seus pr\'oprios comandos com letra min\'uscula. 



\item No \textbf{Mathematica} s\~ao usados diversos tipos de par\^enteses: 





\item As fun\c{c}\~oes utilizam argumentos (separados por v\'\i rgulas) ent\-re par\^enteses rectos, \textsl{e.g.}, \textbf{Sin[x]}, \textbf{Integrate[x$\,\hat{}\, $2,x]}.



\item As chavetas s\~ao usadas para delimitar listas,  \textsl{e.g.}, $\textbf{\{1, 2, 3\}}$. 


\item Os par\^enteses curvos usam-se apenas para agrupar termos. 



\item Os par\^enteses rectos duplos s\~ao usados para indexa\c{c}\~ao, \textsl{e.g.},  o quarto elemento da lista \textbf{a} \'e \textbf{a[[4]]}.



\item Deve utilizar com cuidado s\'\i mbolos como ponto e  v\'\i rgula (;), v\'\i rgula (,), dois pontos (:) e ponto final (.).


\end{itemize}

\pagebreak


\section{Opera\c{c}\~oes B\'asicas  }



O sistema \texttt{Mathematica} usa 4 tipos de n\'umeros. Os inteiros (\texttt{Integer}) e racionais (\texttt{Rational}) s\~ao representados exatamente. Os n\'umeros reais (\texttt{Real}) s\~ao representados aproximadamente. Os complexos (\texttt{Complex}) s\~ao representados na forma \texttt{number} + \texttt{number\, I}.

\begin{itemize}

\item[a)] No seu \textsl{notebook} escreva
\[
2+2
\]
Para avaliar o seu comando prima e mantenha a tecla \textsl{shift}, prima depois \textsl{return}. Aparecer\'a no ecr\~a \texttt{In[1]} e \texttt{Out[1]} como abreviaturas respetivamente do  \texttt{Input} e do \texttt{Output}. 

\item[b)]   Uma alternativa ao menu \texttt{Help} consiste na execu\c{c}\~ao do operador \texttt{?} ou \texttt{??}. Execute as seguintes instru\c{c}\~oes (linha a linha) e observe o resultado
 \[\begin{array}{l}
%?Complex\\
%  ?Complex \ast \\
%?\!\ast \!\!Form \\
?Plot\\
?\!\ast \!Plot \ast\\
% ?\! \ast \!Expand\\
% ?Factor\ast\\
  ??FactorInteger\\
\end{array}
 \]


\item[c)] Avalie
\[
10\ast (3 \ 5+24/3)
\]
 Note que entre o 3 e o 5 n\~ao h\'a sinal de multiplica\c{c}\~ao, mas o sistema
assume a exist\^encia ali desse s\'\i mbolo.
Execute as instru\c{c}\~oes a seguir (linha a linha)
 \[\begin{array}{l}
%FullForm[a\ast(  b \ c + d\ \widehat{}\ 2 /c) ]\\ 
FullForm[a\ast(  bc + d\ \widehat{}\ 2 /c) ]\\ T reeF orm[a\ast(  b \ c + d\ \widehat{}\ 2 /c)]
 \end{array} \]
 
  

O s\'\i mbolo \ $\widehat{}$\ \, indica exponencia\c{c}\~ao. Experimente (linha a linha)


 \[\begin{array}{l}
2\,\hat{}\, 3\,\hat{}\, 4 \\ (2\,\hat{}\, 3)\,\hat{}\,4 
% \\ HoldForm[2\,\hat{}\, 3\,\hat{}\, 4] 
% \\ TreeForm[Hold[2\,\hat{}\, 3\,\hat{}\, 4]]
 \\ TreeForm[Hold[(2\,\hat{}\, 3)\,\hat{}\, 4]]
\\ 4\,\hat{}\, 4\,\hat{}\, 4 \\
N[\%]\\
4\,\hat{}\, 4\,\hat{}\, 4 \,\hat{}\, 4
\end{array} \]

  Consulte o \texttt{Help} sobre os comandos  \texttt{N}, 
  \texttt{Hold},
  % \texttt{HoldForm} 
  e \texttt{FullForm}. O s\'\i mbolo $\%$, refere-se ao \'ultimo \texttt{Output}.


\item[d)] Uma instruc\~ao t\'\i pica do  \texttt{Mathematica}  \'e de forma geral
\[
h[x_1,x_2,\ldots, x_n]
\]
onde \textsl{h}  \'e designado por \textsl{cabe\c{c}a} ou \textsl{cabe\c{c}alho} (\texttt{Head}). A instru\c{c}\~ao pode ter zero, um, ou mais argumentos $x_1, x_2,\ldots, x_n$, os quais por sua vez s\~ao tamb\'em express\~oes.
As express\~oes mais simples, denominadas do tipo at\'omico, constituem os tipos b\'asicos de dados que podemos utilizar: \texttt{Integer, Rational, Real, Complex, Symbol} e \texttt{String}. Experimente
\[
\{Head[x], Head[1], Head[2.0], Head[2/3], Head[\texttt{\char"0D}\!\! \texttt{\char"0D}\texttt{Mathematica}\texttt{\char"0D}\!\! \texttt{\char"0D}], Head[3 + 2 I], 
 Head[\{a, b, c\}]\}
 \]
 
% \item[e)]   Depois de executar as instru\c{c}\~oes
%  \[\begin{array}{l}
% TreeForm[Hold[5/2\ast3-5\ast2\ast9] ] \\
% Trace[5/2 \ast 3 - 5 \ast2 \ast 9]
% \end{array}
% \]
% diga qual  \'e a ordem segundo a qual as opera\c{c}\~oes envolvidas s\~ao executadas.
 
 
\item[e)]   Coment\'arios podem ser escritos entre os s\'\i mbolos $(\ast \quad \ast)$. Experimente
\[
a+b \qquad (\ast \textrm{\ soma  a  com b }\ast)
\]

\item[f)]  Execute o comando

\[\begin{array}{l}
Print[ \texttt{\char"0D}\!\! \texttt{\char"0D}\texttt{A vers\~ao }\texttt{\char"0D}\!\! \texttt{\char"0D} , \$Version,\texttt{\char"0D}\!\! \texttt{\char"0D} \texttt{ do meu Mathematica possui }\texttt{\char"0D}\!\! \texttt{\char"0D} , \\ 
\qquad \qquad  \qquad Length[ Names[\texttt{\char"0D}\!\! \texttt{\char"0D} System\char"012\!\!\ast\!\texttt{\char"0D}\!\! \texttt{\char"0D}  ]], \texttt{\char"0D}\!\! \texttt{\char"0D}  \texttt{ comandos.}\texttt{\char"0D}\!\! \texttt{\char"0D} ]
\end{array}
\]
Uma \textsl{string}  como \texttt{\char"0D}\hspace{-4mm} \texttt{\char"0D}\texttt{A vers\~ao }\texttt{\char"0D}\hspace{-4mm} \texttt{\char"0D} acima n\~ao \'e avaliada. Experimente
\[\begin{array}{l}
 \texttt{\char"0D}\!\! \texttt{\char"0D} 3\ast 2 \texttt{\char"0D}\!\! \texttt{\char"0D}
 \\
Head[\%]\\
ToExpression[ \texttt{\char"0D}\!\! \texttt{\char"0D} 3\ast 2 \texttt{\char"0D}\!\! \texttt{\char"0D}]\\
 Head[\%]
\end{array}
\]

Consulte o \texttt{Help} sobre os comandos  \texttt{Names} e \texttt{Length}. 

% \item[g)]  Execute as seguintes opera\c{c}\~oes com n\'umeros racionais
% \[\begin{array}{l}
% 1/11+3/23\\
%  FullForm[1/11+3/23]\\
% N[1/11+3/23]\\
% N[1/11+3/23,20]
%\end{array}
%\]


\item[g)]  O \texttt{Mathematica} representa os n\'umeros na forma mais exata poss\'\i vel e conhece algumas regras usuais de transforma{c}\~ao de express\~oes alg\'ebricas e trigonom\'etricas. Experimente (linha a linha)
 \[\begin{array}{l}
 Sqrt[2]\\
 Sqrt[20]\\
 % Sqrt[2.0]\ast Sqrt[20]\\
%FullForm[Sqrt[2]]\\
% FullForm[Sqrt[2.0]]\\
\{ArcSin[1],ArcTan[1],ArcCot[0]\}\\
% ArcCot[0.0]\\
% ArcCot[0\char"012\!\char"012\! 50]\\
% \% - Pi/2
 \end{array}
\]


\item[h)] Deve ter reparado na al\'\i nea anterior que algumas constantes irracionais est\~ao predefinidas no sistema. Observe os resultados das seguintes instru\c{c}\~oes
 \[\begin{array}{l}
 % Pi/6\\
% Sin[\%]\\
 N[Pi,50]\\
\{Cos[Pi/5],Cos[Pi/6],Cos[Pi/7]\}
\\
% \{Sin[90\, Degree],Cos[135\, Degree],Tan[-70\, Degree]\}\\
\{Exp[1], Exp[-1], Exp[1.0]\}
 \end{array}
\]



\item[i)] Muitas das fun\c{c}\~oes predefinidas do \texttt{Mathematica} s\~ao \textsl{list\'aveis} (\texttt{Listable}), isto \'e, quando aplicadas a uma lista o resultado \'e uma lista que se obt\'em aplicando a fun\c{c}\~ao a cada elemento da lista. 
Execute os  comandos 
 \[\begin{array}{l}
 ?Listable 
 
 \\ 
 
 ??Cos 
 
 \\ 
 
 Cos[\{Pi/5,Pi/6,Pi/7\}] 
 
% \vspace{2mm} \\ 

% \displaystyle Tan\Big[\Big\{\frac{\pi}{4},\frac{\pi}{3},-178 Degree\Big\}\Big]


\vspace{2mm}
\\

Sqrt[\{2, 9, -1, (-3)\,\hat{}\, 3\}]

\vspace{1mm}
\\

Log[\{1, E,-1\}]

\vspace{1mm}
\\

Log[10, \{1, 10\,\hat{}\, 10, x\}]

\vspace{1mm}
\\

TrigExpand[\{Sin[2 x], Sin[x + y]
% , Cos[2 x], Tan[2 x], 
%  Sin[x]\,\hat{}\, 2 + Cos[x]\,\hat{}\, 2
\}]

 \end{array}
\]


\item[j)] Experimente  as instru\c{c}\~oes a seguir, e diga  se os resultados est\~ao de acordo com o que sabe sobre opera\c{c}\~oes com n\'umeros complexos.
 \[\begin{array}{l}



(1 + 3\ \textup{I})\ast (1 - 2\ \textup{I})

\vspace{1mm}
\\

%FullForm[\%]

% \vspace{1mm}
% \\


Arg[1 + \imath ]  \qquad ( prima \ \texttt{esc}, \ escreva \ {\rm ii} \ e \ prima \ \texttt{esc} \ para \ obter \ o \ s\acute{\imath}mbolo \ \imath)


\vspace{1mm}
\\

z1 = a + b\ \textup{I}; \, z2 = c + I \ d;

% \vspace{1mm}
% \\

% FullForm[z1]


%\vspace{1mm}
%\\





%\vspace{1mm}
%\\

%z1\ast z2

\vspace{1mm}
\\

z1^{\ast} z1 \qquad (prima \ \texttt{esc}, \ escreva  \  {\rm co} \ e \  prima \  \texttt{esc} \ para \ obter \ o \ s\acute{\imath}mbolo \ ^\ast \  do \ conjugado). 

\vspace{1mm}
\\

Column[ComplexExpand[\{Abs[z1],
%z1 + z2,
 z1\ast z2, 
% z1/z2, 
z1^{\ast} z1\}]]


% \vspace{1mm}
% \\


% FullSimplify[ComplexExpand[\{z1\,\hat{}\,n, z2\,\hat{}\,(1/n)\}]] // Column

 \end{array}
\]


Consulte o \texttt{Help} sobre os comandos  \texttt{ComplexExpand} e  \texttt{Conjugate}.
%  \texttt{Simplify} e  \texttt{FullSimplify}.

  \end{itemize}

  
  \section{Express\~oes simb\' olicas}
  
  
  \begin{itemize}
  
  
 % \item[a)] Execute os seguintes comandos sobre express\~oes alg\'ebricas 
  
 % \[\begin{array}{l}
%(x + 3)\ \widehat{}\ 2 + (x + 3)\  \widehat{}\ (-1)
  
%\vspace{1mm}
%\\
  
%  Expand[\%]
  
%    \vspace{1mm}
%\\
  
  
%  Factor[\%\%]
  
 % \vspace{1mm}
%\\
  
 % (5 (2 + x) (x + y) + (x + y)\ \widehat{} \ 2)\ \widehat{} \ 3
%  \vspace{1mm}
%\\
  
%  Factor[\%]
  
% \end{array}
%\]


\item[a)] Podemos atribuir valores a vari\'aveis. Experimente por exemplo
 \[\begin{array}{l}

 z = 7 (a + b)
 
    \vspace{1mm}
\\

pol= z\ \widehat{}\ 3 - 5 z\ \widehat{}\ 2 + 6 z 
 
     \vspace{1mm}
\\


pol /.\,a \rightarrow 1 

     \vspace{1mm}
\\

ReplaceAll[pol,\{a \rightarrow 1,b \rightarrow 2 \}]


  %   \vspace{1mm}
% \\


% pol /. z \rightarrow 1 

     \vspace{1mm}
\\


?/.

 \end{array}
\]
Para apagar a informa\c{c}\~ao retida em mem\'oria para as vari\'aveis $pol$ e $z$   fa\c{c}a
\[
 Clear[pol,z]
 \]
 
 

 
 \item[b)] 
 
 As chavetas \{ \ \} s\~ao usadas para especificar listas, vetores, conjuntos e matrizes. Uma lista consiste numa sucess\~ao de s\'\i mbolos separados por v\'\i rgulas e delimitada por chavetas.  Par\^enteses retos duplos s\~ao usados para indexa\c{c}\~ao.  Experimente
  \[\begin{array}{l}
a = \{1, 2, 3, 4, 5\}

     \vspace{1mm}
\\
 
 b = Table[i, \{i, 5\}]
 
 
      \vspace{1mm}
\\

 c = Range[5]
 
       \vspace{1mm}
\\
 
% \{Head[a], Head[b], Head[c]\}

%        \vspace{1mm}
% \\

%  \{a[[1]], b[[4]], c[[5]],a[[0]]\}

  %      \vspace{1mm}
%\\

% a[[6]]
%        \vspace{1mm}
%\\

Range[5, 10]
 
         \vspace{1mm}
\\

A = \{\{1, 2\}, \{2, 3\}\}

         \vspace{1mm}
\\

MatrixForm[A]

         \vspace{1mm}
\\

Table[i + j - 1, \{i, 2\}, \{j, 2\}] // MatrixForm

 \end{array}
\]
  



 \item[c)]   O comando \texttt{Apply}[\textsl{fun\c{c}\~ao,express\~ao}] substitui o cabe\c{c}alho da \textsl{express\~ao} pela \textsl{fun\c{c}\~ao}. Experimente
  \[\begin{array}{l}

c=Range[5];


        \vspace{1mm}
\\

Apply[Plus,c]

        \vspace{1mm}
\\

Apply[Times,c]

        \vspace{1mm}
\\

Sum[c[[i]], \{i, 5\}]

        \vspace{1mm}
\\

Factorial[5]

        \vspace{1mm}
\\

5!

 \end{array}
\]


 \item[d)]  Consulte o \texttt{Help} sobre o comando \texttt{Sum}. A seguir execute as instru\c{c}\~oes
 \[\begin{array}{l}

Sum[i, \{i, 1, n\}]


        \vspace{1mm}
\\

Sum[a\,\hat{}\,(n), \{n, m\}]

        \vspace{1mm}
\\

Sum[1/n\,\hat{}\,2, \{n, 1, Infinity \}]

        \vspace{1mm}
\\

Sum[1/n, \{n, 1,  \infty \}]

        \vspace{1mm}
\\



Sum[(-1)\,\hat{}\,(n + 1)/n, \{n, 1,\infty \}]





 \end{array}
\]


(O s\'\i mbolo $\infty$ obt\'em-se  escrevendo \texttt{esc} inf \texttt{esc}.)

% Experimente ainda o seguinte c\'odigo consultando o \texttt{Help} sobre comandos cujo significado desconhece
%  \[\begin{array}{l}
% Do[ Print[TraditionalForm[Subscript[s, j, n] == Sum[i\,\hat{}j, \{i, n\}]]], \{j, 4\}]
% \end{array}
% \]

\pagebreak 



   \end{itemize}
   
   
  \section{Gr\'aficos} 
  
    \begin{itemize}
  
  \item[a)] Para desenhar o gr\'afico da fun\c{c}\~ao $y = \cos x $ no intervalo $[0, \pi]$ escreva 
\[
  Plot[Cos[x],\{x,0,\pi\}]
  \]
Desenhe outras fun\c{c}\~oes \`a sua escolha.
  
    \item[b)]   O \texttt{Mathematica} gera gr\'aficos unindo \textsl{primitivas gr\'aficas} tais como \texttt{Line}, \texttt{Point}, \texttt{Circle}, \texttt{Disk}, \texttt{Polygon} e \texttt{JoinedCurve}. Execute (comando a comando) e observe o resultado dos seguintes comandos
 \[\begin{array}{l}
 

Graphics[Circle[\{0, 0\}, 1]]


         \vspace{1mm}
\\

Graphics[Circle[\{0, 0\}, \{2, 1\}], Axes \rightarrow True]


         \vspace{1mm}
\\

Graphics[JoinedCurve[Line[\{\{0, 0\}, \{0, 1\}, \{1, 0\}\}], CurveClosed \rightarrow True]]

         \vspace{1mm}
\\

% Graphics[ Polygon[\{\{0, 0\}, \{1, 0\}, \{1, 1\}, \{0, 1\}\}]]

 %        \vspace{1mm}
% \\

Graphics[\{Red, Polygon[\{\{0, 0\},  \{1, 1\}, \{0, 1\},\{1, 0\}\}]\}]

         \vspace{1mm}
\\

% \displaystyle Show[Graphics[\{Orange, Disk[\{0, 0\}, 1, \{0.25 \pi, 0.75 \pi\}]\}], 
% \\ \qquad \ \  \ Graphics[\{Orange, Disk[\{0, 0\}, 1, \{-0.25\pi, -0.75 \pi\}]\}], 
%  ImageSize \rightarrow Small]
 
 %         \vspace{1mm}
% \\
 
  
% Short[InputForm[Plot[Exp[x], \{x, 0, 1\}]], 4]



 
  \end{array}
\]


 
     \item[c)]  Os comandos \texttt{Plot}, \texttt{Graphics} e \texttt{Show} admitem v\'arias op\c{c}\~oes  (execute   
 \texttt{Options[Graphics]} ou \texttt{Options[Plot]}).  Observe o efeito das diferentes  op\c{c}\~oes  nas instru\c{c}\~oes seguintes 
   \[\begin{array}{l}
 Plot[Cos[x], \{x, -3 \pi, 3 \pi\}, Frame \rightarrow True, AspectRatio \rightarrow Automatic]
     
     
         \vspace{2mm}
\\

Plot[\{Cos[x], Sin[x]\}, \{x, -3 \pi, 3 \pi \}, 
 PlotStyle \rightarrow \{\{Red, Dashed\}, \{Purple, Thick\}\}]
 
          \vspace{2mm}
\\

ListPlot[Map[Cos, Range[-3 \pi, 3 \pi, \pi/6]], 
 PlotStyle \rightarrow PointSize[0.015], 
  \\ 
\qquad \qquad  \ DataRange \rightarrow \{-3 \pi, 3 \pi \}]
 
          \vspace{2mm}
\\

 
 RegionPlot[Abs[x]\ast Abs[y] < 4, \{x, -10, 10\}, \{y, -10, 10\}, 
 Frame \rightarrow False,    \\ 
\qquad \qquad \quad \
PlotStyle \rightarrow Red, 
 BoundaryStyle \rightarrow Thick]
 
         \vspace{2mm}
\\

% Plot[Cos[x], \{x, -3 \pi , 3 \pi\}, PlotRange \rightarrow \{\{-3.5 \pi, 3.5 \pi\}, \{-1.2, 1.2\}\}, 
%  \\ 
% \qquad \ \    PlotStyle \rightarrow \{RGBColor[0.9, 0.1, 0.5], AbsoluteThickness[3]\}, AxesLabel \rightarrow \{x, y\}, \\
% \qquad \ \ 
% Ticks \rightarrow \{Range[-3 \pi, 3 \pi, \pi/2], None\},  Background \rightarrow LightOrange]
 \end{array}
\]
%Experimente  com outros valores para as \textsl{diretivas gr\'aficas}   \texttt{PointSize}, \texttt{AbsoluteThickness} e  \texttt{RGBColor} da op\c{c}\~ao  \texttt{PlotStyle}. 
Consulte o \texttt{Help} sobre as op\c{c}\~oes \texttt{PlotStyle}, \texttt{AspectRatio} e \texttt{DataRange} e sobre os comandos \texttt{ListPlot} e \texttt{RegionPlot} .  
 
  %   \item[d)]  Um objecto gr\'afico em \texttt{Mathematica}  \'e uma express\~ao da forma
% \texttt{Graphics}[\textsl{lista}, \textsl{op\c{c}\~oes}].
% Os elementos da \textsl{lista} s\~ao diretivas ou  primitivas gr\'aficas.  Nas \textsl{op\c{c}\~oes} o utilizador define, atrav\'es de regras de substitui\c{c}\~ao (\texttt{Rule}), o modo como o gr\'afico \'e apresentado. Experimente
     
% \[\begin{array}{l}
% flag=Graphics[\{\{EdgeForm[Dashed], White, Rectangle[\{0, 0\}, \{5, 3\}]\},

%            \vspace{1mm}
% \\  
    
 %  \qquad \qquad   \qquad \qquad  \ \ 
% \{Blue,
%    Rectangle[\{0, 1\}, \{5, 2\}]\}, \{Blue, 
 %  Rectangle[\{1.5, 0\}, \{2.5, 3\}]\}\}, 
   
  %             \vspace{1mm}
% \\  
    
%   \qquad \qquad   \qquad \qquad \ \ 
   
%   ImageSize \rightarrow 300]        
   
 %           \vspace{2mm}
% \\
   
%   FullForm[flag]
   
 %              \vspace{2mm}
% \\
   
 %  TreeForm[flag, ImageSize \rightarrow 800]
   
   %               \vspace{2mm}
% \\     
        
   %     vert = \{\{0, 0\}, \{2, 0\}, \{1, 1\}\}; verthexa = Table[\{Sin[i\ast \pi/3], Cos[i\ast \pi/3]\}, \{i, 0, 5\}];
        
  %                   \vspace{2mm}
% \\          
        
        
    %    Graphics[\{\{Red, Opacity[0.5], PointSize[0.1], Point[vert]\},
 
      %         \vspace{1mm}
% \\  
    
  % \qquad \qquad \ \ 
%  \{Dashed, Thick, JoinedCurve[Line[vert], CurveClosed \rightarrow True]\}\},
  
   %              \vspace{1mm}
% \\  
    
%   \qquad \qquad  \ \  Background  \rightarrow LightGreen]
        
  %                     \vspace{2mm}
% \\          
        
        
  %      TableForm[
% Table[Graphics[\{GrayLevel[t], Polygon[verthexa]\}, 
%   ImageSize \rightarrow 100], 
   
 %                  \vspace{1mm}
% \\  
    
  % \qquad \qquad \qquad \qquad \ \ 
   
%   \{t, 0.1, 1, 0.2\}],
   
   %                   \vspace{1mm}
% \\  
    
 %  \qquad \qquad  \qquad 
   
 %   TableDirections \rightarrow Row]
%  \end{array}
% \]
  Identifique as diretivas, primitivas e op\c{c}\~oes que foram utilizadas nos comandos \texttt{Graphics} e explique o efeito de cada uma delas.
 

 
      \item[d)]   O comando \texttt{Manipulate}[\textsl{express\~ao},\{\textsl{u,umin,umax}\}] gera uma vers\~ao interativa da \textsl{express\~ao} que permita controlar o valor do \textsl{par\^ametro de controle} \textsl{u}$\,\in$[umin,umax]. Experimente
   
   
   \[\begin{array}{l}   
         Manipulate[Plot[Sin[x]-Cos[n\ast x], \{x, 0, \pi\},
  PlotStyle \rightarrow \{Dashed, Thick\},
  
                    \vspace{1mm}
\\  
  
\qquad \qquad  \qquad  \qquad PlotRange \rightarrow \{-1, 2\},
 TicksStyle  \rightarrow  Black,
  Filling  \rightarrow  Axis,
  
  
                      \vspace{1mm}
\\  
  
\qquad \qquad  \qquad \qquad 
FillingStyle \rightarrow  cor,
  PlotLabel  \rightarrow  StringJoin[ \texttt{\char"0D}\!\! \texttt{\char"0D}\texttt{n=} \texttt{\char"0D}\!\! \texttt{\char"0D}, ToString[n]],

                      \vspace{1mm}
\\  
  
\qquad \qquad  \qquad \qquad 

  LabelStyle  \rightarrow   Bold],
  
                        \vspace{1mm}
\\  
\qquad \qquad \qquad

 \{n, 1, 20, 1  \},
 \{cor, LightBlue\}]
 
 
   \end{array}
\]
% Consulte o \texttt{Help} sobre o comando \texttt{Tooltip}.

 
   %   \item[f)]   Consulte o \texttt{Help} sobre os comandos \texttt{Export, Import} 
 %     e \texttt{Flatten}. Execute a seguir as  instru\c{c}\~oes

   
   
%   \[\begin{array}{l}   

% t = Table[\{Sin[i + j], Cos[i + j]\}, \{i, -1, 1, 0.1\}, \{j, -1, 1, 0.5\}];

%                      \vspace{1mm}
% \\  

% Export[ NotebookDirectory[\,] <>\texttt{\char"0D}\!\! \texttt{\char"0D}data.xls \texttt{\char"0D}\!\! \texttt{\char"0D}, t]

%                      \vspace{1mm}
% \\  

% u = Import[ \texttt{\char"0D}\!\! \texttt{\char"0D}data.xls \texttt{\char"0D}\!\! \texttt{\char"0D}];

%                      \vspace{1mm}
% \\  

% Graphics[\{Blue, Map[Circle, Flatten[u, 1]]\}, Frame \rightarrow True]



 
 %  \end{array}
%\]

  \end{itemize}
  
\pagebreak 
  
  \section{Problemas}   
  \begin{itemize}
  
  \item[1.]       \begin{itemize}
  
  \item[a)]  Determine o valor das seguintes  somas 
  
  \[\begin{array}{l}
    (1+2+\ldots+n)^2\vspace{2mm}\\
    1^3+2^3+\ldots+n^3\vspace{2mm}\\
\displaystyle \frac{1}{1}+\frac{1}{1+2}+\ldots + \frac{1}{1+2+\ldots +n} 
\vspace{2mm}\\
\displaystyle  \sum_{i=1}^n \ln \Big( 1+\frac{1}{i}\Big)
\end{array}  \]

\underline{Resposta:}
\[
\frac{1}{4} \, n^2 (1 + n)^2\, , \qquad \frac{1}{4}\, n^2 (1 + n)^2, \qquad \frac{2 n}{1 + n}\, , \qquad  \texttt{Log}[1+n]\, .
\]


  \item[b)]   Estude a converg\^encia (entando calcular a sua soma) das seguintes s\'eries (infinitas)

  
  \[\begin{array}{l}
\displaystyle 1+ \frac{1}{1}+\frac{1}{2!}+\frac{1}{3!} + \ldots 
\vspace{2mm}\\

\displaystyle \frac{1}{1}-\frac{1}{1+2}+\ldots + \frac{(-1)^{n-1}}{1+2+\ldots +n} + \ldots 

\vspace{3mm}\\

\displaystyle 
1+\frac{1}{2}+\frac{1}{3}-\frac{1}{4}-\frac{1}{5}-\frac{1}{6}+\frac{1}{7}+\frac{1}{8}+\frac{1}{9} - - - + + + \ldots
\end{array}  \]

\underline{Resposta:} As s\'eries convergem e as somas s\~ao:
\[
e\approx 2.71828,\qquad 2\, (2\, \texttt{Log}[2]-1)\approx 0.772589,\qquad \frac{1}{9} \, (2\, \texttt{Sqrt}[3] \,\pi+ \texttt{Log} \,[8])\approx 1.44025
\]

\end{itemize}
  
 %   \item[2.]   
    
%      \begin{itemize}
  
%  \item[a)]  Utilize os pontos $\{Cos[i\pi/8], Sin[i\pi/8]\}/Sin[\pi/8], i=1,3,5,\ldots$, para construir uma lista de coordenadas  de v\'ertices de um oct\'ogono regular de lado $2$. 
  
  
 %  \item[b)]    Desenhe o oct\'ogono e no mesmo gr\'afico os c\'\i rculos circunscrito e inscrito.
   
   
%   \item[c)]   Determine as \'areas do oct\'ogono, do  c\'\i rculo circunscrito e do c\'\i rculo inscrito.
   
%   \underline{Resposta:}
%   \[
% 8\,\texttt{Cot}[\pi/8]\approx  19.3137, \qquad \pi\, \texttt{Csc}[\pi]/8]^2\approx 21.4521, \qquad \pi\, \texttt{Cot}[\pi/8]^2\approx 18.3105
%   \]
      
%   \item[d)]  Desenhe  um octagrama regular. (\underline{Sugest\~ao}:  O comando \texttt{Mod} pode ser \'util.)
   
   
%   \item[e)] Execute ainda as instru\c{c}\~oes
   
%   \[\begin{array}{l}
%     Animate[Graphics[\{Darker[Green],                         \vspace{1mm}
%\\  
%\qquad   \qquad Rotate[Polygon[Table[\{Cos[t], Sin[t]\}/Sin[\pi/8], \{t, \pi/8, 6 \pi, 6 \pi/8\}]], \theta, \{0, 0\}]\}, 
 %                        \vspace{1mm}
%\\  
%\qquad \qquad  \qquad  \qquad  PlotRange \rightarrow 3], \{\theta, 0,
%   2 \pi\}, AnimationRunning \rightarrow False]
%   \end{array}
 %  \]
 
  %  \end{itemize}
 
 
  \item[2.]   O objetivo deste exerc\'\i cio \'e aproximar numericamente o valor de $\pi$ de duas  formas diferentes sem utilizar fun\c{c}\~oes
   trigonom\'etricas predefinidas (ou o comando \texttt{Pi}). 
   
  %   \begin{itemize}
  
 % \item[a)] 
 Aproxime $\pi$ como a \'area do c\'\i rculo de raio $1$ das seguintes formas:
   
        \begin{itemize}
        
        
        \item[{i.}] Cobra o quadrado $[-1,1] \times [-1,1]$  por uma grelha uniforme de $N\times N$ pontos
e conte quantos pontos est\~ao dentro do c\'\i rculo (ver a Fig. 1). O comando \texttt{Count} poder\'a ser \'util. 

\bigskip
\bigskip

\begin{minipage}{\textwidth}
\centering 
 %\includegraphics[totalheight=7cm]{circle}\\
Figura 1. Grelha uniforme.
\end{minipage}

\pagebreak 
        \item[{ii.}]  (M\'etodo de Monte Carlo) Em vez de uma grelha uniforme, gere $N$ pontos aleat\'orios no quadrado $[-1,1] \times [-1,1]$ e relacione a percentagem deles dentro do c\'\i rculo com a valor de $\pi$ (ver a Fig. 2). O comando \texttt{RandomReal} poder\'a ser \'util.         
    
    
    \bigskip
\bigskip
    
        
\begin{minipage}{\textwidth}
\centering 
 %\includegraphics[totalheight=7cm]{MonteCarlo}\\
Figura 2. Pontos aleat\'orios.
\end{minipage}

 
 
        \end{itemize}
 
 
 %       \end{itemize}




\end{itemize}
  




 \end{document}
