\documentclass{article}

\usepackage{amsmath}
\usepackage{amssymb}
\usepackage{amsfonts,stmaryrd}
\usepackage{mathtools}

\usepackage[cal=euler]{mathalpha}

\usepackage[thmmarks, amsmath]{ntheorem}
%\usepackage{fullpage}
\usepackage{titling}
\setlength{\droptitle}{-3cm}
\pretitle{\noindent\large}
\posttitle{\par\smallskip}
\preauthor{\noindent}
\postauthor{\par}
\predate{\noindent}
\postdate{\par}

\usepackage{graphicx}


\usepackage{cancel}
\usepackage{interval}
\usepackage{comment}

\usepackage{enumitem}
\usepackage{multicol}

\setlist[enumerate,1]{label=(\alph*)}

\title{Math 27800 / CS 27800, Winter 2024: Problem Session 1}
\author{Duarte Maia}
\date{Tuesday, January 16th 2024}

\theorembodyfont{\upshape}
\theoremseparator{.}
\newtheorem{theorem}{Theorem}
\newtheorem{prop}{Prop}
\renewtheorem*{prop*}{Prop}
\newtheorem{lemma}{Lemma}

\newtheorem{ex}{Exercise}

\theoremstyle{nonumberplain}
\theoremheaderfont{\itshape}
\theorembodyfont{\upshape}
\theoremseparator{:}
\theoremsymbol{\ensuremath{\text{\textit{(End proof of lemma)}}}}
\newtheorem{lemmaproof}{Proof of Lemma}
\theoremsymbol{\ensuremath{\blacksquare}}
\newtheorem{proof}{Proof}
\theoremheaderfont{\bfseries}
\newtheorem{sol}{Solution}

\newcommand{\R}{\mathbb{R}}
\newcommand{\N}{\mathbb{N}}
\newcommand{\Q}{\mathbb{Q}}

\newcommand{\id}{\mathrm{id}}

\DeclarePairedDelimiter{\abs}{\lvert}{\rvert}
\DeclarePairedDelimiter{\norm}{\lvert}{\rvert}
\DeclarePairedDelimiter{\Norm}{\lVert}{\rVert}
\DeclarePairedDelimiter{\braket}{\langle}{\rangle}
\DeclarePairedDelimiter{\card}{\lvert}{\rvert}

\DeclareMathOperator{\powerset}{\mathcal{P}}

\newcommand{\Z}{\mathbf{Z}}
\newcommand{\ZF}{\mathbf{ZF}}
\newcommand{\ZFC}{\mathbf{ZFC}}

%%%% comment below for solution version
\excludecomment{sol}

\begin{document}
\maketitle

\begin{ex}
Prove that the following are well-defined operations in $\ZFC$. First, write down precisely what this means in each case.
%\begin{multicols}{1}
\begin{enumerate}
\item $A \cup B$,
\item $A \cap B$,
\item $A \setminus B$.
\end{enumerate}
%\end{multicols}
\end{ex}

\begin{sol}
We will take note of the axioms we are applying in the following solution. In all, we use extensionality to obtain that the resulting set is uniquely well-defined.

\begin{enumerate}
\item Given two sets $A$ and $B$, $A \cup B$ is the (unique by extensionality) set such that
\begin{equation}
\forall_x (x \in (A \cup B) \leftrightarrow (x \in A \lor x \in B)).
\end{equation}

It exists by applying the axiom of union to the set obtained by applying the axiom of pairing to $A$ and $B$. In other words, $A \cup B := \bigcup\{A,B\}$. Indeed, under this definition, an arbitrary $x$ is in $A \cup B$ iff $\exists_y (y \in \{A,B\} \land x \in y)$ iff $\exists_y ((y = A \lor y = B) \land x \in y)$, and first-order logic proves that the latter is equivalent to $x \in A \lor x \in B$.

\item Given $A$ and $B$, $A \cap B$ is the unique set such that
\begin{equation}
\forall_x (x \in (A \cap B) \leftrightarrow (x \in A \land x \in B)).
\end{equation}

It can be shown to exist by applying the axiom schema of comprehension to create the set $\{x \in A \mid \varphi(x,B) \}$ with $\varphi(x,B) \equiv x \in B$.

\item  Given $A$ and $B$, $A \setminus B$ is the unique set such that
\begin{equation}
\forall_x (x \in (A \setminus B) \leftrightarrow (x \in A \land x \not\in B)).
\end{equation}

It can be shown to exist by applying the axiom schema of comprehension to create the set $\{x \in A \mid \varphi(x,B) \}$ with $\varphi(x,B) \equiv x \not\in B$.
\end{enumerate}
\end{sol}

\begin{ex}
Recall Kuratowski's definition of ordered pair. Denote a pair as $\braket{x,y}$. Prove in $\ZFC$ that if $\braket{x,y} = \braket{x',y'}$ then $x = x'$ and $y = y'$. Take note of the axioms that you need to use to make this definition work.
\end{ex}

\begin{sol}
Kuratowski defines $\braket{x,y} = \{\{x\},\{x,y\}\}$.

Suppose that $\braket{x,y} = \braket{x',y'}$. We show $x = x'$ and $y = y'$, but first we make some observations about $\braket{x,y}$.

First, note that $x \in z$ for every $z \in \braket{x,y}$, and indeed the only set satisfying this property is $x$ itself. From this, we immediately conclude $x = x'$.

Second, note that $y \in z$ for \emph{some} $z \in \braket{x,y}$, and only $x$ and $y$ satisfy this property. Thus, we conclude $y = x'$ or $y = y'$. Likewise, $y' = x$ or $y' = y$. From this we conclude that $y = y'$: the only way for this not to directly be the case would have $y = x' = x = y'$ anyway.

The only axiom needed to make this definition work is the axiom of pairing, to ensure that the pair $\braket{x,y}$ exists. Otherwise, no axioms were used.
\end{sol}

\begin{ex}
John's professor erased the definition from the board too quickly for him to write it down, so he had to jot it from memory. Instead of Kuratowski's definition, he wrote down: $\braket{x,y} = \{x,y\}$. What is wrong with this definition?

Rose suffered a similar issue, but instead she wrote: $\braket{x,y} = \{x, \{x,y\}\}$. Is there anything wrong with this definition?

Dave missed the class entirely, and came up with the following definition on his own: $\braket{x,y} = \{\{0, x\}, \{1, y\}\}$. Is there anything wrong with this definition?

Finally, Jade tried to simplify Dave's definition, and defined $\braket{x,y} = \{x, \{y\}\}$. What is wrong with this definition?

Bonus question: Can you come up with any interesting alternate definitions of your own?
\end{ex}

\begin{sol}
John's definition fails to distinguish the pair $\braket{x,y}$ from $\braket{y,x}$.

\smallskip

Rose's definition works, but, unlike in Kuratowski's definition, we will require the axiom of foundation to do so. We prove now that Rose's definition works.

First, note that $x$ may be recovered from $\braket{x,y}_R$ as the $\in$-minimal element of $\braket{x,y}_R$ (there must be some by regularity, and $x$ is the only possibility). Then, we may also recover the set $\braket{x,y}$ as the non-$\in$-minimal element of $\braket{x,y}_R$, and either this is a singleton set, in which case $y = x$ is recovered, or it is a set with two distinct elements, in which case $y$ is the unique element which is not $x$.

This proof requires foundation in an essential way. To understand why, the reader will have to accept that, without the axiom of foundation, it is consistent that there exists a set $x$ such that $x = \{\{x,0\}, 1\}$. (The choice of $0$ and $1$ are irrelevant; any two distinct sets of the reader's choice would suffice.) Then, if $x' = \{x,0\}$, it is easy to check that $\braket{x, 0} = \braket{x', 1}$, so this is not a good definition of pair in this case!

\smallskip

Dave's definition works, but unlike in Kuratowski's definition, one requires enough axioms of $\ZFC$ to prove that $0$ and $1$ exist. It suffices to know that $0$ exists (a very reasonable demand...), as in this case $1$ exists by pairing, which we will need to use anyway.

To prove that Dave's definition works, we begin by proving a lemma.

\textbf{Lemma X.} If $\{a,b\} = \{a,c\}$ then $b=c$.

\textit{Proof of Lemma.} In this event, we know that $b \in \{a,c\}$, hence either $b=c$ (in which case we're done) or $b = a$. In the latter case, likewise, we have either $c=b$, and so we're done, or $c = a = b$, and so we're also done. \hfill $\square$

Now, suppose that $\{\{0,x\},\{1,y\}\} = \{\{0,x'\},\{1,y'\}\}$. If $\{0,x\} = \{0,x'\}$, then three applications of Lemma X immediately give us $x = x'$ and $y = y'$, so let us suppose that this is not the case.

Then, $\{0,x\} = \{1,y'\}$, and so we conclude $1 \in \{0,x\}$. Since $1 \neq 0$, we have $x = 1$. Likewise, $y' = 0$. Now, an application of Lemma X gives us that $\{1,y\} = \{0,x'\}$, and the same argument again will yield $x' = 1 = x$ and $y = 0 = y'$, and the proof that Dave's definition works is complete.

\smallskip

Jade's definition fails to distinguish $\braket{\{x\}, y}$ from $\braket{\{y\}, x}$.
\end{sol}

\begin{ex}
Given two sets $A$, $B$, define the cartesian product $A \times B$ and prove in $\ZFC$ that it exists.
\end{ex}

\begin{sol}
The cartesian product $A \times B$ is the (unique by extensionality) set satisfying the condition
\begin{equation}
\forall_x (x \in (A \times B) \leftrightarrow \exists_a \exists_b (a \in A \land b \in B \land \braket{a,b} = x)).
\end{equation}

To prove that this set exists, we apply the pairing and union axioms to take $A \cup B$, then apply the power set axiom twice, finally followed by the comprehension axiom. Indeed, we construct by comprehension:
\begin{equation}
A \times B = \{\,x \in \powerset(\powerset(A \cup B)) \mid \exists_a \exists_b (a \in A \land b \in B \land \braket{a,b} = x)\,\}.
\end{equation}

The only thing that we need to show is that every pair is in $A \times B$. This is just a matter of noticing that, for $a \in A$ and $b \in B$, $\braket{a,b} = \{\{a\},\{a,b\}\}$ is in $\powerset(\powerset(A \cup B))$, and thus:
\begin{equation}
\begin{multlined}
x \in \powerset(\powerset(A \cup B)) \land \exists_a \exists_b (a \in A \land b \in B \land \braket{a,b} = x) \iff \\
\iff \exists_a \exists_b (a \in A \land b \in B \land \braket{a,b} = x).
\end{multlined}
\end{equation}
\end{sol}


\end{document}