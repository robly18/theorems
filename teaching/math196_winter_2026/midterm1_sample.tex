\documentclass[addpoints]{exam}

\usepackage{amsmath}
\usepackage{amssymb}
\usepackage{amsfonts}
\usepackage{mathtools}
\usepackage{tikz}

\DeclarePairedDelimiter{\Norm}\lVert\rVert
\newcommand{\R}{\mathbb{R}}
\newcommand{\tr}{\intercal}

%EDIT THESE WHEN NEW CLASS
\newcommand{\themath}{19620}
\newcommand{\thesec}{46}

\pagestyle{headandfoot}
\runningheader{MATH \themath\ Section \thesec}{Midterm 1 Sample}{Duarte Maia}
\runningheadrule
%\footer{}{\ifincomplete{Exercise continues on next page.}{}}{}


\begin{document}

\begin{center}
\Large Quiz 1 -- Sample Version

MATH \themath - Section \thesec

\smallskip

\large Section Instructor: Duarte Maia

\smallskip

No date
\end{center}

\noindent
\makebox[\textwidth]{
Name and UCID: \,\hrulefill
}

\begin{center}
\fbox{
\parbox{6in}{
\begin{itemize}
\item You have 45 minutes to answer the following question in the space given below.

\item You may ask for additional paper if necessary.

\item Your solution should use methods similar to those you used on the project.

\item You do not need to explain or justify your solution. However, you do need to \textbf{show your work}. Someone who knows how to solve this exercise should be able to understand how you reached your solution. An incomplete solution with full work shown is better than a correct solution with no work shown.
\end{itemize}
}
}
\end{center}

\hrule

\qformat{\textbf{Exercise:} \quad}

\begin{questions}
\question Consider the following linear equation with real parameters $p$ and $q$:
\[
\left\{
\begin{array}{rrrrrrrrr}
&& p^2 y &+& p z &+& (1+p^3)w &=& q\\
p x &+& && (1-p)z && &=& 1 \\
x &+& p y &+& &&p w &=& q\\
&& (1+p^2)y &+& (p-1)z &+& 2pw &=& 0
\end{array}
\right.
\]

Some values of the parameters lead to this equation having zero, one, or infinitely many solutions. Provide a full description of what parameters lead to what number of solutions.

\bigskip

\textit{See answer in next page.}
\newpage
\printanswers
\begin{solution}\ 


\ 
\begin{tikzpicture}[scale=3]
\node (base) at (0,0) {};

\node (pn1) at (-1,-1) {};
\node (p1) at (1,-1) {exactly 1 solution};

\draw[->] (base) -- (p1) node[midway, fill=white]{$p=1$};
\draw[->] (base) -- (pn1) node[midway, fill=white]{$p\neq 1$};

\node (qn1) at (-2,-2) {no solutions};
\node (q1) at (0,-2) {infinitely many solutions};

\draw[->] (pn1) -- (qn1) node[midway,fill=white]{$q\neq 1$};
\draw[->] (pn1) -- (q1) node[midway, fill=white]{$q=1$};
\end{tikzpicture}
\end{solution}
\end{questions}
\end{document}