\documentclass[addpoints]{exam}

\usepackage{amsmath}
\usepackage{amssymb}
\usepackage{amsfonts}
\usepackage{mathtools}
\usepackage{tikz}

\DeclarePairedDelimiter{\Norm}\lVert\rVert
\newcommand{\R}{\mathbb{R}}
\newcommand{\tr}{\intercal}

%EDIT THESE WHEN NEW CLASS
\newcommand{\themath}{19620}
\newcommand{\thesec}{46}

\pagestyle{headandfoot}
\runningheader{MATH \themath\ Section \thesec}{Midterm 1 Sample}{Duarte Maia}
\runningheadrule
%\footer{}{\ifincomplete{Exercise continues on next page.}{}}{}


\begin{document}

\begin{center}
\Large Quiz 2

MATH \themath - Section \thesec

\smallskip

\large Section Instructor: Duarte Maia

\smallskip

February 19, 2026
\end{center}

\noindent
\makebox[\textwidth]{
Name and UCID: \,\hrulefill
}

\begin{center}
\fbox{
\parbox{6in}{
\begin{itemize}
\item You have 45 minutes to answer the following question in the space given below.

\item You may ask for additional paper if necessary.

\item Your solution should use methods similar to those you used on the project.

\item You do not need to explain or justify your solution. However, you do need to \textbf{show your work}. Someone who knows how to solve this exercise should be able to understand how you reached your solution. An incomplete solution with full work shown is better than a correct solution with no work shown.
\end{itemize}
}
}
\end{center}

\hrule

\qformat{\textbf{Exercise:} \quad}

\begin{questions}
\question There is exactly one choice of numbers $a$, $b$, and $c$ which make the following linear system of equations have exactly one solution. Determine these numbers, and find the solution in this scenario. Do not row-reduce the system.
\[
\left\{
\begin{array}{rrrrrrrr}
-6 x & + & 7 y & + & 4 z & = & 5 \\
-34 x & + & 39 y & + & 23 z & = & 28 \\
15 x & - & 17 y & - & 10 z & = & -12 \\
46 x & - & 53 y & - & 31z & = & a \\
11 x & - & 14 y & - & 8z & = & b \\
-18 x & + & 20 y & + & 12z & = & c \\
\end{array}
\right.
\]
The following matrix identities may be useful. You don't need to use both of them, they are both here to accommodate different ways of solving this problem.
\[
\begin{array}{ccc}
\begin{bmatrix}
-6 & 7 & 4 \\
-34 & 39 & 23 \\
15 & -17 & -10
\end{bmatrix}^{-1}
=
\begin{bmatrix}
1 & 2 & 5 \\
5 & 0 & 2 \\
-7 & 3 & 4
\end{bmatrix}
&
\begin{bmatrix}
 -6 & 7 & 4 & 0 & 0 & 0 \\
 -34 & 39 & 23 & 0 & 0 & 0 \\
 15 & -17 & -10 & 0 & 0 & 0 \\
 46 & -53 & -31 & 1 & 0 & 0 \\
 11 & -14 & -8 & 0 & 1 & 0 \\
 -18 & 20 & 12 & 0 & 0 & 1 \\
\end{bmatrix}^{-1}
=
\begin{bmatrix}
 1 & 2 & 5 & 0 & 0 & 0 \\
 5 & 0 & 2 & 0 & 0 & 0 \\
 -7 & 3 & 4 & 0 & 0 & 0 \\
 2 & 1 & 0 & 1 & 0 & 0 \\
 3 & 2 & 5 & 0 & 1 & 0 \\
 2 & 0 & 2 & 0 & 0 & 1 \\
\end{bmatrix}
\end{array}
\]
{\footnotesize If you solved the project in a way that I didn't account for, and don't think I gave you enough matrix identities to solve this problem by hand, explain how you'd solve this problem using the same method you used to solve the project. You'll be graded on the correctness of your explanation and on whether it matches your reasoning on the project.}
\end{questions}
\end{document}