\documentclass[11pt]{amsart}

\usepackage{amsmath}
\usepackage{amsfonts}
\usepackage{amsthm}

\usepackage{showlabels}

\usepackage{lipsum}

\newcommand{\R}{\mathbb{R}}
\newcommand{\Q}{\mathbb{Q}}
\newcommand{\Z}{\mathbb{Z}}
\newcommand{\N}{\mathbb{N}}
\newcommand{\HH}{\mathcal{H}}
\newcommand{\BB}{\mathcal{B}}

\newtheorem{lemma}{Lemma}
\newtheorem{definition}{Definition}

\title{\textbf{On optimal bounds for comparing Binary (and $n$-ary) Nest Measures with the Hausdorff Measure on $\R$}}


\author{Duarte Maia}
\address{Department of Mathematics, Instituto Superior Técnico}
\email{duarte.nascimento@tecnico.ulisboa.pt}

\author{Jorge Drumond Silva}
\address{Department of Mathematics, Instituto Superior Técnico}
\email{jsilva@math.tecnico.ulisboa.pt}

\date{}


\begin{document}


\begin{abstract}
\lipsum[1]
\end{abstract}

\maketitle


\section{Introduction}

\lipsum[1]

\section{Hausdorff and Binary measures: a review}

\subsection{Hausdorff}

Let $X$ be a metric space, with distance $d$. We define the \emph{$s$-dimensional Hausdorff measure on $X$} as follows:

First, define, for $E \subseteq X$ and $\delta > 0$, $\HH_s^\delta(E) := \inf \sum_i \lvert U_i \rvert^s$, where:

\begin{itemize}

\item $\lvert E \rvert$ denotes the \emph{diameter of $E$}, defined by the $\sup$ of distances between elements of $E$;

\item The infimum is taken over countable coverings $\{U_i\}_{i \in \N}$ of $E$ such that the diameter of all $U_i$ is less than $\delta$. This mouthful is usually abbreviated to \emph{$\delta$-coverings}.

\end{itemize}

It is easy to see that, as $\delta$ decreases, this quantity increases. Hence, the limit $\lim_{\delta \to 0} \HH_s^\delta(E)$ exists in $\left[ 0, \infty \right]$. It is this quantity that we denote by the \emph{$s$-dimensional Hausdorff measure of $E$}:

\begin{definition}(Hausdorff Measure)

 \[\HH_s(E) := \lim_{\delta \to 0} \HH_s^\delta(E)\]
\end{definition}

\subsection{Dimension}

Fixed any set $E$ in a metric space $X$, an interesting phenomenon occurs: there exists a dimension $\sigma \in \left[0, \infty\right]$ such that:

\begin{itemize}
\item For $s < \sigma$, $\HH_s(E) = \infty$;

\item For $s > \sigma$, $\HH_s(E) = 0$;
\end{itemize}

To see that this is so, the following lemma is key:

\begin{lemma}\label{helperdimension}
Let $s < t$ be real nonnegative numbers. Then, if $\HH_t(E) > 0$ we have $\HH_s(E) = \infty$.
\end{lemma}

\begin{proof}
todo
\end{proof}

With this in mind, define, for a set $E$, its dimension

\begin{definition} (Hausdorff dimension of a set)

\[\sigma_E := \sup \{\, s \mid \HH_s(E) > 0 \,\}\]
\end{definition}

We can now show:

\begin{lemma}
(Well-definedness of the dimension of a set) Let $E \subseteq X$, where $X$ is a metric space.

\begin{itemize}
\item For $s < \sigma_E$, $\HH_s(E) = \infty$;

\item For $s > \sigma_E$, $\HH_s(E) = 0$;
\end{itemize}

\end{lemma}

\begin{proof}
The second inequality is trivial by definition of $\sigma_E$.

As for the first inequality, notice that if $s < \sigma_E$ then there exists $t > s$ such that $\HH_t(E) > 0$. By lemma \ref{helperdimension}, we have $\HH_s(E) = \infty$, as desired.
\end{proof}

This shows that the dimension of a set is the only dimension for which its Hausdorff measure may be nontrivial, that is, different from $0$ or $\infty$. It should be noted, however, that there is no requirement that it is so.

\begin{definition}
We will say $E \subseteq X$ is an \emph{$s$-set} if $\HH_s(E) \in \left]0, \infty \right[$.

It is obvious that an $s$-set has dimension $s$.
\end{definition}

\subsection{Net measures}

Hausdorff measure, and by extension Hausdorff dimension, is sometimes rather unwieldy to calculate, seeing as its definition requires one to look at all possible $\delta$-coverings. As such, there is interest in simplifying this process.

We will restrict our attention to $\R^n$, and later on to $\R$, as unfortunately a lot of what follows is not generalizable to other contexts.

First, notice that it is not without precedent for one to restrict their attention to specific coverings. Indeed, the reader may easily show that, if in the definition of $\HH_s^\delta$ one takes the infimum over \emph{open} $\delta$-coverings instead of arbitrary ones, the value of $\HH_s^\delta$ may change, but the limit as $\delta \to 0$ remains the same. Hence, one could consider the Hausdorff measure defined using open $\delta$-coverings without losing anything.

The notion of net measure arises from the idea of redefining the Hausdorff measure with a more manageable collection of sets to use for coverings. In the context of $\R^n$, Besicovitch (in 1952) used the so-called binary intervals, construction which we will make for the case of $n = 1$, but the reader can satisfy any curiosity they might have regarding the general case in \cite{falconer} \cite{rogers}.

\begin{definition}
Define the so-called \emph{binary intervals} in $\R$ as the set of intervals of the form $\left[ \frac k {2^j}, \frac {k+1} {2^j} \right[$, for $k, j \in \Z$.

The binary measure is defined analogously to the Hausdorff measure, except instead of arbitrary coverings one restricts themselves to $\delta$-coverings by binary intervals. It is denoted by $\BB_s(E) = \lim_{\delta \to 0} \BB_s^\delta(E)$.
\end{definition}

The usefulness of this definition lies in that, even though these measures do \emph{not} coincide in general, one can find bounds for one given the other.


\begin{thebibliography}{1}
\bibitem{falconer}
K. J. Falconer, \textit{The Geometry of Fractal Sets}

\bibitem{rogers}
C. A. Rogers, \textit{Hausdorff Measures}

\end{thebibliography}


\end{document}
