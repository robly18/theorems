\documentclass[11pt]{amsart}

\usepackage{amsmath}
\usepackage{amsfonts}
\usepackage{amsthm}

\usepackage{showlabels}

\usepackage{lipsum}

\newcommand{\R}{\mathbb{R}}
\newcommand{\N}{\mathbb{N}}
\newcommand{\HH}{\mathcal{H}}

\newtheorem{lemma}{Lemma}

\title{\textbf{On optimal bounds for comparing Binary (and $n$-ary) Nest Measures with the Hausdorff Measure on $\R$}}


\author{Duarte Maia}
\address{Department of Mathematics, Instituto Superior Técnico}
\email{duarte.nascimento@tecnico.ulisboa.pt}

\author{Jorge Drumond Silva}
\address{Department of Mathematics, Instituto Superior Técnico}
\email{jsilva@math.tecnico.ulisboa.pt}

\date{}


\begin{document}


\begin{abstract}
\lipsum[1]
\end{abstract}

\maketitle


\section{Introduction}

\lipsum[1]

\section{Hausdorff and Binary measures: a review}

\paragraph{Hausdorff}

Let $X$ be a metric space, with distance $d$. We define the \emph{$s$-dimensional Hausdorff measure on $X$} as follows:

First, define, for $E \subseteq X$ and $\delta > 0$, $\HH_s^\delta(E) := \inf \sum_i \lvert U_i \rvert^s$, where:

\begin{itemize}

\item $\lvert E \rvert$ denotes the \emph{diameter of $E$}, defined by the $\sup$ of distances between elements of $E$;

\item The infimum is taken over countable coverings $\{U_i\}_{i \in \N}$ of $E$ such that the diameter of all $U_i$ is less than $\delta$.

\end{itemize}

It is easy to see that, as $\delta$ decreases, this quantity increases. Hence, the limit $\lim_{\delta \to 0} \HH_s^\delta(E)$ exists in $\left[ 0, \infty \right]$. It is this quantity that we denote by $\HH_s(E)$, the \emph{$s$-dimensional Hausdorff measure of $E$}.

\paragraph{Dimension}

Fixed any set $E$ in a metric space $X$, an interesting phenomenon occurs: there exists a dimension $\sigma \in \left[0, \infty\right]$ such that:

\begin{itemize}
\item For $s < \sigma$, $\HH_s(E) = \infty$;

\item For $s > \sigma$, $\HH_s(E) = 0$;
\end{itemize}

To see that this is so, the following lemma is key:

\begin{lemma}\label{helperdimension}
Let $s < t$ be real nonnegative numbers. Then, if $\HH_t(E) > 0$ we have $\HH_s(E) = \infty$.
\end{lemma}

\begin{proof}
todo
\end{proof}

With this in mind, define, for a set $E$, its dimension

\[\sigma_E := \sup \{\, s \mid \HH_s(E) > 0 \,\}\]

We can now show:

\begin{lemma}
(Well-definedness of the dimension of a set) Let $E \subseteq X$, where $X$ is a metric space.

\begin{itemize}
\item For $s < \sigma_E$, $\HH_s(E) = \infty$;

\item For $s > \sigma_E$, $\HH_s(E) = 0$;
\end{itemize}

\end{lemma}

\begin{proof}
The second inequality is trivial by definition of $\sigma_E$.

As for the first inequality, notice that if $s < \sigma_E$ then there exists $t > s$ such that $\HH_t(E) > 0$. By lemma \ref{helperdimension}, we have $\HH_s(E) = \infty$, as desired.
\end{proof}


\begin{thebibliography}{1}
\bibitem{falconer}
K. J. Falconer, \textit{The Geometry of Fractal Sets}

\end{thebibliography}


\end{document}
