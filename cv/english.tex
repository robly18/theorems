%%%%%%%%%%%%%%%%%%%%%%%%%%%%%%%%%%%%%%%%%
% Compact Academic CV
% LaTeX Template
% Version 2.0 (6/7/2019)
%
% This template originates from:
% https://www.LaTeXTemplates.com
%
% Authors:
% Dario Taraborelli (http://nitens.org/taraborelli/home)
% Vel (vel@LaTeXTemplates.com)
%
% License:
% CC BY-NC-SA 3.0 (http://creativecommons.org/licenses/by-nc-sa/3.0/)
%
%%%%%%%%%%%%%%%%%%%%%%%%%%%%%%%%%%%%%%%%%

%----------------------------------------------------------------------------------------
%	PACKAGES AND OTHER DOCUMENT CONFIGURATIONS
%----------------------------------------------------------------------------------------

\documentclass[11pt]{article} % Default document font size
\usepackage{cite}


\input{structure2.tex} % Include the file specifying the document structure and styling

% Set PDF meta-information
\hypersetup{
	pdftitle={Duarte Maia - CV Eng},
	pdfauthor={Duarte Maia}
}

%----------------------------------------------------------------------------------------

\begin{document}

%----------------------------------------------------------------------------------------
%	CONTACT AND GENERAL INFORMATION
%----------------------------------------------------------------------------------------

{\LARGE\bfseries Curriculum Vitae} % Name
\bigskip % Whitespace

Name: Duarte Luis Maia Nascimento
\smallskip

Email: \href{mailto:duarte.nascimento@tecnico.ulisboa.pt}{duarte.nascimento@tecnico.ulisboa.pt} % Email address
\smallskip

Webpage: \url{http://web.tecnico.ulisboa.pt/~ist189623/en.php}

\smallskip

Phone Number: +351 965 323 339 % Phone number
\smallskip


Address:\\
R. José Afonso, nº 9, 1º esq.\\ % Address
Amadora - 2720-316
\smallskip

Birth Date: 28/03/1999 % Date of birth

%------------------------------------------------

%------------------------------------------------

\section*{Education}

\entry 2017-2020 BSc in Applied Mathematics and Computation
\smallskip

Instituto Superior Técnico, Universidade de Lisboa

Final Grade Average: 19 out of 20

\medskip

\entry 2020-2022 (Expected) MSc in Mathematics and Applications
\smallskip

Instituto Superior Técnico, Universidade de Lisboa

Current Grade Average: 18.3 out of 20

\section*{MSc Thesis}

\years{2021-Ongoing} 
Supervised by professor Miguel Abreu, from Department of Mathematics of Instituto Superior Técnico, in Symplectic Geometry and Persistence Homology\\
Summary: This work is based on recently-released notes by Polterovich et al. \cite{polterovich}, in which a recent tool called Persistence Homology is applied to Metric Geometry, Function Theory, and Symplectic Geometry. I am currently in the process of learning about the applications of this tool to Symplectic Geometry, with the goal of making an original contribution which will serve as my MSc Thesis.

\section*{Research Initiation Scholarships}

\years{2017/2018} Novos Talentos em Matemática (Research Initiation Scholarship)\\
Supervised by professor João Rasga, from Department of Mathematics of Instituto Superior Técnico, in Logic\\
\textit{SAT, Multigraphs, and CNF}\\
Summary: A standard algorithm to solve 2-SAT starts with the construction of a directed graph from a 2-CNF formula. A proposition relating cycles in this graph to the satisfiability of the formula allows one to reduce the problem of 2-SAT to a problem of graphs. In this work, we generalized this proposition to general CNF formulae, relating their satisfiability to the existence of (a generalization of) cycles in a certain directed hypergraph, in which edges could have multiple arrowheads.
\medskip

\years{2018/2019} Novos Talentos em Matemática (Research Initiation Scholarship)\\
Supervised by professor Jorge Drumond Silva, from Department of Mathematics of Instituto Superior Técnico, in Geometric Measure Theory\\
\textit{On Optimal Bounds for Compring Dyadic (and $b$-ary) Net Measures with the Hausdorff Measure on $\mathbf{R}$}\\
(In preparation for publication, preprint available on my webpage)\\
Summary: The Hausdorff measure is a generalization of the Lebesgue measure to non-integer dimensions. A useful tool for its study are the so-called dyadic measures, which allow one to estimate the Hausdorff measure and hence calculate the Hausdorff dimension of sets. In this work, we investigated how good of an estimate the dyadic measure is, and related the dyadic measure (and a few of its cousins, which we called $b$-ary measures) to the theory of non-integer base digit expansions.
\medskip

\years{2020-Ongoing} CAMGSD Research Scholarship\\
Supervised by professor Pedro Resende, from Department of Mathematics of Instituto Superior Técnico, in Quantum Mechanics and Pointless Topology\\
Summary: This work is based on a recent paper \cite{measurement} by Pedro Resende, in which he proposes an axiomatization for the notion of measurement in Quantum Mechanics. This project is still ongoing, but some progress has been made. In another paper \cite{prjps}, Pedro Resende and João Paulo Santos (from the same department) show that, if $A$ is a locally convex vector space, the space $\mathop{\mathrm{Max}} A$ is a sober topological space. This is important in order to construct examples of so-called Measurement Spaces, but it is a difficult result. In an attempt to simplify the proof, I have found an elementary proof for the case where $A$ is finite-dimensional, using simple tools from linear algebra.

\section*{Olympiads}

\years{2015} \entry Bronze medal at the Portuguese Mathematical Olympiad
\smallskip

\entry Silver medal at the Mathematical Olympiad for the Community of Portuguese Speaking Countries
\smallskip

\years{2016} \entry Gold medal at the Portuguese Mathematical Olympiad
\smallskip

\entry Third place at the National Informatics Olympiad
\smallskip

\entry Gold medal at the Regional phase of the National Physics Olympiad
\smallskip

\entry Silver medal at the National phase of the National Physics Olympiad
\smallskip

\entry Honorable mention at the International Mathematical Olympiad
\smallskip

\years{2017} \entry Participation at the International Mathematical Olympiad
\smallskip

\entry Bronze medal at the International Olympiad in Informatics
\smallskip

\entry Gold medal in the Iberoamerican Physics Olympiad
\smallskip

\entry Second place at the National Informatics Olympiad
\smallskip

\years{2018-Ongoing} \entry Occasional participation in university mathematics and informatics olympiads

\section*{Other Activities}

\years{2018} \entry Guest speaker at the Diagonal Seminar. (A group of IST students which organizes seminars by and for undergrad students.)

\years{2018-Ongoing} \entry Member of the Diagonal Seminar team.

\years{2018} \entry Wrote a collection of notes for a class on Linear Optimization.

\years{2018-Ongoing} \entry Instructor, author and coordinator in several math-related activites in the project TreeTree2, whose goal is to complement portuguese students' studies from grades 1 to 12, by offering interested students the opportunity to study STEM subjects in their free time.

\pagebreak

\years{2019/2020} \entry Wrote a collection of notes for a class on Logic. This was done as a project for a class, and it was graded 20/20.

\years{2020} \entry Organizer and lecturer in a study group on Measure Theory

\years{2021} \entry Organizer of a study group on Topology



\bibliographystyle{plain}
\bibliography{bibliography}

\end{document}
