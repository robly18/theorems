%%%%%%%%%%%%%%%%%%%%%%%%%%%%%%%%%%%%%%%%%
% Compact Academic CV
% LaTeX Template
% Version 2.0 (6/7/2019)
%
% This template originates from:
% https://www.LaTeXTemplates.com
%
% Authors:
% Dario Taraborelli (http://nitens.org/taraborelli/home)
% Vel (vel@LaTeXTemplates.com)
%
% License:
% CC BY-NC-SA 3.0 (http://creativecommons.org/licenses/by-nc-sa/3.0/)
%
%%%%%%%%%%%%%%%%%%%%%%%%%%%%%%%%%%%%%%%%%

%----------------------------------------------------------------------------------------
%	PACKAGES AND OTHER DOCUMENT CONFIGURATIONS
%----------------------------------------------------------------------------------------

\documentclass[11pt]{article} % Default document font size
\usepackage{cite}


\input{structure2.tex} % Include the file specifying the document structure and styling

% Set PDF meta-information
\hypersetup{
	pdftitle={Duarte Maia - CV Eng},
	pdfauthor={Duarte Maia}
}

%----------------------------------------------------------------------------------------

\begin{document}

%----------------------------------------------------------------------------------------
%	CONTACT AND GENERAL INFORMATION
%----------------------------------------------------------------------------------------

{\LARGE\bfseries Curriculum Vitae} % Name
\bigskip % Whitespace

Name: Duarte Luis Maia Nascimento (Duarte Maia)
\smallskip

Email: \href{mailto:dmaia@uchicago.edu}{dmaia@uchicago.edu} % Email address
\smallskip

Webpage: \url{http://math.uchicago.edu/~dmaia/}

\smallskip

Phone Number: +1 (312) 686-5246 % Phone number
\smallskip


Address:\\
R. José Afonso, nº 9, 1º esq.\\ % Address
Amadora -- 2720-316
Lisbon, Portugal
\smallskip

Birth Date: 28/03/1999 % Date of birth

%------------------------------------------------

%------------------------------------------------

\section*{Education}

\entry 2017-2020 BSc in Applied Mathematics and Computation
\smallskip

Instituto Superior Técnico, Universidade de Lisboa

\medskip

\entry 2020-2022 MSc in Mathematics and Applications
\smallskip

Instituto Superior Técnico, Universidade de Lisboa

\medskip

\entry 2022-2027 (Expected) PhD in Mathematics
\smallskip

University of Chicago

Advisors: Denis Hirschfeldt and Maryanthe Malliaris

\section*{Graduate Studies}

\years{2024-2025} \entry Organizer of the Graduate Student Logic and Combinatorics Seminar at UChicago

\years{2024} \entry Passed the Topic Examination\\
Supervised by Leonardo Coregliano, Denis Hirschfeldt, and Maryanthe Malliaris

\years{2023-2024} \entry Organizer of the `Pizza Seminar' at the UChicago math department

\years{2023} \entry Completed the first year mathematics graduate courses at UChicago

\section*{Teaching}

\years{2025} \entry Recipient of the 2024–2025 Physical Sciences Division Graduate Teaching Prize at UChicago

\years{2024-2025} \entry Graduate student lecturer at UChicago

\years{2024} \entry Completed the College Fellow training program at UChicago

\years{2022} \entry Teaching Assistant for ``Experimental Mathematics'' at Instituto Superior Técnico

\section*{MSc Thesis}

\years{2021-2022} 
Supervised by professor Miguel Abreu, from Department of Mathematics of Instituto Superior Técnico, in Symplectic Geometry and Persistence Homology

\section*{Research Initiation Scholarships}

\years{2017/2018} Novos Talentos em Matemática (Research Initiation Scholarship)\\
Supervised by professor João Rasga, from Department of Mathematics of Instituto Superior Técnico, in Logic\\
\textit{SAT, Multigraphs, and CNF}

\medskip

\years{2018/2019} Novos Talentos em Matemática (Research Initiation Scholarship)\\
Supervised by professor Jorge Drumond Silva, from Department of Mathematics of Instituto Superior Técnico, in Geometric Measure Theory\\
\textit{On Optimal Bounds for Compring Dyadic (and $b$-ary) Net Measures with the Hausdorff Measure on $\mathbf{R}$}
\medskip

\years{2020-2022} CAMGSD Research Scholarship\\
Supervised by professor Pedro Resende, from Department of Mathematics of Instituto Superior Técnico, in Quantum Mechanics and Pointless Topology

\section*{Olympiads}

\years{2015} \entry Bronze medal at the Portuguese Mathematical Olympiad
\smallskip

\entry Silver medal at the Mathematical Olympiad for the Community of Portuguese Speaking Countries
\smallskip

\years{2016} \entry Gold medal at the Portuguese Mathematical Olympiad
\smallskip

\entry Third place at the National Informatics Olympiad
\smallskip

\entry Gold medal at the Regional phase of the National Physics Olympiad
\smallskip

\entry Silver medal at the National phase of the National Physics Olympiad
\smallskip

\entry Honorable mention at the International Mathematical Olympiad
\smallskip

\years{2017} \entry Participation at the International Mathematical Olympiad
\smallskip

\entry Bronze medal at the International Olympiad in Informatics
\smallskip

\entry Gold medal in the Iberoamerican Physics Olympiad
\smallskip

\entry Second place at the National Informatics Olympiad
\smallskip

\years{2018-2022} \entry Occasional participation in university mathematics and informatics olympiads

\section*{Other Activities}

\years{2018-2022} \entry Member of the Diagonal Seminar team. (A group of IST students which organizes seminars by and for undergrad students.)

\years{2018-2022} \entry Instructor, author and coordinator in several math-related activites in the project TreeTree2, whose goal is to complement portuguese students' studies from grades 1 to 12, by offering interested students the opportunity to study STEM subjects in their free time.




%\bibliographystyle{plain}
%\bibliography{bibliography}

\end{document}
