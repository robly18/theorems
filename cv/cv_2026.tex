
\documentclass[
    9pt,
    english,
    draft = false,
    twoside = false,
]{article}
\usepackage{comment}
\usepackage{amsfonts}
\usepackage{curriculum-vitae}

%\usepackage{cite}


%\input{structure2.tex} % Include the file specifying the document structure and styling

%\hypersetup{
%	pdftitle={Duarte Maia - CV Eng},
%	pdfauthor={Duarte Maia}
%}

\begin{document}
	%	Basic information	
	\setname{Duarte}{Maia}
	\setaddress{\url{http://math.uchicago.edu/~dmaia/}}
	\setmobile{+1(312)686-5246}
	\setmail{dmaia@uchicago.edu}
	
	
	\cvtitle{Curriculum Vitae}
	
	\cvSection{Education}
	\CVBlockWithTime{BSc in Applied Mathematics and Computation}{2017-2020}
	{Instituto Superior Técnico, Universidade de Lisboa}{}{}
	\CVBlockWithTime{MSc in Mathematics and Applications}{2020-2022}
	{Instituto Superior Técnico, Universidade de Lisboa}{}
	{Advisor: Miguel Abreu}
	\CVBlockWithTime{PhD in Mathematics}{2022-2027 (Expected)}
	{University of Chicago}{}
	{Advisors: Denis Hirschfeldt and Maryanthe Malliaris}
	
	\cvSection{Graduate Studies}
	\CVBlockWithTime{Organizer of the GSLCS at UChicago}{2024-Present}
	{Graduate Student Logic and Combinatorics Seminar}{}
	{\url{https://voices.uchicago.edu/grads-logic-combi/}}
	\CVBlockWithTime{Passed the Topic Examination}{2024}
	{Supervised by Leonardo Coregliano, Denis Hirschfeldt, and Maryanthe Malliaris}{}
	{}
	\CVBlockWithTime{Organizer of the 'Pizza Seminar' at the UChicago Math Department}{2023-2024}
	{}{}
	{}
	\cvSection{Teaching}
	\CVBlockWithTime{Graduate Student Lecturer}{2024-Present}
	{University of Chicago}{}{}
	\CVBlockWithTime{Recipient of the 2024-2025 Physical Sciences Division Graduate Teaching Prize}{2025}
	{University of Chicago}{}{}
	\CVBlockWithTime{Completed the College Fellow Training Program at UChicago}{2024}
	{University of Chicago}{}{}
	\CVBlockWithTime{Teaching Assistant for ''Experimental Mathematics''}{2022}
	{Instituto Superior Técnico}{}{}
	\cvSection{Undergraduate Studies}
	\CVBlockWithTime{MSc Thesis}{2021-2022}
	{Symplectic Geometry and Persistence Homology}{}
	{Supervised by Miguel Abreu\\
	Department of Mathematics of Instituto Superior Técnico}
	\CVBlockWithTime{CAMGSD Research Scholarship}{2020-2022}
	 {Quantum Mechanics and Pointless Topology}{}
	 {Supervised by Pedro Resende}
	\CVHeader{Novos Talentos em Matemática (Research Initiation Scholarship)}
	\CVMinorTime{On Optimal Bounds for Compring Dyadic (and $b$-ary) Net Measures with the Hausdorff Measure on $\mathbb{R}$}{2018-2019}{Supervised by Jorge Drumond Silva}
	\break
	\CVMinorTime{SAT, Multigraphs, and CNF}{2017-2018}{Supervised by João Rasga} 
	\cvSection{Olympiads}
	\CVMinorTime{Occasional participation in university mathematics and informatics olympiads}{2018-2022}{}{}{}
	\CVMinorTime{Second place at the Portuguese National Informatics Olympiad}{2017}{}{}{}
	\CVMinorTime{Gold medal in the Iberoamerican Physics Olympiad}{2017}{}{}{}
	\CVMinorTime{Bronze medal at the International Olympiad in Informatics}{2017}{}{}{}
	\CVMinorTime{Participation at the International Mathematical Olympiad}{2017}{}{}{}
	\CVMinorTime{Honorable mention at the International Mathematical Olympiad}{2016}{}{}{}
	\CVMinorTime{Silver medal at the National phase of the Portuguese National Physics Olympiad}{2016}{}{}{}
	\CVMinorTime{Gold medal at the Regional phase of the Portuguese National Physics Olympiad}{2016}{}{}{}
	\CVMinorTime{Third place at the National Informatics Olympiad}{2016}{}{}{}
	\CVMinorTime{Gold medal at the Portuguese Mathematical Olympiad}{2016}{}{}{}
	\CVMinorTime{Silver medal at the Mathematical Olympiad for the Community of Portuguese Speaking Countries}{2015}{}{}{}
	\CVMinorTime{Bronze medal at the Portuguese Mathematical Olympiad}{2015}{}{}{}
	\cvSection{Other Activities}
	\CVBlockWithTime{Member of the Diagonal Seminar team}{2018-2022}
	{}{}{A group of IST students which organizes seminars by and for undergrad students}
	\CVBlockWithTime{Instructor, author and coordinator in several math-related activites TreeTree2}{2018-2022}
	{}{}{TreeTree2 is a Portuguese nonprofit organization whose goal is to complement portuguese students' studies from grades 1 to 12, by offering interested students the opportunity to study STEM subjects in their free time.}



%\bibliographystyle{plain}
%\bibliography{bibliography}

\end{document}
