\documentclass{article}

\usepackage[utf8]{inputenc}
\usepackage{amsfonts}
\usepackage{amsmath}
\usepackage{amssymb}

\usepackage{commath}

\usepackage[margin=2cm,top=1cm]{geometry}

\title{Aula $8^{TM}$}
\author{Duarte Maia}
\date{}

\newcommand{\R}{\mathbb{R}}
\newcommand{\C}{\mathbb{C}}
\newcommand{\Q}{\mathbb{Q}}

\newcommand{\e}{\mathrm{e}}
\newcommand{\I}{\mathrm{i}}


\renewcommand{\cal}[1]{\mathcal{#1}}

\DeclareMathOperator{\ps}{P}
\DeclareMathOperator{\dom}{dom}

\begin{document}
\maketitle

\begin{itemize}
\item Última aula. Teorema de Fubini let's go.
\item Problema: sejam $(X,M,\mu)$ e $(Y,N,\nu)$ espaços de medida, e seja $f : X \times Y \to [0,\infty]$. Queremos de alguma forma argumentar que se $f$ é simpática o suficiente, podemos tomar os integrais
\[\iint f(x,y) \dif \mu(x) \dif \nu(y) \text{ e } \iint f(x,y) \dif \nu(y) \dif \mu(x)\]
e estes são iguais.
\item Ideia para mostrar igualdade assumindo que eles existem: passar a medida-produto. Exemplificar com $I \times I$.
\item Ou seja, o objetivo é mostrar que se $f : X \times Y \to [0,\infty]$ é mensurável em $M \otimes N$, então o integral iterado existe e
\[\iint f(x,y) \dif \mu(x) \dif \nu(y) = \int f \dif \mu \times \nu.\]
(Note-se que por simetria o integral com $\dif \nu \dif \mu$ também é igual a isto.)
\item Agora, estando no mundo das funções $M \otimes N$-mensuráveis, podemos aproximar $f$ por funções simples e efetivamente reduzimos o nosso trabalho a provar uma afirmação sobre conjuntos. Ou seja, pretendemos provar que, se $E \in M \otimes N$, então a função $x \mapsto \nu(E_x)$ está em $L^+(X)$, e para mais
\begin{equation}\label{fub}
\mu \times \nu(E) = \int \nu(E_x) \dif \mu(x).
\end{equation}
\item Queremos provar uma afirmação para todos os $E \in M \otimes N$, por isso já conhecem a estratégia. Definimos, digamos, $\cal F$ como a coleção de conjuntos $E \in M \otimes N$ tal que \eqref{fub}, mostramos que os retângulos estão lá, e mostramos que $\cal F$ é uma $\sigma$-álgebra. Easy peasy, right?
\item Mostrar retângulos é fácil. Vamos agora mostrar $\sigma$-álgebra.
\item Vamos começar por mostrar que $\cal F$ é fechado para complementares. A ideia óbvia é a seguinte:
\[ \mu \times \nu(E) + \mu \times \nu(E^c) = \int \nu(E_x) \dif \mu(x) + \int \nu(E_x^c) \dif \mu(x),\]
porque ambos os lados são iguais a
\[\mu\times\nu(X\times Y) = \int \nu(Y) \dif \mu(x).\]
\item Agora, queremos concluir que se \eqref{fub} é verdade para $E$ então o é para $E^c$. Aqui, precisamos de exigir que coisas sejam finitas! Ou seja, adicionamos a exigência:
\begin{center}
A partir de agora, supomos $X$ e $Y$ finitos.
\end{center}
Claro que mais tarde extenderemos o resultado a $\sigma$-finitos.
\item Ok, já provámos que $\cal F$ é fechado para complementares. E para uniões contáveis? Well... Isso é mais difícil.
\item\relax [Pausa comprida para pessoas pensarem no assunto!!!]
\item (@Duarte: don't forget. Pausa! Para as pessoas perceberem que de facto assim não vai lá. Se conseguires tenta fazer momento de colaboração.)
\item Ok, agora que já viram que provar que isto é uma $\sigma$-álgebra diretamente é pouco exequível, vamos ver como dar a volta à coisa. Vamos ver que género de operações podemos fazer em $\cal F$.
\item Uniões disjuntas. Uniões binárias mesmo muito simpáticas. Uniões de retângulos. Ooh, $\cal F$ contém todos os elementos de $\cal A$, o anel (neste caso álgebra) gerado pelos retângulos.
\item Usando o TCM, podemos mostrar que $\cal F$ é fechado para uniões monótonas. Já estamos mais próximos: já podemos fazer pelo menos uma operação contável. Usando complementares, podemos também fazer interseções monótonas.
\item Temos aqui uma caixa de ferramentas de operações que podemos fazer, mas nenhuma delas é uniões. Reparem, já agora, que se mostrarmos que podemos fazer uniões binárias, pela coisa de uniões monótonas, temos uma $\sigma$-álgebra. Mas as uniões binárias parecem elusivas.
\end{itemize}

\end{document}
