\documentclass{article}

\usepackage[utf8]{inputenc}
\usepackage[portuguese]{babel}
\usepackage{amsfonts}


\title{Currículo para Aulas de Teoria de Medida}
\author{Duarte Maia}
\date{}

\newcommand{\R}{\mathbb{R}}
\newcommand{\C}{\mathbb{C}}

\begin{document}
\maketitle

\section{Conteúdos}

\begin{itemize}
\item Setup: definir rigorosamente comprimento, área, volume
\item Justificar porque é que a coisa naïve não funciona (gloss over Vitali)
\item Dar bash a definição de Riemann
\item Motivar $\sigma$-álgebra como sendo o conjunto no qual é razoável querer definir medidas
\item Definir medida
\item Justificar utilidade de medidas como coisas mais gerais do que `áreas'
\begin{itemize}
\item Distribuições de massa
\item Medidas de probabilidade
\item (Se permitirmos valores negativos) Distribuições de carga
\end{itemize}
\item Dar exemplos simples
\item Motivar noção de construir medidas a partir de coisas mais simples
\begin{itemize}
\item Construir a medida de Lebesgue começando com comprimentos de intervalos
\item Construir uma medida de probabilidade começando com uma função de distribuição
\item Dados dois espaços $X$ e $Y$ com medidas $\mu$ e $\nu$, construir uma medida em $X \times Y$ a partir de retângulos
\item ...Por exemplo, construir uma medida em $X \times \left[0,\infty\right]$ para depois poder definir integral
\item ...Por exemplo, construir a medida de Lebesgue em $\R^n$
\item ...Por exemplo, construir uma medida de probabilidade produto, que corresponde a fazer duas experiências independentes em paralelo.
\end{itemize}
\item Partir desta noção para motivar o conceito de $\sigma$-álgebra gerada por conjunto
\item Falar da $\sigma$-álgebra de Borel, como sendo a gerada por intervalos
\item Várias expressões semelhantes para definir $\mathcal{B}$
\item Como provar coisas para $\sigma$-álgebras geradas
\item A não-simplicidade da $\sigma$-álgebra gerada por um conjunto
\item Motivação para abordar o problema de criar uma medida a partir de coisas mais simples
\item Introduzir ideia de estimativa exterior, e portanto, de medida exterior (Função de conjuntos $\to$ Medida exterior)
\item Ao contrário de Riemann, em vez de restringir com base numa estimativa interior, restringir com base na condição de mensurabilidade.
\item Teorema de Carethéodory (Medida exterior $\to$ Medida)
\item Encontrar condições na nossa função de conjuntos para assegurar que:
\begin{itemize}
\item Todos os conjuntos com que começámos são mensuráveis
\item A medida final coincide com a função inicial
\end{itemize}
(Uma função que satisfaz as condições que encontrarmos chama-se uma pré-medida.)
\item Concretizar com medida de Borel
\item Generalizar para medidas de Lebesgue-Stieltjes
\item Medida dada por função de distribuição
\item Medida produto
\item Definir medida de Borel em $\R^n$

\item (A partir daqui ainda está abreviado)

\item Invariância da medida de Borel sob translações
\item Falar de completude; medida de Lebesgue
\item Relação entre Lebesgue-mensuráveis, abertos e compactos; densidade de elementares em $\mathcal{L}$

\item Definir integral em $L^+$; definição Folland vs Ricou
\item Definir integral em $L^1$
\item Funções simples
\item Mostrar aplicações de definição de integral por funções simples
\item Teoremas de convergência; Fatou
\item (Opcional) Lebesgue extende Riemann
\item Fubini
\item Mudança de variáveis linear

\item (Se der) Mudança de variáveis polar
\end{itemize}

\end{document}
