\documentclass{article}

\usepackage{amsmath}
\usepackage{mathtools}
\usepackage{ntheorem}
\usepackage[utf8]{inputenc}
\usepackage{diffcoeff}
\diffdef{}{op-symbol=\mathrm{d},op-order-sep=0mu}

\usepackage[portuges]{babel}

\title{Projeto de CV\\
\large Multiplicadores de Lagrange e restrições holonómicas e não-holonómicas}
\author{Duarte Maia}
\date{Abril de 2021}

\theoremseparator{.}
\newtheorem{prop}{Proposição}
\newtheorem{lema}{Lema}
\newtheorem{teorema}{Teorema}

\theoremstyle{nonumberplain}
\theorembodyfont{\upshape}
\theoremseparator{.}
\newtheorem{proof}{Dem}

\DeclarePairedDelimiter\abs{\lvert}{\rvert}
\DeclarePairedDelimiter\norm{\lVert}{\rVert}

\begin{document}
	\maketitle

	\section{Introdução}

	todo

	\section{Restrições Holonómicas}

	Um problema de otimização com restrições holonómicas é um problema no qual se pretende minimizar um funcional da forma
	\[I(u) = \int_a^b f(x,u(x),u'(x)) \dl x,\]
	sujeito a restrições pontuais. A forma mais elementar é exigir que o caminho esteja contido numa variedade definida por equações cartesianas, isto é, exigir que, para todo o $x \in [a,b]$,
	\[g_1(u(x)) = c_1, \dots, g_k(u(x)) = c_k,\]
	onde os $g_i$ são funções regulares o suficiente e os $c_i$ são constantes reais. Por exemplo, se quisermos otimizar um funcional sobre os caminhos contidos na esfera, podemos escrever isto como a condição $\norm{u(x)} = 1$.

\end{document}