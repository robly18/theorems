\documentclass{article}

\usepackage{amsmath,amssymb,amsfonts}
\usepackage{mathtools}
\usepackage{ntheorem}
\usepackage[utf8]{inputenc}
\usepackage{diffcoeff}
\diffdef{}{op-symbol=\mathrm{d},op-order-sep=0mu}

\usepackage[portuges]{babel}

\title{Projeto de CV\\
\large Multiplicadores de Lagrange e restrições holonómicas e não-holonómicas}
\author{Duarte Maia}
\date{Abril de 2021}

\theoremseparator{.}
\newtheorem{prop}{Proposição}
\newtheorem{lema}{Lema}
\newtheorem{teorema}{Teorema}

\theoremstyle{nonumberplain}
\theorembodyfont{\upshape}
\theoremseparator{.}
\newtheorem{proof}{Dem}

\newcommand{\R}{\mathbb{R}}

\newcommand{\tr}{\intercal}

\DeclarePairedDelimiter\abs{\lvert}{\rvert}
\DeclarePairedDelimiter\norm{\lVert}{\rVert}

\begin{document}
\maketitle

\section{Introdução}

todo

\section{Restrições Holonómicas}

Um problema de otimização com restrições holonómicas é um problema no qual se pretende minimizar um funcional da forma
\[I(u) = \int_a^b f(t,u(t),u'(t)) \dl2 t,\]
sujeito a restrições pontuais. A forma mais elementar é exigir que o caminho esteja contido numa variedade definida por equações cartesianas, isto é, exigir que, para todo o $x \in [a,b]$,
\[g_1(u(t)) = c_1, \dots, g_k(u(t)) = c_k,\]
onde os $g_i$ são funções regulares o suficiente e os $c_i$ são constantes reais. Por exemplo, se quisermos otimizar um funcional sobre os caminhos contidos na esfera, podemos escrever isto como a condição $\norm{u(t)} = 1$.

\begin{teorema}
Seja $u$ um minimizante do funcional $I$ sob as condições de fronteira usuais ($u(a) = \alpha$, $u(b) = \beta$), juntamente com a restrição $g(u(x)) = c$, em que $g \colon \R^d \to \R^k$ e $c \in \R^k$ é um valor regular de $g$. Impõem-se as hipóteses de regularidade: $u$ é $C^2$, $f$ é $C^1$ e $g$ é $C^2$ num aberto que contém o caminho. Então, existe uma função contínua, $\lambda \colon \left]a,b\right[ \to \R^k$, tal que
\[\diffp f {u_i}(t,u(t),u'(t)) - \diff*{\left(\diffp f {\xi_i}(t,u(t),u'(t))\right)}t = \sum_{j=1}^k \lambda_j(t) \diffp {g_j} {u_i} (u(t)), \quad i = 1, \dots, d.\]
\end{teorema}

\begin{proof}
Vamos demonstrar esta afirmação construindo localmente a função $\lambda$. Para cada $t_0 \in \left]a,b\right[$ mostraremos que existe uma função $\lambda$ definida numa vizinhança de $t_0$ que satisfaz as restrições acima, e fica a partir disto provada a afirmação, visto que o facto de $c$ ser valor regular assume que $\lambda(t)$ é sempre único e portanto a função $\lambda$ ``cola bem'' nas vizinhanças.

Para este efeito, usemos o teorema da função inversa. Como $g'(u(t_0))$ é uma matriz $k \times d$ de característica $k$, uma das suas submatrizes $k \times k$ é invertível. Trocando índices, podemos sem perda de generalidade assumir que é a submatriz ``da esquerda'', isto é, podemos supor que a matriz
$
A = \begin{bmatrix}
g'(u(t_0))\\
\begin{matrix}
0 & I
\end{matrix}
\end{bmatrix}
$
é invertível. Isto permite-nos aplicar o teorema da função inversa à seguinte função. Decomponha-se $u = (v, w)$, em que $v$ são as primeiras $k$ componentes de $u$ e $w$ as últimas $d-k$. Defina-se agora $G(v,w) = (g(v,w),w)$. Então, claramente temos $G'(u(t_0)) = A$, que é invertível, pelo que existem abertos $U$ e $V$ em $\R^d$ tal que:
\begin{itemize}
\item $u(t_0) \in U$,
\item $G(u(t_0)) \in V$, e
\item $G \colon U \to V$ é um difeomorfismo $C^2$.
\end{itemize}

Visto que $G(u(t_0)) \in V$, existe uma bola, digamos de raio $r$, em torno de $G(u(t_0))$ que está contida em $V$. Considere-se um intervalo $\left]t_0-\delta, t_0 + \delta\right[$ tal que $G(u)$ não difere de mais de $r/2$ de $G(u(t_0))$. Seja $v$ uma função de teste de suporte contido neste intervalo. Então, para $\varepsilon$ pequeno o suficiente temos que $G(u) + \varepsilon v$ está contido em $V$ para $t \in \left]t_0-\delta,t_0+\delta\right[$. Consequentemente, é possível definir $u_\varepsilon$ de modo a concordar com $u$ fora deste intervalo, e como $G^{-1}(G(u) + \varepsilon v)$ dentro dele.

Podemos agora fazer o argumento usual para obter as equações de Euler-Lagrange. Como sabemos que $u$ minimiza o funcional de entre os caminhos que satisfazem $g(u) = c$, sabemos que, se $v$ tiver as primeiras $k$ componentes nulas, $I(u_\varepsilon) \geq I(u)$. Assim sendo, definindo $\Phi(\varepsilon) = I(u_\varepsilon)$, obtemos $\Phi'(0) = 0$, e esta derivada pode ser calculada usando a regra de Leibniz, porque, pelas nossas hipóteses de regularidade, a função $f(t, u_\varepsilon(t), u'_\varepsilon(t))$ é $C^1$ em $\varepsilon$ e $t$. A verificação é tediosa, mas resume-se a dividir no caso em que $t$ não está no suporte de $v$ (neste caso a afirmação é trivial) e no caso em que $t$ está no suporte de $v$ (neste caso podemos usar a expressão $u_\varepsilon = G^{-1}(G(u) + \varepsilon v)$ numa vizinhança).

Calcule-se então $\Phi'(0)$. Primeiro que tudo, aplicamos Leibniz para obter
\[\Phi'(0) = \int_a^b \diffp*{f(t, u_\varepsilon(t), u'_\varepsilon(t))}\varepsilon \dl2 t.\]

De seguida, podemos usar o facto que para $t \not \in \left]t_0-\delta,t_0+\delta\right[$ temos $u_\varepsilon = u$, para reduzir o integral a
\[\int_{t_0 - \delta}^{t_0 + \delta} \diffp*{f(t, u_\varepsilon(t), u'_\varepsilon(t))}\varepsilon \dl2 t,\]
e neste intervalo temos o benefício que $u_\varepsilon$ tem uma expressão simples. Nomeadamente, $u_\varepsilon(t) = G^{-1}(G(u(t)) + \varepsilon v(t))$. A expressão para $u'_\varepsilon(t)$ é ligeiramente mais complexa.
\[u'_\varepsilon(t) = (G^{-1})'(G(u(t)) + \varepsilon v(t)) [G'(u(t)) u'(t) + \varepsilon v'(t)].\]

Aplicando a derivada da composta, concluimos
\[\Phi'(0) = \int_{t_0 - \delta}^{t_0 + \delta} \diffp fu (t, u(t), u'(t)) \diffp {u_\varepsilon(t)} \varepsilon(0) + \diffp f \xi (t, u(t), u'(t)) \diffp{u_\varepsilon}{\varepsilon, t}(0,t)\dl2 t,\]
e uma aplicação de integração por partes no segundo termo resulta em
\[\Phi'(0) = \int_{t_0 - \delta}^{t_0 + \delta} \left( \diffp fu (t, u(t), u'(t))  - \diff*{\diffp f \xi (t, u(t), u'(t))}t \right) \diffp {u_\varepsilon(t)} \varepsilon(0)\dl2 t.\]

O termo $\diffp {u_\varepsilon(t)} \varepsilon(0)$ toma nesta prova o lugar do que normalmente chamamos $v$, sendo que normalmente aqui aplicaríamos o lema fundamental do cálculo de variações para concluir Euler-Lagrange. No entanto, os $u_\varepsilon$ estão limitados à nossa variedade, pelo que podemos apenas concluir que o termo nos parênteses pertence ao espaço normal à variedade. Isto faz sentido com o que esperamos, visto que esse espaço é gerado pelos gradientes dos $g_i$.

Para prosseguir, escreva-se melhor o termo $\diffp {u_\varepsilon(t)} \varepsilon(0)$. Este é dado por
\[\diffp {u_\varepsilon(t)} \varepsilon(0) = (G^{-1})'(G(u(t))) v(t) = (G'(u(t)))^{-1} v(t).\]

Seja $A(t) = G'(u(t))^{-1}$, $a_{ij}(t)$ o elemento $i,j$ de $A(t)$, e defina-se a função $\lambda \colon \left]t-\delta,t+\delta\right[ \to \R^d$ como
\[\lambda_i(t) = \sum_j \left( \diffp f{u_j} (t, u(t), u'(t))  - \diff*{\diffp f {\xi_j} (t, u(t), u'(t))}t \right) a_{ji}(t). \]

Então, a equação acima diz-nos que, pelo TFCV, $\lambda_i = 0$ para $i > k$. Para mais, é fácil ver por definição que
\[\lambda(t)^\tr = \left( \diffp fu (t, u(t), u'(t))  - \diff*{\diffp f \xi (t, u(t), u'(t))}t \right) G'(u(t))^{-1},\]
pelo que $G'(u(t))^\tr \lambda(t) = \left( \diffp fu (t, u(t), u'(t))  - \diff*{\diffp f \xi (t, u(t), u'(t))}t \right)^\tr$.

Visto que apenas as primeiras $k$ coordenadas de $\lambda$ são não-nulas, e que as primeiras $k$ coordenadas de $G$ coincidem com $g$, esta expressão pode ser escrita como (abusando da notação e usando agora $\lambda(t)$ para representar o vetor das primeiras $k$ coordenadas de $\lambda$):
\[
\begin{bmatrix}
\nabla g_1(u(t)) & \dots & \nabla g_k(u(t))
\end{bmatrix}
\lambda(t)
=
\left( \diffp fu (t, u(t), u'(t))  - \diff*{\diffp f \xi (t, u(t), u'(t))}t \right)^\tr.
\]

Consequentemente, a função $\lambda(t)$ (com apenas $k$ componentes) é a função que se procura, e a prova está concluída.
\end{proof}

\end{document}