\documentclass{article}

\usepackage{amsmath}
\usepackage{amsthm}
\usepackage{amsfonts}
\usepackage[portuguese]{babel}
\usepackage{tikz}
\usetikzlibrary{patterns}
\usepackage{listings}
\usepackage{ifxetex}
\usepackage{mathrsfs}
\usepackage{enumerate}
\usepackage{courier}
\usepackage{hyperref}
\usepackage{indentfirst}

\lstset{mathescape=true, keepspaces=true, tabsize=2,basicstyle=\fontencoding{T1}\ttfamily,breaklines=true}

\ifxetex
%xetex is recommended!
\else
\usepackage[utf8]{inputenc}
\lstset{
  literate=
  {á}{{\'a}}1
  {à}{{\`a}}1
  {ã}{{\~a}}1
  {é}{{\'e}}1
  {ê}{{\^e}}1
  {í}{{\'i}}1
  {ó}{{\'o}}1
  {õ}{{\~o}}1
  {ú}{{\'u}}1
  {ü}{{\"u}}1
  {ç}{{\c{c}}}1
}
\fi

\addto\captionsportuguese{
	\renewcommand*{\proofname}{Dem}
}

\DeclareMathOperator{\interior}{int}
\DeclareMathOperator{\exterior}{ext}
\DeclareMathOperator{\border}{\partial}
\DeclareMathOperator{\im}{Im}
\DeclareMathOperator{\CC}{CC}

\title{IO para alunos de LMAC}
\newcommand{\myemail}{duarte.nascimento@tecnico.ulisboa.pt}
\newcommand{\myemailref}{\href{mailto:\myemail}{\myemail}}
\author{Duarte Maia \\ \myemailref}
\date{}

\newcommand{\R}{\mathbb{R}}
\newcommand{\N}{\mathbb{N}}

\newtheorem{prop}{Prop}
\newtheorem*{prop*}{Prop}
\newtheorem{conjetura}{Conjetura}

\theoremstyle{definition}
\newtheorem{definition}{Definição}
\newtheorem*{definition*}{Definição}
\newtheorem*{notacao}{Notação}

%todo, make def environment bolder, and bolden the just defined terms

\begin{document}
	\maketitle
	
	\tableofcontents
	
	\section{Introdução}
	
	\subsection{Introdução à versão 1}
	
	Uma das minhas coisas favoritas de fazer quando pego num livro novo de matemática é ler a introdução. Sei que há pessoal que passa isso à frente, porque não serve realmente para nada. No entanto, é algo que me agrada, porque sei que é a única parte do livro que vou ler a um ritmo de mais de 10 palavras por minuto.
	
	Esta versão do ``IO para alunos de LMAC'' (que era originalmente chamado ``IO Done Right'' mas mudei o nome porque não queria insultar demasiado a Sernadas) é uma versão bastante primordial. Ainda não reli metade disto, pelo que inevitavelmente há imensas partes que estarão bastante insatisfatórias. Assim sendo, apelo ao leitor que mande para o meu e-mail (\myemailref) quaisquer críticas que tenha, desde `Falta aqui um ponto final' até `Este teorema é falso', e tudo no meio, incluindo (e desejando-se) coisas para o efeito de `esta prova está mal explicada'. O objetivo é a compreensão, pelo que desejo fazer um texto assessível a qualquer aluno de LMAC que saiba ler.
	
	Uma nota final: a secção de pré-requisitos está, neste momento, muito desarrumada. Estou eventualmente a planear melhorá-la, e escrever anexos a tratar dos conceitos básicos topológicos necessários. De qualquer das formas, se o leitor sentir que há algum conhecimento base que lhe falte, ou que algo está mal explicado, é favor assinalar e informar-me, para o problema ser retificado em versões futuras.
	
	Sem mais atrasos, eis a sebenta em que estou a trabalhar desde o verão de 2018.
	
	-Duarte Maia
	
	\subsection{Introdução à versão 2}
	
	To do.
	
	\section{Pré-requisitos}
	
	Vou aqui escrevendo coisas à medida que elas são necessárias.
	
	Obviamente é preciso saber um pouco de cálculo I e AL.
	
	É recomendado (i.e. necessário) que o leitor saiba os seguintes conceitos topológicos em $\R^n$:
	
	Conceito de conjunto aberto e conjunto fechado; interior, fronteira e exterior de um conjunto; função contínua; teorema de Bolzano e Weierstrass em $\R^n$, definição topológica de continuidade ($f : D \rightarrow \R^m$ é contínua sse para todo conjunto aberto $A$ se tem $f^{-1}(A)$ é aberto relativamente a $D$ sse para todo conjunto fechado $A$ se tem $f^{-1}(A)$ é fechado relativamente a $D$)
	
	No que se segue vão ser usados alguns factos sobre estes conceitos que não vão ser justificados, visto que caem no âmbito da cadeira de Cálculo II. Estes serão devidamente assinalados, e o leitor é encorajado a prová-los se não estiver familiar com eles.
	
	Referências para AL e CDI:
	\begin{itemize}
	\item Linear Algebra Done Right, Sheldon Axler
	\item Calculus on Manifolds, Michael Spivak
	\end{itemize}
	
	Referência na qual me baseei para o conteúdo de IO: as notas da cadeira de IO escritas por: Diogo Gomes, Amílcar Sernadas,Cristina Sernadas, João Rasga, Paulo Mateus; estas notas encontram-se na página do Fénix da cadeira de IO.
	
	\section{Problemas de Otimização Linear}
	
	\subsection{Introdução}
	
	Um problema de otimização é um tipo específico de problema em que o objetivo é minimizar ou maximizar uma certa função (função-objetivo ou pontuação), dentro de um certo domínio, usualmente (no contexto desta cadeira) um subconjunto de $\R^n$ parametrizado por um conjunto de (in)equações.
	
	Mais concretamente, o tipo de problemas com que se lida nesta cadeira são problemas de otimização linear. Estes são problemas em que a função-objetivo é uma função linear $\R^n \rightarrow \R$ e todas as condições no domínio são da forma $ax \leq b$, $ax = b$ ou $ax \geq b$, onde $a$ é um vetor-linha e $b$ é um escalar.
	
	\begin{definition}
	Um \emph{problema de otimização linear} (normalmente abreviado a \emph{pol}) é um problema da forma:
	
	\smallskip
	
	\textbf{Objetivo: } maximizar/minimizar (em função de $x \in \R^n$) o valor de ${c_1 x_1 + c_2 x_2 + \ldots + c_n x_n}$
	
	\smallskip
	
	\textbf{Restrições: } $x$ tem de obedecer a todas as seguintes igualdades

\[
\begin{cases}
	a_{11} x_1 + a_{12} x_2 + \ldots + a_{1n} x_n = b_1 \\ 
	a_{21} x_1 + a_{22} x_2 + \ldots + a_{2n} x_n = b_2 \\
	\vdots \\
	a_{p1} x_1 + a_{p2} x_2 + \ldots + a_{pn} x_n = b_p
\end{cases}
\]

E todas as seguintes desigualdades

\[
\begin{cases}
	a'_{11} x_1 + a'_{12} x_2 + \ldots + a'_{1n} x_n \leq b'_1 \\ 
	\vdots \\
	a'_{q1} x_1 + a'_{q2} x_2 + \ldots + a'_{qn} x_n \leq b'_q
\end{cases}
\]
\[
\begin{cases}
	a''_{11} x_1 + a''_{12} x_2 + \ldots + a''_{1n} x_n \geq b''_1 \\ 
	\vdots \\
	a''_{r1} x_1 + a''_{r2} x_2 + \ldots + a''_{rn} x_n \geq b''_r
\end{cases}
\]
	\end{definition}
	
	Infelizmente, isto é algo trabalhoso de escrever. Como tal, normalmente estas condições são escritas de forma mais compacta usando a linguagem da álgebra linear.
	
	\begin{notacao}
Dados dois vetores $x,y \in \R^n$, dizemos $x \leq y$ se $x_i \leq y_i$ para todo $i$.
	\end{notacao}
	\begin{notacao}
	Considere-se o pol escrito acima. Este é normalmente escrito da seguinte forma:
	
	(Supõe tratar-se de um problema de maximização; caso contrário escreva-se $\min$ no lugar de $\max$)
	
	Seja $c$ o vetor linha $[\,c_1\,c_2\, \cdots \,c_n\,]$,

	$A$ a matriz $p \times n$ cujo $i,j$-ésimo elemento é $a_{ij}$, e considerações análogas para $A'$ a $A''$, com $a'_{ij}$ e $a''_{ij}$ respetivamente
	
	$b$ o vetor coluna $(b_1, b_2, \cdots, b_p)$ e análogamente para $b'$ e $b''$.
	
	\[
	\begin{cases}
	\max\limits_x cx \\
	Ax = b \\
	A'x \leq b' \\
	A''x \geq b''
	\end{cases}
	\]
	\end{notacao}
	
	Normalmente, não é útil considerar um tipo de problema tão geral, pelo que discutímos em baixo formas de passar de um tipo de problemas para outros.
	
	\subsection{Tradução}
	
	Considere-se os seguintes problemas de otimização linear:
	
	\[
	\begin{cases}
	\max\limits_x 2x_1 - 3x_2 \\
	5x_1 + x_2 = 6 \\
	2x_1 + 0x_2 \leq 2
	\end{cases}
	\begin{cases}
	\min\limits_x -2x_1 + 3x_2 \\
	5x_1 + x_2 \leq 6 \\
	5x_1 + x_2 \geq 6 \\
	-2x_1 + 0x_2 \geq -2
	\end{cases}
	\]
	
	Apesar de terem uma aparência diferente, alguma inspeção leva à conclusão que estes são, na realidade, exatamente o mesmo problema. Ou seja, o mesmo problema pode ser representado de mais do que uma forma diferente. Isto leva à possibilidade de considerar `formas canónicas' de expressar os problemas, que sejam mais fáceis de estudar. Se, por exemplo, dissermos que todos os problemas da forma XYZ podem ser resolvidos fazendo ABC, e arranjarmos forma de traduzir qualquer pol para esta forma XYZ, estamos numa boa situação.
	
	\begin{notacao}
	Dado um pol $P$, o conjunto de $x \in \R^n$ que satisfazem as suas condições é denominado de \emph{conjunto admissível de $P$}, representado por $X_P$, ou só $X$ se o pol em questão for óbvio de contexto.
	
	Da mesma forma, o \emph{conjunto solução de $P$}, $S_P$, ou $S$ se $P$ é claro de contexto, é o conjunto de $x \in X$ que maximizam, de facto, a função objetivo. Por outras palavras,
	
	\[S_P = \{\,x \in X_p \mid \forall_{y \in X_p} cx \geq cy\,\}\]
	
	Isto no caso de maximização. No caso de minimização, a desigualdade deve estar trocada.
	\end{notacao}
	
	Há várias formas possíveis de transformar um problema noutro.
	
	\subsubsection{Substituição de equivalências}
	
	Por exemplo, a condição de igualdade $a = b$ pode ser expressa como duas desigualdades: $a \leq b$ e $a \geq b$. Assim sendo, sabemos à partida que qualquer pol que nos venha à cabeça pode ser expresso só com condições de desigualdade.
	
	Para mais, a desigualdade $a \geq b$ é equivalente a $-a \leq -b$. Logo, podemos sempre assumir que as condições que restringem um pol são sempre da forma $Ax \leq b$.
	
	Podemos também assumir sem perda de generalidade que um problema de otimização linear é um problema de maximização, pois minimizar $cx$ é o mesmo que maximizar $-cx$.
	
	Logo, dado um pol qualquer, podemos sempre assumir, sem perda de generalidade, que é da forma
	
	\[
	\begin{cases}
	\max\limits_x cx\\
	Ax \leq b
	\end{cases}
	\]
	
	A esta forma vamos-nos referir, no futuro, como `forma semicanónica', na falta de melhor nome. Sugestões são aceites.
	
	O que foi até agora descrito é o tipo mais simples possível de tradução de um problema a outro: trocar as condições por condições equivalentes, e substituir `maximizar $f$' por `minimizar $-f$'.
	
	Este tipo de tradução tem a propriedade que não muda o conjunto admissível e conjunto solução, mas há traduções que não são tão triviais.
	
	\subsubsection{Adição de positividades (Forma canónica)}
	
	É, em muitos casos, útil estudar um problema de otimização em que sabemos que todas as variáveis são não-negativas. Assim sendo, vamos agora ver que podemos sempre, sem perda de generalidade, assumir que o nosso pol é da forma
	
	\[
	\begin{cases}
	\max\limits_x cx\\
	Ax \leq b\\
	x \geq 0
	\end{cases}
	\]
	
	O truque é modificar o conjunto em que estamos.
	
	Dado um número real $x$, define-se a sua parte positiva, $x^+$, como
	
	\[
	x^+ =
	\begin{cases}
	x & \text{se } x \geq 0 \\
	0 & \text{caso contrário}
	\end{cases}
	\]
	
	e a parte negativa, $x^-$, como
	
	\[
	x^- =
	\begin{cases}
	0 & \text{se } x \geq 0 \\
	-x & \text{caso contrário}
	\end{cases}
	\]
	
	Dado um vetor $x \in \R^n$, define-se $x^+$ como o vetor das partes positivas de $x$, e $x^-$ como o vetor das partes negativas. É claro que $x = x^+ - x^-$, e $x^+, x^- \geq 0$. Isto permite-nos escrever qualquer vetor como a diferença de dois vetores não-negativos.
	
	Assim sendo, considere-se a condição $Ax \leq b$. Esta condição é equivalente a $Ax^+ - Ax^- \leq b$, com $x^+, x^- \geq 0$. Para mais, $cx = cx^+ - cx^-$, pelo que isto sugere a consideração de transformar o pol
	
	\[
	P =
	\begin{cases}
	\max\limits_x cx\\
	Ax \leq b
	\end{cases}
	\]
	
	em
	
	\[
	Q =
	\begin{cases}
	\max\limits_{(x_+, x_-)} cx_+ - cx_-\\
	Ax_+ - Ax_- \leq b\\
	x_+, x_- \geq 0
	\end{cases}
	\]
	
	(Aqui, $(x_+,x_-)$ representa apenas um vetor, de dimensão igual ao dobro da de $x$. O $+$ e $-$ em subscrito é suposto ser sugestivo de que um deles é a parte positiva e o outro a parte negativa de $x$.)
	
	Repare-se que este segundo pol não tem a mesma dimensão que o primeiro: se no primeiro, o vetor $x$ pertence a $\R^n$, no segundo, o vetor $(x_+, x_-)$ pertence a $\R^{2n}$. Portanto, não é imediatamente óbvio que estes são `o mesmo' pol, pelo que vamos fazer a seguinte definição:
	
	\begin{definition}
	Dados dois problemas de otimização linear, $P$ e $Q$, dizemos que \emph{$Q$ é uma $f$-tradução de $P$} se se for possível resolver $P$ sabendo a resposta de $Q$, usando $f$.
	
	Em termos mais precisos, se $f$ é uma função sobrejetiva $X_Q \rightarrow X_P$ tal que $q \in X_Q$ tem uma pontuação melhor que $q' \in X_Q$ (menor se $Q$ é de minimzação, maior se maximização) sse $f(q)$ tem melhor pontuação que $f(q')$.
	\end{definition}
	
	Esta definição tem interesse devido à seguinte proposição:
	
	\begin{prop}
	Se $Q$ é uma $f$-tradução de $P$, $f(S_Q) = S_P$.
	\end{prop}
	\begin{proof}
	Basta reparar que, por definição, $x \in S_Q$ sse $x$ tem pontuação melhor ou igual que $y$ para todo $y \in X_Q$, sse $f(x)$ tem pontuação melhor que $f(y)$ para todo $y \in X_Q$. Queremos então mostrar que isto é sse $f(x) \in S_P$.
	
	Como $f$ é sobrejetivo, isto implica que $f(x)$ tem pontuação melhor ou igual que todo $z$ em $X_P$, o que por definição implica $f(x) \in S_P$. Pelo outro lado, se houver $y \in X_Q$ tal que $f(y)$ tem pontuação melhor que $f(x)$, então $f(x)$ claramente não pode ser solução, pois $f(y)$ é admissível de pontuação melhor que ele.
	\end{proof}
	
	Isto mostra que se resolvermos uma tradução de um pol, é fácil recuperar a solução e conjunto admissível do problema original.
	
	O caso anterior, em que $X_P = X_Q$ e $S_P = S_Q$ é o caso trivial em que $f$ é a identidade. Neste caso, em que pretendemos adicionar a condição de positividade, é ligeiramente mais complicado.
	
	Vamos mostrar que, para o $P$ e $Q$ definidos em cima, $Q$ é uma $f$-tradução de $P$, em que $f$ é a função $f(x_+, x_-) = x_+ - x_-$.
	
	\begin{proof}
	Primeiro, mostre-se que se um vetor pertence a $X_Q$, a sua imagem pertence a $X_P$.
	
	Repare-se que, se $(x_+, x_-) \in X_Q$, então a sua imagem é $f(x_+, x_-) = x_+ - x_-$. Por definição, estes satisfazem $A(x_+ - x_-) \leq b$, donde $A f(x) \leq b$, que é a condição para $f(x) \in X_P$.
	
	Para mostrar sobrejetividade, repare-se que se $x \in X_P$, tem-se $(x^+, x^-) \in X_Q$, e $f(x^+, x^-) = x^+ - x^- = x$.
	
	Agora, queremos mostrar que $(x_+, x_-)$ tem pontuação melhor que $(y_+, y_-)$ sse $f(x_+, x_-)$ tem pontuação melhor que $f(y_+, y_-)$. Mas isto é trivial, pois a pontuação de $f(x_+, x_-)$ é a mesma que a de $(x_+, x_-)$, e o objetivo é, tanto em $P$ como em $Q$, maximizar.
	\end{proof}
	
	Assim sendo, podemos sempre assumir, sem perda de generalidade, que qualquer problema que tenhamos está na chamada \emph{forma canónica}, isto é, é da forma
	
	\[
	\begin{cases}
	\max\limits_x cx\\
	Ax \leq b\\
	x \geq 0
	\end{cases}
	\]
	
	Repare-se que a nossa função de tradução $f$ é linear. Isto vai ser relevante eventualmente. Dizemos que qualquer pol pode ser \emph{traduzido linearmente para um pol na forma canónica}.
	
	\subsubsection{Desigualdades para igualdades (Forma padrão)}
	
	Outra forma útil de expressar um problema de otimização linear é a chamada forma padrão, da forma seguinte:
	
	\[
	\begin{cases}
	\min\limits_x cx\\
	Ax = b\\
	x \geq 0
	\end{cases}
	\]
	
	Já sabemos que podemos supor que os nossos problemas estão na forma canónica, sem perda de generalidade. Portanto, considere-se o pol
	
	
	\[
	P =
	\begin{cases}
	\max\limits_x cx\\
	Ax \leq b\\
	x \geq 0
	\end{cases}
	\]
	
	E vamos reescrevê-lo na forma padrão.
	
	Para fazer a tradução, é preciso representar desigualdades com igualdades. Para isso, repare-se que $a \leq b$ é a mesma coisa que dizer que $a + y = b$, para algum $y$ não-negativo. Assim, adicionamos as chamadas \emph{variáveis de folga}: Considere-se o pol
	
	\[
	Q =
	\begin{cases}
	\min\limits_{(x,y)} -cx\\
	Ax + y = b\\
	x, y \geq 0
	\end{cases}
	\]
	
	Vamos mostrar que $Q$ é uma $f$-tradução de $P$, com $f(x,y) = x$.
	
	\begin{proof}
	Primeiro, verifique-se que se $(x,y) \in X_Q$ se tem $f(x,y) \in X_P$. Repare-se que a condição $f(x,y) \geq 0$ é trivial, pois $x \geq 0$ por hipótese, e a condição $A f(x,y) \leq 0$ é verídica pois $A f(x,y) = Ax \leq Ax + y = b$.
	
	De seguida, é preciso verificar sobrejetividade. Suponhamos que se quer um objeto cuja imagem sob $f$ seja $x \in X_P$. Considere-se, então, o vetor $(x,y)$, com $y = b - Ax$. Repare-se que este vetor pertence a $X_Q$, e a sua imagem é precisamente $x$, como queríamos demonstrar.
	
	Finalmente, provemos que $(x,y)$ tem melhor pontuação que $(x',y')$ em $Q$ sse $x$ tem melhor pontuação que $x'$ em P.
	
	Mas isto é trivial, pois $(x,y)$ tem melhor pontuação que $(x',y')$ sse $-cx < -cx'$ sse $cx > cx'$, como queríamos demonstrar.
	\end{proof}
	
	Isto mostra, então, que qualquer pol tem uma tradução linear para um pol na forma padrão.
	
	\subsubsection{Padrão para Canónico}
	
	Viu-se agora uma forma para traduzir problemas da forma canónica para a forma padrão. É possível, também, passar da forma padrão para a canónica (provou-se, aliás, que é possível passar de qualquer pol para a forma canónica) usando os procedimentos usados antes.
	
	No entanto, é muito mais económico fazer o processo simples de transformar todas as igualdades em desigualdades do tipo $\leq$. Ou seja, passar de:
	
	\[
	\begin{cases}
	\min\limits_x cx\\
	Ax = b\\
	x \geq 0
	\end{cases}
	\]
	
	Para
	
	\[
	\begin{cases}
	\max\limits_x -cx\\
	Ax \leq b\\
	-Ax \leq -b\\
	x \geq 0
	\end{cases}
	\]
	
	Já vimos antes que estas traduções são válidas.
	
	\section{Noções Topológicas}
	
	\subsection{Intuição geométrica}
	
	Esta secção é dedicada a dar intuição geométrica para motivar os teoremas que se seguem. Vamos, a título de exemplo, partir de um pol específico a duas dimensões na forma canónica.
	
	Considere-se o pol
	
	\[
	P =
	\begin{cases}
	\max\limits_{(x,y)} 2x + y\\
	x + 2y \leq 8\\
	x - y \leq 2 \\
	x, y \geq 0
	\end{cases}
	\]
	
	Vamos representar, no plano cartesiano, a região admissível $X_P$ e o vetor $c^T = (2,1)$.
	
	\begin{tikzpicture}
	\draw[->] (-2, 0) -- (6, 0) node[anchor=south] {x};
	\draw[->] (0, -1) -- (0, 5) node[anchor=west] {y};
	\filldraw[color=red, fill=blue!80, very thick] (0,0) node[anchor=north west, color=black]{0} -- (0,4) node[anchor=east, color=black]{(0,4)} -- (4,2) node[anchor=north west, color=black]{\,(4,2)} -- (2,0) node[anchor=north, color=black]{(2,0)} -- (0,0);
	
	\draw[->, color=green, very thick] (4,2) -- (6,3) node[anchor=west, color=black]{$c^T = (2,1)$};
	\draw[very thick, color=green] (3.5,3) -- (4.5,1);
	\end{tikzpicture}
	
	Uma interpretação para o conjunto solução $S_P$ é o conjunto de vetores admissíveis que maximiza o produto escalar com $c^T$. Intuitivamente, os vetores que `estão o mais possível na direção de $c^T$'.
	
	É possível ver (isto em termos intuitivos, claro) que não há nenhum vetor admissível `para lá' da linha verde desenhada, o que indica que o conjunto solução será o conjunto $\{(4,2)\}$.
	
	Neste caso, o conjunto solução tem apenas um elemento, mas é concebível uma situação em que tivesse mais. Por exemplo, se o vetor $c^T$ estivesse perpendicular ao lado $(0,4)-(4,2)$.
	
	Isto é só especulação com base num desenho, mas dita especulação pode ser útil para motivar conclusões.
	
	Por exemplo, com base nestas considerações, seria de esperar que, por exemplo, em geral o conjunto solução estará contido na fronteira do conjunto admissível, o que quer que isso signifique (a área marcada a vermelho na imagem). Também é fácil uma pessoa convencer-se que tem necessariamente de conter pelo menos um dos vértices -- novamente, o que quer que isso signifique.
	
	Seria também de esperar que no caso de o conjunto admissível ser limitado e não vazio, existe, necessariamente, solução. E, já agora, que o conjunto admissível é convexo.
	
	Todos estes factos aleatórios serão provados, e mais, e alguns deles provar-se-ão úteis na procura de formas de resolver problemas de otimização.
	
	\subsection{Factos introdutórios}
	
	Recorde-se o leitor da definição de interior, fronteira e exterior de conjunto, e da definição de conjunto aberto e fechado.
	
	\begin{prop}
		Fixo um pol $P$, $X_P$ e $S_P$ são fechados.
	\end{prop}
	
	\begin{proof}
	Primeiramente, mostre-se que $X_P$ é fechado.
	
	Recorde-se que interseções finitas de fechados são fechadas. De seguida, note-se que, dado um pol $P$ (que podemos supor sem perda de generalidade estar na forma semicanónica) o seu conjunto admissível é o conjunto de $x \in \R^n$ que satisfazem as condições
	
	\[a_i x \leq b_i \text{ para $i = 1, 2, \cdots, m$}\]
	
	Onde os $a_i$ são vetores-linha, e os $b_i$ são escalares.
	
	Ou seja, $X_P$ é a interseção
	
	\[\bigcap_{i=1}^m \{\,x \in \R^n \mid a_i x \leq b_i \,\}\]
	
	Repare-se agora que $\{\,x \in \R^n \mid a_i x \leq b_i \,\}$ é a pré-imagem sob uma função linear (e então contínua) de um conjunto fechado ($]-\infty, b_i]$), pelo que usando Cálculo II se conclui que estes conjuntos são todos fechados, pelo que $X_P$ é fechado.
	
	Para mostrar que $S_P$ é fechado, repare-se que há dois casos: ou $S_P$ é vazio, e então fechado; ou contém pelo menos um elemento $s$. Neste último caso, $S_P = X_P \cap \{\,x \mid c^T x \geq c^T s\,\}$ por definição, e isto é a interseção de dois conjuntos fechados, e então fechado.
	\end{proof}
	
	Recorde-se do teorema de Weierstrass, que deverá ter dado em Cálculo II: uma função contínua definida num subconjunto fechado, limitado e não-vazio de $\R^n$ (compacto não-vazio) tem máximo e mínimo. Assim sendo, conseguimos facilmente o seguinte corolário:
	
	\begin{prop}
	Fixo um problema de otimização linear $P$, se $X_P$ é limitado e não-vazio, $S_P$ é não-vazio.
	\end{prop}
	\begin{proof}
	Basta reparar que se $X_P$ é limitado, por ser fechado (ver acima), é compacto. Assim sendo, como a função objetivo é linear e então contínua, tem valor máximo algures em $X_P$. Os valores maximizantes formam o conjunto solução.
	\end{proof}
	
	Mais um detalhe topológico simples, antes de avançarmos para coisas mais específicas: denomine-se o interior de um conjunto $X$ por $\interior X$, e a sua fronteira por $\partial X$.
	
	\begin{prop}
	Se $P$ é um pol de função objetivo não-nula, $S_P \subseteq \partial X_P$.
	\end{prop}
	
	\begin{proof}
	Basta reparar que, se $x \in \interior  X_P$, então há uma vizinhança $\varepsilon$ de $x$ contida em $X_P$. A ideia é que, então, se $c^T$ não é o vetor nulo, podemos `avançar um bocadinho na direção de $c^T$'. Em termos mais precisos, o vetor $y = x + \frac \varepsilon 2 \frac{c^T}{\lvert c^T \rvert}$ pertence a $X_P$, por distar de $x$ menos de $\varepsilon$. Mas este vetor é `melhor' que $x$, pois $cy = cx + c \frac \varepsilon 2 \frac {c^T}{\lvert c^T \rvert} = cx + \frac{\varepsilon |c^T|} 2 > cx$. Logo, nenhum $x \in \interior X_P$ pertence a $S_P$, donde $S_P$ tem que estar todo contido na fronteira.
	\end{proof}
	
	Só um último detalhe, antes de avançarmos para outras pastagens: uma caraterização do interior de $X_P$ para um pol $P$, que se assume spdg estar na forma semicanónica.
	
	\begin{prop}
	Seja $P$ o pol
	
	\[
	P =
	\begin{cases}
	\max\limits_x cx\\
	Ax \leq b
	\end{cases}
	\]
	
	e assuma-se que nenhuma linha da matriz $A$ é composta somente de zeros. (Isto é denotado por \emph{$A$ é uma matriz própria})
	
	Então, $\interior X_P$ é o conjunto de $x > 0$ tal que $Ax < b$. Consequentemente, $\partial X_P$ é o conjunto de $x \geq 0$ tal que $Ax \leq b$ e há pelo menos uma igualdade.
	\end{prop}
	
	\begin{proof}
	Relembre-se da seguinte identidade: $\interior (A \cap B) = \interior A \cap \interior B$. Mais geralmente, o interior de uma interseção finita é a interceção dos interiores.
	
	Como tal, sendo que
	
	\[X_P = \bigcap_{i=1}^m \{\,x \in \R^n \mid a_i x \leq b_i \,\}\]
	
	onde $a_i$ representa a $i$-ésima linha da matriz $A$,
	
	Para determinar $\interior X_P$ basta determinar os interiores destes conjuntos. Afirmamos que o interior do $i$-ésimo conjunto é
	
	\[\interior \{\,x \in \R^n \mid a_i x \leq b_i \,\} = \{\,x \in \R^n \mid a_i x < b_i \,\}\]
	
	Para justificar a inclusão $\supseteq$, note-se que o lado direito é um conjunto aberto contido em $\{\,x \in \R^n \mid a_i x \leq b_i \,\}$, e então contido no seu interior.
	
	Para justificar a inclusão $\subseteq$, suponha-se que $x$ pertence ao lado esquerdo. Mostrar-se-á que pertence ao lado direito.
	
	Por pertencer ao interior daquele conjunto, sabemos que, para $y$ numa vizinhança $\varepsilon$ de $x$, $a_i y \leq b_i$. Considere-se, então, $y = x + \frac \varepsilon 2 \frac{a_i^T}{\lvert a_i^T \rvert}$. Por hipótese, $a_i y \leq b_i$. Mas tem-se também $a_i y = a_i x + a_i \frac \varepsilon 2 \frac{a_i^T}{\lvert a_i^T \rvert} = a_i x + \frac {\varepsilon \lvert a_i^T \rvert} 2 > a_i x$, donde se conclui $a_i x < a_i y \leq b_i$. Logo, $x$ pertence ao lado direito.
	\end{proof}
	
	\section{Álgebra Linear}
	
	\subsection{Intuição geométrica}
	
	Queremos agora investigar a intuição que o conjunto solução contém sempre um vértice, no caso de este ser não-vazio.
	
	Para este estudo, no entanto, é útil considerar, em vez do pol na forma canónica, o pol na forma padrão, visto que é mais fácil caraterizar o significado de `vértice' neste contexto.
	
	Considere-se o seguinte pol:
	
	\[
	P =
	\begin{cases}
	\min\limits_{(x,y,z)} 2x + y + 5z\\
	3x + 6y + 4z = 24\\
	x, y, z \geq 0
	\end{cases}
	\]
	
	O desenho do conjunto admissível é o seguinte:
	
	\resizebox{\columnwidth}{!}{
	\begin{tikzpicture}
	\draw[->] (0, 0) -- (10, 0) node[anchor=south] {x};
	\draw[->] (0, 0) -- (-5, -5) node[anchor=north west] {y};
	\draw[->] (0, 0) -- (0, 7) node[anchor=west] {z};
	\filldraw[color=red, pattern color=blue, very thick, pattern = north east lines] (8,0) node[anchor=north, color=black]{(8,0,0)} -- (-4,-4) node[anchor=east, color=black]{(0,4,0)} -- (0,6) node[anchor=west, color=black]{(0,0,6)} -- (8,0);
	\end{tikzpicture}
	}
	
	Temos agora de definir o conceito de vértices no contexto de otimização linear, mas a figura deve tornar evidente que se tratam dos pontos com o maior número de coordenadas 0.
	
	Vamos primeiro tentar definir o que queremos dizer por `vértice'.
	
	\subsection{Geometria}
	
	Dado o nosso contexto, é possível atribuir significado aos termos `vértice', `aresta', `lado' e assim por diante. No entanto, por simplicidade, vamos apenas definir o significado de vértice.
	
	Relembre-se da definição de plano em Álgebra Linear. Um plano em $\R^n$ é um conjunto da forma $U = a + V$, em que $a \in \R^n$ e $V$ é um subespaço vetorial de $\R^n$. Dizemos que $U$ é um plano de dimensão $m$ se $V$ tiver dimensão $m$.
	
	Um \emph{hiperplano em $\R^n$} é um plano de dimensão $n-1$.
	
	Relembre-se que os planos podem também ser caraterizados como conjuntos-solução de equações lineares $Ax = b$ a $n$ dimensões. A dimensão do plano é, então, a dimensão do núcleo da matriz $A$. Relembre-se também que, se um plano $a + V$ é definido por $Ax = b$, as colunas de $A^T$ geram o espaço $V^\perp$.
	
	Uma particularidade dos hiperplanos é que o seu espaço ortogonal tem dimensão 1. Assim sendo, um hiperplano pode sempre ser expresso da forma $w \cdot x = b$, para algum vetor não-nulo $w$ e escalar $b$. Para mais, quaisquer duas expressões que representem o mesmo hiperplano têm vetores colineares.
	
	Isso permite fazer a seguinte definição:
	
	\begin{definition}
	Dado um plano $H$ definido por $w \cdot x = b$, dizemos que este separa o espaço em dois semiespaços: $H^+$, definido por $w \cdot x > b$ e $H^-$, definido por $w \cdot x < b$.
	
	Devido à observação anterior, esta definição não depende do $w$ escolhido, a menos de sinal. Ou seja, $H^+$ e $H^-$ não dependem de $w$, exceto para distinguir qual é qual.
	
	Diz-se que $H$ \emph{separa o espaço em $H^+$ e $H^-$}.
	\end{definition}
	
	Estamos agora prontos para caraterizar o que queremos dizer com `vértice'.
	
	\begin{definition}
	Dado um conjunto $S \subseteq \R^n$, dizemos que $v \in S$ é um \emph{vértice exterior de $S$} se existem $w \in \R^n$ e $b \in \R$ tal que $w \cdot v = b$ e $w \cdot x < b$ para todo $x \in S \setminus \{v\}$.
	\end{definition}
	
	Esta definição é equivalente à seguinte:
	
	\begin{definition*}
	Nas mesmas condições, dizemos que $v \in S$ é vértice exterior se existe um hiperplano que separa $v$ do resto do conjunto. Ou seja, se existe um hiperplano $H$ que separe o espaço em $H^+$ e $H^-$ tal que $v \in H$ e $S \setminus \{v\} \subseteq H^-$.
	\end{definition*}
	
	Repare-se que é óbvio que esta segunda definição implica a primeira, devido à forma como se definiram os subespaços. No entanto, a outra implicação não é tão óbvia, pois repare-se que não é exigido do vetor $w$ que seja diferente de zero. Este detalhe tem pequena relevância, mas permitir-nos-á simplificar uma prova algures no futuro. Claro que se o vetor $w$ for $0$, o conjunto $S$ contém apenas o ponto $v$, e portanto a segunda definição aplica-se com qualquer hiperplano que passe por $v$. Isto mostra que as duas definições são equivalentes.
	
	Para exemplificar esta definição, considere-se o conjunto:
	
	\[S = \{\, (x,y) \mid x,y \geq 0, y \leq 2+x, y \geq -2 + 2x\,\}\]
	
	Em baixo, está uma representação da afirmação: ``o ponto $(4, 6)$ é vértice de $S$, pois pondo $w = (1,1)$ e $b = 10$, tem-se $w \cdot (4,6) = b$ e para todos os pontos $(x,y)$ em $S \setminus \{(4,6)\}$ tem-se $w \cdot (x,y) < b$''.
	
	\begin{tikzpicture}
	\draw[->] (-2, 0) -- (6, 0) node[anchor=south] {x};
	\draw[->] (0, -1) -- (0, 8) node[anchor=west] {y};
	\filldraw[color=blue, fill=blue!80, very thick] (0,0) -- (0,2) -- (4,6) -- (1,0) -- (0,0);
	
	\draw[->, color=purple, very thick] (4,6) -- (5,7) node[anchor=west, color=black]{$w = (1,1)$};
	
	\draw[color=green, very thick] (2.5, 7.5) -- (5.5, 4.5);
	\fill[pattern=north east lines, pattern color=green] (-1.5, -0.5) -- (-1.5, 7.5) -- (2.5, 7.5) -- (5.5, 4.5) -- (5.5, -0.5);
	\end{tikzpicture}
	
	A reta verde representa o hiperplano (reta) $H$ definido por ${w \cdot (x,y) = 10}$, e a área pintada a verde representa o semiespaço $H^-$ definido por ${w \cdot (x,y) < 10}$. Repare-se que todo o conjunto $S$, exceto o ponto $(4,6)$, está contido neste semiespaço.
	
	\subsection{Vértices de conjuntos admissíveis}
	
	Estamos agora prontos para tentar caraterizar os vértices de conjuntos admissíveis do pol na forma padrão. A seguinte proposição ser-nos-á útil:
	
	\begin{prop} \label{vertexvpmh}
	Seja $S$ um conjunto em $\R^n$. As seguintes quatro afirmações não podem ser todas verdade ao mesmo tempo:
	
	\begin{itemize}
	\item $v$ é vértice de $S$
	\item $v + h \in S$
	\item $v - h \in S$
	\item $h \neq 0$
	\end{itemize}
	\end{prop}
	
	\begin{proof}
	Para mostrar isto, vamos supor que três destes são verdade, e mostramos que o quarto tem que ser falso. Nomeadamente, suponhamos que $v$ é vértice de $S$, $v+h \in S$ e $h \neq 0$.
	
	Assim sendo, visto que $v$ é vértice, existem $w$ e $b$ tal que $w \cdot v = b$ e $w \cdot x < b$ para todo $x \in S$ diferente de $v$. Em particular, $w \cdot (v + h) < b$. Mas então $w \cdot h < 0$, donde $w \cdot (v-h) > b$, e então concluímos que $v-h$ não pode pertencer a $S$.
	\end{proof}
	
	Na caraterização de vértices é-nos útil falar das coordenadas positivas de vetores, e das colunas correspondentes da matriz $A$. (A razão para isto será evidente em breve.) Assim sendo, adotamos as seguintes notações:
	
	\begin{notacao}
	No nosso contexto, considere-se o seguinte pol a $n$ dimensões:
	
	\[
	\begin{cases}
	\min\limits_x cx\\
	Ax = b\\
	x \geq 0
	\end{cases}
	\]
	
	Em que $x$ é um vetor em $\R^n$, $A$ é uma matriz $m \times n$ e $b$ é um vetor em $\R^m$. Por conveniência, seja $N$ o conjunto $\{1,2,\cdots,n\}$.
	
	Fixo um vetor $x$:
	
	Usamos $P_x$ para denotar o conjunto de índices $i$ tal que $x_i > 0$.
	
	Dado um conjunto de índices $B \subseteq N$, $x_B$ representa o vetor em $\R^{\#B}$ obtido de $x$ resultante apenas das coordenadas em $B$. Por exemplo, se $B$ é o conjunto $\{2, 5, 9\}$, $x_B$ seria o vetor em $\R^3$ dado por $(x_2, x_5, x_9)$. Análogamente, $A_B$ denota a submatriz de $A$ considerando apenas as colunas com índices em $B$.
	\end{notacao}
	
	Antes de fazer a prova `a sério', será feito um esboço de prova para suportar a afirmação feita há bocado: ``os vértices tratam-se dos pontos com o maior número de coordenadas 0''. Este esboço de prova vai-nos servir para motivar a caraterização (completa) dos vértices de $X_P$, para $P$ na forma padrão.
	
	Seja $k$ a caraterística da matriz $A$. Recorde-se que a caraterística de uma matriz equivale à dimensão do seu espaço de colunas. Afirmamos que qualquer vértice de $X_P$ tem no máximo $k$ coordenadas maiores que zero.
	
	Para justificar isto, considere-se um $x \in X_P$ tal que $\# P_x > k$. Queremos justificar que este não pode ser um vértice, e vamos fazer isso usando a proposição \ref{vertexvpmh}.
	
	Nomeadamente, vamos arranjar um $h$ diferente de zero tal que $x+h$ e $x-h$ pertencem a $X_P$, o que, devido a esta proposição, implica que $x$ não é vértice de $X_P$. Para arranjar este $h$, repare-se nas condições que ele tem que obedecer:
	
	\begin{itemize}
	\item $h \neq 0$
	\item $Ah = 0$
	\item $x \pm h \geq 0$
	\end{itemize}
	
	Em particular, a última condição diz-nos algo importante: para todo $i$, se $x_i = 0$, $h_i$ tem que ser $0$. Caso contrário, um daqueles dois vetores teria a $i$-ésima casa negativa.
	
	Assim sendo, isso limita as nossas escolhas de $h$. Este tem necessariamente as casas em $N \setminus P_x$ igual a zero, pelo que resta definir $h_{P_x}$.
	
	Repare-se, agora, que, sob estas condições, $Ah = A_{P_x} h_{P_x}$, e visto que $\# P_x$ é maior do que a caraterística de $A$, a matriz $A_{P_x}$ é singular, e então existe um $h$ diferente de zero (chamemos-lhe $\tilde h$) que satisfaz $A \tilde h = 0$ e $\tilde h_i$ é igual a zero para $i \not \in P_x$. Falta apenas assegurar que $x \pm \tilde h \geq 0$, mas isto pode não ser verdade.
	
	Para o passo final, repare-se que estas duas condições ($A \tilde h = 0, h \neq 0$) não deixam de ser verdade se multiplicarmos $\tilde h$ por um escalar $t$ diferente de zero.
	
	A ideia é a seguinte: para todo $i \in N$ tem-se um dos dois casos: ou $x_i = 0$ e então, como $\tilde h_i = 0$, $x_i + t \tilde h_i \geq 0$ independentemente de $t$, ou $x_i > 0$ e então $x_i + t \tilde h_i \geq 0$ para qualquer $t$ de módulo pequeno o suficiente. Considere-se o $m$ mínimo destes módulos, e tem-se que, para $t$ de módulo menor que $m$, $x \pm t \tilde h_i \geq 0$. Pondo $h = t \tilde h_i$, obtemos o resultado desejado: $x \pm h \in X_P$ e $h \neq 0$. Assim sendo, $x$ não pode ser vértice de $X_P$.
	
	Repare-se que neste esboço de prova, a condução que $\#P_x > k$ foi apenas usada para justificar que as colunas de $A_{P_x}$ são linearmente dependentes. Acontece que esta nova condição, mais fraca, não é só suficiente como necessária. Provamos, agora, a
	
	\begin{prop}
	Seja $P$ o seguinte pol na forma padrão:
	
	\[
	\begin{cases}
	\min\limits_x cx\\
	Ax = b\\
	x \geq 0
	\end{cases}
	\]
	
	Então, $x \in X_P$ é vértice de $X_P$ sse $A_{P_x}$ tem as suas colunas todas linearmente independentes.
	\end{prop}
	
	\begin{proof}
	Primeiro, a parte que já está feita: se as colunas de $A_{P_x}$ não forem linearmente independentes, $x$ não é vértice. ($\rightarrow$) Construa-se $\tilde h \neq 0$ tal que $\tilde h_i = 0$ para $i \not \in P_x$, e $\tilde h_{P_x}$ é tal que $A_{P_x} \tilde h_{P_x} = 0$. Isto pode ser feito porque, por hipótese, a equação $A_{P_x} h = 0$ tem soluções não-triviais.
	
	Considere-se a função $f(t) = x + t \tilde h$. Pretendemos arranjar $t$ tal que \allowbreak ${f(t), f(-t) \geq 0}$, pois assim ter-se-ia que $x \pm t \tilde h \in X_P$ com $t \tilde h \neq 0$, o que justifica $x$ não ser vértice conforme a prop \ref{vertexvpmh}.
	
	Para este objetivo, decomponha-se esta função em $f_{P_x}(t)$ e $f_{N \setminus P_x}(t)$. A segunda é sempre $\geq 0$ (por ser sempre igual a zero), pelo que basta examinar a primeira.
	
	Mas repare-se que, como $f$ é contínua (soma de uma constante com uma função linear), a sua restrição às coordenadas de $P_x$ também. Para mais, note-se que $f_{P_x}(0) > 0$, pelo que, por continuidade, existe uma vizinhança $\varepsilon$ de zero onde $f_{P_x}(t) > 0$, e então $f(t) \geq 0$. Escolhendo um $t$ diferente de zero nesta vizinhança, por exemplo, $t = \varepsilon/2$, temos $f(t), f(-t) > 0$, como desejado.
	
	A partir daqui, aplicando o raciocínio anterior, usando $h = t \tilde h$, concluimos que $x$ não é vértice de $X_P$.
	
	Agora, a segunda parte ($\leftarrow$): mostrar que se $x \in X_P$ e as colunas de $A_{P_x}$ são linearmente independentes, então $x$ é vértice.
	
	Para fazer isto, considere-se o vetor $w$ definido como
	
	\[w_i = (\text{$0$ se $i \in P_x$, $-1$ caso contrário})\]
	
	E o escalar $0$.
	
	Então, claramente temos $w \cdot x = 0$.
	
	Agora, mostre-se que, para todo $y \in X_P \setminus \{x\}$ se tem $w \cdot y < 0$. Isto é claramente equivalente (dado que $y \in X_P$ e então $y \geq 0$) a mostrar que $y_i \neq 0$ para algum $i \not \in P_x$. Ou seja, que $P_y \not \subseteq P_x$.
	
	Para este efeito, suponha-se que $y \in X_P$ e $P_y \subseteq P_x$. Vamos mostrar que $y = x$.
	
	Se $y \in X_P$, temos $Ay = b$. Mas como $P_y \subseteq P_x$, temos que $Ay = A_{P_x} y_{P_x} = b$. Mas como as colunas de $A_{P_x}$ são linearmente independentes, a solução do sistema $A_{P_x} v = b$, se existir (neste caso existe), é única. Assim sendo, $y_{P_x} = x_{P_x}$, e como as outras coordenadas de $y$ são todas 0, como as de $x$, temos $y = x$, como queriamos demonstrar.
	
	Logo, para $y \in X_P \setminus \{x\}$ tem-se sempre $w \cdot y < 0$, como se pretendia demonstrar, e então $x$ é vértice de $X_P$.
	\end{proof}
	
	Estamos agora em condições de justificar o seguinte:
	
	\begin{prop}
	Seja $P$ o pol
	
	\[
	\begin{cases}
	\min\limits_x cx\\
	Ax = b\\
	x \geq 0
	\end{cases}
	\]
	
	Se a função $x \mapsto cx$ é minorada em $X_P$, para todo $v \in X_P$ existe um vértice $z$ de $X_P$ tal que $cz \leq cv$.
	\end{prop}
	
	\begin{proof}
	Se $v \in X_P$, o conjunto $\{\,x \in X_P \mid cx \leq cv\,\}$ é não-vazio. A função $x \mapsto \#P_x$ vai deste conjunto para $\N_0$, pelo que é minimizada nalgum ponto. Chamemos-lhe $z$. Mostraremos que $z$ é vértice de $X_P$.
	
	Para fazer isto, usaremos a caraterização feita há pouco: mostraremos que as colunas de $A_{P_z}$ são linearmente independentes, o que justifica que $z$ é vértice.
	
	A prova faz-se por contrarecíproco. Para mostrar que um $z$ que minimize $\#P_z$ é vértice, supõe-se que não é vértice e mostra-se que não minimiza $\#P_z$.
	
	Suponha-se, então, que $z$ não é vértice, ou seja, que as colunas de $A_{P_z}$ são linearmente dependentes. Vamos arranjar um elemento $\tilde z$ de $S_P$ tal que $\#P_{\tilde z} < \#P_z$
	
	Sabendo que as colunas de $A_{P_z}$ são linearmente dependentes, existe solução não-trivial do sistema $A_{P_z} x = 0$. Seja $\tilde h$ tal que $\tilde h_i = 0$ para $i \not \in P_z$, e $\tilde h_{P_z}$ é solução não-trivial de $A_{P_z} \tilde h_{P_z} = 0$. Suponha-se, sem perda de generalidade, que $c \tilde h \geq 0$. Podemos fazer isto, pois, se não for o caso, substitua-se $\tilde h$ por $-\tilde h$.
	
	Repare-se que, como consequência, podemos também assumir que $\tilde h$ tem pelo menos uma coordenada positiva. Isto pois, se $c \tilde h = 0$ podemos simplesmente trocar $\tilde h$ por $- \tilde h$, e se $c \tilde h > 0$, este vetor tem de certeza pelo menos uma coordenada positiva. Caso contrário, todos os vetores da forma $x - t \tilde h$ seriam admissíveis (verifique) e, fazendo $t \rightarrow +\infty$, temos que a função objetivo toma valores arbitráriamente pequenos em $X_P$, o que contraria o enunciado.
	
	Assim sendo, temos as seguinte duas hipóteses sobre $\tilde h$: $c \tilde h \geq 0$ e $\tilde h$ tem pelo menos uma coordenada positiva.
	
	Assim sendo, para todo o $t$ tal que $z - t \tilde h$ é admissível, este tem pontuação melhor ou igual do que $v$.
	
	A ideia é, agora, escolher $t$ de modo a que $z - t \tilde h$ tenha menos coordenadas positivas do que $z$, mas continue admissível. Para fazer isto, efetivamente, o que se faz é `subir o valor de $t$ continuamente até uma das coordenadas ser zero'. (Isto vai ser formalizado já a seguir.) Essa coordenada ficará zero, e nenhuma coordenada que fosse antes zero deixou de o ser, pelo que o número de coordenadas positivas diminuiu... E o vetor continua a ser admissível porque a condição $Ax = b$ nunca deixa de ser verdade, e a condição $x \geq 0$ continua a ser verdade porque não deixamos nenhuma coordenada ir para os negativos.
	
	Em termos formais: considere-se o conjunto $T = \{\,\frac {z_i} {\tilde h_i} \mid i \in P_{\tilde h}\,\}$. Este conjunto é não-vazio (assumimos que $\tilde h$ tinha pelo menos uma coordenada positiva), finito e todos os seus elementos são positivos (pois $P_{\tilde h} \subseteq P_z$). Considere-se, então, o seu mínimo, $\alpha$. Vamos ver que o vetor $\tilde z = z - \alpha \tilde h$ é admissível e tem menos coordenadas positivas do que $z$.
	
	Como já vimos, obedece à condição $A \tilde z = b$, pelo que basta certificarmo-nos que $\tilde z \geq 0$. Fazemos isto coordenada a coordenada.
	
	Examine-se $\tilde z_i = z_i - \alpha \tilde h_i$. Se $i \not \in P_{\tilde h}$, claramente $\tilde z_i \geq z_i \geq 0$. Pelo outro lado, se $i \in P_{\tilde h}$, temos que $\alpha \leq z_i/\tilde h_i$, e então $-\alpha \geq -z_i/\tilde h_i$, donde $z_i - \alpha \tilde h_i \geq z_i - \frac {z_i}{\tilde h_i} h_i = 0$. Como se pretendia demonstrar.
	
	Assim sendo, o vetor $\tilde z$ é admissível. Já vimos que tem pontuação melhor ou igual do que $v$. E, finalmente, mostramos que tem menos coordenadas positivas do que $z$.
	
	Repare-se que para todo $i$, se $z_i = 0$ então $\tilde h_i = 0$, pelo que nenhumas coordenadas positivas `novas' aparecem. Pelo outro lado, pelo menos uma coordenada positiva fica nula. Nomeadamente, visto que $\alpha$ é o valor mínimo do conjunto $\{\,\frac {z_i} {\tilde h_i} \mid i \in P_{\tilde h}\,\}$, é da forma $\alpha = z_j / \tilde h_j$ para algum $j$. Vamos examinar a coordenada $j$ de $\tilde z$.
	
	Sabemos que $z_j > 0$. Mas também temos que $\tilde z_j = 0$, pois $\tilde z_j = z_j - \alpha \tilde h_j = z_j - \frac {z_j}{\tilde h_j} \tilde h_j = 0$. Isto mostra, então, que $\tilde z$ tem pelo menos uma coordenada positiva a menos que $z$, o que mostra que $\tilde z$ é o elemento que pretendiamos arranjar. Isto conclui a nossa prova.
	\end{proof}
	
	Um corolário útil, cuja demonstração é deixada como exercício ao leitor:
	
	\begin{prop}
	Se $S_P$ é não-vazio, contém pelo menos um vértice de $X_P$.
	\end{prop}
	
	Outro corolário:
	
	\begin{prop}
	Dado um pol $P$ na forma padrão, se o conjunto admissível não for vazio, tem pelo menos um vértice.
	\end{prop}
	
	\begin{proof}
	Já provámos que, dado um elemento $v$ do conjunto admissível, existe um vértice de pontuação menor ou igual que $v$. Em particular, isto significa que existe um vértice.
	\end{proof}
	
	\subsection{O método dos vetores básicos}
	
	Estamos perto de ter um método primitivo de resolução de problemas de otimização linear. Dado um pol, podemos sempre pô-lo na forma padrão. Uma vez nesta forma, podemos examinar os vértices.
	
	A ideia é que, em principio (vamos provar isto) haverá um número finito de vértices. Se o problema tiver solução, o vértice de melhor pontuação será solução. Daí em diante, é preciso verificar que, de facto, $S_P$ é não-vazio. Os detalhes de tal verificação serão feitos mais tarde: por agora, arranjaremos um método de encontrar os vértices de $X_P$ para $P$ na forma padrão.
	
	A ideia é a seguinte: considere-se um conjunto de índices $B \subseteq N$ tal que $A_B$ tem colunas linearmente independentes. Então, a equação $A_B x = b$ tem zero ou uma soluções. Isto dá-nos o primeiro método, mais primitivo possível, de encontrar os vértices de $X_P$:

	\begin{lstlisting}[mathescape=true, keepspaces=true]
Para todo o $B \subseteq N$:
  Se as colunas de $A_B$ são linearmente independentes:
    Resolver o sistema $A_B x = b$.
    Se este tiver solução $x$:
      Se $x \geq 0$:
        Definir $v \in \R^n$ de modo a que $v_B = x$ e $v_{N\setminus B} = 0$
        Adicionar $v$ à lista de vértices.
	\end{lstlisting}
	
	Este método funciona, mas tem um enorme problema: se estamos em dimensão $n$, temos $2^n$ sistemas para resolver. Procuramos, então, uma forma mais computacionalmente eficiente de fazer as coisas.
	
	Lembre-se da Álgebra Linear que qualquer conjunto linearmente independente de vetores pode ser extendido a uma base. Isso é refletido na seguinte proposição:
	
	\begin{prop}
	Dada uma matriz $A$ $m \times n$ e um conjunto de índices $B \subseteq N$, dizemos que $B$ é uma \emph{base de índices} se as colunas de $A_B$ formam uma base de $\im A$.
	
	Se $B \subseteq N$ é tal que as colunas de $A_B$ são linearmente independentes, existe uma base de índices $B' \supseteq B$.
	\end{prop}
	
	\begin{proof}
	Esta prova será feita mostrando que, se $B$ é tal que $A_B$ tem colunas linearmente independentes mas $B$ não é base de índices, é possível adicionar um elemento $b$ a $B$ tal que as colunas de $B \cup \{b\}$ continuem a ser linearmente independentes. Feito isto, o algoritmo para extender $B$ a uma base passa a ser repetir este processo até se ter um conjunto de tamanho igual à caraterística de $A$.
	
	Assim sendo, basta mostrar este passo. Suponha-se que $B$ está sob estas condições, mas não é uma base de índices. Então, tem de existir um vetor-coluna de $A$ que não está em $\im A_B$. Isto pois, caso contrário, $\im A = \im A_B$, e $B$ seria uma base de índices. Assim sendo, escolha-se uma tal coluna, e seja $b$ o seu índice. As colunas de $A_{B \cup \{b\}}$ são linearmente independentes, como o leitor poderá facilmente verificar, o que termina a nossa prova.
	\end{proof}
	
	Dizemos que um vetor $x$ é básico (no contexto de um pol $P$ na forma padrão) se $P_x$ está contido numa base de índices $B$.
	
	O que a proposição anterior mostra é que todo vértice é básico (pegue-se em $P_x$ e extenda-se a uma base). Pelo outro lado, é fácil ver que qualquer vetor básico $x$ tem a propriedade que as colunas de $A_{P_x}$ são linearmente independentes. Temos, então, que os vetores admissíveis básicos e os vértices de $X_P$ são os mesmos.
	
	As bases de índices têm uma propriedade notável: no contexto do pol
	
	\[
	P =
	\begin{cases}
	\min\limits_x cx\\
	Ax = b\\
	x \geq 0
	\end{cases}
	\]
	
	Se a equação $Ax = b$ é impossível, este problema trivialmente não tem solução, visto que o conjunto admissível é vazio. Se isto acontece, dizemos que o pol $P$ é \emph{vazio}.
	
	Pelo outro lado, se $b \in \im A$, dada um base de índices $B$, temos que o sistema de equações $A_B x = b$ tem uma e só uma solução.
	
	Outra observação importante é que as bases de índices de $A$ têm todas tamanho igual à dimensão do espaço de colunas de $A$, ou seja, a sua caraterística.
	
	Assim sendo, considere-se o seguinte (novo) algoritmo.
	
	\begin{lstlisting}[mathescape=true, keepspaces=true]
Verificar se o problema é vazio.
Caso afirmativo, a lista de vértices é vazia.
Caso contrário:
  Seja $k$ a caraterística de $A$.
  Para todo o $B \subseteq N$ tal que $\#B = k$:
    Se as colunas de $A_B$ são linearmente independentes:
      Resolver o sistema $A_B x = b$.
      Se $x \geq 0$:
        Definir $v \in \R^n$ de modo a que $v_B = x$ e $v_{N\setminus B} = 0$
        Adicionar $v$ à lista de vértices.
	\end{lstlisting}
	
	Este algoritmo é muito mais económico do que o anterior. Ao passo que no outro havia $2^n$ possibilidades para verificar, aqui é preciso apenas verificar $\binom{n}{k}$ conjuntos, onde $k$ é a caraterística de $A$.
	
	Para obter um candidato a solução basta, agora, calcular os vértices de $X_P$, $v_1, v_2, \cdots, v_\ell$, calcular $c v_i$ para $i = 1, \cdots, \ell$ e ver qual destes é menor. Pegando, então, num vértice $v_j$ que minimize $c v_j$, se o conjunto solução for não-vazio, contém $v_j$. Precisamos, então, de um método de verificar se um vértice é solução. É a isso que nos dedicamos de seguida. Mas antes, algumas proposições extremamente úteis:
	
	\begin{prop}
	Dado um pol $P$ na forma padrão, se a função-objetivo for minorada e o conjunto admissível não for vazio, o pol tem solução.
	\end{prop}
	
	\begin{proof}
	Já vimos que $X_P$ tem pelo menos um vértice, e já vimos que tem um número finito. Considere-se então um vértice $z$ de menor pontuação. Mostraremos que este é solução.
	
	Fixe-se um ponto $x \in X_P$ arbitrário. Como a função-objetivo é minorada, sabemos que existe um vértice $\tilde z$ de pontuação menor ou igual do que $x$, e sabemos que $z$ tem pontuação menor ou igual do que $\tilde z$. Logo, a pontuação de $z$ é menor igual do que a de $x$. Como $x$ é arbitrário, $z$ é solução.
	\end{proof}
	
	\begin{prop}
	Dado um pol $P$ de minimização, se a função-objetivo for minorada e o conjunto admissível não for vazio, o pol tem solução.
	
	Da mesma forma, se $P$ for de maximização, a função-objetivo for majorada e  o conjunto admissível não for vazio, o pol tem solução.
	\end{prop}
	
	\begin{proof}
	%todo this is awful
	Dado um pol arbitrário, sabemos, pelos conteúdos da primeira secção, que podemos traduzi-lo linearmente para a forma padrão com uma certa função linear sobrejetiva $f$ que é monótona para a pontuação.
	
	Isto significa que se a função objetivo é minorada/majorada no pol original, também será minorada no pol traduzido. Isto pois repare que, para as traduções que fizemos, se começarmos com um pol de minimização, a pontuação de $f(x)$ é igual à pontuação de $x$. Caso contrário, é a sua negação. Assim sendo, se $M$ é minorante da pontuação no pol traduzido, $M$ será minorante, ou $-M$ será majorante, no original, consoante se este é de minimização ou maximização.
	
	Logo, se a função-objetivo for majorada/minorada, será minorada no pol traduzido, pelo que este terá solução, pelo que o pol original tem solução.
	\end{proof}
	
	\section{Cones}
	
	\subsection{Introdução}
	
	No estudo do pol na forma padrão e canónica, é muito comum ver expressões da forma $Ax$ com $x \geq 0$. Ou seja, combinações lineares não-negativas das colunas de $A$.
	
	Este fenómeno é tão comum que tem um nome.
	
	\begin{definition}
	Sejam $a_1, a_2, \cdots, a_r$ vetores de $\R^n$. O \emph{cone formado por $a_1, \cdots, a_r$} é o conjunto:
	
	\[\{\, \sum_{i=1}^r t_i a_i \mid t \in \R^r, t \leq 0\,\}\]
	
	Em particular, é muito comum estes vetores serem as colunas de uma matriz. Assim sendo, dada uma matriz $A$, definimos \emph{o cone das colunas de $A$}, denominado $\CC A$, como:
	
	\[\{\, Ax \mid x \in \R^n, x \geq 0\,\}\]
	
	É fácil verificar que isto é o mesmo que o cone formado pelas colunas de $A$. \footnote{O leitor poderá perguntar-se o que acontece ao cone gerado por zero elementos. Neste caso, o nosso conjunto será o conjunto que contém apenas um elemento ($\R^0$ tem um elemento, e todos os seus índices são $\geq 0$) e este será a soma vazia, que dá zero. Logo, o cone gerado pelo vazio é o conjunto $\{0\}$. Em termos matriciais, isto equivale ao cone gerado por uma matriz $n \times 0$. Tendo isto em conta, é fácil ver que o conjunto vazio \emph{não} é um cone sob esta definição.}
	\end{definition}
	
	Nesta secção, estudaremos os cones, que têm surpreendente aplicação ao problema de otimização linear. O nosso principal resultado será o chamado \emph{lema de Farkas}. Este e as suas variantes serão de grande utilidade para encontrar métodos para descobrir se um vetor é solução.
	
	Um destes métodos é o chamado \emph{método das linhas ativas}, e vamos esboçar metade da sua prova de correção, com o intuito de motivar o que se segue.
	
	\subsection{Método das linhas ativas (Parte 1)}
	
	Suponhamos que temos um pol na forma semicanónica:
	
	\[
	P =
	\begin{cases}
	\max\limits_x cx\\
	Ax \leq b
	\end{cases}
	\]
	
	Já vimos que, para o pol na forma padrão, o conjunto-solução, sempre que não-vazio, contém pelo menos um vértice. Em geral, infelizmente, isto não é verdade. Por exemplo, o pol mais trivial possível a uma dimensão:
	
	\[
	\begin{cases}
	\max\limits_x 0x\\
	\text{(Sem condições...)}
	\end{cases}
	\]
	
	Este pol tem solução (qualquer $x$, na realidade) mas não tem nenhum vértice. Isto mostra que a proposição ``o conjunto solução, se não-vazio, contém pelo menos um vértice'' não é verdade em geral. No entanto, é possível arranjar resultados parecidos, em espírito.
	
	Para arranjar uma medida de `quão na fronteira estamos', fazemos a seguinte definição:
	
	\begin{definition}
	Dado o pol $P$ definido acima, a $i$-ésima linha de $A$ (denotada, por agora, de $a_i$) diz-se \emph{ativa relativamente a $x$} se $a_i x = b$.
	\end{definition}
	
	Já vimos no início que $x$ pertence à fronteira de $X_P$ sse $A$ tem pelo menos uma linha ativa relativamente a $x$. O que vamos ver nesta secção é que é possível averiguar se um vetor é solução examinando as linhas ativas.
	
	\begin{definition}
	Dado um vetor $x$ de $X_P$, $A^x$ representa a matriz das linhas ativas de $x$ em $A$.
	\end{definition}
	
	%todo add motivação?
	
	\begin{prop} \label{linhasativas1}
	Seja $x$ um vetor admissível de $P$. $x$ não é solução sse existe $w \in \R^n$ tal que $cw > 0$ e $A^x w \leq 0$
	\end{prop}
	
	\begin{proof}
	Prove-se primeiro a implicação ($\leftarrow$). Suponha-se que existe $w \in \R^n$ tal que $cw > 0$ e $A^x w \leq 0$. A ideia é `somar $w$ em pequenas quantidades', de modo a obter um vetor admissível de pontuação melhor do que $x$.
	
	É óbvio que se existe $t$ positivo tal que $x + tw \in X_P$, $x$ não é solução, pois este novo vetor tem pontuação melhor do que $x$.
	
	Para arranjar tal $t$, considere-se cada linha de $A(x + tw)$.
	
	Para todo $i$, há dois casos:
	
	Ou $a_i$ é linha ativa, e então $a_i (x + tw) = a_i x + t a_i w \leq a_i x = b_i$ para qualquer $t$,
	
	Ou $a_i$ não é linha ativa, e então $a_i (x + t w) = a_i x + t a_i w$. Como $a_i x < b_i$, temos que, para $t$ de módulo menor que $\lvert \frac{b_i - a_i x}{a_i w} \rvert$, $a_i x + t a_i w \leq b_i$.
	
	Considere-se então $t$ pequeno o suficiente para que $a_i (x + t w) \leq b_i$ para todo $i$ (é fácil ver que tal $t$ existe) e temos, como desejado, que este vetor é admissível e tem pontuação melhor do que $x$.
	
	Para provar a implicação ($\rightarrow$) suponha-se que $x$ não é solução. Então, existe um vetor $x+w$ de pontuação melhor do que $x$. Vamos mostrar que este $w$ obedece às condições do enunciado.
	
	Obviamente $cw > 0$, pois $c(x+w) = cx + cw > cx$ por hipótese. Falta então provar que $A^x w \leq 0$.
	
	Mas se isso não fosse o caso, e houvesse $i$ tal que $a_i$ é linha ativa e $a_i w > 0$, ter-se-ia que a linha $i$ de $A(x+w)$ seria $a_i x + a_i w > a_i x  = b_i$, e então $x+w$ não seria admissível. Contradição.
	\end{proof}
	
	Esta proposição vai-nos ajudar a desenvolver o método das linhas ativas. Falta, agora, caraterizar quando é que existe $w$ tal que $cw > 0$ e $A^x w \leq 0$. Para dar uma ideia, visualize-se a situação.
	
	Repare-se, primeiro que tudo, que a condição $cw > 0$ implica que $w$ é diferente de zero, pelo que $w$ define um hiperplano. Para mais, recorde-se que os hiperplanos dividem o espaço em dois. A condição que temos, então, é que, se considerarmos $c^T$ e as colunas de $A^{xT}$ como vetores em $\R^n$, o hiperplano definido por $w$ separa $c^T$ das colunas da matriz:
	
	\begin{tikzpicture}
	\draw[->] (-4, 0) -- (4, 0) node[anchor=south] {x};
	\draw[->] (0, -4) -- (0, 4) node[anchor=west] {y};
	
	\draw[->, color=green, very thick] (0,0) -- (-0.33333,1) node[anchor=south, color=black]{$w$};
	\draw[color=green, very thick] (-3 * 1.5,-1 * 1.5) -- (3 * 1.5,1 * 1.5);
	
	\draw[->, color=purple, very thick] (0,0) -- (3,1);
	\draw[->, color=purple, very thick] (0,0) -- (1,-2) node[anchor=north, color=black]{(As colunas de $A^{xT}$)};
	\draw[->, color=purple, very thick] (0,0) -- (-1,-3);
	
	
	\draw[->, color=blue, very thick] (0,0) -- (-2,4) node[anchor=south west, color=black]{$c^T$};
	\end{tikzpicture}
	
	Vamos ver esta mesma imagem, mas, a vermelho, está o cone das colunas de $A^{xT}$.
	
	\begin{tikzpicture}
	\fill[->, color=purple!50, very thick] (0,0) -- (3 * 1.5, 1*1.5) --  (3 * 1.5, -3.5) -- (-1 * 3.5/3, -3.5);
	
	\draw[->] (-4, 0) -- (4, 0) node[anchor=south] {x};
	\draw[->] (0, -4) -- (0, 4) node[anchor=west] {y};
	
	\draw[->, color=green, very thick] (0,0) -- (-0.33333,1) node[anchor=south, color=black]{$w$};
	\draw[color=green, very thick] (-3 * 1.5,-1 * 1.5) -- (3 * 1.5,1 * 1.5);
	
	
	
	\draw[->, color=purple, very thick] (0,0) -- (3,1);
	\draw[->, color=purple, very thick] (0,0) -- (1,-2) node[anchor=north, color=black]{(O cone das colunas de $A^{xT}$)};
	\draw[->, color=purple, very thick] (0,0) -- (-1,-3);
	
	\draw[->, color=blue, very thick] (0,0) -- (-2,4) node[anchor=south west, color=black]{$c^T$};
	
	\end{tikzpicture}
	
	A ideia é então a seguinte: mostrar que se $c^T$ está fora deste cone, é possível arranjar tal $w$. Isto porque verificar se $c^T$ está fora do cone é possívelmente mais fácil. Por exemplo, se as linhas de $A^x$ são linearmente independentes, a equação $A^{xT} y = c^T$ tem no máximo uma solução. É então possível tentar encontrar tal solução. Se ela existir e for $\geq 0$, então $c^T$ está no cone desejado, e $x$ é solução. Caso contrário, não está, e $x$ não é solução.
	
	Isto motiva, então, o enunciado do lema de Farkas:
	
	\begin{conjetura} (Lema de Farkas)
	
	Seja $A$ uma matriz $n \times m$, $v$ um vetor de $\R^n$.
	
	$v \not\in \CC A$ sse existe $w$ tal que $w \cdot v > 0$ e $A^T w \leq 0$.
	\end{conjetura}
	
	Focamo-nos, agora, em provar este lema.
	
	\subsection{O Lema de Farkas (Parte 1)}
	
	Para provar o Lema de Farkas vai-nos ser necessário provar vários resultados intermédios. É, no entanto, feito o esboço da prova, para motivar esses resultados.
	
	O objetivo é, dado um cone $C$ e um ponto $v$ fora deste:
	
	\begin{tikzpicture}
	\fill[->, color=purple!50, very thick] (0,0) -- (3.5, 2) --  (3.5, -1.5) -- (1, -1.5);
	
	\draw[->] (-2, 0) -- (4, 0) node[anchor=south] {x};
	\draw[->] (0, -2) -- (0, 3) node[anchor=west] {y};
	
	\draw[->, color=blue, very thick] (0,0) -- (1,2.5) node[anchor=south west, color=black]{$v$};
	
	\end{tikzpicture}
	
	Encontrar um hiperplano que os separe.
	
	\begin{tikzpicture}
	\fill[->, color=purple!50, very thick] (0,0) -- (3.5, 2) --  (3.5, -1.5) -- (1, -1.5);
	
	\draw[->] (-2, 0) -- (4, 0) node[anchor=south] {x};
	\draw[->] (0, -2) -- (0, 3) node[anchor=west] {y};
	
	\draw[color=green, very thick] (-3.5 / 2, -1) -- (3.5, 2);
	
	\draw[->, color=blue, very thick] (0,0) -- (1,2.5) node[anchor=south west, color=black]{$v$};
	
	\end{tikzpicture}
	
	O que vamos fazer é considerar a projeção ortogonal de $v$, digamos $z$, no cone
	
	\begin{tikzpicture}
	\fill[->, color=purple!50, very thick] (0,0) -- (3.5, 2) --  (3.5, -1.5) -- (1, -1.5);
	
	\draw[->] (-2, 0) -- (4, 0) node[anchor=south] {x};
	\draw[->] (0, -2) -- (0, 3) node[anchor=west] {y};
	
	\draw[color=green, very thick] (-3.5 / 2, -1) -- (3.5, 2);
	
	\draw[->, color=blue, very thick] (0,0) -- (1,2.5) node[anchor=south west, color=black]{$v$};
	\draw[->, color=yellow, very thick] (0,0) -- (1.83077,1.04617) node[anchor=east, color=black]{$z$};
	
	\draw[dashed] (1,2.5) -- (1.83077,1.04617);
	
	\draw (1.83077-0.248072*0.5,1.04617+0.43412*0.5)
		-- ++(0.43412*0.5, 0.248072*0.5)
		-- ++(0.248072*0.5,-0.43412*0.5);
	
	\end{tikzpicture}
	
	E consideramos o plano definido por $w = v-z$.
	
	\begin{tikzpicture}
	\fill[->, color=purple!50, very thick] (0,0) -- (3.5, 2) --  (3.5, -1.5) -- (1, -1.5);
	
	\draw[->] (-2, 0) -- (4, 0) node[anchor=south] {x};
	\draw[->] (0, -2) -- (0, 3) node[anchor=west] {y};
	
	\draw[color=green, very thick] (-3.5 / 2, -1) -- (3.5, 2);
	
	\draw[->, color=blue, very thick] (0,0) -- (1,2.5) node[anchor=south west, color=black]{$v$};
	\draw[->, color=yellow, very thick] (0,0) -- (1.83077,1.04617) node[anchor=east, color=black]{$z$};
	
	\draw[->, color=green, very thick] (1.83077,1.04617) -- (1,2.5) node[anchor=west, color=black, pos=0.5]{$w$};
	
	\draw (1.83077-0.248072*0.5,1.04617+0.43412*0.5)
		-- ++(0.43412*0.5, 0.248072*0.5)
		-- ++(0.248072*0.5,-0.43412*0.5);
	
	\end{tikzpicture}
	
	A partir daqui, há dois grandes passos que é preciso justificar:
	
	\begin{itemize}
	\item O que queremos dizer por projeção ortogonal, e como a encontramos?
	\item Como sabemos que o plano assim definido separa o ponto $v$ do cone $C$?
	\end{itemize}
	
	Dedicamo-nos agora ao primeiro.
	
	\subsection{Cones primitivos}
	
	Em geral, a `projeção ortogonal de $x$ em $Y$' pode ser definida como o $y \in Y$ que minimize $\lvert x - y \rvert$. No entanto, é preciso assegurar que esta função pode ser minimizada.
	
	Felizmente, em certas condições, a existência deste mínimo é-nos garantida.
	
	\begin{prop}
	Se $x \in \R^n$ e $Y$ é um subconjunto fechado e não-vazio de $\R^n$, existe $y \in Y$ tal que $\lvert x - y \rvert \leq \lvert x - y' \rvert$ para todo $y' \in Y$.
	\end{prop}
	
	\begin{proof}
	Primeiro que tudo, considere-se o caso em que $Y$ é limitado. Assim sendo, visto que também é fechado, é compacto. Logo, a função $f(y) = \lvert x - y \rvert$ tem mínimo por Weierstrass, o que mostra a nossa afirmação.
	
	No caso de $Y$ não ser limitado, pegue-se num $y'$ qualquer em $Y$. Considere-se $Y' = Y \cap \{\, y \in \R^n \mid \, \lvert x - y \rvert \leq \lvert x - y' \rvert \,\}$. Este conjunto já é compacto, e então o raciocínio anterior aplica-se, e existe $y$ que minimize $f$ em $Y'$.
	
	Mas se $y$ minimiza $f$ em $Y'$, também minimiza $f$ em $Y$, pois para qualquer $y''$ em $Y \setminus Y'$ tem-se $f(y) \leq f(y') \leq f(y'')$.
	\end{proof}
	
	Assim sendo, dado um ponto $x \in \R^n$ e um conjunto fechado $Y$, sabemos que existe a chamada \emph{projeção ortogonal de $x$ em $Y$}, ou seja, o $y \in Y$ que minimiza $\lvert x - y \rvert$.
	
	Precisamos, agora, de assegurar que os cones são fechados.
	
	Repare-se que qualquer cone em $\R^n$ pode ser expresso como $\{\,Ax \mid x \in \R^m, x \geq 0\,\}$ para alguma matriz $m \times n$. (As suas colunas são os vetores que geram o cone.)
	
	Vamos começar por considerar o caso particular de um cone gerado por uma matriz não-singular $A$. Representamos por $\R^n_{+0}$ o conjunto $\{\,x \in \R^n \mid x \geq 0\,\}$. Repare-se que este conjunto é fechado (deixado como exercício ao leitor.)
	
	Assim sendo, queremos considerar o conjunto $A \R^n_{+0}$, e queremos justificar que este é fechado.
	
	Repare-se que a função linear $x \mapsto Ax$ é contínua. Infelizmente, não é necessariamente verdade que a imagem sob uma função contínua de um conjunto fechado seja fechado. No entanto, há um resultado que nos pode ajudar:
	
	\begin{prop}
	Se $f : \R^n \rightarrow \R^n$ é uma bijeção de inversa contínua e $X$ é um conjunto fechado, $f(X)$ é fechado.
	\end{prop}
	
	\begin{proof}
	Seja $g$ a inversa de $f$, que sabemos ser contínua. Temos que $f(X) = g^{-1}(X)$. Como $X$ é fechado, $g^{-1}(X)$ é fechado, e então $f(X)$ é fechado.
	\end{proof}
	
	Repare-se que a nossa função $x \mapsto Ax$ está nestas condições, pois tem inversa contínua (em particular, linear) $x \mapsto A^{-1} x$. Assim sendo, a imagem sob $A$ de $\R^n_{+0}$ é fechada, e o nosso cone é fechado.
	
	Este raciocínio pode ser feito, com algumas modificações, na hipótese mais fraca de $A$ ter colunas linearmente indepentes.
	
	\begin{prop}
	Dizemos que um cone $A \R^m_{+0}$ é \emph{primitivo} se as colunas de $A$ são linearmente independentes.
	
	Qualquer cone primitivo é fechado.
	\end{prop}
	
	\begin{proof}
	Considere-se os vetores (colunas de $A$) $a_1, \cdots, a_m$ linearmente independentes, e extendenda-se este conjunto a uma base $a_1, \cdots, a_m, a_{m+1}, \cdots, a_n$.
	
	Se $A'$ é a matriz cujas colunas são $a_1, \cdots, a_n$, o cone $A \R^m_{+0}$ é igual a \allowbreak ${A' (\R^m_{+0} \times \{0\})}$. A função $x \mapsto A'x$ tem inversa contínua, e o conjunto $\R^m_{+0} \times \{0\}$ é fechado, pelo que o cone original é fechado.
	\end{proof}
	
	Infelizmente, não é fácil extender este raciocínio a qualquer cone. É possível, no entanto escrever qualquer cone como uma união de cones primitivos.
	
	A ideia é a seguinte: suponhamos que tenho o cone gerado por $a_1, \cdots, a_m$. Qualquer elemento $x$ deste cone pode ser escrito como $t_1 a_1 + \cdots + t_m a_m$ para $t_i \geq 0$, $i = 1, \cdots, m$. No entanto, da Álgebra Linear, sabemos que podemos escrevê-lo como $t_1 a_{i_1} + \cdots t_k a_{i_k}$ onde $a_{i_1}, \cdots, a_{i_k}$ são linearmente independentes. Se conseguirmos assegurar que há uma forma de escrever isto de modo a que $t_1, \cdots, t_k$ são todos $\geq 0$, obtemos que $x$ pertence ao cone (primitivo) gerado por $a_{i_1}, \cdots, a_{i_k}$.
	
	Feito isto, considere-se o cone gerado por $B = \{a_1, \cdots, a_n\}$. O que isto provaria seria que qualquer $x$ no cone gerado por $B$ pertence a um cone gerado por um $B' \subseteq B$ tal que $B'$ é linearmente independente, e então o cone gerado por $B$ é a união dos cones gerados pelos subconjuntos linearmente independentes de $B$, e então é união (finita) de cones primitivos, sendo então fechado.
	
	Falta apenas, então, mostrar que é sempre possível escrever um elemento do cone como soma não-negativa de elementos linearmente independentes.
	
	\begin{prop}
	Qualquer cone é união finita de cones primitivos, sendo então fechado.
	\end{prop}
	
	\begin{proof}
	Pelo que já vimos antes, basta mostrar que, para $x$ da forma $t_1 a_1 + \cdots t_m a_m$, ($t_1, \cdots, t_m \geq 0$), existem $i_1, \cdots, i_k$ e $r_1, \cdots, r_k \geq 0$ tal que $a_{i_1}, \cdots, a_{i_k}$ são linearmente independentes e $x = r_1 a_{i_1} + \cdots + r_k a_{i_k}$.
	
	Para fazer isto, suponha-se que $x$ pertence ao cone gerado por $a_1, \cdots, a_m$, e considere-se uma forma de escrever $x$ que minimize os índices diferentes de zero. Ou seja, vamos escrever $x$ como $t_1 a_1 + \cdots + t_m a_m$ de uma forma que faça com que o número de $i$ tal que $t_i > 0$ seja mínimo.
	
	Sejam $i_1, \cdots, i_k$ os índices tais que $t_{i_1}, \cdots, t_{i_k} > 0$. Provamos que $a_{i_1}, \cdots, a_{i_k}$ são linearmente independentes.

	Para este objetivo, suponha-se que não são. Mostraremos que afinal esta forma de escrever $x$ não é o mínimo que tinhamos suposto.
	
	Se não são linearmente independentes, existem $s_1, \cdots, s_k$, nem todos iguais a zero (suponha-se spdg que pelo menos um deles é maior que zero), tal que $s_1 a_{i_1} + \cdots + s_k a_{i_k} = 0$
	
	Comparando as duas equações que temos até agora: (E pondo $r_j = t_{i_j}$)
	
	\begin{align*}
	r_1 a_{i_1} + \cdots + r_k a_{i_k} &= x \\
	s_1 a_{i_1} + \cdots + s_k a_{i_k} &= 0
	\end{align*}
	
	A ideia é subtrair a equação de baixo à de cima de modo a deixar todos os termos $\geq 0$, mas cancelar pelo menos um, e então mostrar que existe uma forma de somar os $a$'s a zero que tem menos termos não-nulos.
	
	Para este efeito, considere-se o conjunto $\{\,\frac{r_j}{s_j} \mid s_j > 0\,\}$. Este conjunto é não-vazio por hipótese, e é finito, pelo que tem um mínimo $\frac{r_\ell}{s_\ell}$. Ponha-se então $r'_j = r_j - \frac{r_\ell}{s_\ell} s_j$.
	
	É fácil verificar que $r'_j \geq 0$ para todo $j$ (o caso em que $s_j \leq 0$ é trivial, o caso em que $s_j > 0$ é deixado ao leitor) e que $r'_\ell = 0$. Finalmente, é trivial ver que $r'_1 a_{i_1} + \cdots + r'_k a_{i_k} = x$, pois o que foi feito foi subtrair a equação de baixo à equação de cima $\frac{r_\ell}{s_\ell}$ vezes, pelo que o resultado dá $x + \frac{r_\ell}{s_\ell} 0 = x$.
	
	Isto mostra, então, que a forma de somar a $x$ que minimiza o número de termos não-zero tem todos os termos linearmente independentes, pelo que $x$ é soma de vetores linearmente independentes e então pertence a um dos cones primitivos gerados por subconjuntos linearmente independentes de $B = \{a_1, \cdots, a_m\}$.
	
	Como o número de tais subconjuntos é finito, o cone gerado por $B$ é uma união finita de cones primitivos. Como todos este são fechados, e uniões finitas de conjuntos fechados são fechados, o nosso cone é fechado.
	\end{proof}
	
	Esta proposição é a principal razão pela qual passámos pelos cones primitivos. Visto que agora sabemos que os cones são todos fechados, sabemos que podemos considerar projeções ortogonais, e avançamos para o lema de Farkas.
	
	\subsection{O Lema de Farkas (Parte 2)}
	
	Vamos, agora, voltar aos nossos desenhos. Relembre-se que temos um ponto $v$ e um cone $C$ (a vermelho na imagem abaixo) e pretendemos arranjar um hiperplano que separasse $v$ de $C$. Tinhamos a conjetura que o $w$ construido abaixo (que já sabemos que podemos construir) poderia ser de uso.
	
	\begin{tikzpicture}
	\fill[->, color=purple!50, very thick] (0,0) -- (3.5, 2) --  (3.5, -1.5) -- (1, -1.5);
	
	\draw[->] (-2, 0) -- (4, 0) node[anchor=south] {x};
	\draw[->] (0, -2) -- (0, 3) node[anchor=west] {y};
	
	\draw[color=green, very thick] (-3.5 / 2, -1) -- (3.5, 2);
	
	\draw[->, color=blue, very thick] (0,0) -- (1,2.5) node[anchor=south west, color=black]{$v$};
	\draw[->, color=yellow, very thick] (0,0) -- (1.83077,1.04617) node[anchor=east, color=black]{$z$};
	
	\draw[->, color=green, very thick] (1.83077,1.04617) -- (1,2.5) node[anchor=west, color=black, pos=0.5]{$w$};
	
	\draw (1.83077-0.248072*0.5,1.04617+0.43412*0.5)
		-- ++(0.43412*0.5, 0.248072*0.5)
		-- ++(0.248072*0.5,-0.43412*0.5);
	
	\end{tikzpicture}
	
	Repare-se que não justificámos o ângulo reto na figura. No entanto, da figura é óbvio (e vamos demonstrar) que, de facto, $z \cdot w = 0$. Isto vai ser relevante na nossa prova.
	
	A ideia para justificar que, para todo $x \in C$ se tem $w \cdot x \leq 0$ é a seguinte: suponhamos que havia $x \in C$ tal que $w \cdot x > 0$.
	
	Repare-se que o cone é convexo (exercício ao leitor), pelo que todos os vetores da forma $(1-t) z + t x$ pertencem ao cone para $t \in [0,1]$. Mas repare-se no aspeto da função $f(t) = \lvert v - ((1-t) z + t x) \rvert ^2$. Expandido com o produto escalar, ficamos com
	
	\begin{align*}
	f(t) &= \lvert(v - z) + t(z - x)\rvert^2 \\
	&= \lvert v-z \rvert^2 + 2 (v - z) \cdot (z - x) t + \lvert z - x \rvert^2 t^2
	\end{align*}
	
	Isto tem a aparência de uma parábola virada para cima. Agora, vejamos como é que esta se parece no intervalo $[0,1]$... Calculando a derivada desta função em zero, é possível verificar que esta dá $2 (v - z) \cdot (z - x) = - 2 w \cdot x < 0$ (as contas serão feitas com mais cuidado em breve) e então a derivada é negativa em zero... O que significa que existe uma vizinhança de zero onde a função toma valores menores do que $f(0)$, o que significa que existe $t_0$ tal que $f(t_0) < f(0)$. Logo, temos um ponto, $(1-t_0) z + t_0 x$ que pertence ao cone, cuja distância a $v$ é menor do que a de $z$, o que contraria a nossa hipótese de minimalidade quanto a $z$.
	
	Isto é o cerne da demonstração. Os detalhes serão feitos agora.
	
	\begin{prop} (Lema de Farkas)
	Seja $C \subseteq \R^n$ um cone, $v \in \R^n$
	
	$v \not \in C$ sse existe $w \in \R^n$ tal que $w \cdot v > 0$ e $w \cdot x \leq 0$ para todo $x \in C$.
	\end{prop}
	
	\begin{proof}
	Primeiro, lide-se com a implicação óbvia: ($\leftarrow$).
	
	Se existe $w$ tal que $w \cdot x \leq 0$ para todo $x \in C$ e $w \cdot v > 0$, é óbvio que $v$ não pode pertencer a $C$. Feito.
	
	O que nos leva à implicação não-óbvia.
	
	Suponhamos que $v \not \in C$. Seja $z$ a projeção ortogonal de $v$ em $C$, que sabemos existir pois $C$ é fechado e não-vazio, e seja $w = v - z$.
	
	Para provar que $w \cdot z = 0$, considere-se a função $g(t) = \lvert v - t z \rvert ^2$. Sabemos, por definição de $z$, que $g$ tem um mínimo em $t = 1$, pois os pontos da forma $tz$ pertencem todos ao cone para $t \in \R^+_0$ (verifique). Logo, $g'(1) = 0$.
	
	Calculando $g(t)$ (expandindo o quadrado da norma com o produto escalar), temos que este dá $\lvert v \rvert ^2 - 2 (v \cdot z) t + \lvert z \rvert ^2 t^2$. A derivada é, então, $2 (-(v \cdot z) + (z \cdot z) t)$. Em particular, sabemos que para $t = 1$ isto dá zero, pelo que $2 (-v \cdot z + z \cdot z) = 0$ e então temos que $(v - z) \cdot z = 0$ donde se conclui $w \cdot z = 0$, como queríamos demonstrar.
	
	Feito isto, continue-se a prova. Pretendemos mostrar que para todo $x \in C$ se tem $w \cdot x \leq 0$, pelo que supomos por contradição que existe $x \in C$ tal que $w \cdot x > 0$.
	
	Visto que o cone é convexo (verifique) temos que para todo $t \in [0,1]$ o vetor $(1-t)z + tx$ pertence ao cone. Assim sendo, consideramos a função
	
	
	\begin{align*}
	f(t) &= \lvert v - ((1-t)z + tx) \rvert^2 \\
	&= \lvert(v - z) + t(z - x)\rvert^2 \\
	&= \lvert v-z \rvert^2 + 2 (v - z) \cdot (z - x) t + \lvert z - x \rvert^2 t^2
	\end{align*}
	
	Esta função tem, como já vimos, derivada negativa em zero:
	
	
	\begin{align*}
	f'(t) &= 2 (v - z) \cdot (z - x) + 2 \lvert z - x \rvert^2 t \\
	f'(0) &= 2 (v - z) \cdot (z - x)\\
	&= 2 w \cdot z - 2 w \cdot x\\
	&= - 2 w \cdot x\\
	&< 0
	\end{align*}
	
	Logo, para $t>0$ pequeno o suficiente tem-se $f(t) < f(0)$. Pondo $t_0$ um destes $t$ pequenos o suficiente, temos que $z' = (1-t_0)z + t_0 x$ é um ponto que pertence ao cone que tem distância a $v$ menor do que $z$, o que contraria a nossa hipótese sobre $z$. Assim sendo, a hipótese que $w \cdot x > 0$ tem que ser falsa, e o lema de Farkas está provado.
	\end{proof}
	
	A seguir está uma versão equivalente do Lema de Farkas, que é mais fácil de verificar na prática, pois basta verificar a condição $w \cdot x \leq 0$ para um número finito de vetores.
	
	\begin{prop} (Lema de Farkas Modificado)
	Seja $A$ uma matriz $n \times m$, $v \in \R^n$
	
	$v \not \in \CC A$ sse existe $w \in \R^n$ tal que $w \cdot v > 0$ e $A^T w \leq 0$.
	\end{prop}
	
	\begin{proof}
	Sabemos que $v \not \in \CC A$ sse existe $w \in \R^n$ tal que $w \cdot v > 0$ e $w \cdot (Ax) \leq 0$ para todo $x \geq 0$.
	
	Em particular, pondo $x = e_1, e_2, \cdots, e_m$ (onde $e_i$ representa o $i$-ésimo vetor da base canónica em $\R^m$), temos que $v \not \in \CC A$ implica que existe $w \in \R^n$ tal que $w \cdot v > 0$ e $w \cdot a_i \leq 0$ para todo $i = 1, \cdots, m$. Isto pode ser reformulado como $w \cdot v > 0$ e $A^T w \leq 0$, o que conclui a implicação ($\rightarrow$).
	
	Basta agora mostrar ($\leftarrow$). Suponha-se que existe $w \in \R^n$ tal que $w \cdot v > 0$ e $A^T w \leq 0$. Mostraremos que este $w$ é tal que $w \cdot (Ax) \leq 0$ para todo $x \geq 0$, o que, por Farkas, mostra que $v \not \in \CC A$.
	
	Ora, de $A^t w \leq 0$ obtemos que $w \cdot a_i \leq 0$ para todo $i$, onde $a_i$ representa a $i$-ésima coluna de $A$. Assim sendo, pondo $x$ um vetor não-negativo, temos $w \cdot (Ax) = w \cdot \sum_{i=1}^m x_i a_i = \sum_{i=1}^m x_i (w \cdot a_i)$. Todos os termos desta soma são $\leq 0$, pelo que obtemos $w \cdot (Ax) \leq 0$, como queríamos demonstrar.
	\end{proof}
	
	\subsection{Método das linhas ativas (Parte 2)}
	
	Recorde-se da proposição \ref{linhasativas1}, que provámos anteriormente:
	
	\begin{prop*}
	Seja $P$ o seguinte pol na forma semicanónica:
	
	\[
	\begin{cases}
	\max\limits_x cx\\
	Ax \leq b
	\end{cases}
	\]
	
	E seja $x$ um vetor admissível de $P$. $x$ não é solução sse existe $w \in \R^n$ tal que $cw > 0$ e $A^x w \leq 0$
	\end{prop*}
	
	Com o lema de Farkas, podemos refinar isto para o seguinte útil critério:
	
	\begin{prop} (Critério das linhas ativas)
	
	Sob as mesmas hipótses, $x$ é solução sse $c^T$ pertence a $\CC A^{xT}$
	\end{prop}
	
	A demonstração disto é uma aplicação trivial do lema de Farkas, que o leitor poderá confirmar por si mesmo.
	
	Visto que qualquer pol pode ser transformado na forma semicanónica, este critério acaba por ser extremamente útil para demonstrar que um dado vetor é solução.
	
	\section{Dualidade}
	
	\subsection{Motivação}
	
	Para motivar o conceito de dualidade, vamos primeiro considerar um exemplo.
	
	Considere-se o seguinte pol, que já apareceu neste documento:
	
	
	\[
	P =
	\begin{cases}
	\max\limits_{(x,y)} 2x + y\\
	x + 2y \leq 8\\
	x - y \leq 2 \\
	x, y \geq 0
	\end{cases}
	\]
	
	Repete-se a representação visual da região admissível, do vetor $c^T$ e da solução (4,2).
	
	\begin{tikzpicture}
	\draw[->] (-2, 0) -- (6, 0) node[anchor=south] {x};
	\draw[->] (0, -1) -- (0, 5) node[anchor=west] {y};
	\filldraw[color=red, fill=blue!80, very thick] (0,0) node[anchor=north west, color=black]{0} -- (0,4) node[anchor=east, color=black]{(0,4)} -- (4,2) node[anchor=north west, color=black]{\,(4,2)} -- (2,0) node[anchor=north, color=black]{(2,0)} -- (0,0);
	
	\draw[->, color=green, very thick] (4,2) -- (6,3) node[anchor=west, color=black]{$c^T = (2,1)$};
	\draw[very thick, color=green] (3.5,3) -- (4.5,1);
	\end{tikzpicture}
	
	Da imagem, é claro que este ponto é solução. Mas uma imagem não é uma demonstração, e vamos então provar isto analíticamente.
	
	Queremos provar que $(4,2)$ é solução, ou seja, que todos os pontos em $X_P$ têm pontuação $\leq 2 \cdot 4 + 2 = 10$.
	
	Para fazer isto, peguemos nas desigualdades que temos: $x + 2y \leq 8$ e $x - y \leq 2$.
	
	Normalmente, dados sistemas de equações lineares, adicionamos múltiplos de umas às outras para obter informação sobre as nossas variáveis. É isso que fazemos aqui.
	
	Por exemplo, adicionando a primeira desigualdade mais duas vezes a segunda, obtemos $(x+2y) + 2(x-y) \leq 8 + 2(2)$, ou seja, $3x \leq 12$, e então $x \leq 4$ para todo $(x,y) \in X_P$.
	
	Observe-se a desigualdade $x + 2y \leq 8$, que é igual a $\frac 1 2 x + y \leq 4$. Somando $\frac 3 2 x$, obtemos $2x + y \leq \frac 3 2 x + 4 \leq 6 + 4 = 10$. Logo, para todo $(x,y) \in X_P$ temos que a sua pontuação é no máximo 10, pelo que, como o nosso ponto (4,2) tem pontuação 10, este é solução.
	
	Reiterando, o que fizemos foi pegar nas nossas desigualdades e somá-las de modo a que o lado esquerdo fosse a expressão da nossa função-objetivo, e então obter um majorante desta.
	
	Vamos fazer outro exemplo, desta vez sem imagem:
	
	Considere-se o pol
	
	\[
	P =
	\begin{cases}
	\max\limits_{(x,y)} x + y\\
	x + 2y \leq 8\\
	5x + y \leq 13 \\
	x, y \geq 0
	\end{cases}
	\]
	
	Usando uma ideia semelhante, obtemos que, somando $4 \times$ a primeira desigualdade a $1 \times$ a segunda, $4(x + 2y) + 1(5x + y) \leq 4 \cdot 8 + 13$, donde $9x + 9y \leq 45$ e então $x+y \leq 5$. Agora, se arranjarmos um elemento admissível que tenha pontuação 5, sabemos que este é solução, e de facto, o ponto $(2, 3)$ é admissível, de pontuação 5, pelo que é solução.
	
	Só mais um exemplo, este trivial para mostrar que às vezes a condição $\geq 0$ é necessária. Remova-se a segunda desigualdade ao pol anterior para obter
	
	\[
	P =
	\begin{cases}
	\max\limits_{(x,y)} x + y\\
	x + 2y \leq 8\\
	x, y \geq 0
	\end{cases}
	\]
	
	Bem, sabendo que $8 \geq x + 2y$, e sabendo que $y \geq 0$, obtemos $8 \geq x + y + y \geq x + y$, donde $x+y \leq 8$.
	
	Assim, se arranjarmos um ponto com pontuação 8, este é solução. E de facto, pondo $(x,y) = (8,0)$, temos a nossa solução desejada.
	
	Isto já são exemplos suficientes para fazer o seguinte ponto: adicionar desigualdades é útil. E é nisso que se baseia a dualidade.
	
	\subsection{Introdução à dualidade}
	
	Considere-se um pol $P$ na forma canónica:
	
	\[
	\begin{cases}
	\max\limits_x cx\\
	Ax \leq b\\
	x \geq 0
	\end{cases}
	\]
	
	Vamos tentar generalizar o método de somar desigualdades. Vamos escrever a desigualdade $Ax \leq b$ como:
	
	\begin{gather*}
	a_1 x \leq b_1\\
	a_2 x \leq b_2\\
	\cdots\\
	a_m x \leq b_m
	\end{gather*}
	
	Sejam $y_1, \cdots, y_m$ escalares $\geq 0$ (isto é necessário porque multiplicar por escalares menores que zero troca as desigualdades). Temos que
	
	\[y_1 (a_1 x) + y_2 (a_2 x) + \cdots + y_m (a_m x) \leq y_1 b_1 + y_2 b_2 + \cdots + y_m b_m\]
	
	Isto pode ser escrito de forma compacta da seguinte forma. Seja $y$ o vetor -coluna cuja $i$-ésima componente é $y_i$. Temos que, se $y \geq 0$,
	
	\[y^T Ax \leq y^T b\]
	
	Visto que isto é um número real, transpôr não faz diferença, pelo que
	
	\[b^T y \geq y^T Ax\]
	
	Relembre-se que esta mixórdia de somar desigualdades umas às outras foi com o intuito de arranjar majorantes de $cx$. Portanto, sendo $p_i$ a $i$-ésima casa de $y^T A$, isto dá-nos que
	
	\[b^T y \geq p_1 x_1 + p_2 x_2 + \cdots + p_n x_n\]
	
	Aqui podemos usar o facto que $x \geq 0$ para obter que, \emph{na condição que $p_i \geq c_i$ para todo $i$},
	
	\begin{align*}
	b^T y &\geq p_1 x_1 + p_2 x_2 + \cdots + p_n x_n\\
	&\geq c_1 x_1 + c_2 x_2 + \cdots + c_n x_n = cx
	\end{align*}
	
	E então $b^T y$ é um majorante da função objetivo.
	
	Em suma: dado o pol
	
	\[
	P =
	\begin{cases}
	\max\limits_x cx\\
	Ax \leq b\\
	x \geq 0
	\end{cases}
	\]
	
	É possível arranjar majorantes de $cx$ somando as desigualdades de $Ax \leq b$. Na condição de $A^T y \geq c^T$, temos que $b^T y$ é majorante de $cx$.
	
	Obviamente estamos interessados no \emph{menor} majorante, daí fazer sentido considerarmos o pol
	
	\[
	\begin{cases}
	\min\limits_y b^T y\\
	A^T y \geq c^T\\
	y \geq 0
	\end{cases}
	\]
	
	Este é o chamado \emph{dual de $P$}.
	
	Um facto interessante: se a mesma estratégia for aplicada a este pol, acabamos com o pol com que começámos!
	
	Nomeadamente, considere-se somar as linhas de $A^T y \geq c^T$ com o vetor $z \geq 0$. Ficamos com $z^T A^T y \geq z^T c^T = cz$, pelo que, na condição de $z^T A^T$ ter todas as respetivas componentes menores ou iguais do que $b^T$, $cz$ é minorante de $b^T y$. Ou seja, para obter o maior minorante, resolvemos o pol
	
	\[
	\begin{cases}
	\max\limits_z cz\\
	Az \leq b\\
	z \geq 0
	\end{cases}
	\]
	
	Que é exatamente o pol com que começámos.
	
	\subsection{Algumas palavras sobre o dual}
	
	Antes de avançar na teoria, gostaria de fazer notar uma coisa.
	
	A técnica usada anteriormente, de somar desigualdades, é útil, mas não é um método rigoroso de descobrir `o dual de um pol'. Até agora foi considerado o pol na forma canónica, mas vamos adiante considerar duais de pols na forma padrão e semicanónica. E um facto importante é que o dual do mesmo pol em qualquer uma destas três formas é diferente. Ou seja, o dual depende da forma como o pol é expresso. E, por exemplo, se tentássemos fazer o dual do pol
	
	\[
	\begin{cases}
	\min\limits_{(x,y)} -x+y\\
	-2x-y \geq 0\\
	x \geq 0\\
	y \geq 0
	\end{cases}
	\]
	
	Teriamos resultados diferentes se interpretássemos este como um pol da forma
	
	\[
	\begin{cases}
	\min\limits_y c y\\
	A y \geq b\\
	y \geq 0
	\end{cases}
	\]
	
	Ou da forma
	
	\[
	\begin{cases}
	\min\limits_y c y\\
	A y \geq b
	\end{cases}
	\]
	
	Portanto o dual não depende só da forma como o pol está expresso, mas também da forma como o interpretamos... Assim sendo, reservamos o termo `o dual de um pol' como um termo informal, cujo significado será óbvio de contexto e definições anteriores, e não como uma definição rigorosa.
	
	Para terminar esta parte, o termo dual tem significado rigoroso no contexto de problemas de max-min e min-max. Ou seja, problemas da forma:
	
	\[\sup\limits_x \inf\limits_y \mathscr{L}(x,y) \text{ ou } \inf\limits_y \sup\limits_x \mathscr{L}(x,y)\]
	
	Em que $\mathscr{L}$ é uma função $X \times Y \rightarrow \R$ e $X$ e $Y$ são conjuntos arbitários.
	
	Neste contexto, o dual é não-ambíguo: estes dois problemas são duais um do outro.
	
	É possível tentar fazer a noção de dual no contexto do pol mais rigorosa, transformando o pol num problema max-min e min-max. No entanto, esta transformação sofre dos mesmos problemas de depender de como interpretamos o pol, e é bastante artificial.
	
	\subsection{Propriedades básicas do dual}
	
	Relembre-se que dado o pol na forma canónica
	
	\[
	P =
	\begin{cases}
	\max\limits_x cx\\
	Ax \leq b\\
	x \geq 0
	\end{cases}
	\]
	
	Definimos o seu dual como
	
	\[
	Q =
	\begin{cases}
	\min\limits_y b^T y\\
	A^T y \geq c^T\\
	y \geq 0
	\end{cases}
	\]
	
	Para simplificar a notação, vamos, de agora em diante, usar $X$ e $S$ para nos referirmos ao conjunto admissível e solução do pol em questão, e $Y$ e $R$ para os respetivos conjuntos do problema dual.
	
	Primeiro que tudo, a proposição que motivou a consideração do problema dual.
	
	\begin{prop} (Teorema fraco da dualidade)
	
	Se $x \in X$ e $y \in Y$, então $cx \leq b^T y$.
	
	Como consequência, se $cx = b^T y$, $x$ e $y$ são soluções dos respetivos problemas.
	\end{prop}
	
	\begin{proof}
	Se $x$ é admissível, $Ax \leq b$, donde, por $y$ ser $\geq 0$, temos $y^T A x \leq y^T b$. Devido a estes serem números reais, podemos transopô-los, para obter $x^T A^T y \leq b^T y$. Como $c^T \leq A^T y$ e $x \geq 0$, temos $x^T c^T \leq x^T A^T y \leq b^T y$ e então, transpondo, obtemos $cx \leq b^T y$, como pretendiamos demonstrar.
	
	A demonstração de que se $cx = b^T y$ então estes são soluções é trivial e deixada como exercício ao leitor.
	\end{proof}
	
	O pol dual pode também ser utilizado para servir de prova de existência de solução.
	
	\begin{prop}
	Se $X$ e $Y$ forem ambos não-vazios, tanto o pol original como o seu dual têm solução.
	\end{prop}
	
	\begin{proof}
	Recorde-se que, visto que o pol original é um pol de maximização, se a função-objetivo for majorada, tem solução. Se $Y$ é não-vazio, qualquer seu elemento tem pontuação que majora a dos elementos de $X$, pelo que o pol original tem pontuação majorada e então tem solução.
	
	A prova que o dual tem solução é semelhante e deixada para o leitor.
	\end{proof}
	
	Recorde-se que nos exemplos que fizemos, era sempre possível somar as desigualdades de modo a obter um majorante minimal. Ou seja, nesses exemplos, 
conseguimos sempre descobrir um upper bound que corresponde à solução.

	Não é inconcebível de um cenário em que isto é impossível. Ou seja, um pol em que a solução é, digamos, 2, mas com a estratégia do dual só conseguimos provar que é no máximo 3.
	
	Acontece que isto é, de facto, impossível. Ou seja, tendo como base a estratégia do dual, o menor majorante corresponde exatamente à solução.
	
	Para provar isto, provamos primeiro uma coisa parecida: caso exista solução, é possível arranjar majorantes tão `apertados' quanto queiramos.
	
	\begin{prop}
	Seja $s$ uma solução do pol original. Então, para todo $\varepsilon > 0$ existe $y \in Y$ tal que $cs \leq b^T y < cs + \varepsilon$
	
	Da mesma forma, se $r$ é solução do dual, para todo $\varepsilon$ existe $x \in X$ tal que $b^T y - \varepsilon < cx \leq b^T y$.
	\end{prop}
	
	\begin{proof}
	Suponha-se que o pol original é
	
	\[
	P =
	\begin{cases}
	\max\limits_x cx\\
	Ax \leq b\\
	x \geq 0
	\end{cases}
	\]
	
	E que este tem uma solução $s$.
	
	A ideia para criar um majorante arbitráriamente próximo é usar o lema de Farkas de forma inteligente.
	
	Repare-se que não existe $x$ tal que
	
	\begin{gather*}
	Ax \leq b\\
	x \geq 0\\
	cx \geq cs + \varepsilon
	\end{gather*}
	
	Repare-se que isto é a mesma coisa que dizer que não existe $x \geq 0$ tal que
	
	\[
	\begin{bmatrix}A \\
	-c\end{bmatrix}
	x \leq
	\begin{bmatrix} b \\
	-cs - \varepsilon \end{bmatrix}
	\]
	
	Isto pode ser escrito como uma afirmação sobre cones: adicionando as variáveis de folga $f$, temos que isto é igual a dizer que não existe $x, f \geq 0$ tal que
	\[
	\begin{bmatrix}A & I & \\
	-c & & 1\end{bmatrix}
	\begin{bmatrix}x \\
	f\end{bmatrix}
	=
	\begin{bmatrix} b \\
	-cs - \varepsilon \end{bmatrix}
	\]
	
	Isto é o mesmo que dizer que o vetor $(b, -cs-\varepsilon)$ não pertence a $\CC \begin{bmatrix}A & I & \\
	-c & & 1\end{bmatrix}$.
	
	Isto permite-nos aplicar Farkas e dizer que existe um vetor $(w,z)$ tal que
	
	\[(w, z) \cdot (b, -cs-\varepsilon) < 0\]
	
	E
	
	\[\begin{bmatrix}A^T & -c^T\\
	I &  \\
	  & 1\end{bmatrix} (w,z) \geq 0\]
	  
	Repare-se que trocámos as desigualdades do lema de Farkas. Isto foi por uma razão de conveniência, pelo que se vai ver no que se segue. Não há problema em fazer isto, pois basta pegar no vetor assegurado pelo lema de Farkas e considerar a sua negação.
	
	Rearranjando os termos acima usando muliplicação de matrizes por blocos, obtemos que $(w,z)$ satisfaz:
	
	\begin{gather*}
	b^T w < z(cx + \varepsilon) \\
	A^T w \geq z c^T\\
	w \geq 0\\
	z \geq 0
	\end{gather*}
	
	É fácil, agora, verificar que, se por acaso $z$ for diferente de zero, o vetor $y = \frac 1 z w$ satisfaz o que é dele desejado. Basta, então, mostrar que $z \neq 0$.
	
	Para ver isto, repare-se que, se $z$ fosse zero, ter-se-ia que
	
	\begin{gather*}
	b^T w < 0 \\
	A^T w \geq c^T\\
	w \geq 0\\
	\end{gather*}
	
	O que, usando o lema de Farkas, implicaria que $b \not \in \CC \begin{bmatrix}A & I \end{bmatrix}$. Usando o truque das variáveis de folga, é fácil ver que isto é o mesmo que dizer que não existe $x \geq 0$ tal que $Ax \leq b$, mas sabemos que isto é falso, pois por hipótese, o nosso vetor $s$ é admissível. Isto mostra que $z$ não pode ser igual a zero, e então existe o nosso vetor $y$ desejado.
	
	A prova da segunda parte pode ser feita de forma análoga, ou pode ser feita reparando que o pol dual é equivalente a
	
	\[
	\begin{cases}
	\max\limits_y -b^Ty\\
	-A^T y \leq -c^T\\
	y \geq 0
	\end{cases}
	\]
	
	E que o dual deste (interpretando este como um pol na forma canónica) é equivalente ao pol original.
	
	Então, usando esta proposição, sendo $r$ uma solução do dual, temos que existe $x$ admissível no pol original tal que $-b^T y \leq -cx < -b^T y + \varepsilon$, ou seja, existe $x$ admissível tal que $b^T y - \varepsilon < cx \leq b^T y$, como pretendiamos demonstrar.
	\end{proof}
	
	Isto leva diretamente a:
	
	\begin{prop} (Teorema forte da dualidade)
	
	Fixo o pol na forma canónica e o seu dual, um dos seguintes quatro casos acontece necessariamente:
	
	\begin{enumerate}[i]
	\item $X = \emptyset = Y$ e então nenhum dos pols tem solução.
	
	\item $X = \emptyset \neq Y$ e então a função-objetivo não é minorada no pol dual e nenhum deles tem solução.
	
	\item $X \neq \emptyset = Y$ e então a função-objetivo não é majorada no pol original e nenhum deles tem solução.
	
	\item $X \neq \emptyset \neq Y$ e então ambos os pols têm solução. Se $s \in S$ e $r \in R$, temos $cs = b^T r$.
	\end{enumerate}
	
	\end{prop}
	
	\begin{proof}
	Esta prova vai examinar os quatro casos separadamente.
	
	i) Este caso é trivial.
	
	ii) Sabendo que $Y \neq \emptyset$, se a função-objetivo fosse minorada no dual ter-se-ia que este tem solução. Pela prop anterior, isto implicaria $X \neq \emptyset$.
	
	iii) Análogo ao anterior.
	
	iv) Suponha-se que $X$ e $Y$ são não-vazios. Então pelo teorema fraco da dualidade, ambos os pols têm solução, digamos $s$ e $r$.
	
	Temos que $cs = b^T r$ pois, caso contrário, seja $\varepsilon = b^T r - cs$. Sabemos pela prop anterior que existe $y$ tal que $b^T y < cs + \varepsilon = b^T r$, o que contraria a hipótese de  $r$ ser solução. Logo, $cs = b^T r$.
	\end{proof}
	
	\subsection{Outros duais}
	
	Até agora, já tivemos bastante sucesso a arranjar formas de verificar se um vetor é solução de um pol: o método das linhas ativas para pols na forma semicanónica e dualidade para pols na forma canónica. Isto ser-nos-ia útil se tivessemos uma forma de \emph{adivinhar} soluções. Temos um método (vetores básicos), mas este só funciona para o pol na forma padrão. Assim sendo, faz sentido tentar extender os nossos métodos de verificação ao pol na forma padrão.
	
	Vamos investigar dualidade para o pol na forma padrão. Claro que é sempre possível transformar o pol na forma padrão para o pol na forma canónica e tomar  o dual deste -- E vamos investigar isto na sequência -- mas vamos tentar fazer primeiro as coisas de raíz como fizemos com o canónico.
	
	Para exemplificar e depois generalizar, vamos considerar um pol na forma padrão:
	
	\[
	\begin{cases}
	\min\limits_{(x,y,z)} x + 2y - z\\
	x + y + z = 1\\
	x,y,z \geq 0
	\end{cases}
	\]
	
	Da primeira restrição, sabemos que $x+y+z = 1$ e então $-x-y-z = -1$. Ora, $x + 2y - z \geq -x-y-z = -1$, donde a função-objetivo é sempre maior ou igual que $-1$.
	
	Como o ponto $(0,0,1)$ toma este valor, sabemos que este é solução.
	
	Para um exemplo menos trivial, vamos adicionar uma restrição a este problema.
	
	\[
	\begin{cases}
	\min\limits_{(x,y,z)} x + 2y - z\\
	x + y + z = 1\\
	-y  + z = 0\\
	x,y,z \geq 0
	\end{cases}
	\]
	
	Desta vez, repare-se que $x + 2y - z \geq \frac 1 2 x + 2y - z = \frac 1 2 (x + y + z) - \frac 3 2 (-y + z) = \frac 1 2$. Logo, se arranjarmos um elemento que tenha pontuação $\frac 1 2$ este tem que ser solução, e o vetor $(0,\frac 1  2, \frac 1 2)$ é admissível e tem esta pontuação, sendo então solução.
	
	Vamos agora generalizar. Suponhamos que temos um pol na forma padrão:
	
	\[
	\begin{cases}
	\min\limits_x cx\\
	Ax = b\\
	x \geq 0
	\end{cases}
	\]
	
	Somando as linhas de $Ax = b$, obtemos igualdades da forma $y^T Ax = y^T b$. Ao contrario do pol na forma canónica, não nos precisamos aqui de preocupar com que $y$ seja $\geq 0$, pois não há sinais de desigualdades a trocar.
	
	Queremos arranjar minorantes da função objetivo, portanto queremos que $y^T A x \leq cx$. Para isso, é suficiente que $y^T A \leq c$, e então temos o nosso minorante: $b^T y = y^T A x \leq cx$ desde que $A^T y \leq c^T$
	
	Claro que queremos arranjar o maior minorante possível, pelo que ficamos que o dual do pol na forma padrão:
	
	\[
	\begin{cases}
	\max\limits_y b^T y\\
	A^T y \leq c^T
	\end{cases}
	\]
	
	O leitor poderá reparar que isto é um pol na forma semicanónica.
	
	Fazemos a mesma estratégia na direção oposta, para justificar que o dual do pol na forma semicanónica é o pol original na forma padrão.
	
	Podemos somar linhas da desigualdade $A^T y \leq c^T$ para obter $x^T A^T y \leq x^T c^T$. Se $Ax = b$, temos que $cx \geq (Ax)^T y = b^T y$, donde $cx$ é majorante de $b^T y$. Queremos o menor majorante possível, pelo que voltamos a
	
	\[
	\begin{cases}
	\min\limits_x cx\\
	Ax = b\\
	x \geq 0
	\end{cases}
	\]
	
	Como desejado.
	
	Queremos agora justificar que estes dois pols têm propriedades que já associámos ao dual. Nomeadamente, queremos demonstrar algo parecido com o teorema forte da dualidade.
	
	Podemos fazer isto diretamente, claro, mas podemos ser mais económicos aproveitando o teorema forte da dualidade que já temos.
	
	Vamos fazer isto usando traduções: vamos traduzir ambos os pols para pols de formas em que podemos dizer que estes são o dual um do outro.
	
	Primeiro que tudo, repare-se que o pol
	
	\[
	P =
	\begin{cases}
	\min\limits_x cx\\
	Ax = b\\
	x \geq 0
	\end{cases}
	\]
	
	É equivalente ao pol na forma canónica
	
	
	\[
	P' =
	\begin{cases}
	\max\limits_x -cx\\
	\begin{bmatrix}
	-A \\
	A
	\end{bmatrix}
	x \leq
	\begin{bmatrix}
	-b \\
	b
	\end{bmatrix}\\
	x \geq 0
	\end{cases}
	\]
	
	Que por sua vez tem dual
	
	\[
	Q' =
	\begin{cases}
	\min\limits_{(y_+,y_-)} -b^T (y_+ - y_-)\\
	-A^T (y_+ - y_-) \geq -c^T\\
	y_+, y_- \geq 0
	\end{cases}
	\]
	
	Mas este pol é uma tradução do pol
	
	\[
	Q =
	\begin{cases}
	\max\limits_y b^T y\\
	A^T y \leq c^T
	\end{cases}
	\]
	
	Como descrito na secção Tradução - Adição de positividades.
	
	Assim sendo, temos que $X_P = \emptyset$ sse $X_{P'} = \emptyset$, $S_P = \emptyset$ sse $S_{P'} = \emptyset$, e coisas semelhantes para $Q$ e $Q'$. Assim sendo, aplicando o teorema forte da dualidade a $P'$ e $Q'$, temos que um se algum de $X_P$, $X_Q$ for vazio, nenhum dos quatro pols acima tem solução. Se $X_P$ e $X_Q$ forem não-vazios, $X_{P'}$ e $X_{Q'}$ são não-vazios e então, por dualidade, $S_{P'}$ e $S_{Q'}$ são não-vazios, e por tradução temos que $P$ e $Q$ têm ambos solução.
	
	Sabendo que o valor mínimo de $P$ é a negação do valor máximo de $P'$, que por sua vez é igual ao valor mínimo de $Q'$ que é a negação do valor máximo de $Q$. Logo, se $s \in S_P$ e $r \in S_Q$ temos que $cs = b^T r$
	
	Isto acaba o nosso esboço de demonstração do teorema forte da dualidade para o pol na forma padrão. O leitor é, no entanto, encorajado a verificar os detalhes, e/ou, se tiver pachorra (o autor não teve), a fazer a demonstração com o lema de Farkas.
	
	Resumindo isto tudo:
	
	\begin{prop} (Teorema forte da dualidade para o pol na forma padrão)
	
	Seja $P$ o pol
	
	\[
	\begin{cases}
	\min\limits_x cx\\
	Ax = b\\
	x \geq 0
	\end{cases}
	\]
	
	E $Q$ o pol
	
	\[
	\begin{cases}
	\max\limits_y b^T y\\
	A^T y \leq c^T
	\end{cases}
	\]
	
	Dizemos que estes são duais um do outro.
	
	Se algum destes não tiver vetores admissíveis, nenhum dos dois tem solução. Se ambos tiverem vetores admissíveis, têm ambos solução e se $s$ é solução de $P$ e $r$ é solução de $Q$ tem-se $cs = b^T r$.
	\end{prop}
	
	\section{Simplexo}
	
	\subsection{Introdução}
	
	O algoritmo do Simplexo é bastante importante, visto que permite resolver problemas de otimização linear muito mais rapidamente do que os métodos que o precederam. E apesar de haver outros algoritmos, na teoria mais rápidos do que o Simplexo\footnote{Os detalhes serão especificados quando se falar sobre complexidade de algoritmos, mas em termos técnicos, o algoritmo do Simplexo é \emph{worst-case} exponencial. Existem algoritmos que resolvem pols em \emph{worst-case} polinomial. No entanto, repare-se que os casos em que o Simplexo corre lentamente são raros, de modo a que, em média (para diversos sentidos da palavra), o Simplexo é polinomial. Estes assuntos estão, no entanto, bastante fora do âmbito da cadeira, pelo que ficam apenas aqui como trivia para o leitor.}, este acaba por ser usado ainda hoje em aplicações práticas, dado que é, na vasta maioria dos casos, extremamente eficiente.
	
	A ideia base por trás do Simplexo é razoavelmente simples. Considere-se o algoritmo dos vetores básicos para o pol na forma padrão. Visto que sabemos que, se houver solução, existe solução básica, basta verificar os vetores básicos todos, e destes o que tiver melhor pontuação é o nosso candidato a solução.
	
	Este algoritmo tem dois problemas: primeiro que tudo, não nos diz se um dado pol tem de facto solução, apenas indica o melhor candidato para tal. Segundo, um pol pode ter imensos vetores admissíveis básicos, pelo que pode ser muito ineficiente verificá-los a todos.
	
	Veremos que, na tentativa de resolver um destes problemas, acabamos por resolver o outro, e o resultado final será o algoritmo do Simplexo.
	
	\subsection{Motivação}
	
	A nossa procura começa com o seguinte: tentaremos aplicar dualidade para obter um teste que nos diga se, dado um pol e um vetor admissível $x$, se $x$ é solução.
	
	Recorde-se que já temos o método das linhas ativas, mas esse tem um problema: requer uma `testemunha'. Ou seja, ele permite-me justificar que $x$ é solução, mas para isso preciso de mostrar que $c^T$ pertence ao cone das colunas de $A^{xT}$. Para isso, preciso de arranjar um $w \geq 0$ tal que $A^{xT} w = c^T$. Arranjar este $w$ não é fácil em geral.
	
	Vamos, então, procurar um método que nos permita, dado um $x$, sob certas condições, determinar se $x$ é solução ou não.
	
	Consideremos então o pol na forma padrão:
	
	\[
	\begin{cases}
	\min\limits_x cx\\
	Ax = b\\
	x \geq 0
	\end{cases}
	\]
	
	Tendo em conta o teorema forte da dualidade, pretendemos descobrir se existe $y$ tal que $A^T y \leq c^T$ e $b^T y = cx$. Tendo em conta que o nosso candidato a solução $x$ é admissível, temos $b^T = x^T A^T$, donde temos que pretendemos descobrir $y$ tal que $x^T A^T y = x^T c^T$ e $A^T y \leq c^T$.
	
	Uma possível ideia seria tentar descobrir $y$ tal que $A^T y = c^T$. Tal $y$ certamente obedeceria a ambas as condições. Infelizmente, em geral, este $y$ não existe, visto que a matriz $A^T$ pode ser muito alta, e então a equação $A^T y = c^T$ pode não ter solução.
	
	Suponhamos, no entanto, que $x$ é básico. Então, $cx = x_{P_x}^T c_{P_x}^T$. Da mesma forma, $b^T y = x^T A^T y = x_{P_x}^T (A^T y)_{P_x}$, e este último, como o leitor poderá facilmente verificar, é igual a $x_{P_x}^T A_{P_x}^T y$.\footnote{Para evitar ambiguidade: na expressão $A_{P_x}^T$, o subscrito toma precedência. Ou seja, por $A_{P_x}^T$ entende-se: a matriz obtida por transposição a partir da matriz $A_{P_x}$. Por outras palavras, a matriz obtida de $A^T$ apagando as linhas $i$ tal que $x_i = 0$.}  Assim sendo, a condição $b^T y = cx$ acaba por ser equivalente a $x_{P_x}^T A_{P_x}^T y = x_{P_x}^T c_{P_x}^T$. Ficamos então com a ideia de resolver $A_{P_x}^T y = c_{P_x}^T$, e esta equação sim é solúvel, pois, por definição de vetor básico, as linhas de $A_{P_x}^T$ são linearmente independentes, e então a caraterística desta matriz é igual à sua altura.
	
	Infelizmente, resolver esta equação não nos assegura nada sobre se $x$ é solução ou não, visto que também é preciso verificar $A^T y \leq c^T$. No entanto, \emph{se} se encontrar uma solução de $A_{P_x}^T y = c_{P_x}^T$ que satisfaz $A^T y \leq c^T$, de certeza que $x$ é solução.
	
	A proposição seguinte diz-nos que não só a existência de tal $y$ é suficiente para que $x$ seja solução, é também necessária.
	
	\begin{prop}
	Dado o pol
	
	\[
	\begin{cases}
	\min\limits_x cx\\
	Ax = b\\
	x \geq 0
	\end{cases}
	\]
	
	E um vetor $x$ admissível,\footnote{Repare-se que não é tomado como hipótese que $x$ seja básico.} temos que $x$ é solução sse existe $y$ tal que $A_{P_x}^T y = c_{P_x}^T$ e $A^T y \leq c^T$.
	\end{prop}
	
	\begin{proof}
	A implicação ($\leftarrow$) é trivial, visto que um $y$ tal que $A^T y \leq c^T$ é admissível do dual, e se $A_{P_x}^T y = c_{P_x}^T$ temos que $x_{P_x}^T A_{P_x}^T y = x_{P_x}^T c_{P_x}^T$ e então (verifique) $x^T A^T y = cx$ e então, visto que $x$ é admissível, temos que $b^T y = cx$, o que, por dualidade, nos indica que $x$ é solução.
	
	Para a implicação ($\rightarrow$), aplique-se o teorema da dualidade forte. Sabemos que se $x$ é solução existe $y$ tal que $A^T y \leq c^T$ e $b^T y = cx$. Mas é fácil ver que esta ultima condição é equivalente a $x_{P_x}^T A_{P_x}^T y = x_{P_x}^T c_{P_x}^T$.
	
	Mostraremos agora que $A_{P_x}^T y = c_{P_x}^T$. Sabemos já que $A_{P_x}^T y \leq c_{P_x}^T$, por $y$ ser admissível do dual, e vejamos então o que acontece se não houver igualdade.
	
	Visto que $x^T A^T y = x^T c^T$, temos que
	
	\[\sum_{i = 1}^n x_i (A^T y)_i = \sum_{i = 0}^n x_i c_i\]
	Supondo que não se tem $A_{P_x}^T y = c_{P_x}^T$, existirá $j \in P_x$ tal que $(A^T y)_j \neq c_j$, e então $(A^T y)_j < c_j$. Separando este termo do resto, temos
	
	\[x_j (A^T y)_j + \sum_{i \neq j} x_i (A^T y)_i = x_j c_j + \sum_{i \neq j} x_i c_i\]
	
	Mas visto que todos os $x_i$ são maiores ou iguais que zero, e como $A^T y \leq c^T$, temos que $x_i (A^T y)_i \leq x_i c_i$ para todo $i$, e então
	
	\[\sum_{i \neq j} x_i (A^T y)_i \leq \sum_{i \neq j} x_i c_i\]
	
	E como $x_j$ é positivo e, por hipótese, $(A^T y)_j < c_j$, temos que $x_j (A^T y)_j < x_j c_j$. Somando estas duas desigualdades, obtemos
	
	\[x_j (A^T y)_j + \sum_{i \neq j} x_i (A^T y)_i < x_j c_j + \sum_{i \neq j} x_i c_i\]
	
	O que contradiz a igualdade acima. Logo, não pode existir tal $j$, donde obtemos que $A_{P_x}^T y = c_{P_x}^T$, como pretendiamos demonstrar.
	\end{proof}
	
	Este novo critério para verificar se um dado $x$ é solução é, em geral, não muito mais útil do que os que temos até agora. Existe um caso, no entanto, em que funciona na perfeição.
	
	Suponha-se que $x$ é um vetor admissível básico. Então, decerto que a equação $A_{P_x}^T y = c^T_{P_x}$ é possível. Verificar se existe uma solução que obedece a $A^T y \leq c^T$ é, em geral, difícil
, mas se houver uma única solução, fica uma tarefa trivial. Ora, isto acontece precisamente quando $A_{P_x}^T$ é quadrada, ou seja, $\# P_x = m$, onde $m$ é o número de linhas da matriz $A$.

	Se $x$ é um vetor básico tal que $\#P_x = m$, dizemos que $x$ é um \emph{vetor básico não-degenerado}. O que vimos agora é um teste rápido e eficaz de testar se um vetor básico não-degenerado é solução:
	
	\begin{lstlisting}[mathescape=true, keepspaces=true]
Resolver a equação $A_{P_x}^T y = c_{P_x}^T$
Verificar se $A^T y \leq c^T$
Caso afirmativo, $x$ é solução
Caso contrário, $x$ não é solução \end{lstlisting}

	A restrição que $x$ seja não-degenerado não é tão restritiva quanto possa parecer. Em geral, vetores degenerados são raros, especialmente quando se trabalha com aproximações e dados reais.
	
	Para ter uma ideia de porquê (mas repare-se que o que se segue não é uma prova formal) vamos considerar o menor caso não-trivial de um vetor básico degenerado: considere-se um pol em $\R^3$ cuja matriz $A$ seja uma matriz $2 \times 3$ de linhas linearmente independentes. O conjunto-admissível será então uma reta.
	
	Neste contexto, um vetor básico degenerado seria um vetor com $0$ ou $1$ coordenadas iguais a zero. Ora, a existência de tal vetor requer que a que a reta que representa o conjunto admissível passe por um dos eixos. O leitor pode facilmente convencer-se que, na prática, é preciso muita pontaria para que a reta toque nos eixos. Acontece (mas não será justificado) que isto é verdade em geral, ou seja, é muito raro haver vetores básicos degenerados. Para mais, mesmo que haja, a mais pequena disturbância (e.g. um erro de arredondamento) pode transformar o problema de volta num problema não-degenerado. Assim sendo, vamos desenvolver o algoritmo do Simplexo ignorando a sua existência. É possível adaptar o algoritmo para contextos em que vetores degenerados aparecem, mas isto encontra-se fora do âmbito da cadeira.
	
	Vamos, então, trabalhar sob a seguinte hipótese: dizemos que um pol $P$ é \emph{não-degenerado no sentido lato} se não tiver nenhum vetor básico degenerado.
	
	Neste contexto, o teste acima, para além de eficaz e eficiente, dá-nos uma forma de, em caso negativo, arranjar um vetor melhor, como se verá no seguimento. É, aliás, nisto que se baseia o algoritmo do Simplexo. Os detalhes serão feitos a seguir, mas a ideia é um ciclo da forma:
	
	\begin{lstlisting}[mathescape=true, keepspaces=true]
Começar com um vetor básico $x$
$x$ é solução?
Se sim, parar.
Se não, arranjar um vetor melhor.
  Conseguimos arranjar vetores arbitráriamente bons?
  Se sim, parar. Não há solução.
  Se não, (provaremos) é possível arranjar um vetor
    básico $v$ de pontuação melhor do que $x$.
    Voltar ao início, com $x = v$
	\end{lstlisting}
	
	\subsection{Concretização}
	
	Para finalizar o algoritmo, falta-nos apenas um ingrediente: se soubermos que $x$ não é solução, como arranjar um vetor de melhor pontuação?
	
	Por outras palavras, o que desejamos é arranjar um vetor $\tilde x$ tal que $\tilde x$ seja também admissível e tal que $c \tilde x < cx$.
	
	Para arranjar um candidato, voltemos à nossa forma de justificar que um vetor $x$ \emph{não} é solução. Supondo que $x$ é um vetor básico não-degenerado, resolvemos a equação $A_{P_x}^T y = c^T_{P_x}$ e verificámos que ${A^T y \not\leq c^T}$. Assim sendo, existe uma casa, digamos a $j$-ésima, tal que ${(A^T y)_j > c_j}$. Sabemos decerto que $j \not \in P_x$, pois para estes índices há igualdade.
	
	Considere-se o que acontece quando tentamos `andar um pouco na $j$-ésima direção'. Seja $e_j$ o vetor unitário com 1 na casa $j$. Queremos que o nosso $h$ seja `parecido com` algo da forma $t e_j$, mas infelizmente o vetor $x + h$ assim obtido pode não ser admissível.
	
	Para o ser, é preciso o requerimento que $Ah = 0$ (verifique), portanto tente-se modificar o vetor $e_j$ ligeiramente para isto acontecer. Considere-se o vetor $\hat h$ tal que $h_i = 0$ para $i \not \in P_x$ e tal que $A_{P_x} \hat h_{P_x} = A e_j$. Então, o vetor $h = \hat h - e_j$ satisfaz $A h = 0$. Para mais, se se considerar $\tilde x = x - \varepsilon h$ para $\varepsilon > 0$ pequeno o suficiente, $\tilde x \geq 0$. Para ver porquê, repare-se que todas as casas $i$ de $\tilde x$ obedecem a um dos três casos seguintes:
	
	\begin{enumerate}
	\item Se $i \not \in P_x$ e $i \neq j$, $\tilde x_i = x_i = 0 \geq 0$
	
	\item Se $i = j$, $\tilde x_i = x_i + \varepsilon = \varepsilon \geq 0$
	
	\item Se $i \in P_x$, $\tilde x_i = x_i - \varepsilon h_i$, e como $x_i > 0$, para $\varepsilon$ pequeno o suficiente isto continua a ser $\geq 0$.
	\end{enumerate}
	
	Um detalhe importante: se $h_i \geq 0$ para todo $i \in P_x$, \emph{qualquer $\varepsilon$ funciona}. No caso contrário, existe um $\varepsilon$ maximal, que será útil na sequência.
	
	Sabendo que $\tilde x$ é admissível, basta agora mostrar que a sua pontuação é melhor do que a de $x$. Para isso, basta mostrar que $ch > 0$. Para esse efeito, eis as seguintes contas:
	
	\begin{align*}
	ch =& c \hat h - c_j \\
	   =& c_{P_x} \hat h_{P_x} - c_j\\
	   =& y^T A_{P_x} \hat h_{P_x} - c_j\\
	   =& y^T A e_j - c_j\\
	   =& (A^T y)_j - c_j\\
	   >& 0
	\end{align*}
	
	O que, agradávelmente, é precisamente o desejado. Assim sendo, temos uma forma de arranjar vetores admissíveis melhores que $x$.
	
	\medskip
	
	Para terminar rapidamente um caso, repare-se no seguinte: se $h \leq 0$, $x - \varepsilon h$ é admissível \emph{para qualquer $\varepsilon$ positivo}. Assim sendo, é possível arranjar vetores arbitrariamente bons: basta fazer $\varepsilon \rightarrow \infty$, obtendo-se que $x - \varepsilon h$ é sempre admisível, e a sua pontuação é $cx - \varepsilon ch \rightarrow - \infty$. Disto se conclui o seguinte:
	
	\textbf{Se o vetor $h$ obtido anteriormente for tal que $h \leq 0$, o pol não tem solução.}
	
	Basta agora terminar o outro caso: supondo que $h \not \leq 0$, como arranjar um vetor melhor? De preferência, um vetor básico não-degenerado, de modo a podermos repetir o algoritmo.
	
	Recorde-se que a razão pela qual, neste caso, não se pode tomar valores de $\varepsilon$ arbitrariamente grandes é porque para $\varepsilon$ grande o suficiente o vetor $x - \varepsilon h$ pode não ser $\geq 0$.
	
	Suponha-se que $i$ é tal que $h_i > 0$. Então, sendo $\xi_i = \frac{x_i}{h_i}$, o sinal de $x_i - \varepsilon h_i$ depende de se $\varepsilon$ é menor, maior, ou igual a $\xi_i$.
	
	O maior $\varepsilon$ tal que $x_i - \varepsilon h_i \geq 0$ é precisamente $\xi_i$. Considere-se, então, $\xi = \min_{i \in P_h} \xi_i$, suponhamos $\xi_k$. Então, é facil ver que, para $\varepsilon = \xi$ se tem $x - \varepsilon h \geq 0$.
	
	Para ver isto, considere-se a divisão em casos feita acima. Para $i \not \in P_x$ é trivialmente verdade que $x_i - \varepsilon h_i \geq 0$; para $i \in P_x$ isto é verdade desde que $h_i \leq 0$, e finalmente, no caso em que $h_i > 0$ temos que
	
	\[x_i - \varepsilon h_i = x_i - \xi h_i \geq x_i - \xi_i h_i = 0\]
	
	Em particular, para $i = k$, $x_i - \varepsilon h_i = 0$.
	
	Assim sendo, repare-se que não só é o vetor obtido admissível, mas também é tal que as suas coordenadas positivas são as de $x$, com a adição do índice $j$ e com a exclusão do índice $k$ e possívelmente mais. Assim sendo, o número de coordenadas positivas não aumentou, e há esperanças que o vetor resultante seja básico e não-degenerado. É este o nosso candidato para `sucessor'.
	
	Isto deixa-nos, no entanto, com uma primeira versão do algoritmo do simplexo:
	
	\begin{lstlisting}
Começar com um vetor básico não-degenerado $x$
Seja $y$ a solução de $A_{P_x}^T y = c_{P_x}^T$
Se $A^T y \leq c^T$:
	$x$ é solução. Parar.
Caso contrário:
	Seja $j$ um índice tal que $(A^T y) j > c_j$
	Seja $h$ o vetor definido da seguinte forma:
		$h_j = -1$
		$h_{P_x}$ é a solução de $A_{P_x} h_{P_x} = A e_j$
		$h_i = 0$ para o resto dos índices
	Se $h \leq 0$:
		Há vetores admissíveis arbitráriamente bons, pelo que o pol não tem solução. Parar.
	Caso contrário:
		Seja $\xi$ o menor número da forma $x_i/h_i$, para $i \in P_h$
		Seja $\tilde x = x - \xi h$
		Se $\tilde x$ é básico não-degenerado:
			Começar o algoritmo de novo com $x = \tilde x$
		Caso contrário:
			Desistir. :(
	\end{lstlisting}
	
	Seria agora desejável provar que desistências não acontecem. Infelizmente, em geral, isto não acontece. Provaremos na seguinte secção que $\tilde x$ é sempre básico, mas nada o proibe, em geral, de ser degenerado.
	
	Claro que se, por acaso, o nosso pol for não-degenerado, isso nunca acontecerá. E como confiamos que na prática pols degenerados quase nunca acontecem, ficamos satisfeitos com o nosso algoritmo.
	
	\subsection{Pontas soltas}
	
	Nesta secção provaremos que, se o nosso pol for não-degenerado, o algoritmo descrito acima termina sempre e nunca nos tomamos por desistentes. Ilustramos também um critério ligeiramente parvo, mas frequentemente útil em testes da Sernadas, para justificar que um pol é não-degenerado.
	
	Para mais, descrevemos duas formas possíveis de encontrar um vetor básico admissível: num dos casos, supondo que temos à nossa disposição um vetor admissível arbitrário, e noutra supondo que o nosso pol provém de um pol na forma canónica, com o vetor $b \geq 0$. Esta última condição, apesar de parecer restritiva, acaba por ser extremamente comum se pensarmos no pol no contexto económico.
	
	Comecemos por finalizar a prova de completude do simplexo:
	
	\begin{prop}
	O vetor $\tilde x$ obtido após uma iteração do simplexo é sempre básico.
	\end{prop}
	
	\begin{proof}
	Sejam $a_1, \cdots, a_n$ os vetores-coluna de $A_{P_x}$ (que sabemos, por hipótese, formar uma base de $\R^n$) e seja $\tilde a$ o vetor $A e_j$.
	
	Repare-se que os índices de coordenadas positivas de $\tilde x$ são precisamente aqueles de $x$, mais o índice $j$, e retirando um ou mais dos índices correspondentes a $a_1, \cdots, a_n$. Mais precisamente, se $\tilde a = \sum_{i=1}^n t_i a_i$, os índices retirados têm \emph{necessariamente} o $t_i$ correspondente diferente de zero, pois os índices associados a $t_i=0$ não mudam de $x$ para $\tilde x$.
	
	Considere-se agora o que acontece quando, da lista $a_1, \cdots, a_n$, se adiciona $\tilde a$ e são retirados um ou mais vetores correspondentes a $t_i \neq 0$. A lista resultante é necessariamente linearmente independente: caso contrário, visto que os $a_1, \cdots, a_n$ são linearmente independentes, ter-se-ia que $\tilde a$ se poderia escrever em função do resto dos vetores. Por estes formarem uma base, existe apenas uma forma de escrever $\tilde a$ em função destes, e pelo menos um dos vetores necessários para isto foi removido da lista, pelo que entramos em contradição e a lista é linearmente independente. Isto prova que $\tilde x$ é básico.
	\end{proof}
	
	Isto permite-nos, então, afirmar:
	
	\begin{prop}
	Se um pol é não-degenerado no sentido lato, a aplicação do algoritmo do simplexo termina sempre.
	\end{prop}
	
	É, então, útil assegurar que um pol dado é não-degenerado. A seguinte proposição dá um critério teórico para assegurar esta condição.
	
	\begin{prop}
	Considere-se o seguinte pol:
	
	\[
	\begin{cases}
	\min\limits_x cx\\
	Ax = b\\
	x \geq 0
	\end{cases}
	\]
	
	Onde a matriz $A$ é $m \times n$. Dizemos que este é \emph{não-degenerado no sentido estrito} se $b$ não pode ser escrito como combinação linear de menos de $m$ colunas de $A$.
	
	Um pol não-degenerado no sentido estrito também o é no sentido lato.
	\end{prop}
	
	\begin{proof}
	Isto é, na realidade, uma trivialidade. Afinal, se houvesse um vetor básico degenerado $x$, $b$ poderia ser escrito como combinação linear das (menos que $m$) colunas de $A_{P_x}$, pois $b = A_{P_x} x_{P_x}$.
	\end{proof}
	
	[O fim desta secção está ainda em construção]
	
	%todo show that x basic iff #Px = m
	
	\appendix
	
	\section{Conceitos de Álgebra Linear}
	
	%i dont know if this is all necessary.
	
	\begin{definition*}
	Sejam $x, y$ pontos em $\R^n$. Define-se o produto interno de $x$ e $y$, denotado $\langle x, y \rangle$, como $x_1 y_1 + \cdots + x_n y_n$.
	
	Repare-se que, para qualquer $x$, $\langle x, x \rangle$ é sempre maior ou igual a zero, por ser uma soma de quadrados.
	
	A norma de um ponto, $\lvert x \rvert$, é definida como $\sqrt{\langle x, x \rangle}$. A distância entre dois pontos, $x, y$, denotada $d(x, y)$, é definida como $\lvert x - y \rvert$.
	\end{definition*}
	
	\begin{prop}
	(Desigualdade de Cauchy-Schwarz) $\lvert \langle x, y \rangle \rvert \leq \lvert x \rvert \lvert y \rvert$
	
	Igualdade acontece sse $x = \lambda y$ ou $y = \lambda x$, para algum $\lambda \in R$.
	\end{prop}
	
	\begin{proof}
	Considere-se a função $f(\lambda) = \langle y - \lambda x, y - \lambda x \rangle$. Isto é uma função real de variável real, que corresponde a uma parábola. De facto, aplicando a propriedade distributiva, obtemos:
	
	\[f(\lambda) = \langle y, y \rangle - \lambda 2 \langle x, y \rangle + \lambda^2 \langle x, x \rangle\]
	
	Ora, sabemos que a função $f$ só toma valores maiores ou iguais a zero, pelo que, inspecionando a formula resolvente, concluimos que, no caso em que $x \neq 0$ (para não dividirmos por 0) o discriminante (o valor que fica dentro da raíz quadrada; $b^2 - 4 a c$) é necessariamente menor ou igual a zero. Assim sendo,
	
	\[(2 \langle x, y \rangle) ^2 - 4 \langle x, x \rangle \langle y, y \rangle \leq 0\]
	
	Disto se conclui rapidamente a desigualdade desejada. O caso não considerado, em que $x = 0$, é trivial.
	
	Finalmente, inspecionemos o caso de igualdade. Se $x = 0$ temos igualdade, e de facto temos $x = \lambda y$ com $\lambda = 0$.
	
	Pelo outro lado, se $x \neq 0$, igualdade acontece sse o discriminante de $f$ é zero, sse a função $f$ tem uma raíz $\lambda$. Mas, ora, se $f(\lambda) = 0$ temos que $y = \lambda x$, como queriamos demonstrar.
	\end{proof}
	
	\begin{prop}
	(Desigualdade triangular) $\lvert x + y \rvert \leq \lvert x \rvert + \lvert y \rvert$
	\end{prop}
	
	\begin{proof}
	Repare-se que
	
	\begin{align*}
	\lvert x + y \rvert ^2 &= \lvert x \rvert ^2 + 2 \langle x, y \rangle + \lvert y \rvert ^2 \\
	&\leq \lvert x \rvert ^2 + 2 \lvert \langle x, y \rangle \rvert + \lvert y \rvert ^2 \\
	&\leq \lvert x \rvert ^2 + 2 \lvert x \rvert \lvert y \rvert + \lvert y \rvert ^2 \\
	&= (\lvert x \rvert + \lvert y \rvert)^2
	\end{align*}
	
	Tomando a raíz quadrada dos dois lados obtemos o desejado.
	\end{proof}
	
	\section{Conceitos Topológicos}
	
	\begin{definition*}
	Seja $x \in \R^n$. Define-se a \emph{vizinhança-$\varepsilon$ de $x$}, denotada $V_\varepsilon(x)$, como o conjunto de pontos $y$ tal que $d(x, y) < \varepsilon$. Implícito estará sempre que $\varepsilon > 0$.
	\end{definition*}
	
	\begin{definition*}
	Seja $S \subseteq \R^n$. Definem-se o interior, exterior, borda e fecho de $S$, respetivamente, da seguinte forma:
	
	\[\interior S = \{\, x \in S \mid \exists_\varepsilon V_\varepsilon (x) \subseteq S\,\}\]
	
	\[\exterior S = \interior (\R^n \setminus S)\]
	
	\[\border S = \R^n \setminus \interior S \setminus \exterior S\]
	
	\[\overline S = S \cup \border S = \interior S \cup \border S = \R^n \setminus \exterior S\]
	
	Um conjunto $S$ diz-se aberto se $S = \interior S$. Um conjunto $S$ diz-se fechado se o seu complementar, $\R^n \setminus S$, for aberto, ou, equivalentemente, se $S = \overline S$.
	
	Usando a desigualdade triangular é possível provar que $V_\varepsilon(x)$ é sempre aberto.
	\end{definition*}
	
	\begin{definition*}
	Seja $D \subseteq \R^n$, $f : D \to \R^m$.
	\end{definition*}
	
	\begin{prop}
	Seja $D \subseteq \R^n$, $f : D \to \R^m$.
	\end{prop}
	
\end{document}