\documentclass{article}

\usepackage{amsmath}
\usepackage{amsthm}
\usepackage{amssymb}
\usepackage[utf8]{inputenc}
\usepackage[portuguese]{babel}

\title{Introdução à Lógica}
\author{}
\date{}

\theoremstyle{definition}
\newtheorem{definicao}{Definição}

\theoremstyle{remark}
\newtheorem{obs}{Observação}

\addto\captionsportuguese{
	\renewcommand{\proofname}{Dem}
}

\newcommand{\V}{\mathbf{V}}
\newcommand{\F}{\mathbf{F}}
\newcommand*{\lneg}{\mathord{\sim}}

\begin{document}
	\maketitle

	\section{Introdução}

	A lógica é, entre outras coisas, o estudo de linguagens e demonstrações. Normalmente, ao fazer matemática, usamos linguagem natural (português, inglês, etc.), mas vários paradoxos descobertos no início do século passado fizeram-nos entender que linguagem natural não serve de fundamento adequado. Eis alguns exemplos de paradoxos para ilustrar o que se quer dizer:
	
	\begin{itemize}
	\item (Paradoxo do mentiroso, antiguidade) Um homem diz a frase ``Esta frase é falsa''. Ele está a mentir ou a dizer a verdade?
	
	\item (Paradoxo de Berry, c. 1900) Seja $n$ o menor número natural que não pode ser descrito em menos que 20 palavras. (Acabámos de descrever $n$ em 16 palavras.) 
	%todo, explicar paradoxo de russel
	\end{itemize}
	
	Estes paradoxos atormentaram filósofos e matemáticos durante alguns anos, e foi para os evitar que os matemáticos começaram a pensar em linguagens feitas para o propósito de fazer matemática, nas quais não se conseguisse expressar estes paradoxos. Na prática, continua-se a usar linguagem natural, mas de forma que possa ser traduzida para estas linguagens.
	
	Hoje vamos falar sobre uma destas linguagens. É uma das mais simples, mas tem utilidade em aplicações, por exemplo, em ciência de computadores, e chama-se \textit{lógica proposicional}.
	
	\section{Proposições}
	
	Na linguagem de lógica proposicional, as frases são construídas a partir de
	
	\begin{itemize}
	\item Variáveis ($p$, $q$, $a$, $b$, $c$, $x$, $y$, $z$, ...)
	
	\item Conectivos ($\land$, $\lor$, $\lneg$, ...)
	
	\item Parênteses
	\end{itemize}
	
	Normalmente, referimo-nos a frases por nome de letras gregas. Por exemplo, podemos considerar a frase $\varphi$ (fi): $a \lor \lneg(b \land c)$.
	
	Cada variável é suposto representar uma afirmação, que pode ser verdadeira ou falsa. Os conectivos significam, respetivamente, `e', `ou' e `não'. Em breve introduziremos o `implica' e `equivalente'.
	
	Esta linguagem não é muito expressiva. Por exemplo, não podemos expressar frases como `A soma dos ângulos internos de um triângulo é $180^\circ$'. Podemos, no entanto, expressar relações entre afirmações.
	
	Por exemplo, se $C$ representa a afirmação `Está a chover' e $S$ a afirmação `Saí de casa', posso expressar `Saí de casa e não está a chover' como $S \land \lneg C$.
	
	\section{Semântica}
	
	Dada uma frase $\varphi$, podemos tentar averiguar se (sabendo que variáveis são verdadeiras e quais são falsas) $\varphi$ é verdadeira ou falsa. Por exemplo, se sabemos que a afirmação $a$ é verdadeira mas $b$ é falsa, então a afirmação $a \lor b$ é verdadeira, mas $a \land b$ é falsa.
	
	À ``veracidade'' de uma frase (isto é, se ela é verdadeira ou falsa) chamamos o seu \emph{valor de verdade}. Ou seja, o valor de verdade de uma frase é `verdadeiro' se esta é verdadeira, e `falso' se esta é falsa.
	
	O interesse disto, no entanto, não é no caso em que sabemos o valor de verdade das variáveis, mas sim ver todos os casos possíveis. Para formalizarmos isso, vamos establecer notação.
	
	\begin{definicao}
	Uma \emph{valoração} é uma atribuição de valores de verdade a variáveis. Ou seja, associamos a cada variável o valor `verdadeiro' ou `falso'.
	\end{definicao}
	
	\begin{obs}
	Escrever `verdadeiro' e `falso' constantemente é inconveniente, e como tal estes termos são normalmente abreviados. Escritores em português normalmente abreviam-nos para $\V$ e $\F$; em inglês isto é normalmente mudado para $\mathbf{T}$ e $\mathbf{F}$ (\textit{true} e \textit{false}), e alguns autores usam notações independentes de linguagem, por exemplo $\top$ para verdadeiro e $\bot$ para falso, ou $1$ e $0$.
	
	A partir de agora, usaremos a notação $\V$ $\F$.
	\end{obs}
	
	Normalmente usamos o nome $\rho$ (ró) para nos referirmos a valorações.
	
	Por exemplo, seja $\varphi$ a frase $(a \land b) \lor \lneg (b \lor c)$. Podemos considerar a valoração $\rho$ que atribui os valores $a:\V$, $b:\F$ e $c:\V$.
	
	Uma valoração corresponde, intuitivamente, a um `universo', onde algumas coisas são verdadeiras e outras são falsas. Fixa, então, uma valoração, faz sentido perguntar se uma certa frase complexa é verdadeira.
	
	Averiguemos se $\varphi$ é verdade sob a valoração $\rho$. Para o fazer, substituimos cada variável pelo seu valor de verdade. Neste caso:
	\[(\V \land \F) \lor \lneg (\F \lor \V).\]
	
	Depois, avaliamos a expressão resultante, tendo em conta o nosso conhecimento dos operadores $\land$, $\lor$ e $\lneg$:
	
	\begin{itemize}
	\item (`e') A afirmação $x \land y$ é verdade quando são ambos verdade. Isto é:
	\begin{align*}
	\V \land \V &\text{ é } \V\\
	\V \land \F &\text{ é } \F\\
	\F \land \V &\text{ é } \F\\
	\F \land \F &\text{ é } \F
	\end{align*}
	
	\item (`ou') A afirmação $x \lor y$ é verdade quando pelo menos um destes dois é verdade. Isto é:
	\begin{align*}
	\V \lor \V &\text{ é } \V\\
	\V \lor \F &\text{ é } \V\\
	\F \lor \V &\text{ é } \V\\
	\F \lor \F &\text{ é } \F
	\end{align*}
	
	\item (`não') A afirmação $\lneg x$ é verdade quando $x$ é falso. Isto é:
	\begin{align*}
	\lneg \V &\text{ é } \F\\
	\lneg \F &\text{ é } \V
	\end{align*}
	\end{itemize}
	
	Assim sendo, a expressão acima pode ser simplificada para $\F \lor \lneg \V$, que por sua vez é simplificado para $\F \lor \F$, ou seja, $\F$. Ou seja, sob a valoração $\rho$, a frase $\varphi$ é falsa.
	
	\textbf{Exercício 0:} Tenta encontrar uma valoração que faça com que $\varphi$ fique verdadeira.
	
	\begin{definicao}
	Se, sob a valoração $\rho$, a frase $\varphi$ é verdadeira, dizemos que $\rho$ satisfaz $\varphi$. Simbolicamente:
	\[\rho \Vdash \varphi.\]
	\end{definicao}
	
	Vamos fazer mais um exemplo. Seja $\varphi$ a frase $(a \land b) \lor (\lneg a \lor \lneg c)$, e considere-se a valoração $\rho$ que faz:
	\[
	\begin{array}{c|c|c}
	a & b & c\\
	\hline
	\V & \F & \F
	\end{array}
	\]
	
	Vamos averiguar se $\rho \Vdash \varphi$. Substituindo as variáveis pelos seus valores de verdade, obtemos
	\[(\V \land \F) \lor (\lneg \V \lor \lneg \F),\]
	que, simplificando, dá
	\begin{gather*}
	\F \lor (\F \lor \V)\\
	\F \lor \V\\
	\V
	\end{gather*}
	
	E concluímos, então, $\rho \Vdash \varphi$.
	
	\smallskip
	
	De vez em quando, é-nos dado uma fórmula $\varphi$ e queremos saber para que valorações é que ela é verdade.\footnote{Um dos problemas mais estudados de ciência de computadores é até mais `simples' do que isto: dada uma fórmula $\varphi$, será que existe \emph{pelo menos uma} valoração que a satisfaz? Este problema é chamado SAT (de `satisfiability') e descobrir algoritmos para o resolver eficientemente é uma área de pesquisa muito ativa.} Uma forma primitiva mas eficaz de fazer isso é simplesmente tentar todas as valorações possíveis.
	
	Para este efeito, é costume organizar a informação no que se chamam \emph{tabelas de verdade}. Para as construir, começamos por, nas primeiras $n$ colunas, organizar todos os valores possíveis das nossas variáveis. Por exemplo, para $\varphi: (a \lor b) \land \lneg (a \land b)$, as variáveis são $a$ e $b$, e nas primeiras $2$ colunas descrevemos todas as 4 formas possíveis de lhes atribuir valores de verdade:
	\[
	\begin{array}{c|c}
	a & b\\
	\hline
	\V & \V\\
	\V & \F\\
	\F & \V\\
	\F & \F
	\end{array}
	\]
	
	Depois, nas colunas ao lado, escrevemos contas intermédias que nos sejam úteis no cálculo de $\varphi$. Neste caso, é-nos útil calcular, por exemplo, $(a \lor b)$ e $(a \land b)$.
	\[
	\begin{array}{c|c|c|c|c|c}
	a & b & a \lor b & a \land b & \lneg (a \land b) & (a \lor b) \land \lneg (a \land b)\\
	\hline
	\V & \V &&&\\
	\V & \F &&&\\
	\F & \V &&&\\
	\F & \F &&&
	\end{array}
	\]
	
	Depois vamos preenchendo a tabela da esquerda para a direita com a informação que já temos. Tenta preenchê-la sem olhar para a solução. No final deverás ficar com uma tabela igual à seguinte.
	
	\[
	\begin{array}{c|c|c|c|c|c}
	a & b & a \lor b & a \land b & \lneg (a \land b) & (a \lor b) \land \lneg (a \land b)\\
	\hline
	\V & \V &\V&\V&\F&\F\\
	\V & \F &\V&\F&\V&\V\\
	\F & \V &\V&\F&\V&\V\\
	\F & \F &\F&\F&\V&\F
	\end{array}
	\]
	
	Concluímos, então, que as valorações que satisfazem $\varphi$ são precisamente aquelas que dão valores diferentes a $a$ e $b$.
	
	\medskip
	
	\textbf{Exercício 1:} Averigua para que valorações são verdadeiras as seguintes proposições:
	
	\begin{enumerate}
	\item $a \land b \land c$
	
	\item $a \land (b \lor c)$
	
	\item $(a \land b) \lor \lneg (b \lor c)$
	
	\item $(a \land \lneg b) \lor (\lneg a \lor b)$
	
	\item $((a \lor b) \land \lneg (a \land c)) \lor (d \lor b \lor \lneg d)$
	
	\item $(a \lor b) \land (\lneg a \lor c \lor d) \land (\lneg c) \land (\lneg d \lor b) \land (\lneg b \lor (a \land \lneg d))$
	\end{enumerate}
	
	(Sugestão: podes resolver estes exercícios tentando todos os casos possíveis. No entanto, isto é mesmo chato com mais do que três variáveis! Em particular, és desencorajado de fazer os últimos desta forma. Assim sendo, tenta encontrar formas mais fáceis de os fazer.)
	
	\section{Tautologias}
	
	Se tiveres feito o exercício 1, hás-de ter reparado que algumas das proposições são verdadeiras independentemente da valoração que escolhas. Estas afirmações são verdadeiras em qualquer universo, e têm um nome especial: tautologias.
	
	\begin{definicao}
	Dada uma proposição $\varphi$, dizemos que $\varphi$ é uma tautologia se $\rho \Vdash \varphi$ para toda a valoração $\rho$. Simbolicamente:
	\[\vDash \varphi.\]
	\end{definicao}
	
	Vamos ver alguns exemplos de tautologias.
	
	O exemplo mais simples é uma afirmação da forma $\varphi : a \lor \lneg a$. Vamos verificar que ela é de facto uma tautologia.
	
	Para isto, vamos enumerar todas as valorações possíveis. Neste caso, a única variável é $a$, por isso só há dois casos: $a$ é $\V$ (chamemos a esta valoração $\rho$) ou $a$ é $\F$ (chamemos a esta $\rho'$).
	
	Sob a valoração $\rho$, $\varphi$ fica $\V \lor \lneg \V$, que é $\V$. Da mesma forma, sob a valoração $\rho'$, $\varphi$ fica $\F \lor \lneg \F$, que é $\V$. Assim sendo, todas as valorações satisfazem $\varphi$, ou seja, $\varphi$ é uma tautologia.
	
	Vamos fazer um exemplo mais complicado. Seja $\varphi$ a frase $(a \lor (b \lor \lneg a))$.
	
	Para verificar que esta frase é uma tautologia, vamos enumerar todas as valorações, e para cada uma delas avaliar o valor de verdade de $\varphi$.
	
	\[
	\begin{array}{c|c|c}
	a & b & a \lor (b \lor \lneg a)\\
	\hline
	\V & \V &\\
	\V & \F &\\
	\F & \V &\\
	\F & \F &
	\end{array}
	\]
	
	\textbf{Exercício 2:} Preenche a tabela acima. (Adiciona mais colunas para cálculos intermédios se achares necessário.)
	
	\smallskip
	
	Após fazeres o exercício 2, deverás obter o resultado
	
	\[
	\begin{array}{c|c|c}
	a & b & a \lor (b \lor \lneg a)\\
	\hline
	\V & \V &\V\\
	\V & \F &\V\\
	\F & \V &\V\\
	\F & \F &\V
	\end{array}
	\]
	
	Ou seja, para todas as valorações, $\varphi$ é verdadeira. Ou seja, $\varphi$ é uma tautologia.
	
	\smallskip
	
	\textbf{Exercício 3:} Verifica quais das seguintes afirmações são verdadeiras. (Recorda-te que `$\vDash \varphi$' significa `$\varphi$ é uma tautologia'.
	
	\begin{enumerate}
	\item $\vDash (\lneg a \lor \lneg (\lneg a))$
	
	\item $\vDash \lneg a \lor \lneg b \lor (a \lor c) \land (b \lor c)$
	
	\item $\vDash a$
	
	\item $\vDash (a \lor \lneg a) \lor \lneg (a \land (c \lor \lneg c)) \lor (b \land (d \lor \lneg d))$
	
	\item $\vDash a \land b \land c \land d$
	
	\item $\vDash \lneg (a \land \lneg b) \lor \lneg(c \lor b) \lor (a \land c) \lor (a \land (b \lor \lneg c))$
	\end{enumerate}
	
	
	Às vezes, no entanto, estamos interessados em restringir a nossa atenção a um conjunto limitado de universos. Por exemplo, suponhamos que quero restringir a minha atenção aos universos onde está a chover, e onde sei que não saio de casa sem guarda-chuva. Se $C$ representa a afirmação `Está a chover' e $G$ a afirmação `Saio de casa sem guarda-chuva', posso averiguar o que é verdade num universo onde sei $C$ e $\lneg (C \land G)$.
	
	Por exemplo, num tal universo, consigo de certeza concluir $\lneg G$. Para justificar isto, vamos ver em que universos é que tanto $C$ como $\lneg (C \land G)$ são verdadeiros:
	
	\begin{tabular}{|c|c|c|c|}
	\hline
	$C$ & $G$ & $\lneg(C \land G)$ & Conclusão \\
	\hline
	$\V$ & $\V$ & $\F$ & Não \\
	$\V$ & $\F$ & $\V$ & Sim \\
	$\F$ & $\V$ & $\V$ & Não \\
	$\F$ & $\F$ & $\V$ & Não \\
	\hline
	\end{tabular}
	\smallskip
	
	Portanto, neste caso, há apenas um universo em que ambas as premissas são verdadeiras, e de facto, neste universo, $G$ é falso.
	
	\begin{definicao}
	Seja $\Gamma$ (gama) um conjunto de proposições, $\varphi$ uma proposição. Dizemos que \emph{$\varphi$ é consequência semântica de $\Gamma$} se todas as valorações $\rho$ que satisfazem todos os elementos de $\Gamma$ também satisfazem $\varphi$. Simbolicamente:
	
	\[\Gamma \vDash \varphi.\]
	
	Podes também ler isto como `Se $\Gamma$, então $\varphi$'.
	\end{definicao}
	
	Sob esta notação, a conclusão de acima podia ter sido escrita como
	\[\{\, C, \lneg(C \land G) \,\} \vDash G.\]
	
	É comum, ao escrever coisas desta forma, omitir a notação de conjunto e escrever só
	\[C, \lneg(C \land G) \vDash G.\]
	
	\begin{obs}
	Recorda-te que, na matemática, uma afirmação da forma `para todo $x$' é sempre verdadeira se não houver nenhum $x$. Neste caso, isto reflete-se de duas formas:
	
	\begin{enumerate}
	\item Se $\Gamma$ é tal que nenhuma valoração satisfaz todas as suas fórmulas, $\Gamma \vDash \varphi$ é sempre verdade. (Dizemos que é verdade \emph{por vacuosidade}.) Por exemplo, verifica o seguinte:
	\[ a, \lneg a \vDash b.\]
	
	\item Se $\Gamma$ é o conjunto vazio, \emph{qualquer valoração} satisfaz todas as fórmulas em $\Gamma$. Assim sendo, para qualquer proposição $\varphi$, temos $\emptyset \vDash \varphi$ se e só se $\vDash \varphi$.
	\end{enumerate}
	
	Verifica ambas estas afirmações, e certifica-te que fazem sentido na tua cabeça.
	\end{obs}
	
	Vamos fazer mais um exemplo. Vamos verificar que $a \lor b, \lneg a \lor c \vDash b \lor c$.
	
	Vamos fazer isto de duas formas: a forma chata, que é verificar todos os casos possíveis; ou a forma fácil, por um raciocínio que nos poupa muito trabalho.
	
	\textbf{Forma chata:} Vamos fazer a tabela em que organizamos todos os valores possíveis de verdade de $a$, $b$ e $c$, vemos em quais destas valorações se verificam $a \lor b$ e $\lneg a \lor c$, e verificamos que em todas essas temos $b \lor c$.
	
	\[
	\begin{array}{c|c|c|c|c|c}
	a&b&c&a\lor b&\lneg a \lor c& b \lor c\\
	\hline
	\V&\V&\V&\V&\V&\V\\
	\V&\V&\F&\V&\F&\\
	\V&\F&\V&\V&\V&\V\\
	\V&\F&\F&\V&\F&\\
	\F&\V&\V&\V&\V&\V\\
	\F&\V&\F&\V&\V&\V\\
	\F&\F&\V&\F&\V&\\
	\F&\F&\F&\F&\V&\\
	\end{array}
	\]
	Na última coluna, só preenchemos os casos em que $a\lor b$ e $\lneg a \lor c$ se verificam.
	
	Visto que em todas estas linhas verificam $b \lor c$, concluímos que $b \lor c$ é consequência semântica de $a\lor b$ e $\lneg a \lor c$.
	
	\medskip
	
	\textbf{Forma simples:} Agora, em vez de fazermos a tabela com oito linhas, vamos pensar um bocadinho. Vamos imaginar que temos uma valoração $\rho$ que satisfaz $a\lor b$ e $\lneg a \lor c$, e mostraremos que também satisfaz $b \lor c$.
	
	Bem, a nossa valoração $\rho$ tem que atribuir um valor de verdade a $a$. Ou seja, ou $a$ é $\V$, ou $a$ é $\F$.
	
	No caso em que $a$ é $\V$, como $\rho$ satisfaz $\lneg a \lor c$, temos necessariamente que ter que $c$ é $\V$ (verifica). Como consequência, $b \lor c$ é $(\text{algo}) \lor \V$, que é $\V$.
	
	No caso em que $a$ é $\F$, como $\rho$ satisfaz $a \lor b$, temos necessáriamente que $b$ é $\V$. Pelo mesmo raciocínio, $b \lor c$ é $\V \lor (\text{algo})$, que é $\V$.
	
	Como, independentemente do caso, chegamos a que $\rho \Vdash b \lor c$, temos que qualquer $\rho$ que satisfaça $a\lor b$ e $\lneg a \lor c$ também satisfaz $b \lor c$. Ou seja, esta última é consequência semântica das primeiras.
	
	\bigskip
		
	\textbf{Exercício 2:} Verifica quais das seguintes afirmações são verdadeiras.
	
	\begin{enumerate}
	\item $a, b \vDash (a \lor c) \land (b \lor c)$
	
	\item $a \lor c, a \lor \lneg c \vDash a$
	
	\item $a \lor b, \lneg( (a \land (b \lor \lneg c)) \lor \lneg d) \vDash (a \land (c \lor \lneg c)) \lor (b \land (d \lor \lneg d))$
	
	\item $a, b, c \vDash a \land b \land c \land d$
	
	\item $a \land \lneg b, c \lor b, \lneg(a \land c) \vDash a \land (b \lor \lneg c)$
	\end{enumerate}
	
	\textbf{Exercício 3:} Mostra que, para qualquer proposição $\varphi$, temos $\varphi \vDash \varphi$.
	
	\smallskip
	
	\textbf{Exercício 4:} Sejam $\Gamma$ e $\Gamma'$ dois conjuntos de proposições tal que $\Gamma'$ contém todas as proposições que estão em $\Gamma$, e talvez algumas mais. Seja, também, $\varphi$ uma proposição. Será que, sabendo $\Gamma \vDash \varphi$, podes concluir $\Gamma' \vDash \varphi$? E ao contrário?

	\smallskip
	
	
\end{document}