\documentclass{report}

\usepackage{amsmath}
\usepackage{amssymb}
\usepackage{amsthm}
\usepackage{amsfonts}
\usepackage{comment}
\usepackage[utf8]{inputenc}
\usepackage[portuguese]{babel}
\usepackage{hyperref}
\usepackage{indentfirst}

%para desenhar árvores sintáticas
\usepackage{tikz}
\usepackage{tikz-qtree}
\tikzset{level distance=2em}

\usepackage{graphicx}


\title{Lógica}
\author{}
\date{}

\newtheorem{prop}{Proposição}	
\newtheorem*{prop*}{Proposição}

\theoremstyle{definition}
\newtheorem{definicao}{Definição}
\newtheorem*{definicao*}{Definição}

\theoremstyle{remark}
\newtheorem{obs}{Obs}

\addto\captionsportuguese{
	\renewcommand{\proofname}{Dem}
}

\renewcommand{\bf}[1]{\mathbf{#1}}

\newcommand{\F}{\mathrm{F}}

\newcommand{\lt}{\mathrm{T}}
\newcommand{\lf}{\mathrm{F}}

\DeclareMathOperator{\var}{Var}


\DeclareMathOperator{\pnot}{\texttt{not}}
\newcommand{\pand}{\mathbin{\texttt{and}}}
\newcommand{\por}{\mathbin{\texttt{or}}}

\begin{document}
	\maketitle
	\newpage
	
	\tableofcontents
	\newpage
	
	\chapter{Lógica Proposicional}
	
	A lógica proposicional é um sistema que nos permite expressar raciocínios sobre afirmações e relações entre elas. Mais concretamente, é um sistema no qual as variáveis representam afirmações, que tomam o valor `verdadeiro' e `falso', juntamente com um conjunto de operações lógicas com as quais o leitor já estará familiar, por exemplo o `ou' e o `não'.
	
	Não é preciso procurar muito para descobrir que este sistema tem interesse e aplicações práticas. Qualquer linguagem de programação em uso regular terá este sistema contido no seu funcionamento. Tome-se o exemplo da linguagem Python. Nesta linguagem, as variáveis podem ser do tipo \texttt{bool}. Uma variável deste tipo toma um de dois valores: \texttt{True} e \texttt{False}. Existem também os operadores \texttt{and}, \texttt{or} e \texttt{not}, que recebem valores booleanos e retornam valores booleanos. O uso de parênteses permite-nos agrupar expressões, de modo a formar expressões mais complexas. Por exemplo, \texttt{(a and b) or not (b or not c)} é uma expressão válida em Python (assumindo que as variáveis \texttt{a}, \texttt{b} e \texttt{c} estão definidas e são do tipo \texttt{bool}.)
	
	Formalizaremos e estudaremos este sistema, usando-o como `caixa de areia' para nos preparar para a lógica de primeira ordem, que apesar de semelhante, é significativamente mais complexa.
	
	\section{Noções básicas (Preliminar)}
	
	Esta subsecção tem o sufixo `Preliminar' porque não é final. Isto é, o propósito desta subsecção é apenas dar intuição para o significado das coisas, antes de avançar para as definições secas e rigorosas.
	
	\bigskip
	
	Começamos por introduzir a noção de fórmula. Para os nossos propósitos, uma fórmula é uma expressão composta por variáveis, operadores lógicos, e parênteses. Por exemplo, a expressão $(a \land b) \lor \neg (b \lor \neg c)$ é um exemplo de uma fórmula. Fórmulas são normalmente denotadas por letras gregas minúsculas, e.g. `A fórmula $\varphi$'.
	
	As variáveis (neste caso, $a$, $b$ e $c$) podem tomar os valores de verdade `verdadeiro' e `falso', que aqui denotamos por $\lt$ e $\lf$.
	
	Existem diversos operadores lógicos conhecidos, mas no que se segue usaremos os operadores existentes no Python: $\pnot$, $\pand$ e $\por$. O leitor já deverá estar familiar com o comportamento destes, mas para evitar qualquer confusão, apresentamos as respetivas tabelas de verdade.
	
	\[\label{tabela:operadores}
	\begin{array}{|c|c||c|c|c|}
	\hline
	a & b & \pnot a & a \pand b & a \por b\\
	\hline
	\lt & \lt & \lf & \lt & \lt\\
	\lf & \lt & \lt & \lf & \lt\\
	\lt & \lf &     & \lf & \lt\\
	\lf & \lf &     & \lf & \lf\\
	\hline
	\end{array}
	\]
	
	Dada uma fórmula $\varphi$, podemos tentar `interpretá-la'. Isto é, se atribuirmos valores de verdade às variáveis podemos `substituir esses valores na fórmula' e avaliá-la.
	
	A título de exemplo, consideremos a fórmula $\varphi$ descrita acima.
	\[\varphi : (a \land b) \lor \neg (b \lor \neg c).\]
	
	Suponha-se que establecemos $a = \lt$ e $b = c = \lf$. Então, substituindo na fórmula (e escrevendo os operadores \textit{a la} Python), ficamos com
	\begin{gather*}
	(\lt \pand \lf) \por \pnot (\lf \por \pnot \lf)\\
	\lf \por \pnot (\lf \por \lt)\\
	\lf \por \pnot \lt\\
	\lf \por \lf\\
	\lf.
	\end{gather*}
	
	Podemos tornar o processo mais rigoroso descrevendo-o da seguinte forma: definimos uma \emph{valoração} como sendo uma função $\rho : \{\text{Variáveis}\} \to \{\lt, \lf\}$. Afirmamos que qualquer valoração pode ser extendida de forma natural para uma função ${\overline\rho : \{\text{Fórmulas}\} \to \{\lt, \lf\}}$, da forma que descrevemos: substitui-se cada variável $x$ pelo valor de $\rho(x)$ e `faz-se as contas'.
	
	Existe uma classe de fórmulas que tem particular interesse, que são as chamadas tautologias.
	
	Uma fórmula $\varphi$ diz-se uma tautologia se para qualquer valoração $\rho$ temos $\rho(\varphi) = \lt$. Isto é, $\rho$ é sempre verdadeira.
	
	Parte do objetivo deste capítulo é caracterizar as tautologias, de modo a podermos indentificá-las em tempo finito. Claro que, no caso de lógica proposicional, isto já está feito: dada uma fórmula $\varphi$, para determinar se esta é uma tautologia, basta listar todas as valorações possíveis nas variáveis de $\varphi$ (há aqui um detalhe escondido, mas já voltamos a ele) e fazer as contas para todas elas. Se todas elas derem $\lt$, então $\varphi$ é uma tautologia. Caso contrário, não é!
	
	A razão pela qual procuramos outra forma de caracterizar estas fórmulas é porque eventualmente atacaremos o problema mais geral, onde as variáveis, em vez de serem só $\lt$ e $\lf$, podem ser elementos de um universo qualquer arbitrário: naturais, reais, grupos... Assim sendo, deixa de ser possível testar todos os casos possíveis. Como tal, passa a ser necessário um método `sintático', isto é, que manipule as afirmações sem as tentar interpretar. É daí que nasce a formalização de `demonstração', com fim a esclarecer a relação entre o que é verdade e o que é demonstrável.
	
	Voltando a lógica proposicional, consideremos o problema de `testar todos os casos possíveis' para valores das variáveis numa fórmula. Ao fazermos isto estamos a assumir implícitamente que existe um número finito de casos, mas repare-se que, da forma que o definímos, existe um número potencialmente infinito de valorações! De facto, existe um número infinito de variáveis (por exemplo, $x_1$, $x_2$, \dots) pelo que haverá também um número (não-contável!) infinito de valorações. Assim sendo, é impossível testá-las todas.
	
	Claro que, na prática, isto é uma parvoíce. Fixa uma fórmula $\varphi$, só existe um número finito de variáveis a considerar, visto que apenas um número finito de variáveis consta em $\varphi$, e o valor dado a variáveis que não estas é irrelevante. Estamos habituados a tomar estes princípios como garantidos, mas aproveitá-los-emos como veículo para introduzir a noção de indução na estrutura.
	
	A observação essencial é que podemos decompôr uma fórmula arbitrária em fórmulas mais pequenas. Por exemplo, a fórmula $\varphi : (a \land b) \lor \neg (b \lor \neg c)$ pode ser decomposta como $\varphi_1 \lor \varphi_2$, onde $\varphi_1$ e $\varphi_2$ são fórmulas mais pequenas do que a inicial. Isto permite-nos usar o princípio de indução (forte) no tamanho de uma fórmula, em que o passo de indução corresponde a separar uma fórmula como $\varphi_1 \lor \varphi_2$, $\varphi_1 \land \varphi_2$ ou $\neg \varphi_1$. Este processo para nas fórmulas que não podem ser simplificadas mais: chamamos a estas de fórmulas atómicas, e são aquelas compostas por apenas uma variável. Por exemplo, $x$, ou $a$.
	
	Exemplificaremos o princípio, começando por definir indutivamente a noção de `variáveis em fórmula', e demonstrando, com base nesta definição, que, se $\rho$ é uma valoração e $\varphi$ é uma fórmula, $\overline\rho(\varphi)$ depende apenas do valor de $\rho$ nas variáveis em $\varphi$.
	
	\begin{definicao*}
	Seja $\varphi$ uma fórmula. Definimos $\var \varphi$ indutivamente da seguinte forma:
	\begin{itemize}
	\item Se $\varphi$ é da forma `$x$', então $\var \varphi = \{x\}$.
	
	\item Se $\varphi$ é da forma $\neg \varphi_1$, então $\var \varphi = \var \varphi_1$.
	
	\item Se $\varphi$ é da forma $\varphi_1 \lor \varphi_2$ ou $\varphi_1 \land \varphi_2$, então $\var \varphi = \var \varphi_1 \cup \var \varphi_2$.
	\end{itemize}
	\end{definicao*}
	
	\begin{prop*}
	Seja $\varphi$ uma fórmula, $\rho$ uma valoração. Então, $\overline \rho(\varphi)$ depende apenas de $\rho|_{\var \varphi}$. Por outras palavras, se $\rho$ e $\rho'$ são duas valorações tal que para todo $x \in \var \varphi$ se tem $\rho(x) = \rho'(x)$, então $\overline \rho(\varphi) = \overline \rho'(\varphi)$.
	\end{prop*}
	
	\begin{proof}
	Como dito antes, usaremos esta demonstração para exemplificar o conceito de indução em estrutura.
	
	O caso base são as fórmulas atómicas, $\varphi : x$. Como $\overline\rho(\varphi) = \rho(x)$ neste caso, de facto $\overline\rho(\varphi)$ depende apenas de $\rho(x)$, isto é, o valor que $\rho$ toma nos elementos de $\var \varphi = \{x\}$. A demonstração do caso base está terminada.
	
	Fazemos agora o passo de indução. Seja $\varphi$ uma fórmula não-atómica, e suponha-se que o enunciado é verdadeiro para todas as fórmulas mais pequenas do que $\varphi$. Então, partimos do princípio que $\varphi$ é da forma $\neg \varphi_1$, $\varphi_1 \lor \varphi_2$ ou $\varphi_1 \land \varphi_2$. Em qualquer um destes casos, as fórmulas nas quais decompômos $\varphi$ são estritamente mais pequenas do que $\varphi$, pelo que podemos usar nelas a hipótese de indução.
	
	Usemos o caso $\varphi_1 \lor \varphi_2$ como exemplo, sabendo que os outros dois casos são idênticos.
	
	Por hipótese de indução, $\overline\rho(\varphi_1)$ depende apenas de $\rho$ aplicado a $\var \varphi_1$. Da mesma forma, $\overline\rho(\varphi_2)$ depende apenas de $\rho$ aplicado a $\var \varphi_2$. Logo, $\overline\rho(\varphi) = \overline\rho(\varphi_1) \por \overline\rho(\varphi_2)$ depende apenas de $\rho$ aplicado a $\var \varphi_1 \cup \var \varphi_2 = \var \varphi$, o que conclui a demonstração.
	\end{proof}
	
	Voltemos ao problema de identificar tautologias. Existe um conceito mais geral, que é o conceito de `consequência semântica'. A ideia é que às vezes é possível afirmar, \emph{sob certas condições}, que uma fórmula é sempre verdadeira.
	
	Por exemplo, considere-se $\varphi : (a \land c) \lor b \lor (a \land \neg c)$. Esta fórmula não é uma tautologia: por exemplo, a valoração que leva tudo em $\lf$ faz com que esta fórmula fique falsa. No entanto, se uma valoração $\rho$ satisfaz $\overline\rho(a \lor b) = \lt$, então é fácil verificar que de certeza que $\overline\rho(\varphi) = \lt$. Dito de outra forma: sabendo que $\rho$ dá o valor $\lt$ a $a \lor b$, concluímos que $\rho$ dá o valor $\lt$ a $\varphi$. Às vezes, por abuso de linguagem, omitimos as referências a valorações e dizemos apenas `se $a \lor b$ então $\varphi$'.
	
	Antes de generalizarmos este conceito, introduzimos alguma linguagem para nos ajudar a expressar.
	
	\begin{definicao*}
	Seja $\gamma$ uma fórmula, $\rho$ uma valoração. Dizemos que $\rho$ satisfaz $\gamma$, denotado $\rho \Vdash \gamma$, se $\overline\rho(\gamma) = \lt$.
	
	Se em vez de uma fórmula tivermos um conjunto de fórmulas $\Gamma$ (normalmente usamos letras gregas maiúsculas para conjuntos de fórmulas) dizemos que $\rho$ satisfaz $\Gamma$ se $\rho$ satisfizer todas as fórmulas de $\Gamma$. Isto é, simbolicamente,
	\[\rho \Vdash \Gamma \text{ se para todo $\gamma \in \Gamma$ temos } \rho \Vdash \gamma.\]
	\end{definicao*}
	
	\begin{definicao*}
	Seja $\varphi$ uma fórmula e $\Gamma$ um conjunto de fórmulas. Então, dizemos que \emph{$\varphi$ é consequência semântica de $\Gamma$}, representado simbolicamente como
	\[\Gamma \vDash \varphi\]
	se para todas as valorações $\rho$ tal que $\rho \Vdash \Gamma$ temos $\rho \Vdash \varphi$.
	
	Por outras palavras, `se todas as afirmações em $\Gamma$ são verdade, concluímos que $\varphi$ é verdade'.
	
	Normalmente, quando $\Gamma$ é um conjunto finito, abreviamos a notação, omitindo chavetas. Por exemplo, se $\Gamma = \{\gamma_1, \gamma_2, \gamma_3\}$, escreveríamos $\gamma_1, \gamma_2, \gamma_3 \vDash \varphi$.
	
	Isto sugere uma notação para tautologia. De facto, $\varphi$ é uma tautologia sse $\emptyset \vDash \varphi$ (verifique). Escrevendo $\emptyset = \{\}$ e usando a convenção de omitir chavetas, chegamos à notação para tautologias: $\vDash \varphi$.
	
	Outra convenção que seguiremos será abreviar uniões usando vírgulas. Por exemplo, escrever $\Gamma, \alpha, \beta \vDash \varphi$ em vez de $\Gamma \cup \{\alpha, \beta\} \vDash \varphi$.
	\end{definicao*}
	
	A noção de consequência semântica é útil porque, juntamente com uma regra básica que iremos discutir agora, nos permite deduzir verdades novas a partir de verdades conhecidas.
	
	Suponhamos que temos uma afirmação da forma $a \lor b$. Isto significa que pelo menos um dos termos é verdadeiro. Assim sendo, se nos for dito que uma destas duas afirmações é falsa, concluímos que a outra é necessariamente verdadeira, isto é, $a$ falso implica $b$ verdadeiro. Por outras palavras, $a \lor b, \neg a \vDash b$.
	
	Substituindo $a$ por $\neg a$ e partindo do princípio que $\neg \neg a$ é a mesma coisa que $a$, chegamos à conclusão
	\[\neg a \lor b, a \vDash b.\]
	
	Assim sendo, a afirmação $\neg a \lor b$ representa, de alguma forma, `$a$ implica $b$'. Como tal, definimos o símbolo `$a \Rightarrow b$' como sinónimo de $\neg a \lor b$, e concluímos a chamada regra de \textit{modus ponens}:
	\[a \Rightarrow b, a \vDash b.\]
	
	Esta regra será a fundação do nosso cálculo dedutivo. De facto, veremos que existe um conjunto razoavelmente pequeno de tautologias $T$ tal que qualquer outra tautologia pode ser obtida a partir de aplicação repetida de \textit{modus ponens} a tautologias em $T$. A implicação tem um papel tão central, de facto, que quando definirmos fórmulas rigorosamente usaremos como base os operadores `não' e `implica'.
	
	\smallskip
	
	Terminamos esta secção preliminar com uma introdução ao conceito de metateorema.
	
	A noção de implicação e consequência semântica estão intrinsicamente ligadas. De facto, mostraremos que, em certo sentido, `$a \Rightarrow b$ sse $a \vDash b$'. Isto diz-se um metateorema porque relaciona duas `camadas' de verdade: relaciona uma verdade `dentro do sistema' ($a \Rightarrow b$) com uma verdade `sobre o sistema' ($a \vDash b$).
	
	\begin{prop*}
	(Metateorema da dedução) Seja $\Gamma$ um conjunto de fórmulas, $a$ e $b$ proposições. Então,
	\[\Gamma, a \vDash b \text{ sse } \Gamma \vDash a \Rightarrow b.\]
	\end{prop*}
	
	\begin{proof}\label{dem:mtd}
	($\rightarrow$) Suponha-se que $\Gamma, a \vDash b$. Desejamos mostrar $\Gamma \vDash a \Rightarrow b$, isto é, $\Gamma \vDash \neg a \lor b$. Assim sendo, vamos supor que $\rho$ é uma valoração tal que $\rho \Vdash \Gamma$ e mostraremos que $\rho \Vdash \neg a \lor b$.
	
	Existem dois casos: ou $\rho(a) = \lt$ ou $\rho(a) = \lf$.
	
	Se $\rho(a) = \lf$, então $\overline\rho(\neg a \lor b) = (\pnot \lf) \por \rho(b) = \lt \por \rho(b) = \lt$.
	
	Se $\rho(a) = \lt$, então temos que $\rho \Vdash \Gamma \cup \{a\}$, donde, como $\Gamma, a \vDash b$ por hipótese, concluímos que $\rho(b) = \lt$. Concluímos então que $\overline\rho(\neg a \lor b) = (\pnot \lt) \por \lt = \lt$, como desejado.
	
	($\leftarrow$) Suponha-se que $\Gamma \vDash a \Rightarrow b$. Seja $\rho$ uma valoração tal que $\rho \Vdash \Gamma \cup \{a\}$. Então, em particular, $\rho \Vdash \Gamma$, donde $\rho \Vdash a \Rightarrow b$. Sabemos também que $\rho \Vdash a$, pelo que concluímos $\rho \Vdash \{a \Rightarrow b, a\}$. Pela regra de \textit{modus ponens}, como $a \Rightarrow b, a \vDash b$, concluímos que $\rho \Vdash b$.
	
	Como partimos do princípio que $\rho$ era uma valoração arbitrária tal que $\rho \Vdash \Gamma \cup \{a\}$ e concluímos $\rho \Vdash b$, temos que $\Gamma \cup \{a\} \vDash b$.
	\end{proof}
	
	Este metateorema marca a primeira ocasião em que a nossa linguagem diz algo sobre si mesma. Frases autorreferenciais terão um papel central quando falarmos dos teoremas de incompletude de Gödel, mas isso terá que esperar até ao terceiro capítulo.
	
	\section{Semântica}
	
	O que se segue assume parcialmente que o leitor já leu a secção anterior. Apesar de não haver precedência explícita ou implícita, algumas das ideias essenciais já foram explicadas, e como tal não serão detalhadas novamente. Por outras palavras, o leitor poderá ler esta secção sem ter lido a anterior, mas se der por si perdido nas definições sem entender o seu significado poderá querer voltar atrás e ler a secção introdutória.
	
	\medskip
	
	Começamos por definir o conceito de fórmula proposicional.
	
	Fixe-se, primeiro, um conjunto, que usaremos em tudo o que se segue, chamado o conjunto das variáveis. Isto é apenas um conjunto infinito contável\footnote{Algo estranhamente, a cardinalidade exata deste conjunto é relevante. Bastantes argumentos que faremos de futuro necessitam explicitamente da contabilidade de $X$!} de símbolos $X$. Normalmente usamos variáveis como $x$, $y$, $p$, $q$, e permitimos a modificação de símbolos como a adição de apóstrofos ou asteriscos. Os símbolos $c$, $c'$, $c^*$ são considerados distintos. Usaremos a convenção que variáveis serão representadas por letras romanas minúsculas.
	
	\begin{obs}
	Há a necessidade de distinguir uma variável de uma `meta-variável'. Isto é: se falamos na variável $x$, poderá ser ambíguo se nos referimos ao elemento $x \in X$ ou se a letra $x$ é uma incógnita que pode significar uma variável arbitrária.
	
	Para evitar esta ambíguidade, representamos meta-variáveis a negrito: $\bf{x}$. Ou seja: $x$ é o elemento de $X$, enquanto que $\bf x$ é uma incógnita que pode ser substituída por qualquer variável: $x$, $y$, $z$, \dots
	\end{obs}
	
	Há quem defina, agora, fórmulas como sequências de símbolos. Isto parece ser uma definição intuitiva, visto que é assim que representamos fórmulas: sequências de caracteres. No entanto, visto que no futuro teremos que escrever programas que lêm e interpretam estas fórmulas, é mais conveniente definirmos fórmulas pelas respetivas árvores semânticas.
	
	Para esclarecer o que se entende por isto, considere-se a expressão $a \lor b$. Isto consiste de um operador (o operador `ou') aplicado a duas variáveis. Podemos representar isto como uma árvore na seguinte forma:
	
	\begin{center}
	\Tree [.\texttt{or} $a$ $b$ ]
	\end{center}
	
	Podemos, no entanto, considerar expressões mais complexas. Por exemplo, considere-se a expressão $(a \land b) \lor \neg (b \lor \neg c)$. Interpretada como uma árvore, esta expressão fica
	
	\begin{center}
	\Tree [.\texttt{or} [.\texttt{and} $a$ $b$ ] [.\texttt{not} [.\texttt{or} $b$ [.\texttt{not} $c$ ] ] ] ]
	\end{center}
	
	Para os nossos propósitos, é mais fácil manipular àrvores do que sequências de caracteres. Assim sendo, é com base nesta perspetiva que definimos o conjunto das fórmulas proposicionais.
	
	\begin{definicao}
	Definimos o conjunto das fórmulas booleanas $\F_b$ (sobre o conjunto $X$, que deixamos implícito) indutivamente.
	
	Qualquer variável $\bf x$ é uma fórmula.
	
	Se $\alpha$ e $\beta$ são fórmulas, ambos os seguintes são fórmulas:
	
	\begin{center}
	\Tree [.\texttt{not} $\alpha$ ]
	\hspace{3em}
	\Tree [.\texttt{implies} $\alpha$ $\beta$ ]
	\end{center}
	
	Esta forma de representar fórmulas não é particularmente eficiente tipográficamente, pelo que, no que se segue, continuaremos a representá-las com a notação linear $\neg \alpha$ e $\alpha \Rightarrow \beta$.
	
	Ao darmos nomes a fórmulas, associaremos o nome ao conteúdo usando dois pontos. Por exemplo, para associar o nome $\varphi$ à fórmula $a \lor b$, escreveremos $\varphi : a \lor b$.
	\end{definicao}
	
	\begin{definicao}
	Definimos uma valoração $\rho$ como uma função $\rho : X \to \{\lt,\lf\}$.
	
	Dada uma valoração $\rho$, definimos uma função $\overline\rho : \F_b \to \{\lt, \lf\}$ indutivamente:
	
	\begin{itemize}
	\item Se $\varphi : \bf x$, definimos $\overline\rho(\varphi) = \rho(\bf x)$.
	
	\item Se $\varphi : \neg \alpha$, definimos $\overline\rho(\varphi) = \pnot \overline\rho(\alpha)$.\footnote{Se o leitor não perceber o uso dos operadores $\pnot$ e $\por$, ver página \pageref{tabela:operadores}.}
	
	\item Se $\varphi : \alpha \Rightarrow \beta$, definimos $\overline\rho(\varphi) = (\pnot \overline\rho(\alpha)) \por \overline\rho(\beta)$.
	\end{itemize}
	
	Dada uma valoração $\rho$, dizemos que $\rho \Vdash \varphi$ ($\rho$ satisfaz $\varphi$) se $\overline\rho(\varphi) = \lt$. Se em vez de uma fórmula $\varphi$ tivermos um conjunto de fórmulas $\Gamma$, dizemos que $\rho \Vdash \Gamma$ se $\rho \Vdash \gamma$ para todo $\gamma \in \Gamma$.
	\end{definicao}
	
	\begin{definicao}
	Seja $\varphi \in \F_b$, $\Gamma \subseteq \F_b$. Dizemos que $\Gamma \vDash \varphi$ se todas as valorações que satisfazem $\Gamma$ também satisfazem $\varphi$. Por outras palavras, se para todas as valorações $\rho$ tal que $\rho \Vdash \Gamma$ temos $\rho \Vdash \varphi$.
	\end{definicao}
	
	\begin{prop}
	(Metateorema da dedução)
	
	Sejam $\alpha, \beta \in \F_b$, $\Gamma \subseteq \F_b$. Então
	\[\Gamma, \alpha \vDash \beta \text{ sse } \Gamma \vDash \alpha \Rightarrow \beta.\]
	\end{prop}
	
	\begin{proof}
	A demonstração é semelhante à demonstração feita na página \pageref{dem:mtd}, com um pequeno detalhe. Essa demonstração foi feita para variáveis $a, b$ em vez de fórmulas $\alpha, \beta$. No entanto, substituindo $\rho(a)$ por $\overline\rho(\alpha)$ e $\rho(b)$ por $\overline\rho(\beta)$, a demonstração funciona sem modificações.
	\end{proof}
	
	\section{Sintática}
	
	Usar as regras de dedução da secção anterior para motivar a ideia de uma sequência de regras puramente sintáticas para discernir verdade de falsidade.
	
	Realçar a importância futura de regras sintáticas: em universos mais complicados, verificar todos os casos é impossível (visto que há frequentemente um número bastante infinito de casos). Como tal, é útil ter um método puramente sintático que nos permite descobrir verdades (e todas as verdades!).
	
	Concluír que, como objetivo, desejamos provar que algo é possível de provar sse é verdade.
	
	\subsection{Definições básicas}
	
	Expôr as definições de: demonstração, teorema, teoria. Introduzir a ideia de indução em fórmulas. Demonstrar a correção do cálculo.
	
	\subsection{Intuição}
	
	Secção opcional que discute os axiomas escolhidos e o porquê da sua escolha, elaborando como um matemático curioso poderia ter a eles chegado por si mesmo.
	
	\subsection{Metateoremas}
	
	Traduzir as regras de dedução feitas anteriormente para o contexto de sintaxe.
	
	\subsection{Completude}
	
	Demonstrar a completude do cálculo, dados os axiomas apropriados.
	
	\chapter{Lógica de primeira ordem}
	
	Introdução à lógica de primeira ordem, culminando no teorema da completude de Gödel.
	
	\chapter{Introdução à computação}
	
	Introdução à teoria das funções computáveis, ligando-a à lógica de primeira ordem, culminando nos teoremas de incompletude de Gödel.
	

\end{document}