\documentclass{report}

\usepackage{amsmath}
\usepackage{amsthm}
\usepackage{amsfonts}
\usepackage{comment}
\usepackage[utf8]{inputenc}
\usepackage[portuguese]{babel}
\usepackage{hyperref}
\usepackage{indentfirst}

%para desenhar árvores sintáticas
\usepackage{tikz}
\usepackage{tikz-qtree}
\tikzset{level distance=2em}

\usepackage{graphicx}


\title{Lógica}
\author{}
\date{}

\newtheorem{prop}{Proposição}

\theoremstyle{definition}
\newtheorem{definicao}{Definição}
\newtheorem*{definicao*}{Definição}

\theoremstyle{remark}
\newtheorem{obs}{Obs}

\addto\captionsportuguese{
	\renewcommand{\proofname}{Dem}
}

\renewcommand{\bf}[1]{\mathbf{#1}}

\newcommand{\F}{\mathrm{F}}

\begin{document}
	\maketitle
	\newpage
	
	\tableofcontents
	\newpage
	
	\chapter{Lógica Proposicional}
	
	\section{Semântica}
	
	A lógica proposicional é um sistema que nos permite expressar raciocínios sobre afirmações e relações entre elas.
	
	Mais concretamente, é um sistema no qual as variáveis representam afirmações, que tomam o valor `verdadeiro' e `falso', juntamente com um conjunto de operações lógicas com as quais o leitor já estará familiar, por exemplo o `ou' e o `não'.
	
	Não é preciso procurar muito para descobrir que este sistema tem interesse e aplicações práticas. Qualquer linguagem de programação em uso regular terá este sistema contido no seu funcionamento. Tome-se o exemplo da linguagem Python. Nesta linguagem, as variáveis podem ser do tipo \texttt{bool}. Uma variável deste tipo toma um de dois valores: \texttt{True} e \texttt{False}. Existem também os operadores \texttt{and}, \texttt{or} e \texttt{not}, que recebem valores booleanos e retornam valores booleanos. O uso de parênteses permite-nos agrupar expressões, de modo a formar expressões mais complexas. Por exemplo, \texttt{(a and b) or not (b or not c)} é uma expressão válida em Python (assumindo que as variáveis \texttt{a}, \texttt{b} e \texttt{c} estão definidas e são do tipo \texttt{bool}.)
	
	Formalizaremos e estudaremos este sistema, usando-o como `caixa de areia' para nos preparar para a lógica de primeira ordem, que apesar de semelhante, é significativamente mais complexa.
	
	\subsection{Fórmulas (Preliminar)}
	
	De modo a conseguirmos raciocinar sobre afirmações, precisamos de saber o que é uma fórmula. Intuitivamente, uma fórmula é uma sequência de variáveis, operadores e parênteses. Por exemplo, a expressão $(a \land b) \lor \neg (b \lor \neg c)$ é um exemplo de uma fórmula.
	
	Claro que esta `definição' deixa muito a desejar. Apresentamos agora uma definição mais rigorosa.
	
	Fixe-se, primeiro, um conjunto, que usaremos em tudo o que se segue, chamado o conjunto das variáveis. Isto é apenas um conjunto infinito contável\footnote{Algo estranhamente, a cardinalidade exata deste conjunto é relevante. Bastantes argumentos que faremos de futuro necessitam explicitamente da contabilidade de $X$!} de símbolos $X$. Normalmente usamos variáveis como $x$, $y$, $p$, $q$, e permitimos a modificação de símbolos como a adição de apóstrofos ou asteriscos. Os símbolos $c$, $c'$, $c^*$ são considerados distintos. Usaremos a convenção que variáveis serão representadas por letras romanas minúsculas.
	
	\begin{obs}
	Há a necessidade de distinguir uma variável de uma `meta-variável'. Isto é: se falamos na variável $x$, poderá ser ambíguo se nos referimos ao elemento $x \in X$ ou se a letra $x$ é uma incógnita que pode significar uma variável arbitrária.
	
	Para evitar esta ambíguidade, representamos meta-variáveis a negrito: $\bf{x}$. Ou seja: $x$ é o elemento de $X$, enquanto que $\bf x$ é uma incógnita que pode ser substituída por qualquer variável: $x$, $y$, $z$, \dots
	\end{obs}
	
	Há quem defina, agora, fórmulas como sequências de símbolos. Isto parece ser uma definição intuitiva, visto que é assim que representamos fórmulas: sequências de caracteres. No entanto, visto que no futuro teremos que escrever programas que lêm e interpretam estas fórmulas, é mais conveniente definirmos fórmulas pelas respetivas árvores semânticas.
	
	Para esclarecer o que se entende por isto, considere-se a expressão $a \lor b$. Isto consiste de um operador (o operador `ou') aplicado a duas variáveis. Podemos representar isto como uma árvore na seguinte forma:
	
	\begin{center}
	\Tree [.\texttt{or} $a$ $b$ ]
	\end{center}
	
	Podemos, no entanto, considerar expressões mais complexas. Por exemplo, considere-se a expressão $(a \land b) \lor \neg (b \lor \neg c)$. Interpretada como uma árvore, esta expressão fica
	
	\begin{center}
	\Tree [.\texttt{or} [.\texttt{and} $a$ $b$ ] [.\texttt{not} [.\texttt{or} $b$ [.\texttt{not} $c$ ] ] ] ]
	\end{center}
	
	Para os nossos propósitos, é mais fácil manipular àrvores do que sequências de caracteres.
	
	Estamos agora prontos para dar uma definição preliminar de fórmula.
	
	\begin{definicao*}
	\textbf{Nota: Esta não é a definição final e está aqui para o propósito de motivar o que se segue.} Um leitor que queira ver a definição final poderá saltar para a página \pageref{def:formulaproposicional}
	
	Definimos o conjunto das fórmulas booleanas $\F_b$ (sobre o conjunto $X$, que deixamos implícito) indutivamente.
	
	Qualquer variável $\bf x$ é uma fórmula.
	
	Se $\alpha$ e $\beta$ são fórmulas, todos os seguintes são fórmulas:
	
	\begin{center}
	\Tree [.\texttt{not} $\alpha$ ]
	\hspace{3em}
	\Tree [.\texttt{and} $\alpha$ $\beta$ ]
	\hspace{3em}
	\Tree [.\texttt{or} $\alpha$ $\beta$ ]
	\hspace{3em}
	\scalebox{2}{$\dots$}
	\end{center}
	\end{definicao*}
	
	O leitor poderia adicionar mais operadores se assim o desejasse. Pelo momento, trabalharemos apenas com estes três.
	
	Apesar de, neste texto, definirmos fórmulas como árvores sintáticas, este formato não é prático para efeitos tipográficos. Assim sendo, continuaremos a escrever fórmulas de forma linear, tendo sempre em mente que as nossas sequências de caracteres representam uma árvore sintática.
	
	\subsection{Interpretação}
	
	\pagebreak
	
	\begin{definicao}\label{def:formulaproposicional}
	inserir def a serio
	\end{definicao}
	
	\subsection{Regras de dedução}
	
	Expôr regras que nos permitem concluir que certas coisas são consequências de outras sem ter de verificar todos os casos possíveis.
	
	\section{Sintática}
	
	Usar as regras de dedução da secção anterior para motivar a ideia de uma sequência de regras puramente sintáticas para discernir verdade de falsidade.
	
	Realçar a importância futura de regras sintáticas: em universos mais complicados, verificar todos os casos é impossível (visto que há frequentemente um número bastante infinito de casos). Como tal, é útil ter um método puramente sintático que nos permite descobrir verdades (e todas as verdades!).
	
	Concluír que, como objetivo, desejamos provar que algo é possível de provar sse é verdade.
	
	\subsection{Definições básicas}
	
	Expôr as definições de: demonstração, teorema, teoria. Introduzir a ideia de indução em fórmulas. Demonstrar a correção do cálculo.
	
	\subsection{Intuição}
	
	Secção opcional que discute os axiomas escolhidos e o porquê da sua escolha, elaborando como um matemático curioso poderia ter a eles chegado por si mesmo.
	
	\subsection{Metateoremas}
	
	Traduzir as regras de dedução feitas anteriormente para o contexto de sintaxe.
	
	\subsection{Completude}
	
	Demonstrar a completude do cálculo, dados os axiomas apropriados.
	
	\chapter{Lógica de primeira ordem}
	
	Introdução à lógica de primeira ordem, culminando no teorema da completude de Gödel.
	
	\chapter{Introdução à computação}
	
	Introdução à teoria das funções computáveis, ligando-a à lógica de primeira ordem, culminando nos teoremas de incompletude de Gödel.
	

\end{document}