\documentclass{report}

\usepackage{amsmath}
\usepackage{amsthm}
\usepackage[utf8]{inputenc}
\usepackage[portuguese]{babel}
\usepackage{hyperref}
\usepackage{indentfirst}

\title{Lógica}
\author{}
\date{}

\newtheorem{prop}{Proposição}

\theoremstyle{definition}
\newtheorem{definicao}{Definição}

\theoremstyle{remark}
\newtheorem{obs}{Obs}

\addto\captionsportuguese{
	\renewcommand{\proofname}{Dem}
}

\begin{document}
	\maketitle
	\newpage
	
	\tableofcontents
	\newpage
	
	\chapter{Lógica Proposicional}
	
	\section{Semântica}
	
	A lógica proposicional é um sistema que nos permite expressar raciocínios sobre relações entre afirmações.
	
	Mais concretamente, é um sistema no qual as variáveis representam afirmações, que tomam o valor `verdadeiro' e `falso', juntamente com um conjunto de operações lógicas com as quais o leitor já estará familiar, por exemplo o `ou' e o `não'.
	
	Não é preciso procurar muito para descobrir que este sistema não tem apenas interesse teórico. Qualquer linguagem de programação em uso regular terá este sistema contido no seu funcionamento. Exemplificamos com a linguagem Python. Nesta linguagem, as variáveis podem ser do tipo \texttt{bool}. Uma variável deste tipo toma um de dois valores: \texttt{True} e \texttt{False}. Existem também os operadores \texttt{and}, \texttt{or} e \texttt{not}, que recebem valores booleanos e retornam valores booleanos. O uso de parênteses permite-nos agrupar expressões, de modo a formar expressões mais complexas. Por exemplo, \texttt{(a and b) or not (b or not c)} é uma expressão válida em Python (assumindo que as variáveis \texttt{a}, \texttt{b} e \texttt{c} estão definidas e são do tipo \texttt{bool}.)
	
	Formalizaremos e estudaremos este sistema, usando-o como `caixa de areia' para nos preparar para a lógica de primeira ordem, que apesar de semelhante, é significativamente mais complexa.
	
	\subsection{Definições básicas}
	
	De modo a conseguirmos raciocinar sobre afirmações, precisamos de saber o que é uma fórmula. Intuitivamente, uma fórmula é uma sequência de variáveis, operações e parênteses. Por exemplo, a expressão $(a \land b) \lor \neg (b \lor \neg c)$ é um exemplo de uma fórmula.
	
	Claro que esta `definição' deixa muito a desejar. Apresentamos agora uma definição mais rigorosa.
	
	Fixe-se, primeiro, um conjunto, que usaremos em tudo o que se segue, chamado o conjunto das variáveis. Isto é apenas um conjunto infinito contável\footnote{Algo estranhamente, a cardinalidade exata deste conjunto é relevante. Bastantes argumentos que faremos de futuro necessitam explicitamente da contabilidade de $X$!} de símbolos $X$. Normalmente usamos variáveis como $x$, $y$, $p$, $q$, e permitimos a modificação de símbolos como a adição de apóstrofos ou asteriscos. Os símbolos $c$, $c'$, $c^*$ são considerados distintos.
	
	
	
	\subsection{Regras de dedução}
	
	Expôr regras que nos permitem concluir que certas coisas são consequências de outras sem ter de verificar todos os casos possíveis.
	
	\section{Sintática}
	
	Usar as regras de dedução da secção anterior para motivar a ideia de uma sequência de regras puramente sintáticas para discernir verdade de falsidade.
	
	Realçar a importância futura de regras sintáticas: em universos mais complicados, verificar todos os casos é impossível (visto que há frequentemente um número bastante infinito de casos). Como tal, é útil ter um método puramente sintático que nos permite descobrir verdades (e todas as verdades!).
	
	Concluír que, como objetivo, desejamos provar que algo é possível de provar sse é verdade.
	
	\subsection{Definições básicas}
	
	Expôr as definições de: demonstração, teorema, teoria. Introduzir a ideia de indução em fórmulas. Demonstrar a correção do cálculo.
	
	\subsection{Intuição}
	
	Secção opcional que discute os axiomas escolhidos e o porquê da sua escolha, elaborando como um matemático curioso poderia ter a eles chegado por si mesmo.
	
	\subsection{Metateoremas}
	
	Traduzir as regras de dedução feitas anteriormente para o contexto de sintaxe.
	
	\subsection{Completude}
	
	Demonstrar a completude do cálculo, dados os axiomas apropriados.
	
	\chapter{Lógica de primeira ordem}
	
	Introdução à lógica de primeira ordem, culminando no teorema da completude de Gödel.
	
	\chapter{Introdução à computação}
	
	Introdução à teoria das funções computáveis, ligando-a à lógica de primeira ordem, culminando nos teoremas de incompletude de Gödel.
	

\end{document}