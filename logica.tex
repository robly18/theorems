\documentclass{report}

\usepackage{amsmath}
\usepackage{amsthm}
\usepackage[utf8]{inputenc}
\usepackage[portuguese]{babel}

\title{Lógica}
\author{}
\date{}

\newtheorem{prop}{Proposição}

\theoremstyle{definition}
\newtheorem{definicao}{Definição}

\addto\captionsportuguese{
	\renewcommand{\proofname}{Dem}
}

\begin{document}
	\maketitle
	\newpage
	
	\tableofcontents
	\newpage
	
	\chapter{Lógica Booleana}
	
	\section{Semântica}
	
	Definir, em termos intuitivos, a noção de lógica booleana, usando linguagens de programação como exemplos.
	
	\subsection{Definições básicas}
	
	Expôr as definições básicas: fórmula, valoração, satisfabilidade, tautologias e consequência semântica. Introduzir a noção de definições por abreviatura.
	
	\subsection{Regras de dedução}
	
	Expôr regras que nos permitem concluir que certas coisas são consequências de outras sem ter de verificar todos os casos possíveis.
	
	\section{Sintática}
	
	Usar as regras de dedução da secção anterior para motivar a ideia de uma sequência de regras puramente sintáticas para discernir verdade de falsidade.
	
	Realçar a importância futura de regras sintáticas: em universos mais complicados, verificar todos os casos é impossível (visto que há frequentemente um número bastante infinito de casos). Como tal, é útil ter um método puramente sintático que nos permite descobrir verdades (e todas as verdades!).
	
	Concluír que, como objetivo, desejamos provar que algo é possível de provar sse é verdade.
	
	\subsection{Definições básicas}
	
	Expôr as definições de: demonstração, teorema, teoria. Introduzir a ideia de indução em fórmulas. Demonstrar a correção do cálculo.
	
	\subsection{Intuição}
	
	Secção opcional que discute os axiomas escolhidos e o porquê da sua escolha, elaborando como um matemático curioso poderia ter a eles chegado por si mesmo.
	
	\subsection{Metateoremas}
	
	Traduzir as regras de dedução feitas anteriormente para o contexto de sintaxe.
	
	\subsection{Completude}
	
	Demonstrar a completude do cálculo, dados os axiomas apropriados.
	
	\chapter{Lógica de primeira ordem}
	
	Introdução à lógica de primeira ordem, culminando no teorema da completude de Gödel.
	
	\chapter{Introdução à computação}
	
	Introdução à teoria das funções computáveis, ligando-a à lógica de primeira ordem, culminando nos teoremas de incompletude de Gödel.
	

\end{document}