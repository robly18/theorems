\documentclass{article}

\usepackage{amsmath}
\usepackage{amssymb}
\usepackage{amsfonts}
\usepackage{mathtools}

\usepackage[thmmarks, amsmath]{ntheorem}

\usepackage{fullpage}

\usepackage{enumitem}


\usepackage[cal=euler]{mathalpha}
%\usepackage{ dsfont }

\setlist[enumerate,1]{label=\alph*)}

\title{Morley Rank as a Game}
\author{Duarte Maia}
%\date{}

\theorembodyfont{\upshape}
\theoremseparator{.}
\newtheorem{theorem}{Theorem}[subsection]
\newtheorem{prop}[theorem]{Proposition}
\renewtheorem*{prop*}{Proposition}
\newtheorem{lemma}[theorem]{Lemma}
\newtheorem{corollary}[theorem]{Corollary}
\newtheorem{remark}[theorem]{Remark}
\newtheorem{example}[theorem]{Example}

\theoremsymbol{\ensuremath{\square}}
\newtheorem{prelimdef}[theorem]{Preliminary Definition}
\newtheorem{definition}[theorem]{Definition}

\theoremstyle{nonumberplain}
\theoremsymbol{}
\newtheorem{convention}{Convention}

\theoremheaderfont{\itshape}
\theorembodyfont{\upshape}
\theoremseparator{:}
\theoremsymbol{\ensuremath{\blacksquare}}
\newtheorem{proof}{Proof}

\theoremsymbol{\ensuremath{\square}}
\newtheorem{sketch}{Proof Sketch}

\newcommand{\N}{\mathbb{N}}
\newcommand{\Z}{\mathbb{Z}}
\newcommand{\Q}{\mathbb{Q}}
\newcommand{\R}{\mathbb{R}}
\newcommand{\C}{\mathbb{C}}

\newcommand{\Lang}{\mathcal{L}}
\newcommand{\calN}{\mathcal{N}}
\newcommand{\Stone}{\mathrm{S}}

\DeclarePairedDelimiter{\braket}{\langle}{\rangle}
\DeclarePairedDelimiter{\abs}{\lvert}{\rvert}



\begin{document}
\maketitle

%\tableofcontents

\section{Introduction}

This document provides an alternate way to see and define the notion of Morley rank, as defined in his thesis \cite{morley}. It is written by, and for, someone to whom the relevant definition (Definition 2.2) comes across as random and mysterious. For the sake of brevity, this document does assume some prior knowledge, namely that the reader is familiar with the contents of chapter 1 of \cite{morley}. For the reader who may not be as familiar, or has perhaps forgotten some details, we present very quickly the main required bullet points:
\begin{itemize}
\item We are working with a complete theory $T$ over a countable language $\Lang$, which is assumed to be a relational language,
\item The symbol $\calN(T)$ denotes the collection of submodels of models of $T$,
\item Hypotheses are placed on $T$ such that $\calN(T)$ satisfies several `model-merging' properties. For most practical purposes, these properties may be summarized as follows: Given a collection $\{A_i\}_{i \in I}$ of elements of $\calN(T)$, assuming that these are pairwise compatible (i.e. the relations are well-defined on the intersections $A_i \cap A_j$), the union $\cup_{i \in I} A_i$, possibly with some identifications\footnote{To be more precise about these identifications, it is guaranteed that if $x$ and $y$ are in the same $A_i$ and are distinct there, they will not be identified in the union. The identification phenomenon happens e.g. if the theory says that for some $x \in A_i \cap A_j$ there exists a single $y$ satisfying some property, and both $A_i$ and $A_j$ contain a distinct value for $y$. In this instance, these two values would have to be identified in the union.}, is also in $\calN(T)$.
\end{itemize}

At the moment, I am not certain to what extent the following exposition uses these assumptions. Some proofs certainly use them in an essential way, but it is not yet clear to me up to what point the statements that follow break down when the assumptions are not present.

\section{The Main Definition and its Consequences}

\subsection{The Motivation}

Suppose that a third party has in their hands a theory $T$ (under the hypotheses of the introduction), a set $A \in \calN(T)$, and a type $p \in \Stone(A)$,\footnote{The Stone space $\Stone(A)$ is the set of types in one variable with parameters in $A$.} and this third party wishes to find out and measure how `slippery' this type is. As a way to measure this, a wager is arranged, between two players. One of which, which we shall imagine as being ourselves, will be dubbed `the Cat', and our goal is to `catch' the type. Our opponent, dubbed `the Mouse', will be controlling the type, and their goal is to run away.

To be more precise, let us elaborate on the rules of the game. At each step, we, the Cat, are allowed to replace $A$ by a bigger set $A' \supseteq A$ in $\calN(T)$. Once we do so, $p \in \Stone(A)$ ceases to be a type, because there are a lot of new sentences which $p$ has to make a decision on. It is the job of the Mouse to make these decisions, and hence decide on an extension $p'$ of $p$ which is a type in $A'$.

Our goal as the Cat is to make clever choices of $A'$, in order to `pin down' the type. The meaning of this expression is not very clear from the outset, and in fact we will see that its ambiguity is the source of a lot of richness in this definition. 

\subsection{An Alternate Phrasing}

\subsection{An Obvious Generalization}

\section{Equivalent Definitions}



\bibliographystyle{plain}
\bibliography{bibliography}

\end{document}