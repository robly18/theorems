\documentclass{article}

\usepackage{amsmath}
\usepackage{amssymb}
\usepackage{amsfonts}
\usepackage{mathtools}

\usepackage[thmmarks, amsmath]{ntheorem}

\usepackage{fullpage}

\usepackage{enumitem}

\setlist[enumerate,1]{label=\alph*)}

\title{A Proof of the Low Basis Theorem}
\author{Duarte Maia}
%\date{}

\theorembodyfont{\upshape}
\theoremseparator{.}
\newtheorem*{theorem}{Theorem}[subsection]
\newtheorem{prop}[theorem]{Proposition}
\renewtheorem*{prop*}{Proposition}
\newtheorem*{lemma}[theorem]{Lemma}
\newtheorem{corollary}[theorem]{Corollary}
\newtheorem{remark}[theorem]{Remark}
\newtheorem{example}[theorem]{Example}

\theoremsymbol{\ensuremath{\square}}
\newtheorem{prelimdef}[theorem]{Preliminary Definition}
\newtheorem{definition}[theorem]{Definition}

\theoremstyle{nonumberplain}
\theoremsymbol{}
\newtheorem{convention}{Convention}

\theoremheaderfont{\itshape}
\theorembodyfont{\upshape}
\theoremseparator{:}
\theoremsymbol{\ensuremath{\blacksquare}}
\newtheorem{proof}{Proof}

\theoremsymbol{\ensuremath{\square}}
\newtheorem{sketch}{Proof Sketch}

\newcommand{\N}{\mathbb{N}}
\newcommand{\Z}{\mathbb{Z}}
\newcommand{\Q}{\mathbb{Q}}
\newcommand{\R}{\mathbb{R}}
\newcommand{\C}{\mathbb{C}}

\newcommand{\CF}{\mathrm{CF}}
\newcommand{\halt}{\mathrm{Halt}}

\newcommand{\LA}{L_A}
\newcommand{\PA}{\mathrm{PA}}
\newcommand{\WPA}{\PA^-}


\DeclareMathOperator{\len}{\mathrm{len}}

\DeclarePairedDelimiter{\braket}{\langle}{\rangle}
\DeclarePairedDelimiter{\abs}{\lvert}{\rvert}



\begin{document}
\maketitle

\begin{lemma}
Let $T \subseteq 2^{<\omega}$ be a tree which is computable in the Halting problem, by an oracle program $P$ which is inversely monotone in the oracle, in the following sense: If $A \subseteq B$ are oracles, then $P^A \geq P^B$ pointwise.

Then, there is a computable tree $T'$ which contains the same paths. Moreover, $T'$ can be obtained effectively from $P$.
\end{lemma}

\begin{proof}
Let $P$ be a program which, using the Halting problem as an oracle, returns whether a finite string $\sigma$ is in $T$ or not. Then, we determine whether $\sigma$ is in $T'$ as follows. First, we pick an increasing approximation to the Halting problem $H$ which improves with depth, let's say $H_{\len \sigma}$. Then, we verify that, with this oracle, $P$ thinks that $\sigma$, and all its initial segments, are in $T$.

The resulting set $T'$ is computable. We verify that it is a tree. Indeed, suppose that
\begin{equation}
P^{H_{\len \sigma}}(\varepsilon) = P^{H_{\len \sigma}}(\sigma_0) = P^{H_{\len \sigma}}(\sigma_0 \sigma_1) = \dots = P^{H_{\len \sigma}}(\sigma) = 1.
\end{equation}

Then, by monotonicity of $P$ as a function of the oracle, all of the above equalities still hold when using any $n < \len \sigma$. Thus, in particular, any prefix of $\sigma$ is in $T'$.

Now, let us verify that $T$ and $T'$ have the same paths. One inclusion is due to the fact that, due to a reasoning similar to the one we just did, $T \subseteq T'$, and so any path in $T$ is also in $T'$. So, we focus on the opposite inclusion.

Let $f$ be a path in $T'$, and let $f_n$ be its $n$-th initial segment. We wish to verify, for arbitrary $n$, that $P^H(f_n) = 1$. Note that the computation of $P^H(f_n)$, whatever the output may be, uses the oracle only finitely many times. Thus, there is some approximation $H_N$ such that $P^{H_N}(f_n) = P^H(f_n)$. Now, since the entirety of $f$ is in $T'$, then in particular $f_{\max(N,n)}$ is in $T'$, and so by definition we have $P^{H_{\max(N,n)}}(f_n) = 1$. By monotonicity in the oracle, we conclude that
\begin{equation}
P^H(f_n) = P^{H_N}(f_n) \geq P^{H_{\max(N,n)}}(f_n) = 1.
\end{equation}
\end{proof}

\begin{theorem}
Let $T \subseteq 2^{<\omega}$ be an infinite computable binary tree. Then, there exists a path $A$ through $T$ of low degree.
\end{theorem}

\begin{proof}
We will construct $A$ at the same time as we conclude (or force) things about its jump. More precisely, we will iterate over all programs $P$ that make use of an oracle, and whenever possible we will ensure that $A$ is built such that $P$ does not halt when given $A$ as an oracle.

Given the computable tree $T$, and fixed an oracle program $P$, we construct the subtree $T_0$ given as follows. To decide whether a string $\sigma \in T$ is in $T_0$, we check whether $P$ halts (on the empty tape) when given $\sigma$ as an oracle, with the stipulation that if $P$ tries to access past the bounds of $\sigma$, it crashes and loops indefinitely. If $P^\sigma$ runs forever (or crashes), we declare that $\sigma$ is in $T_0$. It is trivial that $T_0$ is a tree. Moreover, $P^\sigma$ requires no oracle to execute, and so $T_0$ is computable in the Halting problem, and furthermore in a way that is inversely monotone in the oracle. Thus, there is a computable tree $T'_0 \supseteq T_0$ whose paths are the same as those of $T_0$, and moreover by intersecting with $T$ we may in fact assume that $T'_0 \subseteq T$. (Note that intersecting with $T$ does not remove any paths.)

Now that we have performed the construction in the previous paragraph, we construct $A$ itself, by constructing a decreasing sequence $T_n$ of infinite trees. The first tree is $T$ itself. Then, we iterate over all oracle programs $P$, and at each step construct the computable subtree (say $U_{n+1}$) of the current tree (say $T_n$) given by the above procedure. Now, there are two cases. Either $U_{n+1}$ is infinite, or $U_{n+1}$ is finite, and with an oracle for the Halting problem it is possible to tell effectively which of these two cases holds.

If the first case holds, and $U_{n+1}$ is infinite, then there is an oracle in $T_n$ which causes $P$ never to halt, and indeed any oracle in $U_{n+1}$ does this. So we set $T_{n+1} = U_{n+1}$. If the second case holds, then any oracle in $T_n$ \emph{will} cause $P$ to halt in finite time. Thus, we set $T_{n+1} = T_n$, confident that whatever future restrictions we place on $A$, $P$ will halt when given $A$ as an oracle.

We thus obtain a decreasing sequence of trees, all of which contain at least one path, and so we may find a path $f$ which is in all of them. [Note: This path is unique, because we can, by considering specific programs $P$, force the process to make a decision on whether any particular $n \in \N$ is in $A$.] The jump of $f$ is determined by the above process, which is effective in the Halting problem, and so $f$ has low degree.
\end{proof}

%\bibliographystyle{plain}
%\bibliography{bibliography}

\end{document}