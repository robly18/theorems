% !TEX spellcheck = Portugues
% !TEX root = main.tex

\chapter{Matemática Inversa: O Meu Teorema Implica o Teu Teorema}

\section*{Abstract}



\section*{Introdução}

O leitor certamente terá encontrado nas suas aventuras matemáticas afirmações da forma `o teorema $X$ implica o teorema $Y$'. O ponto de partida deste texto consiste na observação de que isto é uma trivialidade.

\begin{theorem}
Quaisquer dois teoremas são equivalentes.
\end{theorem}

\begin{proof}
Duas afirmações $X$ e $Y$ dizem-se equivalentes se são verdadeiras exatamente nas mesmas circunstâncias. Em particular, se $X$ e $Y$ são teoremas, são verdadeiros independentemente das circunstâncias, e portanto são equivalentes.
\end{proof}

Neste texto, apresentamos uma breve introdução à área da lógica chamada ``Matemática Inversa'', que fornece uma definição alternativa e não-trivial de implicações entre teoremas, e mostramos algumas consequências (algumas expectáveis, outras surpreendentes) desta definição.

Devido a limitações de espaço, este texto será de natureza puramente expositória. De modo a poder apresentar as ideias principais sem as ofuscar com detalhes desnecessários, será ocasionalmente necessário distorcer a verdade. No que se segue, usaremos o seguinte símbolo para indicar que um dado parágrafo contém falsidades. 

\dnote{Este texto está inteiramente correto.}

Ao leitor interessado, que queira aprender o conteúdo deste texto sem falsidades, indico como bibliografia principal o livro ``Subsystems of Second Order Arithmetic'' escrito por Stephen Simpson \cite{simpson}. As suas primeiras 60 páginas consistem numa introdução extensa, mas clara e sucinta relativamente ao conteúdo que expõe. Dito isto, podemos finalmente começar a responder à questão: O que significa um teorema implicar outro?

\section*{Lógica de Primeira Ordem}

De modo a dar uma definição precisa de implicação entre teoremas, vale a pena pensar um pouco mais nas noções de veracidade e demonstração. Um formalismo possivel para este conceitos consiste na chamada `Lógica de Primeira Ordem'. Os dois conceitos principais que nos são relevantes são o conceito de `afirmação (de primeira ordem)' e `demonstração'.

\begin{enumerate}
\item Uma afirmação de primeira ordem consiste numa expressão `bem-formada' composta a partir de variáveis (e.g. $x,y,z$), conectivos lógicos (e.g. $\neg, \land$), parênteses, quantificadores $\forall$ e $\exists$, o sinal de igual $=$, e alguns outros símbolos dependentes de contexto, e.g. $1$, $\times$, $\in$.

\item Dado um conjunto de afirmações $\Gamma$ e uma outra afirmação $\varphi$, uma `demonstração de $\varphi$ a partir de $\Gamma$' é uma sequência finita de afirmações, cujo último elemento é $\varphi$, onde cada afirmação é um elemento de $\Gamma$ (diz-se uma hipótese), um `axioma lógico' (consulte-se um livro sobre lógica para saber quais são estes), ou segue logicamente das afirmações anteriores (num sentido preciso que não será especificado).

Se houver uma demonstração de $\varphi$ a partir de $\Gamma$, escrevemos $\Gamma \vdash \varphi$, lido `$\Gamma$ demonstra $\varphi$'. Neste contexto, $\Gamma$ deverá ser interpretado como um conjunto de axiomas ou hipóteses, e $\varphi$ um teorema que foi demonstrado sob essas hipóteses. Apresentaremos exemplos em breve.
\end{enumerate}

Antes de apresentar exemplos, gostaria de fazer uma nota filosófica. Não pretendo afirmar que a Lógica de Primeira Ordem formaliza toda a matemática. Isso não significa que essa afirmação não pode ser feita, e há quem diga que toda a matemática está construída sobre lógica de primeira ordem, mas dado que esta é em si parte da matemática isto levanta diversas questões de circularidade que são difíceis de responder. Para os nossos propósitos devemos ver lógica de primeira ordem apenas como uma tentativa de modelar o pensamento matemático, não de o definir.

\subsection*{Teoria de Grupos}

A título de primeiro exemplo, apresentamos os axiomas de teoria de grupos. Neste contexto, as nossas afirmações poderão usar os símbolos $\cdot$, ${}^{-1}$, e $\mathrm{id}$, e os axiomas são
\begin{equation}
\mathrm{Grp} = \left\{
\begin{aligned}
&\forall_x \forall_y \forall_z (x \cdot y) \cdot z = x \cdot (y \cdot z),\\
&\forall_x (x \cdot x^{-1} = \mathrm{id} \land x^{-1} \cdot x = \mathrm{id}),\\
&\forall_x (x \cdot \mathrm{id} = x \land \mathrm{id} \cdot x = x)
\end{aligned}
\right\}.
\end{equation}

Um exemplo de um teorema em teoria de grupos é a afirmação: Se $xy = \mathrm{id}$, então $y = x^{-1}$. No nosso contexto, escrevemos
\begin{equation}
\mathrm{Grp} \vdash \forall_x \forall_y (x \cdot y = \mathrm{id} \Rightarrow y = x^{-1}).
\end{equation}

Fornecemos agora dois não-exemplos de teoremas. O primeiro é a afirmação: Para todo o $x$ e $y$, $xy = yx$. Isto não é um teorema pelo facto que existem grupos não-comutativos, e portanto esta afirmação não segue dos axiomas de grupo. Escrevemos
\begin{equation}
\mathrm{Grp} \nvdash \forall_x \forall_y (x\cdot y = y\cdot x).
\end{equation}

Outro não-exemplo de teorema é o chamado Teorema de Lagrange. Este diz o seguinte: se $H$ é um subgrupo de $G$, então a cardinalidade de $H$ divide a cardinalidade de $G$. Apesar de ser, no sentido matemático usual, um teorema da teoria de grupos, não é um teorema que siga dos axiomas de grupo dados acima, pelo simples facto que \emph{não é sequer possível expressar este teorema em lógica de primeira ordem} (neste contexto). Isto é por várias razões, mas a razão que nos é mais relevante, por revelar uma subtileza das definições subjacentes, é que \emph{lógica de primeira ordem não permite quantificar sobre conjuntos}.

Mais precisamente, as variáveis $x, y, z$ representam elementos de um universo subjacente, no nosso caso de um grupo $G$, e sempre que quantificamos e.g. $\forall_x$, está implícito que o $x$ está a variar sobre os elementos do grupo em jogo. Não há, então, forma de quantificar sobre subconjuntos do grupo, ou sobre outros grupos. Isto será relevante mais tarde.

\subsection*{Teoria de Conjuntos}

\section*{Aritmética de Segunda Ordem}

\section*{Codificações}

\section*{Exemplo: Teorema de Bolzano}

\section*{Exemplo: Teorema de Weierstrass}

\section*{O Lema Fraco de König}

\section*{Exemplo: Ideais Primos}

\section*{Conclusão e `State of the Art'}

\thechapterbibliography{}