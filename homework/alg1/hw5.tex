\documentclass{article}

\usepackage{amsmath}
\usepackage{amssymb}
\usepackage{amsfonts}
\usepackage{mathtools}

\usepackage[thmmarks, amsmath]{ntheorem}

\usepackage{graphicx}

\usepackage{diffcoeff}
\diffdef{}{op-symbol=\mathrm{d},op-order-sep=0mu}

\usepackage{cancel}
\usepackage{interval}

\usepackage{enumitem}

\setlist[enumerate,1]{label=(\roman*)}

\title{Algebra Homework 5}
\author{Duarte Maia}
%\date{}

\theorembodyfont{\upshape}
\theoremseparator{.}
\newtheorem{theorem}{Theorem}
\newtheorem{prop}{Prop}
\renewtheorem*{prop*}{Prop}
\newtheorem{lemma}{Lemma}

\newtheorem{ex}{Exercise}

\theoremstyle{nonumberplain}
\theoremheaderfont{\itshape}
\theorembodyfont{\upshape}
\theoremseparator{:}
\theoremsymbol{\ensuremath{\blacksquare}}
\newtheorem{proof}{Proof}
\newtheorem{sol}{Solution}

\newcommand{\R}{\mathbb{R}}
\newcommand{\C}{\mathbb{C}}
\newcommand{\Z}{\mathbb{Z}}
\newcommand{\Q}{\mathbb{Q}}

\newcommand{\kk}{\Bbbk}

\newcommand{\PP}{\mathbb{P}}
\newcommand{\FF}{\mathcal{F}}
\newcommand{\DD}{\mathcal{D}}

\newcommand{\I}{\mathrm{i}}
\newcommand{\e}{\mathrm{e}}
\newcommand{\id}{\mathrm{id}}

\newcommand{\conj}[1]{\overline{#1}}

\DeclareMathOperator{\inte}{int}
\DeclareMathOperator{\codim}{codim}
\newcommand{\grad}{\nabla}
\newcommand{\schur}{\mathbf{s}}
\newcommand{\reg}{\mathit{reg}}
\newcommand{\regl}{{\mathit{reg}_\ell}}
\newcommand{\cg}{\vee}

\DeclareMathOperator{\vol}{vol}
\DeclareMathOperator{\Av}{Av}
\DeclareMathOperator{\trace}{tr}
\DeclareMathOperator{\sign}{sign}


\DeclareMathOperator{\Aff}{Aff}
\newcommand{\GL}{\mathrm{GL}}
\newcommand{\SL}{\mathrm{SL}}
\newcommand{\Hp}{\mathrm{H}}

\newcommand{\HH}{\mathcal{H}}

\let\Im\relax
\DeclareMathOperator{\Im}{Im}
\let\Re\relax
\DeclareMathOperator{\Re}{Re}

\DeclarePairedDelimiter{\abs}{\lvert}{\rvert}
\DeclarePairedDelimiter{\norm}{\lvert}{\rvert}
\DeclarePairedDelimiter{\Norm}{\lVert}{\rVert}
\DeclarePairedDelimiter{\braket}{\langle}{\rangle}


\begin{document}
\maketitle

\begin{ex}
\leavevmode
\begin{enumerate}
\item Show that if $\lambda \in \Lambda$ the following statements are equivalent:
\begin{gather}
\nabla_\lambda \neq 0,\label{eq:c1}\\
\text{$\lambda$ is strictly decreasing},\label{eq:c2}\\
\exists_{\mu \in \Lambda} \lambda = \rho + \mu.\label{eq:c3}
\end{gather}
\item Prove that the polynomials $\nabla_\lambda$ \emph{with strictly decreasing $\lambda$} are orthogonal and have norm equal to $n!$, under the inner product on the polynomials which makes the monic monomials an orthonormal basis.
\end{enumerate}
\end{ex}

\begin{sol}
\leavevmode
\begin{enumerate}
\item It is obvious that \eqref{eq:c1} implies \eqref{eq:c2}, because if $\lambda$ were not strictly decreasing, then the matrix whose determinant $\nabla_\lambda$ is would have two equal columns, and therefore $\nabla_\lambda$ would be zero.

Now, we show that \eqref{eq:c2} implies \eqref{eq:c3}. Indeed, if $\lambda$ is strictly decreasing, then $\mu = \lambda - \rho$ is positive nonstrictly decreasing. This can be seen as follows: first, note that $\mu_n = \lambda_n - \rho_n = \lambda_n \geq 0$, and so it suffices to show that it is nonstrictly decreasing. To this effect, note that
\begin{equation}
\mu_{k+1} = \lambda_{k+1} + k < \lambda_k + k,
\end{equation}
and the last inequality is equivalent to saying (because everything here is integers)
\begin{equation}
\mu_{k+1} \leq \lambda_k + k - 1 = \mu_k.
\end{equation}

Thus, $\lambda = \mu + \rho$ with $\mu \in \Lambda$.

Now, note that \eqref{eq:c3} implies \eqref{eq:c2} is obvious (adding a strictly decreasing sequence to one which is nonstrictly decreasing yields a strictly decreasing sequence) so it suffices to show that \eqref{eq:c2} implies \eqref{eq:c1}. To do this, recall the formula
\begin{equation}\label{eq:nabla}
\nabla_\lambda = \sum_{s \in S_n} \sign(s) x^{s(\lambda)},
\end{equation}
and since $\lambda$ is strictly decreasing, all monomials $x^{s(\lambda)}$ are distinct, and a sum of more than zero distinct nonzero monomials can never be zero.

\item Once again we use expression \eqref{eq:nabla}. Expanding the expression $\braket{\nabla_\lambda, \nabla_\mu}$ in terms of \eqref{eq:nabla} we get
\begin{equation}
\braket{\nabla_\lambda, \nabla_\mu} = \sum_{s \in S_n} \sum_{t \in S_n} \sign(s) \sign(t) \braket{x^{s(\lambda)}, x^{t(\mu)}}.
\end{equation}

Now, the only nonzero terms in the sum are when $s(\lambda) = t(\mu)$. In other words, $\lambda = s^{-1} t(\mu)$. But both $\lambda$ and $\mu$ are strictly decreasing, and the only permutation of a strictly decreasing sequence which is strictly decreasing itself is the identity. Therefore, the only nonzero terms are precisely when $s = t$, so we get
\begin{equation}
\braket{\nabla_\lambda, \nabla_\mu} = \sum_{s \in S_n} \sign(s)^2 \braket{x^{\lambda}, x^{\mu}}.
\end{equation}

Now, note that the sign squared is always 1, and $\braket{x^\lambda, x^\mu}$ is nonzero exactly when $\lambda = \mu$. Thus, we conclude that $\braket{\nabla_\lambda, \nabla_\mu} = n!$ precisely when $\lambda = \mu$, and zero otherwise.
\end{enumerate}
\end{sol}

\begin{ex}
Prove that the Schur polynomials $\schur_\mu$ form a $\Z$-basis of the symmetric polynomials with integer coefficients.
\end{ex}

\begin{sol}
Recall that multiplication by $\nabla_\rho$ is an isomorphism between $P^{S_n}$ and $P^{\sign_n}$. Thus, it suffices to show that the polynomials $\schur_\mu \nabla_\rho = \nabla_{\mu + \rho}$ form a $\Z$-basis of the skew-symmetric polynomials.

Linear independence of these polynomials is guaranteed by the previous exercise, using the same proof that shows that an orthonormal set in an Euclidean vector space is linearly independent. (Given a null linear combination of the vectors, take the inner product with one of the vectors, obtain that the corresponding coefficient is null.)

To show that they are a generating set, let $p(x)$ be an arbitrary skew-symmetric polynomial. Given any monomial in $p(x)$, we may consider its orbit under the signed action $s \cdot q(x) = \sign(s) q(s(x))$. This orbit is composed entirely of monomials, and by the skew-symmetry of $p(x)$, all their coefficients in $p(x)$ are the same up to sign. Thus, we may write $p(x)$ as a linear combination of the $\nabla_{\lambda + \rho}$ as follows: look at all monomials in $p$ of the form $x^{\lambda + \rho}$, and let $c_\lambda$ be the corresponding (integer!) coefficient. Then, we obtain by the above argument
\begin{equation}
p(x) = \sum c_\lambda \nabla_{\lambda + \rho}(x).
\end{equation}

This proves that these polynomials are a $\Z$-basis of the skew-symmetric polynomials with integer coefficients, and hence that the Schur polynomials are a $\Z$-basis of the symmetric polynomials.
\end{sol}

\setcounter{ex}{3}

\begin{ex}
Find the character associated to $\reg$.
\end{ex}

\begin{sol}
Given the basis $\{\delta_g\}$ of $R$, we let $\{\omega^g\}$ be the corresponding dual basis. We find $\trace(\reg(g,h))$ explicitly, for $(g,h) \in G \times G$.

This trace is given by the expression
\begin{equation}
\trace(\reg(g,h)) = \sum_{z \in G} \omega^z(\reg(g,h)(\delta_z)) = \sum_{z \in G} \omega^z \delta_{g z h^{-1}}.
\end{equation}

Now, by definition of the dual basis, this is precisely the number of values of $z$ for which $z = g z h^{-1}$, or equivalently, $g = z h z^{-1}$.
\end{sol}

\begin{ex}
Show that $\braket{\trace_\reg, \trace_{\rho \boxtimes \sigma^\cg}}$ equals $\braket{\sigma, \rho}$ when $\rho$ and $\sigma$ are reps of the finite group $G$. (This leads directly to what we wish to prove by the orthogonality relations.)
\end{ex}

\begin{sol}
We just expand the definition of the bracket in the case of a finite group:
\begin{equation}
\braket{\trace_\reg, \trace_{\rho \boxtimes \sigma^\cg}} = \frac1{\abs{G}^2} \sum_{g, h \in G} \left( \sum_{\substack{z \in G \\ g = z h z^{-1}}} 1 \right) \conj{\trace(\rho(g) \otimes \sigma^{\cg}(h))}
\end{equation}

Now, we simplify $\trace(\rho(g) \otimes \sigma^\cg(h)) = \trace(\rho(g)) \trace(\sigma^\cg(h))$, and we use lemma 4.6.4 to get $\trace(\sigma^\cg(h)) = \conj{\trace(\sigma(h))}$. As such, we obtain
\begin{equation}
\braket{\trace_\reg, \trace_{\rho \boxtimes \sigma^\cg}} = \frac1{\abs{G}^2} \sum_{g, h \in G} \left( \sum_{\substack{z \in G \\ g = z h z^{-1}}} 1 \right) \conj{\trace(\rho(g))}\trace(\sigma(h))
\end{equation}

Now, let us play around with the sums. We can get rid of the parentheses and have the sum be over triples $g, h, z \in G$, satisfying the condition $g = z h z^{-1}$. Now, this is the same as indexing only over pairs $h, z \in G$ and replacing all instances of $g$ with $z h z^{-1}$. However, $g$ only appears inside the character of $\rho$, and the character is invariant under conjugation, so we get
\begin{equation}
\braket{\trace_\reg, \trace_{\rho \boxtimes \sigma^\cg}} = \frac1{\abs{G}^2} \sum_{h, z \in G} \conj{\trace(\rho(h))}\trace(\sigma(h)).
\end{equation}

Thus, since $z$ is not being referred to in the summand, we may take the sum as being only over $h$, as long as we multiply by $\abs{G}$ to make up for the fact that terms are being summed with redundance because of $z$. Therefore, we obtain
\begin{equation}
\braket{\trace_\reg, \trace_{\rho \boxtimes \sigma^\cg}} = \frac1{\abs{G}} \sum_{h \in G} \conj{\trace(\rho(h))}\trace(\sigma(h)) = \braket{\sigma, \rho}.
\end{equation}
\end{sol}

\begin{ex}
\leavevmode
\begin{enumerate}
\item Show that $\braket{\trace_\regl, \trace_\rho} = \dim(\rho)$.
\item Show that $\abs{G} = \sum_{\rho \in \textrm{Irr}(G)} (\dim \rho)^2$.
\end{enumerate}
\end{ex}

\begin{sol}
\leavevmode
\begin{enumerate}
\item We compute it directly.
\begin{equation}
\braket{\trace_\regl, \trace_\rho} = \frac1{\abs{G}} \sum_{g \in G} \trace(\regl(g)) \conj{\trace(\rho(g))}
\end{equation}

Now, the trace of $\regl(g)$ can be easily computed as the trace of $\reg(g,h)$ was, as counting the number of values of $z \in G$ such that $zg = z$. Clearly this is null if $g \neq e$, and $\abs{G}$ if $g = e$, so we get
\begin{equation}
\braket{\trace_\regl, \trace_\rho} = \conj{\trace(\rho(e))} = \trace(\id_V) = \dim(\rho).
\end{equation}

\item We compute $\braket{\trace_\regl, \trace_\regl}$ in two ways. One of them consists of applying the previous item directly to get $\dim \regl = \abs{G}$.

For the second way, write $\regl$ as a sum of irreducibles. By the previous item, and using the fact that $\trace_{\rho \oplus \sigma} = \trace_\rho + \trace_\sigma$, we get that $\braket{\trace_\regl, \trace_\regl} = \sum_{\rho} \dim \rho$, where the sum is taken over all irreducibles in the decomposition of $\regl$. By the formula for the multiplicity of an irrep in a representation, this is the same as
\begin{equation}
\braket{\trace_\regl, \trace_\regl} = \sum_{\rho \in \mathrm{Irr}(G)} \braket{\trace_\regl, \rho} \times \dim \rho = \sum_{\rho \in \mathrm{Irr}(G)} (\dim \rho)^2.
\end{equation}

This concludes the proof.
\end{enumerate}
\end{sol}

\end{document}