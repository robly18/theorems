\documentclass{article}

\usepackage{amsmath}
\usepackage{amssymb}
\usepackage{amsfonts}
\usepackage{mathtools}

\usepackage[thmmarks, amsmath]{ntheorem}

\usepackage{graphicx}

\usepackage{diffcoeff}
\diffdef{}{op-symbol=\mathrm{d},op-order-sep=0mu}

\usepackage{cancel}
\usepackage{interval}

\usepackage{enumitem}

\setlist[enumerate,1]{label=(\roman*)}

\title{Algebra Homework 3}
\author{Duarte Maia}
%\date{}

\theorembodyfont{\upshape}
\theoremseparator{.}
\newtheorem{theorem}{Theorem}
\newtheorem{prop}{Prop}
\renewtheorem*{prop*}{Prop}
\newtheorem{lemma}{Lemma}

\newtheorem{ex}{Exercise}

\theoremstyle{nonumberplain}
\theoremheaderfont{\itshape}
\theorembodyfont{\upshape}
\theoremseparator{:}
\theoremsymbol{\ensuremath{\blacksquare}}
\newtheorem{proof}{Proof}
\newtheorem{sol}{Solution}

\newcommand{\R}{\mathbb{R}}
\newcommand{\C}{\mathbb{C}}
\newcommand{\Z}{\mathbb{Z}}
\newcommand{\Q}{\mathbb{Q}}

\newcommand{\kk}{\Bbbk}

\newcommand{\PP}{\mathbb{P}}
\newcommand{\FF}{\mathcal{F}}

\newcommand{\I}{\mathrm{i}}
\newcommand{\e}{\mathrm{e}}

\newcommand{\conj}[1]{\overline{#1}}

\DeclareMathOperator{\inte}{int}
\DeclareMathOperator{\codim}{codim}
\newcommand{\grad}{\nabla}

\DeclareMathOperator{\vol}{vol}
\DeclareMathOperator{\Av}{Av}
\DeclareMathOperator{\trace}{tr}


\DeclareMathOperator{\Aff}{Aff}
\newcommand{\GL}{\mathrm{GL}}
\newcommand{\SL}{\mathrm{SL}}
\newcommand{\Hp}{\mathrm{H}}

\newcommand{\HH}{\mathcal{H}}

\let\Im\relax
\DeclareMathOperator{\Im}{Im}
\let\Re\relax
\DeclareMathOperator{\Re}{Re}

\DeclarePairedDelimiter{\abs}{\lvert}{\rvert}
\DeclarePairedDelimiter{\norm}{\lvert}{\rvert}
\DeclarePairedDelimiter{\Norm}{\lVert}{\rVert}
\DeclarePairedDelimiter{\braket}{\langle}{\rangle}


\begin{document}
\maketitle

\begin{ex}
Prove that $\bar M$ is finite dimensional as a $\kk$-vector space iff $M$ is finitely generated as an $A$-module.
\end{ex}

\begin{sol}
($\rightarrow$) Suppose that $\bar M$ is finite dimensional, and let $[m_1], \dots, [m_n]$ be a homogeneous basis for it. We claim that the collection $m_1, \dots m_n$ generates $M$ as an $A$-module.

To this effect, we show by induction that all elements of $M_k$ can be written as an $A$-linear combination of the $m_i$. To prove the base case, let $m \in M_0$. Since $[m] \in \bar M$, we may write it as a $\kk$-linear combination of the $[m_i]$, and hence there exist scalars $k_1, \dots, k_n$ such that
\begin{equation}
m - \sum k_i m_i \in A_{>0} M.
\end{equation}

Now, if we remove the elements from the left-hand side whose degree is not zero, or equivalently take the zeroth homogeneous component of the linear combination\footnote{Here we are using the fact that $A_{>0} M$ is a graded subalgebra of $M$.} we obtain
\begin{equation}
m - \sum_{m_i \in M_0} k_i m_i \in M_0 \cap A_{>0} M,
\end{equation}
and the only element of this intersection is $0$. Thus, we conclude that $m$ is a $\kk$-linear combination of the $m_i$, and hence in the $A$-module generated by them (because $\kk \subseteq A$).

We will now do the induction step. Suppose that we have shown that all $m \in M$ of degree $r$ or less may be written as an $A$-linear combination of the $m_i$. We will show that this also holds at degree $r+1$.

Given $m \in M$ homogeneous of degree $r+1$, begin by writing
\begin{equation}
[m] = \sum k_i [m_i] \in \bar M.
\end{equation}

Then, $\Delta = m - \sum k_i m_i \in A_{>0} M$. As in the previous case, we may take the homogeneous component of degree $r+1$, and thus
\begin{equation}
\Delta \in A_{>0} M \cap M_{r+1}.
\end{equation}

As such, we may write $\Delta$ as a sum of the form $\sum a_i \mu_i$, with each $a_i \in A_{>0}$ and each $\mu_i \in M$, with $\deg \mu_i = (r+1) - \deg a_i \leq r$. Now we apply the induction step and write each $\mu_i$ as an $A$-linear combination of the $m_i$, and the proof is complete, for we may write $m = \sum k_i m_i + \sum a_i \mu_i$.

($\leftarrow$) Let $m_1, \dots, m_n$ be a finite set which generates $M$ as an $A$-module. Then, we will show that $[m_1], \dots, [m_n]$ generates $\bar M$ as a vector space.

To this effect, let $[m] \in \bar M$. Then, by hypothesis, $m = \sum a_i m_i$ for some $a_i \in A$. Write each $a_i$ as $k_i + a'_i$, with $k_i \in \kk$ and $a'_i \in A_{>0}$. Then,
\begin{equation}
m = \sum k_i m_i + \sum a'_i m_i,
\end{equation}
hence it is easy to check that $[m] = \sum k_i [m_i]$. By arbitraryness of $m$, we conclude that the $[m_i]$ generate $\bar M$ as a $\kk$-vector space and the proof is complete.
\end{sol}

\begin{ex}
Prove that $\phi \colon B \otimes H \to A$ induced by multiplication is surjective.
\end{ex}

\begin{sol}
We will prove by strong induction on degree that all $a \in A$ are in the image of $\phi$.

Suppose that we have shown that all the image of $\phi$ covers all elements of $A$ of degree strictly less than some fixed $r$, and let $a \in A$ be homogeneous of degree $r$. Since $A = I \oplus H$, write $a = i + h$, with $p \in I$ and $h \in H$. We remark that both $I$ and $H$ are graded ($H$ by hypothesis, and $I$ can be checked directly), so $p$ and $h$ may be chosen homogeneous of degree $r$.

Note that $B$ is a subalgebra of $A$, hence must contain $1_A$, so we may write $h = \phi(1_A \otimes h)$. Thus, it suffices to show that $p$ is in the image of $\phi$.

Since $p \in I$, we may write
\begin{equation}
p = \sum \alpha_i b_i,
\end{equation}
with each $b_i \in B_{>0}$ and $\deg \alpha_i + \deg b_i = \deg p = r$. Now, since $\deg b_i > 0$ we conclude $\deg \alpha_i < r$, whence the strong induction hypothesis applies. As such, all $\alpha_i$ are in the image of $\phi$. Finally, we conclude that all $b_i \alpha_i$ are in the image of $\phi$ (if $\alpha_i = \phi(t)$ then $b_i \alpha_i = \phi(bt)$)\footnote{This requires some routine checking, which is easily performed by reducing to the case where $t$ is a generator $b \otimes h$ and unfurling the definition of $\phi$.} and hence $p$ is too, which concludes the proof.
\end{sol}

\begin{ex}
Show that $A_{>0}$ is generated as an ideal by homogeneous elements $a_1, \dots, a_k$ iff these elements generate $A$ as an algebra.
\end{ex}

\begin{sol}
($\rightarrow$) Induction. Suppose all things of degree less than $\deg a$ with $a \in A$ are in $\braket{a_1, \dots, a_k}_A$ (the subscript means generated as an algebra). Then, write $a = \sum x_i a_i$, with each $x_i \in A$, which is possible because of the hypothesis.\footnote{Actually, is only possible assuming that $\deg a > 0$. The case where $a \in A_0$ uses $A_0 = \kk$. Indeed, $a = k 1_A$ for some $k \in \kk$, and $1_A$ is in the algebra generated by the $a_i$ by definition of `algebra generated by'.} Then, each $x_i$ has degree less than the degree of $a$ by the same argument used in previous exercises, and thus we apply the induction hypothesis to it to show that each $x_i$ is in the subalgebra generated by the $a_i$. This shows that $a$ itself is too, which completes the proof.

($\leftarrow$) Let $a \in A_{>0}$, homogeneous without loss of generality. Since the $a_i$ generate $A$ as an algebra, we may write $a$ as
\begin{equation}
a = \sum_{\abs{\alpha} = \deg a} k_\alpha a^\alpha,
\end{equation}
where we are using multiindex notation from analysis: $\alpha$ ranges over $k$-uples of nonnegative integers, $\abs{\alpha}$ is their sum, $k_\alpha$ is an $\alpha$-indexed scalar, and $a^\alpha$ is shorthand for $a_1^{\alpha_1} \dots a_k^{\alpha_k}$.

Now, since $\deg a \geq 1$, we may from each $\alpha$ pick some positive entry $\alpha_i$ and thus every term of the form $k_\alpha a^\alpha$ is a multiple of $a_i$, and thus in the ideal $\braket{a_1, \dots, a_k}$. As a consequence, $a$ itself is in this idea, which completes the proof.
\end{sol}

\begin{ex}
Prove that $H_V = H'_V$.
\end{ex}

\begin{sol}
As the problem statement says, it is obvious that $H_V \subseteq H'_V$, so we focus on the other inclusion.

Let $f \in H'_V$. We wish to show, given $v \in V$, that $v(f) = 0$. To do so, we show that all coefficients of $v(f)$ are null. Well, if $\alpha$ is a multiindex, the coefficient of $x^\alpha$ in $v(f)$ is given by $\partial^\alpha v(f)(0)$. Well, $\partial^\alpha v \in I_V$ obviously, so $\partial^\alpha v(f)(0) = 0$ because $f \in H'_V$. This completes the proof.
\end{sol}

\begin{ex}
Construct a $G$-invariant inner product $\beta$ on $P_n$ which makes $P^i_n \perp P^j_n$ and $\HH(G)$ the orthogonal complement of $I = P (P^G)_{>0}$.
\end{ex}

\begin{sol}
We set
\begin{equation}
\beta(p,q) = \int_G 
\end{equation}
\end{sol}

\begin{ex}
Show that
\begin{equation}
\frac1{\vol(G)} \int_G \trace(\rho(g)) \dl2 g = \trace(\Av_\rho) = \dim(V^G).
\end{equation}
\end{ex}

\begin{sol}

\end{sol}

\end{document}