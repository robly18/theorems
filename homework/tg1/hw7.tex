\documentclass{article}

\usepackage{amsmath}
\usepackage{amssymb}
\usepackage{amsfonts}
\usepackage{mathtools}

\usepackage[thmmarks, amsmath]{ntheorem}

\usepackage{graphicx}
\usepackage{float}
\usepackage{tikz-cd}
\usepackage{adjustbox}

\usepackage{diffcoeff}
\diffdef{}{op-symbol=\mathrm{d},op-order-sep=0mu}

\usepackage{cancel}
\usepackage{interval}

\usepackage{array}

\usepackage{enumitem}

\setlist[enumerate,1]{label=(\alph*)}

\title{Algebraic Topology Homework 6}
\author{Duarte Maia}
%\date{}

\theoremstyle{plain}
\theorembodyfont{\upshape}
\theoremseparator{.}
\newtheorem{theorem}{Theorem}
\newtheorem{prop}{Prop}
\renewtheorem*{prop*}{Prop}
\newtheorem{lemma}{Lemma}
\newtheorem*{ex}{Exercise}

\theoremstyle{nonumberplain}
\theoremheaderfont{\itshape}
\theorembodyfont{\upshape}
\theoremseparator{:}
\theoremsymbol{\ensuremath{\blacksquare}}
\newtheorem{proof}{Proof}
\newtheorem{sol}{Solution}

\theoremsymbol{\text{\textit{(End proof of lemma)}}}
\newtheorem{lemmaproof}{Proof of lemma}

\newcommand{\R}{\mathbb{R}}
\newcommand{\C}{\mathbb{C}}
\newcommand{\Z}{\mathbb{Z}}
\newcommand{\Q}{\mathbb{Q}}

\newcommand{\RP}{\mathbb{RP}}

\newcommand{\kk}{\Bbbk}

\newcommand{\PP}{\mathbb{P}}
\newcommand{\FF}{\mathcal{F}}

\newcommand{\I}{\mathrm{i}}
\newcommand{\e}{\mathrm{e}}

\newcommand{\id}{\mathrm{id}}
\newcommand{\GL}{\mathrm{GL}}

\newcommand{\conj}[1]{\overline{#1}}
\newcommand{\close}[1]{\overline{#1}}

\DeclareMathOperator{\inte}{int}
\DeclareMathOperator{\codim}{codim}
\DeclareMathOperator{\trace}{tr}
\newcommand{\grad}{\nabla}


\DeclareMathOperator{\Ext}{Ext}
\DeclareMathOperator{\Hom}{Hom}

\DeclarePairedDelimiter{\abs}{\lvert}{\rvert}
\DeclarePairedDelimiter{\norm}{\lvert}{\rvert}
\DeclarePairedDelimiter{\Norm}{\lVert}{\rVert}
\DeclarePairedDelimiter{\braket}{\langle}{\rangle}


\begin{document}
\maketitle

\begin{ex}[3.2:4]
Show that every map $f \colon \C P^n \to \C P^n$ has a fixed point if $n$ is even, and show that this also holds if $n$ is odd unless $f^*(\alpha) = -\alpha$ for $\alpha$ a generator of $H^2(\C P^n, \Z)$.
\end{ex}

\begin{sol}
The first part of the proof follows whether $n$ is even or odd.

Let $f^*(\alpha) = d \alpha$ for $d \in \Z$. Moreover, recall that (this is in Hatcher) the cohomology ring of $\C P^n$ is given by $\Z[\alpha]/\alpha^{n+1}$ and that all homologies of $\C P^n$ are free (indeed, $\Z$ in even dimensions up to $2n$), hence $h \colon H^* \to (H_*)^*$ is an isomorphism.

Now, we know then that $f^*(\alpha^k) = (f^*(\alpha))^k = d^k \alpha^k$. Moreover, using the naturality of $h$, we get that the map $f_* \colon H_k \to H_k$ (with $k = 0, \dots, 2n$) is given by multiplication by $d^k$. As such, the trace of $f_* \colon H_k \to H_k$ is itself $d^k$, and so the Lefschetz constant of $f$ is given by
\begin{equation}
\tau(f) = 1 + d + \dots + d^n.
\end{equation}

Now, the Lefschetz fixed point theorem guarantees that $f$ will have a fixed point so long as $\tau(f)$ is nonzero. If $d = 1$ this is obvious, and likewise for all $d \neq {-1,1}$ (as $d$ will not divide $\tau(f)$ but $d \mid 0$). The only remaining case is that of $d = -1$, and this is where the evenness of $n$ matters, as if $n$ is even $\tau(f)$ will be a telescopic sum totaling $\tau(f) = 1$, and only if $n$ is odd may $\tau(f)$ be zero after all, and so may $f$ have a fixed point. And again, this only happens if $f^*(\alpha) = -\alpha$, as desired.
\end{sol}

\begin{ex}[3.2:14]
I'm not doing this one. Sorry.
\end{ex}

\begin{sol}
N/A
\end{sol}

\begin{ex}[3.2:15]
Show that $p(X \times Y) = p(X) p(Y)$. Compute the Poincaré series for [a bunch of spaces].
\end{ex}

\begin{sol}
There is a missing hypothesis in the theorem, namely that the cohomologies of $X$ and $Y$ are finite-dimensional in all degrees, otherwise their Poincaré series will not make sense.

That said, since we're over a field, the cohomologies are all free, so the Künneth formula applies, and hence
\begin{equation}
H^i(X \times Y) \cong \bigoplus_{a+b=i} H^a(X) \otimes H^b(Y),
\end{equation}
whose dimension is precisely $\sum_{a+b=i} (\dim H^a(X))(\dim H^b(Y))$, which is exactly the $i$-th coefficient of $p(X) p(Y)$.

\medskip

I'm not going to compute the Poincaré series for eight different spaces. I mean, for $S^n$ it is trivially $1+x^n$, and for the projective spaces it will be a nice geometric sum (in finite dimension) or geometric series (in infinite dimension). But I'm not doing the details.
\end{sol}

\begin{ex}[3.3:3]
Show that every covering space of an orientable manifold is itself orientable.
\end{ex}

\begin{sol}
We do so by inducing an orientation on $\tilde M$ using an orientation in $M$.

Let $\tilde x \in \tilde M$, $x = p(\tilde x) \in M$. The generator of $H_n(M \mid x)$ induces a canonical generator of $H_n(U \mid x)$ by excision, where $U$ is a small enough evenly covered neighborhood of $x$. Take the preimage of $U$, and let $\tilde U$ be a subspace of $p^{-1}(U)$ which contains $\tilde x$ and such that $p|_{\tilde U}$ is a homeomorphism onto $U$. Then, $p$ induces an isomorphism $H_n(U \mid x) \cong H_n(\tilde U \mid \tilde x)$. Again, by excision, the latter is canonically isomorphic to $H_n(\tilde M \mid \tilde x)$. Thus, we have a canonical isomorphism (induced by $p$) between $H_n(\tilde M \mid \tilde x)$ and $H_n(M \mid x)$, and so picking a generator of the latter induces a natural choice of generator on the former.

Naturality properties and the fact that $p$ is a covering space will show that these choices of local orientations on $\tilde M$ are coherent and so induce an orientation on $\tilde M$.
\end{sol}

\begin{ex}[3.3:3]
\end{ex}

\begin{sol}
\end{sol}

\begin{ex}[3.3:3]
\end{ex}

\begin{sol}
\end{sol}

\begin{ex}[3.3:3]
\end{ex}

\begin{sol}
\end{sol}

\begin{ex}[3.3:3]
\end{ex}

\begin{sol}
\end{sol}

\end{document}