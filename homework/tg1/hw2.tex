\documentclass{article}

\usepackage{amsmath}
\usepackage{amssymb}
\usepackage{amsfonts}
\usepackage{mathtools}

\usepackage[thmmarks, amsmath]{ntheorem}

\usepackage{graphicx}
\usepackage{float}

\usepackage{diffcoeff}
\diffdef{}{op-symbol=\mathrm{d},op-order-sep=0mu}

\usepackage{cancel}
\usepackage{interval}

\usepackage{enumitem}

\setlist[enumerate,1]{label=(\alph*)}

\title{Algebraic Topology Homework 2}
\author{Duarte Maia}
%\date{}

\theoremstyle{plain}
\theorembodyfont{\upshape}
\theoremseparator{.}
\newtheorem{theorem}{Theorem}
\newtheorem{prop}{Prop}
\renewtheorem*{prop*}{Prop}
\newtheorem{lemma}{Lemma}
\newtheorem*{ex}{Exercise}

\theoremstyle{nonumberplain}
\theoremheaderfont{\itshape}
\theorembodyfont{\upshape}
\theoremseparator{:}
\theoremsymbol{\ensuremath{\blacksquare}}
\newtheorem{proof}{Proof}
\newtheorem{sol}{Solution}

\theoremsymbol{\text{\textit{(End proof of lemma)}}}
\newtheorem{lemmaproof}{Proof of lemma}

\newcommand{\R}{\mathbb{R}}
\newcommand{\C}{\mathbb{C}}
\newcommand{\Z}{\mathbb{Z}}
\newcommand{\Q}{\mathbb{Q}}

\newcommand{\kk}{\Bbbk}

\newcommand{\PP}{\mathbb{P}}
\newcommand{\FF}{\mathcal{F}}

\newcommand{\I}{\mathrm{i}}
\newcommand{\e}{\mathrm{e}}

\newcommand{\id}{\mathrm{id}}

\newcommand{\conj}[1]{\overline{#1}}

\DeclareMathOperator{\inte}{int}
\DeclareMathOperator{\codim}{codim}
\newcommand{\grad}{\nabla}


\DeclareMathOperator{\spec}{spec}

\DeclarePairedDelimiter{\abs}{\lvert}{\rvert}
\DeclarePairedDelimiter{\norm}{\lvert}{\rvert}
\DeclarePairedDelimiter{\Norm}{\lVert}{\rVert}
\DeclarePairedDelimiter{\braket}{\langle}{\rangle}


\begin{document}
\maketitle

\begin{ex}[1.2:9]
Show that $M'_h$ does not retract onto $C$, and hence $M_g$ does not retract onto $C$. Show that $M_g$ does retract onto $C'$.
\end{ex}

\begin{sol}
We begin by computing the fundamental group of $M'_h$, using some point $p \in C$ as a basepoint.

To do so, we recall that $M_h$ may be represented as a quotient of a $4h$-gon, with its $4h$ faces identified in a certain way. I don't have to write the details out because they can be found in Hatcher, page 5 of chapter 0.

Now, $M'_h$ may be obtained by removing an open disk $M_h$. Pictorially, it is obvious that we can assume that the removed disk is small, and in this case it is clear that we may shuffle it around a bit (this can be made precise using a local topological chart). Thus, we may assume that the removed disk is found in the interior of the $4h$-gon. While we're at it, we may as well assume that it is the center of the polygon, as it is evident that we may move the puncture inside it without changing the figure up to homeomorphism. Thus, we have a picture like the following. (This picture is for $h = 3$, but it evidently generalizes.)
\begin{figure}[H]
\centering
\includegraphics{mph1}
\end{figure}

Now, it is evident that this figure deformation retracts to just its border, which is a wedge of $2h$ circles. Thus, we conclude that the fundamental group of $M'_h$ (with basepoint equal to the vertex, say) is the free group on $2h$ generators, with each edge in the $1$-skeleton of the CW representation of $M_h$ being a generator.

Now, let us look at the following element of $\pi(M'_h)$ (with basepoint equal to the vertex $p$). Go in a straight line from $p$ to the edge of $C$, then follow $C$ clockwise, say, for one full rotation. Then return to $p$ following the reverse of the path previously taken. Call this path $\gamma$.

Then, by homotoping $\gamma$ to the edge of the polygon by a straight-line homotopy, it is easy to see that $[\gamma] \in \pi(M'_h)$ is a product of the generators of $\pi(M'_h)$ in some order, which is not relevant, but crucially, \emph{each generator appears twice in this expression, once with exponent $1$ and once with exponent $-1$}.

Now, suppose that there were a retraction $r \colon M'_h$ to $C$. Then, let us look at $r_*(\gamma)$. If we look at it directly, it is a path that starts in $r(p)$, does some things, does a full rotation, then undoes the things it did at the start. This is a conjugation of a noncontractible path, and hence is noncontractible itself. On the other hand, if we decompose $\gamma$ as a product of generators of $\pi(M'_h, p)$, we see that $r_*(\gamma)$ should be written as a product of a bunch of terms, each of which appears twice, one of which is inverted. Now, since the fundamental product of $C$, a circle, is abelian, this means that $r_*(\gamma)$ must be zero, a contradiction! This proves that such an $r$ may not exist.

\smallskip

This proves that $M_g$ does not retract onto $C$ because if it did, we could restrict such a retraction to $M'_h$.

\medskip

To show that $M_g$ retracts to $C'$, we build such a retraction. To do so, embed $M_g$ into $\R^3$ as in the following figure and separate it into `inside parts', call these $A$, and `outside parts', call them $B$, colored respectively in blue and orange.
\begin{figure}[H]
\centering
\includegraphics{mph2}
\end{figure}

Note: the gray parts (which represent the border) consist precisely of the points of the surface that maximize and minimize the $z$ coordinate in this embedding.

With this decomposition, we build the retraction as follows. Given a point in the curve, if it is in $A$, there is exactly one set in $C' \cap A$ with the same $z$ coordinate. Define its image under $r$ to be this point. Likewise for $B$.

This map is continuous by the pasting lemma. Indeed, restricted to either $A$ or $B$, it coincides with horizontal projection on a half-circumference, which is continuous, and on the points of $A \cap B$ this definition agrees whether we see the points as points of $A$ or points of $B$. Thus, by the pasting lemma, the resulting map is a continuous map from $M_g$ to $C'$.

Of course, this map fixes all points of $C'$ effectively by definition.
\end{sol}

\begin{ex}[1.2:22]
\leavevmode
\begin{enumerate}
\item Show that $\pi(\R^3 \setminus K)$ has a presentation with one generator $x_i$ for each strip $R_i$ and one relation of the form $x_i x_j x_i^{-1} = x_k$ for each square $S_\ell$.
\item Show that the abelianization of the fundamental group of the complement of a knot is $\Z$.
\end{enumerate}
\end{ex}

\begin{sol}
\leavevmode
\begin{enumerate}
\item We apply the van Kampen theorem to the following decomposition of $X$. Let $A = T \cup R$, where $T$ is the tabletop and $R$ is the union of the (open) rectangles $R_i$, and let $B = T \cup S$, where $S$ consists of a small enough neighborhood of the union of the squares $S_i$; the meaning of `small enough' is made clear in the following figure\footnote{Actually, there is a thing that the figure does not show, which is that $S$ should also be bleeding a bit into the rectangle $R$ under $S$.}, which represents $A$, $B$, and $A \cap B$ in a neighborhood of a crossing.
\begin{figure}[H]
\centering
\includegraphics[width=\linewidth]{knot1}
\end{figure}

Pick a basepoint $p$ in $T$. Then, $A$ consists of taking $T$ (which is contractible) and adding a bunch of `thick loops', so it is homotopically equivalent to a wedge of circles, with one circle per arc $\alpha_i$. Thus, its fundamental group is free with one generator, say $x_i$, per arc $\alpha_i$, with this generator going from $p$ to the arc $R_i$ and following it. For the sake of concreteness, \emph{fix an orientation on $K$} (and on $\R^3$), and let $x_i$ follow this loop in the direction given by the right-hand rule, with the thumb following $K$ in the positive direction.

Now, let us look at the fundamental group of $B$. At each crossing, we are (up to homotopy type) appending a square which connects to $T$ via its four corners. By stretching and collapsing stuff, we see that appending a square in this manner is equivalent to wedging with three circles, and so for each crossing the fundamental group of $B$ has three free generators, represented in the following figure. (Not shown in the figure: the paths going from $p$ to the start and end of the pictured loops.)
\begin{figure}[H]
\centering
\includegraphics{knot2}
\end{figure}
Note that the orientation of $K$ is pictured in the figure. The convention being followed is that paths are always going around the loop following the right-hand rule.

Finally, we inspect the fundamental group of $A \cap B$. At each crossing we are appending four handles, so the fundamental group of the intersection is free with four generators for each crossing. They are represented in the following figure.
\begin{figure}[H]
\centering
\includegraphics{knot3}
\end{figure}

Finally, we are ready to present the fundamental group of the complement of $K$. To do so, we apply van Kampen's theorem. Note that all spaces involved ($A$, $B$, and $A \cap B$) are path connected because they consist of the tabletop $T \cong \R^2$ with some appendages attached.

The presentation given to us by van Kampen is as follows. We have one generator $x_i$ for each loop, and three generators $a_i$, $b_i$, and $c_i$ for each crossing. Moreover, we have four relations for each crossing, given by $d_i$, $e_i$, $f_i$, and $g_i$.

We will simplify this presentation by inspection. In the following, let $\ell$ be the name of a certain crossing, let $i$ be the name of the arc passing under the crossing, $j$ the arc `going into the crossing' and $k$ the arc `coming out of the crossing'.

The relation given by $d_\ell$ is that $x_i = a_\ell$. This shows that the generators $a_\ell$ are redundant.

The relation given by $f_\ell$ is that $c_\ell = x_k$. This shows that the generators $c_\ell$ are redundant.

The generator $g_\ell$ evidently corresponds to $x_i$ in $A$. It is a little harder to see what it corresponds to in $B$, but inspection will show that $g_\ell = c_\ell^{-1} b_\ell$. Thus, $b_\ell = c_\ell x_i = x_k x_i$. Hence, $b_\ell$ is also redundant.

We are very close to our goal! We have shown that our presentation needs only the generators $x_i$, and we have one relation per crossing, given by $e_\ell$. We now inspect it.

From the perspective of $A$, $e_\ell$ is evidently $x_j$. From the perspective of $B$, a bit of mental gymnastics (deform $b_\ell$ `upwards') will show that $e_\ell = a_\ell^{-1} b_\ell$, and thus we have the relation
\begin{equation}
x_j = x_i^{-1} x_k x_i.
\end{equation}

Solving for $x_k$ we obtain the desired relation. This computes the proof that the fundamental group of the complement of a knot has the presentation we sought.

\item Upon abelianizing, the relations $x_i x_j x_i^{-1} = x_k$ become $x_j = x_k$. Thus, we can show that all $x_i$ are the same as follows. Start at some arc $\alpha_i$, to which corresponds a generator $x_i$. Then, follow along the knot in some direction until the arc ends. This means that you've run into a crossing and that $\alpha_i$ is `jumping over' some arc to become, say, $\alpha_j$. Now, the one relation we have on this crossing tells us that $x_i = x_j$. Now continue to get that $x_j$ equals whatever other generator comes next, and so on. After you've performed a full revolution, you have a chain of equalities between all the generators (this uses the fact that the knot is connected).

At the end, we end up with a group with exactly one generator and no relations, i.e. $\Z$. 
\end{enumerate}
\end{sol}

\begin{ex}[1.3:1]
Let $p \colon \tilde X \to X$ be a covering space, $A \subseteq X$. Show that the restriction of $p$ to $\tilde A = p^{-1}(A)$ is a covering space.
\end{ex}

\begin{sol}
We show this directly. I mean, obviously $p$ is continuous by definition of subspace topology on $\tilde A$. So all that remains to show is that every point has a uniformly covered neighborhood.

Pick $a \in A$. Pick a uniformly covered neighborhood of $a$ in $X$, call it $U$, and set $V = U \cap A$. By definition, $V$ is open in $A$. We claim that $V$ is evenly covered in $A$.

Write $p^{-1}(U) = \coprod W_\alpha$. Then, $p^{-1}(V) = \coprod (W_\alpha \cap \tilde A)$; it is evident that this is a disjoint union.

Finally, since $p$ is a homomorphism restricted to $W_\alpha$, it restricts to a homomorphism restricted to $W_\alpha \cap \tilde A$. We prove this claim now:
\begin{itemize}
\item Injectivity is obvious.
\item Surjectivity: given $v \in V$, it is $p(w)$ for some $w \in W_\alpha$, and evidently $w \in \tilde A$ as well.
\item Continuity: Restricting continuous functions to subspaces preserves continuity.
\item Openness: Perform the above argument for $p^{-1}$.
\end{itemize}

This completes the proof that $p$ restricted to $\tilde A$ is a covering space of $A$.
\end{sol}

\begin{ex}[1.3:6]
Construct a two-sheeted covering space $Y \to \tilde X$ such that the composition $Y \to X$ is not a covering space.
\end{ex}

\begin{sol}

\end{sol}

\begin{ex}[1.3:10]
Find all connected $2$-sheeted and $3$-sheeted covering spaces of $S^1 \wedge S^1$, up to isomorphism without basepoint.
\end{ex}

\begin{sol}
We do this process geometrically, starting with the $2$-sheeted covers. We know that $S^1 \wedge S^1$ is a graph, and therefore so is any of its covering spaces, with vertices in the cover corresponding to vertices in the base space and likewise for edges. Moreover, two sheetedness means that the number of edges and vertices are doubled, so we have two vertices and four edges to `distribute between the vertices'.

For the sake of definiteness, let the two (oriented) edges of $S^1 \wedge S^1$ be labeled $a$ and $b$. Then, in the covering space, there are two (oriented) edges also labeled $a$ and $b$. Since the graph is connected, there must be an edge going from one of the vertices to the other. Without loss of generality, we suppose that it is an $a$ edge; at the end we may obtain the missing covering spaces by swapping $a$ and $b$. Now, both vertices look locally like the vertex of $S^1 \wedge S^1$, so they must both have one $a$ going in and one $a$ going out (and likewise for $b$). Hence, the one $a$ edge we have determines the other, so so far our covering space looks like this.
\begin{figure}[H]
\centering
\includegraphics{cov1}
\end{figure}

Now, all that remains is to complete the $b$ edges. There are evidently two, and only two, possible arrangements. Taking those two arrangements and taking the symmetric wrt $a$ and $b$, we obtain a total of three two-sheeted connected coverings of $S^1 \wedge S^1$, pictured below.
\begin{figure}[H]
\centering
\includegraphics{cov2}
\end{figure}

\medskip

Let us move on to three-sheeted coverings. Arguing as above, there must be three vertices and three copies of each edge. Let us call the vertices $1$, $2$, and $3$. By connectedness, there must be an edge (suppose $a$ as above) going from (without loss of generality) $1$ to $2$. On the other hand, there must be an outgoing edge $a$ from $2$, which cannot return to $2$ because it already has an incoming $a$. This edge must hence go to $1$ or $3$, so any covering looks partially like one of the following two.
\begin{figure}[H]
\centering
\includegraphics[width=\linewidth]{cov3}
\end{figure}

Of course, it is evident that the third $a$ edge is determined, so it suffices to inspect the $b$ edges now.

On the left case (case L), by connectedness, the $b$ edges of $3$ must connect to $1$ or $2$. Without loss of generality, by symmetry, suppose the outgoing edge is $b \colon 3 \to 1$, so the question is whether the other edge is $1 \to 3$ or $2 \to 3$. As it happens, this choice determines the covering, and the possible outcomes are displayed in the following figure.
\begin{figure}[H]
\centering
\includegraphics[width=\linewidth]{cov4}
\end{figure}

Now, let us inspect case $R$. In this case, the $a$ edges form a loop. (continue)
\end{sol}

\begin{ex}[1.3:13]

\end{ex}

\begin{sol}

\end{sol}

\begin{ex}[1.3:31]

\end{ex}

\begin{sol}

\end{sol}

\end{document}