\documentclass{article}

\usepackage{amsmath}
\usepackage{amssymb}
\usepackage{amsfonts,stmaryrd}
\usepackage{mathtools}

\usepackage[thmmarks, amsmath]{ntheorem}
\usepackage{fullpage}

\usepackage{graphicx}

\usepackage{tikz-cd}
%\usepackage[T1]{fontenc}
\usepackage[cal=zapfc]{mathalpha}
\usepackage{ dsfont }


\usepackage{diffcoeff}
\difdef{f}{}{
outer-Ldelim = \left. ,
outer-Rdelim = \right| ,
sub-nudge = 0 mu
}

\usepackage{cancel}
\usepackage{interval}

\usepackage{enumitem}

\setlist[enumerate,1]{label=(\alph*)}

\title{Analysis Homework 8}
\author{Duarte Maia}
%\date{}

\theorembodyfont{\upshape}
\theoremseparator{.}
\newtheorem{theorem}{Theorem}
\newtheorem{prop}{Prop}
\renewtheorem*{prop*}{Prop}
\newtheorem{lemma}{Lemma}

\newtheorem{ex}{Exercise}

\theoremstyle{nonumberplain}
\theoremheaderfont{\itshape}
\theorembodyfont{\upshape}
\theoremseparator{:}
\theoremsymbol{\ensuremath{\blacksquare}}
\newtheorem{proof}{Proof}
\newtheorem{sol}{Solution}
\theoremsymbol{\ensuremath{\text{\textit{(End proof of lemma)}}}}
\newtheorem{lemmaproof}{Proof of Lemma}

\newcommand{\R}{\mathbb{R}}
\newcommand{\C}{\mathbb{C}}
\newcommand{\Z}{\mathbb{Z}}
\newcommand{\N}{\mathbb{N}}
\newcommand{\Q}{\mathbb{Q}}
\newcommand{\K}{\mathbb{K}}


\newcommand{\ind}{\mathds{1}}
\newcommand{\kk}{\Bbbk}

\newcommand{\Gr}{\mathrm{Gr}}

\newcommand{\I}{\mathrm{i}}
\newcommand{\e}{\mathrm{e}}
\newcommand{\id}{\mathrm{id}}

\newcommand{\conj}[1]{\overline{#1}}
\newcommand{\closed}[1]{\overline{#1}}
\newcommand{\transp}{\top}

\newcommand{\grad}{\nabla}
\DeclareMathOperator{\Ix}{Ix}
\DeclareMathOperator{\coker}{coker}

\DeclareMathOperator{\sign}{sign}
\DeclareMathOperator{\image}{im}
\DeclareMathOperator{\ord}{ord}


\DeclareMathOperator{\diam}{diam}
\DeclareMathOperator{\dist}{d}


\newcommand{\HH}{\mathcal{H}}
\newcommand{\bbH}{\mathbb{H}}

\let\Im\relax
\DeclareMathOperator{\Im}{Im}
\let\Re\relax
\DeclareMathOperator{\Re}{Re}

\DeclarePairedDelimiter{\abs}{\lvert}{\rvert}
\DeclarePairedDelimiter{\norm}{\lvert}{\rvert}
\DeclarePairedDelimiter{\Norm}{\lVert}{\rVert}
\DeclarePairedDelimiter{\braket}{\langle}{\rangle}
\DeclarePairedDelimiter{\floor}{\lfloor}{\rfloor}
\DeclarePairedDelimiter{\bparens}{[}{]}

\newcommand{\EVg}{\mathbb{E}}
\newcommand{\EV}{\EVg\bparens}
\newcommand{\PPg}{\mathbb{P}}
\newcommand{\PP}{\PPg\bparens}

\newcommand{\Brwn}{\mathcal{B}}
\newcommand{\sa}[1]{\mathcal{#1}}


\begin{document}
\maketitle

\begin{ex}
Show that $M_t = u(\Brwn_t)$ is a martingale wrt $\sa F_t = \sigma(\Brwn|_{\interval0t})$.
\end{ex}

\begin{sol}
We need to verify three properties. The first is that $M_t$ is $\sa F_t$-measurable. This is trivial, as $M_t$ is a function of $\Brwn_t$, which is in turn a function of $\Brwn|_{\interval0t}$. The second is that $M_t$ is $L^1$, which is true because:
\begin{equation}
\EV{\abs{M_t}} = \EV{\abs{u(\Brwn_t)}} \leq C \EV{\exp(\abs{\Brwn_t})}.
\end{equation}

Now we simply observe that a normally-distributed variable (even in multiple dimensions) has $L^1$ exponent. This is known to be true for $d = 1$ because (if $X$ is normally distributed in $\R$)
\begin{equation}\label{eq:1}
\EV{\exp(\abs X)} \leq \EV{\exp(X)} + \EV{\exp(-X)} = \phi_X(-\I) + \phi_X(\I),
\end{equation}
and since the characteristic function of the normal is total, \eqref{eq:1} is finite. For multiple dimensions, a simple triangular inequality and independence argument reduces to the $d = 1$ case:
\begin{equation}
\EV{\exp(\abs X)} \leq \EV[\Big]{\exp({\textstyle\sum} \abs{X_i})} = \EV[\Big]{{\textstyle\prod} \exp(\abs{X_i})} = \prod \EV{\exp(\abs{X_i})} < \infty.
\end{equation}

Finally, we prove the condition on the expectation. To do this we use the following technical lemma:
\begin{lemma}
Let $X$ and $Y$ be independent r.v., taking values in $\mathcal{X}$ and $\mathcal{Y}$ respectively, and let $f \colon \mathcal{X} \times \mathcal{Y} \to \R$ be nice enough that all the quantities below exist. Then, we have
\begin{equation}\label{eq:2}
\EV{f(X,Y) \mid Y} = \EVg_X[f(X,Y)],
\end{equation}
where the right-hand side of \eqref{eq:2} means: Apply the (deterministic) function $y \mapsto \EV{f(X,y)}$ to the random variable $Y$, obtaining a $Y$-measurable r.v.
\end{lemma}

With this lemma in hand, we apply it to $Y = \Brwn|_{\interval0t}$, $X = \Brwn_s - \Brwn_t$, and $f(x,y) = u(y(t) + x)$. By independent increments, assuming $s > t$, we have that $X$ is independent from $Y$. Thus,
\begin{equation}
\EV{M_s \mid \sa F_t} = \EV{f(X,Y) \mid Y} = \EVg_X[f(X,Y)] = \EVg_X[u(\Brwn_t + X)].
\end{equation}

Now, we claim that the latter expression is precisely equal to $u(\Brwn_t) = M_t$. This is because the distribution of $X$ is known to be a normal, and in particular it is rotationally invariant. Thus, we may write an expression for $\EVg_X[u(\Brwn_t + X)]$ using an integral, which we then write in polar coordinates, and use the `average on a circle of a harmonic function equals the harmonic function at the center' property to just obtain $u(\Brwn_t)$. And thus, the exercise is done.
\end{sol}

\begin{ex}
Assume that $\phi_n \to \phi$ pointwise. Show that $u_n \to u$ pointwise.
\end{ex}

\begin{sol}
We use the explicit formula for the solution of the Dirichlet equation:
\begin{equation}
u_n(z) = \int_{\partial U} \phi_n(\Brwn_{\tau^z(\omega)}^z(\omega)) \dl \PPg(\omega) \xrightarrow{\text{dct}} \int_{\partial U} \phi(\Brwn_{\tau^z(\omega)}^z(\omega)) \dl \PPg(\omega) = u(z),
\end{equation}
where in the middle step we used the dominated convergence theorem, and the fact that $\phi_n$ are uniformly bounded by $1$, which is integrable.
\end{sol}

\begin{ex}
Show that $\abs{f_\varepsilon(z)} \to 1$ for every $z$ in the shaded area.
\end{ex}

\begin{sol}
First, we investigate the function $P(y)$ from the construction of the Riemann map. Pick an arbitrary $x_0$ in the boundary of the right-hand side of the rectangle to be taken to $1$; it does not matter which, because this changes the Riemann map by a rotation, so absolute values are preserved.

Then, for $y$ in the boundary of the right-hand side, $P(y)$ is pretty small (or close to $1$ if $y$ is to the left of $x_0$), because for a path starting at $0$ to hit the interval between $x_0$ and $y$ it needs to pass through the slit, which is unlikely. In other words,
\begin{equation}
\text{For $y$ in the RHS (to the right of $x_0$): } P(y) \leq \PP{\text{$\Brwn_t$ passes through the slit}} \xrightarrow{\varepsilon \to 0} 0.
\end{equation}

Thus, we have that, for $y$ in the boundary of the right-hand side, $\varphi(y) = \exp(2\I\pi P(y))$ converges uniformly to $1$.

Now, we solve the Dirichlet problem for the boundary condition $\varphi(y)$, as to construct $f_\varepsilon(z)$. To do so, one constructs brownian motion starting at $z$ and takes the expected value of $\varphi(\Brwn_\tau)$. But observe that the probability that some $z$ in the right-hand side leaves the right-hand side is small, so the expected value will be near the values of $\varphi(y)$ for $y$ in the border of the right-hand side. Moreover, as we already saw, for small $\varepsilon$ these are very close to $1$. Thus, we conclude: for small $\varepsilon$, $f_\varepsilon(z)$ will be close to $1$. In particular its absolute value converges to one, and we are done.
\end{sol}

\begin{ex}
Order the figures $U_1$ to $U_4$ by order of $G$.
\end{ex}

\begin{sol}
We use the following fact, from the explicit construction of the Riemann mapping. The image of some interval in the boundary is taken to an interval with length given by $2 \pi$ times the probability that a brownian motion starting at zero hits that interval. Thus, to order the figures by their value of $G$, it suffices to order them by the probability that a brownian motion starting at the origin hits the boundary at a point with positive real part.

The case for $U_1$ is obviously $1/2$, because the figure is left-right symmetric. In turn, $U_3$ is the least likely, because any path that hits the right-hand side first in $U_3$ also does so in $U_1$, but there are some paths (with positive probability, by the annulus thingy from classes) that first hit the indent on the left and only after hit the right half-circle. Thus, $G(U_3) < G(U_1)$.

Now, by a similar (exactly the same) argument, $G(U_4) < G(U_3)$. The only thing that remains to verify is that $G(U_1) < G(U_4)$, and we do so by checking actually that $G(U_4) > 1/2$. This is because, let $U_5$ be given by a copy of $U_4$, but the right spoke is shorter, to have its length matching the left spoke. The resulting figure is left-right symmetric, so $G(U_5) = 1/2$. But again, we will have $G(U_5) < G(U_4)$ because any path that hits the right-hand side in $U_5$ also hits the right-hand side in $U_4$, but there are some paths that hit the extended spoke in $U_4$ before hitting the left-hand side in $U_5$, so indeed $G(U_4) > G(U_5) = 1/2$. In conclusion,
\begin{equation}
G(U_3) < G(U_1) = \frac12 < G(U_4) < G(U_2).
\end{equation}
\end{sol}

\end{document}