\documentclass{article}

\usepackage{amsmath}
\usepackage{amssymb}
\usepackage{amsfonts,stmaryrd}
\usepackage{mathtools}

\usepackage[thmmarks, amsmath]{ntheorem}
\usepackage{fullpage}

\usepackage{graphicx}

\usepackage{tikz-cd}
%\usepackage[T1]{fontenc}
\usepackage[cal=zapfc]{mathalpha}
\usepackage{ dsfont }


\usepackage{diffcoeff}
\difdef{f}{}{
outer-Ldelim = \left. ,
outer-Rdelim = \right| ,
sub-nudge = 0 mu
}

\usepackage{cancel}
\usepackage{interval}

\usepackage{enumitem}

\setlist[enumerate,1]{label=(\alph*)}

\title{Analysis Homework 7}
\author{Duarte Maia}
%\date{}

\theorembodyfont{\upshape}
\theoremseparator{.}
\newtheorem{theorem}{Theorem}
\newtheorem{prop}{Prop}
\renewtheorem*{prop*}{Prop}
\newtheorem{lemma}{Lemma}

\newtheorem{ex}{Exercise}

\theoremstyle{nonumberplain}
\theoremheaderfont{\itshape}
\theorembodyfont{\upshape}
\theoremseparator{:}
\theoremsymbol{\ensuremath{\blacksquare}}
\newtheorem{proof}{Proof}
\newtheorem{sol}{Solution}
\theoremsymbol{\ensuremath{\text{\textit{(End proof of lemma)}}}}
\newtheorem{lemmaproof}{Proof of Lemma}

\newcommand{\R}{\mathbb{R}}
\newcommand{\C}{\mathbb{C}}
\newcommand{\Z}{\mathbb{Z}}
\newcommand{\N}{\mathbb{N}}
\newcommand{\Q}{\mathbb{Q}}
\newcommand{\K}{\mathbb{K}}


\newcommand{\ind}{\mathds{1}}
\newcommand{\kk}{\Bbbk}

\newcommand{\Gr}{\mathrm{Gr}}

\newcommand{\I}{\mathrm{i}}
\newcommand{\e}{\mathrm{e}}
\newcommand{\id}{\mathrm{id}}

\newcommand{\conj}[1]{\overline{#1}}
\newcommand{\closed}[1]{\overline{#1}}
\newcommand{\transp}{\top}

\newcommand{\grad}{\nabla}
\DeclareMathOperator{\Ix}{Ix}
\DeclareMathOperator{\coker}{coker}

\DeclareMathOperator{\sign}{sign}
\DeclareMathOperator{\image}{im}
\DeclareMathOperator{\ord}{ord}


\DeclareMathOperator{\diam}{diam}
\DeclareMathOperator{\dist}{d}


\newcommand{\HH}{\mathcal{H}}
\newcommand{\bbH}{\mathbb{H}}

\let\Im\relax
\DeclareMathOperator{\Im}{Im}
\let\Re\relax
\DeclareMathOperator{\Re}{Re}

\DeclarePairedDelimiter{\abs}{\lvert}{\rvert}
\DeclarePairedDelimiter{\norm}{\lvert}{\rvert}
\DeclarePairedDelimiter{\Norm}{\lVert}{\rVert}
\DeclarePairedDelimiter{\braket}{\langle}{\rangle}
\DeclarePairedDelimiter{\floor}{\lfloor}{\rfloor}

\newcommand{\EV}{\mathbb{E}}
\newcommand{\PP}{\mathbb{P}}

\newcommand{\Brwn}{\mathcal{B}}
\newcommand{\sa}[1]{\mathcal{#1}}


\begin{document}
\maketitle

\begin{ex}
Show that almost surely $\Brwn$ is not $\alpha$-Hölder continuous for any $\alpha > 1/2$. Show also that for each $t > 0$, almost surely $\Brwn$ is not differentiable at $t$.
\end{ex}

\begin{sol}
For the first part, we use Brownian motion to justify the following lemma. In the following, we use the notation: `$f(t)$ is $C$-$\alpha$-HC on $I$' to mean: for all $s_1,s_2 \in I$, $\abs{f(s_1)-f(s_2)} \leq C \abs{s_1-s_2}^\alpha$.
\begin{lemma}
If $\Brwn$ is a standard Brownian motion and $t > 0$ is an arbitrary scaling factor,
\begin{equation}\label{eq:1}
\PP(\text{$\Brwn$ is $C$-$\alpha$-HC in $\interval01$}) = \PP(\text{$\Brwn$ is $(t^{\frac12 - \alpha} C)$-$\alpha$-HC in $\interval0t$}).
\end{equation}
\end{lemma}
\begin{lemmaproof}
Use the fact that $s \mapsto \Brwn_s$ has the same distribution as $s \mapsto t^{-1/2} \Brwn_{ts}$, and expand the definition of $C$-$\alpha$-Hölder continuity.
\end{lemmaproof}

This is one of the two main ingredients of our proof. Now, we apply it to $t = 2$, and introduce the second ingredient: If $\Brwn$ is $C'$-$\alpha$-HC on $\interval02$, then it is so $C'$-$\alpha$-HC on $\interval01$ and on $\interval12$. Morevoer, by independent increments, these two events are independent and have the same probability, hence
\begin{equation}\label{eq:2}
\PP(\text{$\Brwn$ is $C'$-$\alpha$-HC in $\interval02$}) \leq \PP(\text{$\Brwn$ is $C'$-$\alpha$-HC in $\interval01$})^2.
\end{equation}

Combining \eqref{eq:1} and \eqref{eq:2}, we get
\begin{equation}
\PP(\text{$\Brwn$ is $C$-$\alpha$-HC in $\interval01$}) \leq \PP(\text{$\Brwn$ is $(2^{\frac12 - \alpha} C)$-$\alpha$-HC in $\interval01$})^2.
\end{equation}

Finally, note that $2^{\frac12 - \alpha} C \geq C$, and decreasing the constant makes the event more likely, hence
\begin{equation}
\PP(\text{$\Brwn$ is $C$-$\alpha$-HC in $\interval01$}) \leq \PP(\text{$\Brwn$ is $C$-$\alpha$-HC in $\interval01$})^2.
\end{equation}

Thus, if we let $p$ be the probability that $\Brwn$ is $C$-$\alpha$-HC in $\interval01$, we have $p \leq p^2$, whence $p$ is either zero or one. But this probability cannot be one, because the event that $\Brwn$ is not $C$-$\alpha$-HC has nonzero probability; Indeed, $\Brwn_1 - \Brwn_0$ has a (nontrivial) normal distribution, and hence has a nonzero probability to exceed $C$ in absolute value.

Anyhow, so for each $C > 0$ we have $\PP(\text{$\Brwn$ is $C$-$\alpha$-HC in $\interval01$}) = 0$, and taking the union as $C = 1, 2, \dots$ and using $\sigma$-additivity,
\begin{equation}
\PP(\exists_C \text{$\Brwn$ is $C$-$\alpha$-HC in $\interval01$}) = 0,
\end{equation}
or in other words the probability that $\Brwn$ is $\alpha$-Hölder continuous is zero.

\medskip

Now let us do the second part. By independent increments, it suffices to verify that almost surely $\Brwn$ is not differentiable at $t = 0$. Now, if a given function $f$ with $f(0) = 0$ is differentiable at zero, it must satisfy a bound of the form $\abs{f(t)} \leq C \abs t$, for $t$ small enough, say less than $\varepsilon$. (Proof: set $C = \abs{f'(0)} + 1$ and apply elementary calculus.)

In conclusion, we have the bound
\begin{equation}
\PP(\text{$\Brwn$ diff. at $0$}) \leq \PP(\exists_{n \in \N} \forall_{t \in \interval0{\frac1n}} \abs{\Brwn_t} \leq n  t).
\end{equation}

I don't know how to prove that the right-hand is zero. A plausibility argument is that $\abs{\Brwn_t}$ looks `roughly like $\sqrt t$, which is much bigger than $n t$ for small enough $t$.
\end{sol}

\begin{ex}
Find $C,c > 0$ such that
\begin{equation}
\PP[\abs{\Brwn_t} \leq \varepsilon, \forall_{t \in \interval01}] \leq C \exp(-c/\varepsilon^2).
\end{equation}
\end{ex}

\begin{sol}
Split the interval $\interval01$ into $\floor{\varepsilon^{-2}}$ intervals of size $\varepsilon^2$ (plus one small interval which I will ignore). Then, if $\Brwn$ is always in the band $\interval{-\varepsilon}\varepsilon$, in none of these intervals can it vary by $2\varepsilon$ or more. Thus, we have the bound
\begin{equation}
\PP[\abs{\Brwn_t} \leq \varepsilon, \forall_{t \in \interval01}] \leq \PP[\text{$\Brwn_{\text{end}} - \Brwn_{\text{start}} \leq 2\varepsilon$ for all these intervals}].
\end{equation}
Now, by independent increments, we know that the right-hand side is precisely equal to
\begin{equation}
\PP[\Brwn_{\varepsilon^2} \leq 2 \varepsilon]^{\floor{\varepsilon^{-2}}} = \PP[\frac1\varepsilon N(0,\varepsilon^2) \leq 2 ]^{\floor{\varepsilon^{-2}}} = \PP[N(0,1) \leq 2]^{\floor{\varepsilon^{-2}}}.
\end{equation}

Now we conclude the proof by setting $c = - \log(\PP[N(0,1) \leq 2])$ and $C = \e^c$, to get
\begin{equation}
\PP[\abs{\Brwn_t} \leq \varepsilon, \forall_{t \in \interval01}] \leq \PP[N(0,1) \leq 2]^{\floor{\varepsilon^{-2}}} = \exp(-c \floor{\varepsilon^{-2}}) \leq \exp(-c(\varepsilon^{-2} - 1)) = C \exp(-c/\varepsilon^2),
\end{equation}
as desired.
\end{sol}

\begin{ex}
Show that the following are martingales:
\begin{enumerate}
\item $\Brwn_t$,
\item $M_t = \Brwn_t^2 - t$,
\item $N_t = \exp(\Brwn_t - t/2)$.
\end{enumerate}
\end{ex}

\begin{sol}
Let us first discuss some generalities about our circumstances. The requirement that $M_t$ be measurable is met in all our three instances, because indeed they are all actually measurable in the $\sigma$-algebra generated by $\Brwn_t$, which is larger than the $\sigma$-algebra generated by $\{\Brwn_s\}_{s \leq t}$.

Now, the requirement that they be $L^1$ is met in all three instances. For the first one this is because the normal distribution is $L^1$, for the second because it is $L^2$, and for the third because you just sort of do the math and find out that (this will be used later)
\begin{equation}\label{eq:3}
\EV[\abs{\exp(\Brwn_t - t/2)}] = \EV[\exp(\Brwn_t - t/2)] = \phi_{\Brwn_t - t/2}(-\I) = \exp(-\frac12 t + \frac12 t) = 1.
\end{equation}

Finally, we need to compute the conditional expectation of $M_t$ given $\sa{F}_s$, for $s \leq t$. The essential observation is the following: $\Brwn_t = \Brwn_s + X$, where $X$ is a random variable with normal distribution (mean $0$ and variance $t-s$), which, by independence of increments, \emph{is independent from the $\sigma$-algebra $\sa{F}_s$,} for $s \leq t$. This allows for the following computations to be performed:
\begin{equation}
\begin{gathered}
\EV[\Brwn_t \mid \sa F_s] = \EV[\Brwn_s \mid \sa F_s] + \EV[X \mid \sa F_s] = \Brwn_s + \EV[X] = \Brwn_s + 0,\\[1em]
\begin{multlined}
\EV[\Brwn_t^2 - t \mid \sa F_s] = \EV[\Brwn_s^2  \mid \sa F_s] + 2 \EV[\Brwn_s X \mid \sa F_s] + \EV[X^2 \mid \sa F_s] - t\\
= \Brwn_s^2 + 2 \EV[\Brwn_s \mid \sa F_s] \cancel{\EV[X \mid \sa F_s]} + \EV[X^2] - t = \Brwn_s^2 + (t-s) - t = \Brwn_s^2 - s,\end{multlined}\\[1em]
\begin{multlined}
\EV[\exp(\Brwn_t - t/2)\mid \sa F_s] = \EV[\exp(\Brwn_s - s/2) \exp(X - \frac12(t-s)) \mid \sa F_s] \\
= \EV[\exp(\Brwn_s - s/2) \mid \sa F_s] \EV[\exp(X - \frac12(t-s)) \mid \sa F_s] = (\Brwn_s - s/2) \times\EV[\exp(X - \frac12(t-s))] = \Brwn_s - s/2,
\end{multlined}
\end{gathered}
\end{equation}
where in the above computations we used independence a few times to separate conditional expectation of a product as a product of expectations, and in the last equality we used computations similar to \eqref{eq:3}.
\end{sol}

\begin{ex}
\leavevmode
\begin{enumerate}
\item Show that for $r \geq s \geq 0$, the distribution of $T_r - T_s$ is the same as the distribution of $T_{r-s}$.
\item Prove that for $0 \leq r_0 \leq r_1 \leq \dots \leq r_n$, the increments $T_{r_j} - T_{r_{j-1}}$ are independent.
\item Compute the density of the distribution of $T_r$.
\end{enumerate}
\end{ex}

\begin{sol}
\leavevmode
\begin{enumerate}
\item First, we note that by continuity we have $T_r \geq T_s$. Thus, we may compute $T_s$ as $T_r + T'_{s-r}$, where $T'_q$ is defined analogously to $T$, but instead for the random function $\Brwn'_t := \Brwn_{t-T_s} - s$.

If we show that $\Brwn'$ has the same distribution as $\Brwn$, then the proof will be done, because then $T'$ has the same distribution as $T$.

To do this we use the fact that $T_r$ is a stopping time (it is the smallest time for which $\Brwn$ is in $A = \{r\}$), and so by a property given in class, $\Brwn'_t = \Brwn_{t-T_s} - \Brwn_{T_s}$ is a standard Brownian. Ok we are done.

\item We begin with a lemma:
\begin{lemma}
Suppose that $A_1, \dots, A_n$ is a collection of independent r.v., and let $A_{n+1}$ be another r.v. which is independent from the vector $(A_1, \dots, A_n)$. Then, $A_1,\dots,A_{n+1}$ is a collection of independent r.v.
\end{lemma}
\begin{lemmaproof}
Direct computation:
\begin{equation}
\PP[A_1 \in B_1 \land \dots \land A_{n+1} \in B_{n+1}] = \PP[A_1 \in B_1 \land \dots \land A_n \in B_n] \PP[A_{n+1} \in B_{n+1}] = \prod_{i=1}^{n+1} \PP[A_i \in B_i].
\end{equation}
\NoEndMark
\hfill\lemmaproofSymbol
\end{lemmaproof}

Now we just apply this lemma, and induction in $n$. The lemma is applicable because $T_{r_{j+1}} - T_{r_{j}}$ depends only on $\Brwn'_t = \Brwn_{t - T_{r_{j}}} - \Brwn_{T_{r_j}}$, which since $T_{r_j}$ is a stopping time is independent from $\Brwn|_{\interval0{T_{r_j}}}$, and since all $T_{r_i}$ for $i \leq j$ depend only on $\Brwn|_{\interval0{T_{r_j}}}$, we have independence of $T_{r_{j+1}} - T_{r_j}$ wrt all previous increments.

\item The reflection property tells us that $\max_{0 \leq s \leq r} \Brwn_s$ has the same distribution as $\sqrt r \abs X$, where $X$ is a standard normal r.v. Thus, we may compute
\begin{equation}\label{eq:4}
\PP[T_r \leq a] = \PP[\max_{0 \leq s \leq a} \Brwn_s \geq r] = 2 \PP[X \geq r/\sqrt a] = 2 (1 - F(r/\sqrt a)),
\end{equation}
where $F(t) = \PP[X \leq t]$ has the property that $F'(t) = \frac1{\sqrt{2\pi}} \exp\left(-\frac12 t^2\right)$. Now, we may compute the derivative in $a$ of \eqref{eq:4} using the chain rule, to get
\begin{equation}
\diff*{\PP[T_r \leq a]}a = -2\frac1{\sqrt{2\pi}} \exp\left(-\frac12 \frac{r^2}a\right) \, \left(-\frac12 r a^{-3/2} \right) = \frac1{\sqrt{2\pi}} r a^{-3/2} \exp\left(-\frac12 \frac{r^2}a\right).
\end{equation}

Thus, the above is the density function of $T_r$, for $a \geq 0$. Obviously the density is null for $a \leq 0$.
\end{enumerate}
\end{sol}

\end{document}