\documentclass{article}

\usepackage{amsmath}
\usepackage{amssymb}
\usepackage{amsfonts,stmaryrd}
\usepackage{mathtools}

\usepackage[thmmarks, amsmath]{ntheorem}
\usepackage{fullpage}

\usepackage{graphicx}

\usepackage{diffcoeff}
\diffdef{}{op-symbol=\mathrm{d},op-order-sep=0mu}

\usepackage{cancel}
\usepackage{interval}

\usepackage{enumitem}

\setlist[enumerate,1]{label=(\roman*)}

\title{Analysis Homework 2}
\author{Duarte Maia}
%\date{}

\theorembodyfont{\upshape}
\theoremseparator{.}
\newtheorem{theorem}{Theorem}
\newtheorem{prop}{Prop}
\renewtheorem*{prop*}{Prop}
\newtheorem{lemma}{Lemma}

\newtheorem{ex}{Exercise}

\theoremstyle{nonumberplain}
\theoremheaderfont{\itshape}
\theorembodyfont{\upshape}
\theoremseparator{:}
\theoremsymbol{\ensuremath{\blacksquare}}
\newtheorem{proof}{Proof}
\newtheorem{sol}{Solution}
\theoremsymbol{\ensuremath{\text{\textit{(End proof of lemma)}}}}
\newtheorem{lemmaproof}{Proof of Lemma}

\newcommand{\R}{\mathbb{R}}
\newcommand{\C}{\mathbb{C}}
\newcommand{\Z}{\mathbb{Z}}
\newcommand{\N}{\mathbb{N}}
\newcommand{\Q}{\mathbb{Q}}
\newcommand{\K}{\mathbb{K}}

\newcommand{\kk}{\Bbbk}

\newcommand{\PP}{\mathbb{P}}
\newcommand{\Gr}{\mathrm{Gr}}

\newcommand{\I}{\mathrm{i}}
\newcommand{\e}{\mathrm{e}}
\newcommand{\id}{\mathrm{id}}

\newcommand{\conj}[1]{\overline{#1}}
\newcommand{\closed}[1]{\overline{#1}}
\newcommand{\transp}{\top}

\newcommand{\grad}{\nabla}
\DeclareMathOperator{\Ix}{Ix}
\DeclareMathOperator{\coker}{coker}

\DeclareMathOperator{\sign}{sign}
\DeclareMathOperator{\image}{im}
\DeclareMathOperator{\ord}{ord}

\DeclareMathOperator{\EV}{\mathrm{EV}}

\newcommand{\HH}{\mathcal{H}}
\newcommand{\bbH}{\mathbb{H}}

\let\Im\relax
\DeclareMathOperator{\Im}{Im}
\let\Re\relax
\DeclareMathOperator{\Re}{Re}

\DeclarePairedDelimiter{\abs}{\lvert}{\rvert}
\DeclarePairedDelimiter{\norm}{\lvert}{\rvert}
\DeclarePairedDelimiter{\Norm}{\lVert}{\rVert}
\DeclarePairedDelimiter{\braket}{\langle}{\rangle}


\begin{document}
\maketitle

\begin{ex}
Assume that for each $w \in \C$ there is some $n$ such that $f^{(n)}(w) = 0$. Show that $f$ is a polynomial.
\end{ex}

\begin{sol}
For each $n \in \N_0$, set $A_n$ as the set of $w \in \C$ such that $f^{(n)}(w) = 0$. Note that all $A_n$ are closed, and moreover, by hypothesis we have $\bigcup A_n = \C$. We may now apply the Baire category theorem, which guarantees that at least one $A_n$ has nontrivial interior. Without loss of generality, by translation assume that it contains a neighborhood of the origin. Then, if we write the powerseries of $f$ about the origin, it has null $n$th coefficient, but also all coefficients above it are null, because in this neighborhood $f^{(n)} \cong 0$, hence $f^{(n+1)} \cong 0$, and so on. Thus, the powerseries of $f$ about the origin in this neighborhood terminates, and since powerseries of entire holomorphic functions are valid for the whole domain, $f$ itself equals the polynomial given by its powerseries about zero.
\end{sol}

\begin{ex}
Let $f$ be nonconstant in a neighborhood of $z_0$. Show that near $z_0$, $f$ is an $N$-to-one function, where $N$ is the smallest positive integer such that $f^{(N)}(z_0) \neq 0$.
\end{ex}

\begin{sol}
In the proof of the open mapping theorem, one shows that, near $z_0$, $f(z)$ may be written in the form $f(0) + g(z)^N$ for some analytic function $g$. Moreover, this function $g$ has the property that $g'(z_0) \neq 0$ and $g(0) = 0$. As such, by the inverse mapping theorem for analytic functions, it is one-to-one in a neighborhood of $z_0$, and takes a ball around $z_0$ without the center injectively to $\C \setminus 0$. Moreover, the `power $N$' function is $N$-to-one on $\C \setminus 0$, and so their composition, $g(z)^N = f(z)$, is $N$-to-one on this neighborhood of $z_0$ with $z_0$ removed.
\end{sol}

\begin{ex}
Let $f$ be an injective holomorphic function on the unit disk such that $f(0) = 0$. Show that there is an injective holomorphic function $h$ such that $(h(z))^2 = f(z^2)$.
\end{ex}

\begin{sol}
By the previous exercise, we see that $f'(0) \neq 0$. As such, the power series corresponding to $f(z^2)$ is of the form $0 + 0 z + a_1 z^2 + \dots$, and therefore has a formal square root power series $h(z) = \sqrt{a_1} z + \dots$. (Two, in fact, depending on the choice of $\sqrt{a_1}$.) It is an algebraic exercise to find this power series $h(z) = \sum b_n z^n$; each $b_n$ is determined from the previous by solving the equation $\sum_{i+j = n+1} b_i b_j = a_{n+1} $; aside from the case $n = 1$ this is a linear equation in $b_n$.

Now, we claim that the radius of convergence of $h(z)$ is the same as that of $f(z)$; once this is done the exercise will have been solved. In fact, it suffices to show that whenever $f(z)$ converges absolutely so does $h(z)$, and in any case the other implication is trivial.

Suppose $f(z)$ converges absolutely. Then, for some $r > \abs z$, we have that $\abs{a_n} < C r^{-n}$. Then, it can be shown by induction (but it's gross and I don't want to do it) that $\abs{b_n}$ is dominated by something like $C' R^{-n}$, for any $R > r$. This would make use of the inductive formula for $b_n$,
\begin{equation}
b_n = \frac1{2 b_1} \left( a_{n+1} - \sum_{i=2}^{n-1} b_i b_{n+1-i} \right),
\end{equation}
followed by using the bounds for $a_{n+1}$ and for the $b_i$.
\end{sol}

\begin{ex}
Show that a harmonic function over $\C$ which grows slower than linearly is constant.
\end{ex}

\begin{sol}
Given $z$ and $w$, we use the following property of Harmonic functions to show that $u(z) = u(w)$: For all positive radii $R$, we have
\begin{equation}
u(z) = \frac1{\pi R^2} \int_{B_R(z)} u.
\end{equation}

Thus, $u(z) - u(w)$ can be written as the difference of two integrals (of the same function) in large balls, divided by $\pi R^2$. This difference can be bounded by the integral of $\abs u$ on the symmetric difference of the two balls, and in turn this can be bounded by the integral of $\abs u$ on an annulus with large radius, and thickness $\abs{z-u}$.

Now, roughly speaking: the value of $\abs u$ on this annulus is $o(R)$, and the area of the annulus is $O(R)$. Thus, the integral would be $o(R^2)$, which, dividing by $\pi R^2$, yields an $o(1)$ quantity. Thus, taking the limit, we obtain $\abs{u(z) - u(w)} = 0$, and so $u$ is constant.
\end{sol}

\begin{ex}
Let $u, v \colon U \to \R$ be harmonic. Under what conditions is $uv$ also harmonic?
\end{ex}

\begin{sol}
We compute the Laplacian of $uv$. For brevity, we use the notation from PDEs where, e.g. $u_xy$ means $\partial_x \partial_y u$.
\begin{equation}
\begin{aligned}
\Delta(uv) &= (uv)_{xx} + (uv)_{yy}\\
&= ( u_x v + u v_x)_x + (u_y v + u v_y)_y\\
&= u_{xx} v + 2 u_x v_x  + u v_{xx} + u_{yy} v + 2 u_y v_y + u v_{yy}\\
&= 2 \nabla u \cdot \nabla v.
\end{aligned}
\end{equation}

Thus, it is a necessary and sufficient condition that $\nabla u$ be always orthogonal to $\nabla v$. This happens for example if one of these is null, i.e. either $u$ or $v$ is constant. But the following example shows that the condition of orthogonality is a bit more general: consider $u = x^2 - y^2$ and $v = xy$. Both of these are harmonic. Moreover, $\nabla u = 2 (x,-y)$ and $\nabla v = (x,y)$, which are evidently orthogonal, so $uv$ is also harmonic.
\end{sol}

\end{document}