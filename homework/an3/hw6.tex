\documentclass{article}

\usepackage{amsmath}
\usepackage{amssymb}
\usepackage{amsfonts,stmaryrd}
\usepackage{mathtools}

\usepackage[thmmarks, amsmath]{ntheorem}
\usepackage{fullpage}

\usepackage{graphicx}
\usepackage{tikz-cd}

\usepackage{diffcoeff}
\difdef{f}{}{
outer-Ldelim = \left. ,
outer-Rdelim = \right| ,
sub-nudge = 0 mu
}

\usepackage{cancel}
\usepackage{interval}

\usepackage{enumitem}

\setlist[enumerate,1]{label=(\alph*)}

\title{Analysis Homework 6}
\author{Duarte Maia}
%\date{}

\theorembodyfont{\upshape}
\theoremseparator{.}
\newtheorem{theorem}{Theorem}
\newtheorem{prop}{Prop}
\renewtheorem*{prop*}{Prop}
\newtheorem{lemma}{Lemma}

\newtheorem{ex}{Exercise}

\theoremstyle{nonumberplain}
\theoremheaderfont{\itshape}
\theorembodyfont{\upshape}
\theoremseparator{:}
\theoremsymbol{\ensuremath{\blacksquare}}
\newtheorem{proof}{Proof}
\newtheorem{sol}{Solution}
\theoremsymbol{\ensuremath{\text{\textit{(End proof of lemma)}}}}
\newtheorem{lemmaproof}{Proof of Lemma}

\newcommand{\R}{\mathbb{R}}
\newcommand{\C}{\mathbb{C}}
\newcommand{\Z}{\mathbb{Z}}
\newcommand{\N}{\mathbb{N}}
\newcommand{\Q}{\mathbb{Q}}
\newcommand{\K}{\mathbb{K}}

\newcommand{\kk}{\Bbbk}

\newcommand{\Gr}{\mathrm{Gr}}

\newcommand{\I}{\mathrm{i}}
\newcommand{\e}{\mathrm{e}}
\newcommand{\id}{\mathrm{id}}

\newcommand{\conj}[1]{\overline{#1}}
\newcommand{\closed}[1]{\overline{#1}}
\newcommand{\transp}{\top}

\newcommand{\grad}{\nabla}
\DeclareMathOperator{\Ix}{Ix}
\DeclareMathOperator{\coker}{coker}

\DeclareMathOperator{\sign}{sign}
\DeclareMathOperator{\image}{im}
\DeclareMathOperator{\ord}{ord}


\DeclareMathOperator{\diam}{diam}
\DeclareMathOperator{\dist}{d}


\newcommand{\HH}{\mathcal{H}}
\newcommand{\bbH}{\mathbb{H}}

\let\Im\relax
\DeclareMathOperator{\Im}{Im}
\let\Re\relax
\DeclareMathOperator{\Re}{Re}

\DeclarePairedDelimiter{\abs}{\lvert}{\rvert}
\DeclarePairedDelimiter{\norm}{\lvert}{\rvert}
\DeclarePairedDelimiter{\Norm}{\lVert}{\rVert}
\DeclarePairedDelimiter{\braket}{\langle}{\rangle}

\newcommand{\EV}{\mathbb{E}}
\newcommand{\PP}{\mathbb{P}}


\begin{document}
\maketitle

\begin{ex}
\leavevmode
\begin{enumerate}
\item Let $E_j$ be a sequence of independent events such that $\sum \PP[E_j] = \infty$. Then, almost certainly, infinitely many $E_j$ occur.

\item If $E_j$ are not independent, then the probability that infinitely many $E_j$ occur may be null.
\end{enumerate}
\end{ex}

\begin{sol}
\leavevmode
\begin{enumerate}
\item The thing that we want to show is that $\PP[\cap_n \cup_{j > n} E_j] = 1$. Since probability measures take decreasing intersections to limits, we wish to show that $\lim_n \PP[\cup_{j>n} E_j] = 1$. Of course, this is equivalent to showing that for all $n$ we have $\PP[\cup_{j>n} E_j] = 1$, and we may as well restrict to the $n = 0$ case, as if $\sum_{j\geq 1} \PP[E_j] = \infty$ we also have $\sum_{j > n} \PP[E_j] = \infty$.

So, let us compute $\PP[\cup E_j]$. By taking the converse, this is
\begin{equation}\label{eq:p1}
\PP\left[\bigcup E_j\right] = 1 - \PP\left[\bigcap E_j\right] = 1 - \prod (1 - \PP[E_j]),
\end{equation}
where the last step used independence of the $E_j$. Now, we have the bound
\begin{equation}
1 - \PP[E_j] \leq \exp(-\PP[E_j]),
\end{equation}
(Proof of bound: We show $1-x \geq \exp(-x)$ for $x \in [0,1]$. If $x = 0$ it is obvious. Otherwise, apply the mean value theorem to show $\frac{\exp(-x)-\exp(0)}x = -\exp(-\xi)$ for some $\xi$ between $1$ and $x$, hence $\exp(-\xi) \leq 1$. Thus, by rearranging $\frac{\exp(-x)-1}x \geq -1$, we obtain the desired inequality.)

So, plugging this bound into \eqref{eq:p1}, we have
\begin{equation}
\PP\left[\bigcup E_j\right] = 1 - \prod (1 - \PP[E_j]) \geq 1 - \prod \exp(-\PP[E_j]) = 1 - \exp(-\sum \PP[E_j]) = 1 - \exp(-\infty) = 1 - 0 = 1.
\end{equation}

\item Let $X$ be a uniformly chosen real number in $[0,1]$, and let $E_j$ be the event that $X < \frac1j$. Then, $\sum \PP[E_j] = \sum \frac1j = \infty$, and yet the only way that infinitely many $E_j$ occur is if $X = 0$, which happens with probability zero.
\end{enumerate}
\end{sol}

\begin{ex}
\end{ex}

\begin{sol}
\end{sol}

\begin{ex}
\end{ex}

\begin{sol}
\end{sol}

\begin{ex}
\end{ex}

\begin{sol}
\end{sol}

\begin{ex}
\end{ex}

\begin{sol}
\end{sol}

\begin{ex}
\end{ex}

\begin{sol}
\end{sol}

\end{document}