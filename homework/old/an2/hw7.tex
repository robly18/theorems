\documentclass{article}

\usepackage{amsmath}
\usepackage{amssymb}
\usepackage{amsfonts,stmaryrd}
\usepackage{mathtools}

\usepackage[thmmarks, amsmath]{ntheorem}
\usepackage{fullpage}

\usepackage{graphicx}

\usepackage{diffcoeff}
\diffdef{}{op-symbol=\mathrm{d},op-order-sep=0mu}

\usepackage{cancel}
\usepackage{interval}

\usepackage{enumitem}

\setlist[enumerate,1]{label=(\roman*)}

\title{Analysis Homework 7}
\author{Duarte Maia}
%\date{}

\theorembodyfont{\upshape}
\theoremseparator{.}
\newtheorem{theorem}{Theorem}
\newtheorem{prop}{Prop}
\renewtheorem*{prop*}{Prop}
\newtheorem{lemma}{Lemma}

\newtheorem{ex}{Exercise}

\theoremstyle{nonumberplain}
\theoremheaderfont{\itshape}
\theorembodyfont{\upshape}
\theoremseparator{:}
\theoremsymbol{\ensuremath{\blacksquare}}
\newtheorem{proof}{Proof}
\newtheorem{sol}{Solution}
\theoremsymbol{\ensuremath{\text{\textit{(End proof of lemma)}}}}
\newtheorem{lemmaproof}{Proof of Lemma}

\newcommand{\R}{\mathbb{R}}
\newcommand{\C}{\mathbb{C}}
\newcommand{\Z}{\mathbb{Z}}
\newcommand{\N}{\mathbb{N}}
\newcommand{\Q}{\mathbb{Q}}
\newcommand{\K}{\mathbb{K}}

\newcommand{\kk}{\Bbbk}

\newcommand{\PP}{\mathbb{P}}
\newcommand{\Gr}{\mathrm{Gr}}

\newcommand{\I}{\mathrm{i}}
\newcommand{\e}{\mathrm{e}}
\newcommand{\id}{\mathrm{id}}

\newcommand{\conj}[1]{\overline{#1}}
\newcommand{\closed}[1]{\overline{#1}}

\newcommand{\grad}{\nabla}
\DeclareMathOperator{\Ix}{Ix}
\DeclareMathOperator{\coker}{coker}

\DeclareMathOperator{\sign}{sign}
\DeclareMathOperator{\image}{im}
\DeclareMathOperator{\ord}{ord}

\DeclareMathOperator{\EV}{\mathrm{EV}}

\newcommand{\HH}{\mathcal{H}}
\newcommand{\bbH}{\mathbb{H}}

\let\Im\relax
\DeclareMathOperator{\Im}{Im}
\let\Re\relax
\DeclareMathOperator{\Re}{Re}

\DeclarePairedDelimiter{\abs}{\lvert}{\rvert}
\DeclarePairedDelimiter{\norm}{\lvert}{\rvert}
\DeclarePairedDelimiter{\Norm}{\lVert}{\rVert}
\DeclarePairedDelimiter{\braket}{\langle}{\rangle}


\begin{document}
\maketitle

\begin{ex}
\leavevmode
\begin{enumerate}
\item Show that if $f \in W^{1,p}(0,1)$ then
\begin{equation}
\abs{f(x) - f(y)} \leq \Norm{f}_{1,p} \abs{x-y}^{1/q}.
\end{equation}
\item Show that if $f \in W^{1,p}_0(0,1)$ then $f(0) = f(1) = 0$.
\item Show that for all $f \in C^\infty_0(0,1) \cap W^{1,p}_0(0,1)$ there is $f_k \in C^\infty_c(0,1)$ so that $f_k \to f$ in $W^{1,p}$. Conclude that $C^\infty_0 \cap W^{1,p}_0$ is dense in $W^{1,p}_0$.
\item Show that if $f \in W^{1,p}(0,1)$ is such that $f(0) = f(1) = 0$ then $f \in W^{1,p}_0$.
\end{enumerate}
\end{ex}

\begin{sol}
\leavevmode
\begin{enumerate}
\item This inequality follows from Hölder, because (since $f$ is continuous, by the FTC) we have (for $x \leq y$)
\begin{equation}
\begin{aligned}
\abs{f(x) - f(y)}
&= \abs*{\int_x^y f'}\\
&\leq \Norm{\chi_{\interval xy}}_{L^q} \Norm{f'}_{L^p}\\
&\leq \abs{x-y}^{1/q} \Norm{f}_{W^{1,p}}.
\end{aligned}
\end{equation}
\item Let $\varphi_n$ be a sequence of $C^\infty_c$ functions which converge to $f \in W^{1,p}_0$. Then, there exists a uniform bound for the $W^{1,p}$ norm of the $\varphi_n$, call it $C$. By the previous item, then, we have
\begin{equation}
\abs{\varphi_n(x)} = \abs{\varphi_n(x) - \varphi_n(0)} \leq C \abs{x}^{1/q},
\end{equation}
hence this bound also holds for $f$ (because up to subsequence we have almost everywhere convergence). In principle it only holds a.e. but continuity guarantees it holds everywhere. Plugging in particular $x = 0$ we obtain $f(0) = 0$. A similar bound holds near $x = 1$, with $\abs{\varphi_n(x)} \leq C \abs{1-x}^{1/q}$, which shows likewise that $f(1) = 0$.
\item By Brezis' definition of $W^{1,p}_0$, this question is obvious: A function $f$ is in $W^{1,p}_0$ exactly if there exists a sequence $f_k$ in the given conditions which converges to $f$. The fact that $f$ is also in $C^\infty_0$ is irrelevant.

The punchline of the problem is also obvious, as $C^\infty_0 \cap W^{1,p}_0$ contains a set which is dense in $W^{1,p}_0$, namely $C^\infty_c$.

\item Let $f_n$ be a sequence of $C^\infty$ functions converging to $f$ in $W^{1,p}$. Then, up to subsequence, they converge a.e., and by continuity and using the bound in (i) we get convergence everywhere, in particular at zero and at one.

Modify the sequence $f_n$ by setting $\phi_n(x) = f_n(x) - f_n(0) - f_n(1) x$. Since $f_n(0), f_n(1) \to 0$, we have that $\phi_n - f_n \to 0$, hence $\phi_n \to f$.

Since all $\phi_n$ are in $W^{1,p}_0$ and this set is closed in $W^{1,p}$, we get that their limit, $f$, is also in $W^{1,p}_0$, which completes the proof.
\end{enumerate}
\end{sol}

\begin{ex}
\leavevmode
\begin{enumerate}
\item Show that $B_c$ is coercive for $c > \pi^2$. Conclude that there exists a bounded linear map $S_c \colon L^2(0,1) \to H^1_0(0,1)$ such that
\begin{equation}
B_c(S_c(f),v) = -\int_0^1 f v, \; \forall_{v \in H^1_0(0,1)}.
\end{equation}
\item Show that $T(f) \in H^1(0,1)$. Then show that $T(f) \in H^1_0(0,1)$.
\item Compute the eigenvalues of $T$ and conclude that $T+\lambda I$ is invertible for all $\lambda > 0$.
\item With $c > \pi^2$ set $\lambda = \frac1{c-\pi^2}$. Check that
\begin{equation}
S_c(f) = (T-\lambda I)^{-1} T(-\lambda f).
\end{equation}

Conclude that $S_c \colon L^2 \to L^2$ is compact.

\item Pick $c' \neq c$ so that $\sigma = (c' - c)^{-1}$ is not an eigenvalue of $S_c$. Show that $D(f) = (S_c - \sigma I)^{-1} S_c(-\sigma f)$ is such that
\begin{equation}
B_{c'}(D(f), v) = - \int_0^1 f v, \; f \in L^2, v \in H^1_0.
\end{equation}

Conclude that $S_{c'} = D$.

\item Compute $\EV(S_c)$ and show that the set $\{c' \mid (c'-c)^{-1} \notin \EV(S_c)\}$ does not depend on $c$.
\end{enumerate}
\end{ex}

\begin{sol}
\leavevmode
\begin{enumerate}
\item We want to apply the Lax-Milgram theorem, so we need to ensure that $B_c$ (which is obviously bounded) satisfies an inequality of the form
\begin{equation}
B_c(u,u) \geq \theta \Norm{u}^2_{H^1_0}.
\end{equation}

So, we expand:
\begin{equation}
\begin{aligned}
B_c(u,u)
&= \int (u')^2 + (c-\pi^2) u^2\\
&= \Norm{u'}^2_{L^2} + (c-\pi^2) \Norm{u}^2_{L^2},\\
\text{and if $c > \pi^2$, } &\geq (c-\pi^2) \Norm{u}_{H^1}.
\end{aligned}
\end{equation}

So we have coercivity, and thus may apply Lax-Milgram. To do so, we note that the operator $v \mapsto -\int_0^1 f v = - \braket{f,v}_{L^2}$ is bounded in $L^2$, and therefore bounded in $H^1_0$. Thus, by the Lax-Milgram theorem, there is an operator $S_c$ such that
\begin{equation}
B_c(S_c(f),v) = - \int_0^1 f v,
\end{equation}
as desired.

\item We prove that $Tf$ is in $H^1$ by finding a weak derivative, namely
\begin{equation}
S(f)(x) = \int_0^1 \partial_x k(x,y) f(y) \dl3 y.
\end{equation}

We show that this is a weak derivative to $f$. Given $\varphi \in C^\infty_c(0,1)$, we have
\begin{equation}
\begin{aligned}
\int_0^1 Sf(x) \varphi(x) \dl3 x
&= \int_0^1 \int_0^1 \partial_x k(x,y) f(y) \varphi(x) \dl3 y \dl3 x\\
&= \int_0^1 f(y) \int_0^1 \partial_x k(x,y) \varphi(x) \dl3 x \dl3 y\\
\text{(This requires a little check but not hard)} &= - \int_0^1 f(y) \int_0^1 k(x,y) \varphi'(x) \dl3 x \dl3 y\\
&= \text{(Undo the steps we did so far...)}\\
&= - \int_0^1 Tf(x) \varphi'(x) \dl3 x.
\end{aligned}
\end{equation}

Moreover, $S(f)$ is in $L^2$ because it is integration against an $L^2$ kernel.

To see that $Tf$ is moreover in $H^1_0$, we note that it is continuous (because it is in $H^1$) and by the previous problem, to verify that it is in $H^1_0$ it suffices to compute $Tf(0)$ and $Tf(1)$. We do one of them; the other is the same:
\begin{equation}
Tf(0) = \int_0^1 k(0,y) f(y) \dl3 y = \int_0^1 0 = 0.
\end{equation}

Thus, $Tf \in H^1_0$.

(Side note: We also showed that $T$ is a bounded operator $L^2 \to H^1$, because $\Norm{Tf}_{H^1} \leq \Norm{Tf}_{L^2} + \Norm{Sf}_{L^2} \leq (\Norm{T} + \Norm{S}) \Norm{f}_{L^2}$.)

\item We want to solve, for $\lambda \neq 0$, the equation
\begin{equation}\label{eq:ode1}
\begin{gathered}
Tf(x) = \lambda f(x),\\
\text{that is, } \int_0^1 k(x,y) f(y) \dl3 y = \lambda f(x).
\end{gathered}
\end{equation}

If \eqref{eq:ode1} holds, then the second derivative on both sides is also the same. Upon taking it, we end up with a second order ODE, where fortunately the right-hand side has simplified significantly (thanks to the particular formula for $k$)
\begin{equation}
\begin{cases}
\lambda f''(x) = - f(x)\\
f(0)=f(1)=0 & \text{(Because $f \in T(L^2) \subseteq H^1_0$)}.
\end{cases}
\end{equation}

Classical theory solves this ODE: For $\lambda \leq 0$ there are no nonzero solutions, and for $\lambda > 0$ we have the solution (unique up to scaling)
\begin{equation}
f(x) = \sin(\sqrt\lambda x).
\end{equation}

In order for the boundary conditions at $x=1$ to hold, we must have $\lambda$ of the form $k^2 \pi^2$, $k \in \Z$, thus all elements of the spectrum are of this form. Conversely, one verifies by plugging into $T$ that all such values are eigenvalues, with eigenvector $\sin(k \pi x)$.

\item First of all, we remark that $(T-\lambda I)^{-1}$ commutes with $T$ (because $(T-\lambda I)$ does). Thus, by changing variable to $g = (T-\lambda I)^{-1} f$, we reduce the question to proving that, for all $g \in L^2$ and $v \in H^1_0$,
\begin{equation}\label{eq:bc}
B_c(-\lambda T g, v) = - \int_0^1 (T-\lambda I)g(x) v(x) \dl3 x.
\end{equation}

By density, it suffices to prove this for $v \in C^\infty_c(0,1)$. This is good because then $v$ is a test function, which lets us use the definition of derivative when we're working with $B_c$, so that we may write
\begin{equation}
B_c(-\lambda T g, v) = - \int \lambda Tg(x) v''(x) \dl3 x - \cancel{\lambda (c-\pi^2)} \int Tg(x) v(x) \dl3x.
\end{equation}

Thus, upon cancelling some terms and expanding the definition of $T$, equation \eqref{eq:bc} simplifies to
\begin{equation}
\int \int k(x,y) g(y) v''(x) \dl3y \dl3x = \int g(x) v(x) \dl3x.
\end{equation}

This equation can be verified to be true as follows: on the left-hand side integrate by parts in $x$ twice, to obtain
\begin{equation}
\int \int k(x,y) g(y) v''(x) \dl3y \dl3x = \int \int \partial_x^2 k(x,y) g(y) v(x) \dl3y \dl3x,
\end{equation}
where the derivatives are in the sense of distributions. One readily sees by computing that $\partial_x^2 k(x,y) = \delta(x-y)$, and so the left-hand side simplifies to $\int g(x) v(x) \dl3x$, which is precisely what we want.

\item The problem is really about showing that $S_{c'} = D$, or equivalently that
\begin{equation}
(S_c - \sigma I) S_{c'} = -\sigma S_c.
\end{equation}

This can be done by using the previous item to expand $S_c$ and $S_{c'}$ in terms of $T$, using the same trick as in the previous item to show that everything commutes, multiplying both sides out by $T-\lambda I$ and $T-\lambda' I$ to get rid of inverses, expanding, and ruthlessly simplifying. As André would call it, it is a `content-free proof'.


\item
\end{enumerate}
\end{sol}

\end{document}