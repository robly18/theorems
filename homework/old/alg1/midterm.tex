\documentclass{article}

\usepackage{amsmath}
\usepackage{amssymb}
\usepackage{amsfonts}
\usepackage{mathtools}

\usepackage[thmmarks, amsmath]{ntheorem}

\usepackage{graphicx}

\usepackage{diffcoeff}
\diffdef{}{op-symbol=\mathrm{d},op-order-sep=0mu}

\usepackage{cancel}
\usepackage{interval}

\usepackage{enumitem}

\setlist[enumerate,1]{label=(\roman*)}

\title{Algebra Midterm}
\author{Duarte Maia}
%\date{}

\theorembodyfont{\upshape}
\theoremseparator{.}
\newtheorem{theorem}{Theorem}
\newtheorem{prop}{Prop}
\renewtheorem*{prop*}{Prop}
\newtheorem{lemma}{Lemma}

\newtheorem{ex}{Exercise}

\theoremstyle{nonumberplain}
\theoremheaderfont{\itshape}
\theorembodyfont{\upshape}
\theoremseparator{:}
\theoremsymbol{\ensuremath{\blacksquare}}
\newtheorem{proof}{Proof}
\newtheorem{sol}{Solution}

\newcommand{\R}{\mathbb{R}}
\newcommand{\C}{\mathbb{C}}
\newcommand{\Z}{\mathbb{Z}}
\newcommand{\Q}{\mathbb{Q}}

\newcommand{\kk}{\Bbbk}

\newcommand{\PP}{\mathbb{P}}
\newcommand{\FF}{\mathcal{F}}
\newcommand{\DD}{\mathcal{D}}

\newcommand{\I}{\mathrm{i}}
\newcommand{\e}{\mathrm{e}}
\newcommand{\id}{\mathrm{id}}

\newcommand{\conj}[1]{\overline{#1}}

\DeclareMathOperator{\inte}{int}
\DeclareMathOperator{\codim}{codim}
\newcommand{\grad}{\nabla}
\newcommand{\schur}{\mathbf{s}}
\newcommand{\reg}{\mathit{reg}}
\newcommand{\regl}{{\mathit{reg}_\ell}}
\newcommand{\cg}{\vee}

\DeclareMathOperator{\vol}{vol}
\DeclareMathOperator{\Av}{Av}
\DeclareMathOperator{\trace}{tr}
\DeclareMathOperator{\sign}{sign}


\DeclareMathOperator{\Aff}{Aff}
\newcommand{\GL}{\mathrm{GL}}
\newcommand{\SL}{\mathrm{SL}}
\newcommand{\Hp}{\mathrm{H}}

\newcommand{\HH}{\mathcal{H}}

\let\Im\relax
\DeclareMathOperator{\Im}{Im}
\let\Re\relax
\DeclareMathOperator{\Re}{Re}

\DeclarePairedDelimiter{\abs}{\lvert}{\rvert}
\DeclarePairedDelimiter{\norm}{\lvert}{\rvert}
\DeclarePairedDelimiter{\Norm}{\lVert}{\rVert}
\DeclarePairedDelimiter{\braket}{\langle}{\rangle}


\begin{document}
\maketitle

\begin{ex}
Let $G$ be an arbitrary abelian group. Prove that a representation of $G$ in a finite dimensional complex vector space is an irrep iff it is one dimensional.
\end{ex}

\begin{sol}
It is obvious that one-dimensional reps are irreps, so we show that any irrep is one-dimensional.

Let $\rho \colon G \to \GL(V)$. Let $S \subseteq \GL(V)$ be the image of $\rho$. Obviously all elements of $S$ commute, as $\rho(g) \rho(h) = \rho(gh) = \rho(hg) = \rho(h) \rho(g)$, hence we can apply PSet 1 to find a nonzero eigenvector $v$ common to all elements of $S$. Then, the subspace generated by $v$ is obviously invariant under the action of $G$, hence, by the hypothesis that $V$ is an irrep, $V = \braket{v}$, which is one dimensional. This completes the proof.
\end{sol}

\pagebreak

\begin{ex}
Let $G$ be a finite subgroup of $\GL_n(\C)$. Prove that if $\sum \trace(g) = 0$ then $\sum g = 0$.
\end{ex}

\begin{sol}
Let $b = \sum_{g \in G} g$. The hypothesis is equivalent to the statement that $\trace b = 0$.

It is obvious from the expression that $b$ is $G$-invariant, that is, for all $g \in G$ we have $gb = b$. In particular, if we take the sum over all elements of $g$, we obtain $b^2 = n b$, with $n = \abs{G}$. As a consequence, $P = \frac1n b$ is a projection operator. But the trace of $P$ is null, and the trace of a projection operator coincides with the dimension of the image, so we get that $P = 0$ and thus $b = 0$, which is what we wanted to show.
\end{sol}

\pagebreak

\begin{ex}
Prove that all elements of the Specht module $V(\lambda)$ are $S_n$-harmonic.
\end{ex}

\begin{sol}
Let $D$ be a nonconstant homogeneous (wlog) differential operator which is $S_n$-invariant. Then, it is an intertwiner $V(\lambda) \to P^d$ (this is direct by definition of $S_n$-invariant), where $d$ is $d_\lambda$ minus the degree of $D$, which is nonzero by hypothesis, hence $d < d_\lambda$. Thus, by lemma 5.2.6, $D$ is null when restricted to $V(\lambda)$. By arbitraryness of $D$, we get that $V(\lambda)$ is annihilated by any nonconstant $S_n$-invariant polynomial, hence all elements of $V(\lambda)$ are $S_n$-harmonic.
\end{sol}

\pagebreak

\begin{ex}
Show that if $f \colon V \to W$ is a nonzero intertwiner, with $V$ and $W$ irreducible Euclidian finite dimensional complex unitary reps of an arbitrary group $G$, then there exists an intertwiner $F \colon V \to W$ which is an isometry.
\end{ex}

\begin{sol}
By the Schur lemma (prop 4.5.1 in the notes) we get that $f$ is an isomorphism. Moreover, we may consider the adjoint map $g = f^* \colon W \to V$.

We will show that $g$ is itself an intertwiner. In other words, that for $a \in G$ (sorry for the bad notation) and $w \in W$ we have $a g(w) = g(a w)$. Equivalently, we can show that for any $v \in V$ we have
\begin{equation}
\braket{a g(w), v} = \braket{g(a w), v},
\end{equation}
so we will show this now, using the fact that both reps are unitary:
\begin{multline}
\braket{a g(w), v} = \braket{g(w), a^{-1} v} = \braket{w, f(a^{-1} v)}\\ = \braket{w, a^{-1} f(v)} = \braket{a w, f(v)} = \braket{g(a w), v}.
\end{multline}

Now that we've shown that $g$ is an intertwiner, we may apply the Schur lemma again, and we get that $g f$ is a constant multiple of identity. Let $C$ be this constant. Note that $C \neq 0$, because $f$ is an isomorphism, and thus so is $g$. Now, let $Z$ be a complex square root of $C$. Set $F = \frac1Z f$. This is very obviously an intertwiner.

This new map $F$ satisfies $F^* F = \id_V$, and thus
\begin{equation}
\braket{v_1, v_2} = \braket{F^* F v_1, v_2} = \braket{F v_1, F v_2},
\end{equation}
and so $F$ is the map we desired.
\end{sol}

\pagebreak

\begin{ex}
Find a triple of homogeneous generators of $\C[x,y]^G$, with $G$ as in the problem statement. Find a nonzero polynomial $p(t_1, t_2, t_3)$ such that $p(a,b,c) = 0$.
\end{ex}

\begin{sol}
Let $a = x^3$, $b = y^3$, and $c = xy$. Evidently, all of these are $G$-invariant, so now we show that they generate $\C[x,y]$. That they are not algebraically independent is obvious: set $p(t_1, t_2, t_3) = t_1 t_2 - t_3^3$.

To see that $a, b, c$ generate $\C[x,y]^G$, pick an arbitrary $G$-invariant polynomial. Since this is a diagonal action, it preserves the degree in every variable, so a polynomial is $G$-invariant if and only if all its monomials are $G$-invariant, so we may without loss of generality show that we can make any $G$-invariant monomial using only $a$, $b$, and $c$.

A $G$-invariant monomial is a multiple of some $x^N y^M$, where $N$ and $M$ satisfy (by $G$-invariance) $\zeta^{N-M} = 1$, i.e. $N$ is congruent with $M$ modulo $n$.

Without loss of generality, suppose that $N \geq M$; otherwise, swap the roles of $x$ and $y$. Then, $N = M + 3k$, for some $k \geq 0$. As such, we have $x^N y^M = x^{3k} (xy)^M = a^k c^M$. (If $N \leq M$ there would be a $b$ instead of $a$.)

This proves that any $G$-invariant monomial may be written in terms of $a$, $b$, and $c$, and by the above considerations, this suffices to show that these generate $\C[x,y]^G$.
\end{sol}

\pagebreak

\begin{ex}
Let $T \subseteq \GL_n(\C)$ be an $r$-dimensional torus. Show that for any closed convex $T$-stable subset $X \subseteq \C^n$ one has $\Av(X) \subseteq X$; in particular, (if $X$ is nonempty!) $X$ contains a $T$-fixed point.
\end{ex}

\begin{sol}
Let $\rho \colon (S^1)^r \to \GL_n(\C)$ be the isomorphism between $T$ and the torus. We begin by noting that we have a very explicit expression for the averaging operator:
\begin{equation}
\Av(x) = \frac1{(2\pi)^r} \int_0^{2\pi} \cdots \int_0^{2\pi} \rho(\e^{\I \theta_1}, \dots, \e^{\I \theta_r}) x \, \dl2 \theta_1 \dots \dl2 \theta_n.
\end{equation}

Now, it is a known fact from calculus (which may be proven using Riemann sum approximations) that the average value of $f(z)$ for $z$ in some region $R \subseteq \R^n$ and $f$ continuous $\R^n \to \R^m$ is in the closed convex hull of $f(R)$. As a consequence, $\Av(x)$ is in the closed convex hull of the set $A = \{\rho(z) x \mid z \in (S^1)^r\}$, and since $x \in X$ and $X$ is $T$-stable we have $A \subseteq X$, hence $\Av(x) \in X$. This proves the first part.

For the second part, pick any $x_0 \in X$, and set $x_1 = \Av(x) \in X$. We know that (averaging lemma) $\Av(x) \in (\C^n)^G$, i.e. it is a $T$-fixed point. This is the fixed point we sought.
\end{sol}

\end{document}