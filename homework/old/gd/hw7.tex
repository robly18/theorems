\documentclass{article}

\usepackage{amsmath}
\usepackage{amssymb}
\usepackage{amsfonts}
\usepackage{mathtools}

\usepackage{graphicx}

\usepackage[thmmarks, amsmath]{ntheorem}

\usepackage{diffcoeff}
\diffdef{}{op-symbol=\mathrm{d},op-order-sep=0mu}
\usepackage{cancel}

\usepackage{enumitem}

\setlist[enumerate]{label=\alph*)}

\title{Differential Geometry Homework 7}
\author{Duarte Maia}
\date{}

\theorembodyfont{\upshape}
\theoremseparator{.}
\newtheorem{ex}{Exercise}

\theoremstyle{nonumberplain}
\theoremheaderfont{\itshape}
\theorembodyfont{\upshape}
\theoremseparator{:}
\theoremsymbol{\ensuremath{\blacksquare}}
\newtheorem{sol}{Solution}

\newcommand{\R}{\mathbb{R}}
\newcommand{\C}{\mathbb{C}}
\newcommand{\Z}{\mathbb{Z}}

\newcommand{\PP}{\mathbb{P}}
\newcommand{\FF}{\mathcal{F}}

\newcommand{\I}{\mathrm{i}}
\newcommand{\e}{\mathrm{e}}


\DeclareMathOperator{\inte}{int}
\DeclareMathOperator{\codim}{codim}
\DeclareMathOperator{\Lie}{Lie}
\DeclareMathOperator{\Ad}{Ad}
\DeclareMathOperator{\ad}{ad}
\DeclareMathOperator{\sign}{sign}
\DeclareMathOperator{\im}{im}
\newcommand{\grad}{\nabla}
\newcommand{\into}{\mathbin{\lrcorner}}
\newcommand{\id}{\mathrm{id}}

\DeclarePairedDelimiter{\norm}{\lvert}{\rvert}
\DeclarePairedDelimiter{\abs}{\lvert}{\rvert}

\newcommand{\pc}{P^1_\C}
\newcommand{\conj}[1]{\overline{#1}}
\newcommand{\J}{\mathrm{j}}

\begin{document}
\maketitle

\begin{ex}
Let $L \to \pc$ be the tautological line bundle. Let $U_z$ and $U_w$ be charts on $\pc$, both with coordinates living in $\C$, and with correspondences
\[z \mapsto [z,1], \quad w \mapsto [1,w].\]

\begin{enumerate}
\item Consider the trivializations on $U_z$ and $U_w$ induced by the (complex) frames: $s_z \colon [z,1] \mapsto (z,1)$, $s_w \colon [1,w] \mapsto (1,w)$. Determine the corresponding transition function $g_{zw}$.

\item Define the following elements of the dual bundle
\[ \omega_z = \frac{-\overline{z}}{1+\abs{z}^2} \dl z, \quad \omega_w = \frac{-\overline{w}}{1+\abs{w}^2} \dl w.\]

Show that these forms are connections forms for some connection on $L^*$.

\item Compute the curvature form on $U_z$.

\item Compute the first Chern number.

\item Compute $\underline{c_1}(L^{\otimes k} \otimes (L^*)^{\otimes r})$.

\item Show that there exist two linearly independent sections $\sigma, \eta \in \Gamma(L^*)$ given by
\[\sigma|_{U_z} = s^*_z, \sigma|_{w=0} = 0; \quad \eta|_{U_w} = s^*_w, \eta|_{z=0} = 0.\]

Show also that they are holomorphic, in the sense that their covariant derivatives in the directions of $\overline{z}$ and $\overline{w}$ are null.

\item Show that any section of the form $a \sigma + b \eta$, with $a,b \in \C$, has one simple zero.
\end{enumerate}
\end{ex}

\begin{sol}
a) The transition function $g_{zw}$ satisfies: If a section is of the form $s = \alpha s_w$, then it is of the form $s = g_{zw} \alpha s_z$. Consequently, if we set $\alpha = 1$, we may write $g_{zw} s_z = s_w$, so we need only write $s_w$ in terms of $s_z$.

Well, at a point $[z,w]$, we get that
\[s_z = (\frac zw, 1), \quad s_w = (1,\frac wz) = \frac wz s_z,\]
so we conclude
\[g_{zw}[z,w] = \frac wz.\]

This notation isn't very good for the following computations, because we will only be using either the representations $[z,1]$ or $[1,w]$. Under this notation,
\[g_{zw} = w = \frac1z.\]

\medskip

b) The first step is to compute the dual cocycles $g_{zw}^*$, so that afterwards we may verify the condition $\omega_z = g^*_{zw} \omega_w g^*_{wz} + g^*_{zw} \dl g^*_{wz}$. As such, I will deduce the formula for dual cocycles.

Suppose that we have a formula for some cocycle $g_{12}$, which can be used to transform from a frame $s_1$ to a frame $s_2$ in the manner
\[(s_2)_j = \sum_i (s_1)_i (g_{12})_{ij}.\]

This allows us to compute the dual frame $s_2^*$ in terms of $s_1^*$. Let $v$ be a vector on the intersection of the domains of the two frames. Then,
\[v = \sum_j (s_2^*)_j v \, (s_2)_j = \sum_{i,j} (s_2^*)_j v \, (g_{12})_{ij} (s_1)_i.\]

As a consequence, $(s_1^*)_i v = \sum_j (s_2^*)_j v \, (g_{12})_{ij}$, i.e.
\[(s_1^*)_j = \sum_i (s_2^*)_i (g_{12})_{ji}.\]

In other words, $(g_{21}^*) = (g_{12})^T = ((g_{21})^{-1})^T$. (I think that this is surprising, because I was expecting a conjugate transpose, but the math seems to check out.) In our case, this translates to
\[g_{zw}^* = \frac1{g_{zw}} = \frac 1w; \quad g_{wz}^* = w.\]

Therefore, we compute $g^*_{zw} \omega_w g^*_{wz} + g^*_{zw} \dl g^*_{wz}$. Note that since we are in a line bundle, the first term is simply $\omega_w$, so it suffices to compute
\begin{equation*}
\omega_w + g^*_{zw} \dl g^*_{wz} = \frac{-\overline{w}}{1+\abs{w}^2} \dl2 w + z \dl2 w
\end{equation*}

To continue the computations, we need to write $\dl z$ in terms of $\dl w$. We know that (where both are defined) $zw = 1$, and therefore
\[\dl(zw) = 0\text{, i.e. } \dl z \, w + z \dl w = 0.\]

(Note: In principle, I should take care here, because I'm taking rules that are true in the real realm and applying them to complex-valued stuff. So I took my pen and paper and verified the above identity, looking at complex stuff as though they lived in $\R^2$ and doing the math on the components, and it checked out.)

Therefore, $\dl w = -\frac wz \dl z$. Note also that as a consequence of the condition $zw=1$, all instances of $w$ can be replaced by $1/z$, and so

\begin{align*}
\omega_w + g^*_{zw} \dl g^*_{wz} &=\frac{-\overline{w}}{1+\abs{w}^2} \dl2 w + z \dl2 w\\
&= \frac{-1/\overline{z}}{1+1/\abs{z}^2} \frac{-w}z \dl z - z \frac wz \dl z\\
&= \frac{1}{1+\abs{z}^2} \frac1z \dl z - \frac1z \dl2 w\\
&= \frac1z \left(\frac{1}{1+\abs{z}^2} - 1\right) \dl2 w\\
&= \frac1z \frac{-\abs{z}^2}{1+\abs{z}^2} \dl2 w\\
&= \frac{-\overline{z}}{1+\abs{z}^2} \dl2 w = \omega_z.
\end{align*}

In conclusion, the forms $\omega_w$ and $\omega_z$ transform as required, and therefore induce a connection on $\pc$.

\medskip

c) The curvature form is given by $\Omega_z = \dl \omega_z + \omega_z \wedge \omega_z$. Since we're in a line bundle, the second term vanishes, so it suffices to compute
\begin{align*}
\dl \omega_z &= - \dl \frac{\overline{z}}{1+\abs{z}^2} \wedge \dl z\\
&= - \left( \frac{\dl \overline{z}}{1+\abs{z}^2} - \overline{z} \frac{\dl(1+\abs{z}^2)}{(1+\abs{z}^2)^2} \right) \wedge \dl z\\
&= - \frac1{1+\abs{z}^2} \dl \overline{z} \wedge \dl z + \left(\overline{z} \frac{\dl(1+\abs{z}^2)}{(1+\abs{z}^2)^2} \right) \wedge \dl z\\
&= - \frac1{1+\abs{z}^2} \dl \overline{z} \wedge \dl z + \left(\overline{z} \frac{z \dl \overline{z} + \cancel{\overline{z} \dl z}}{(1+\abs{z}^2)^2} \right) \wedge \dl z\\
&= - \frac1{1+\abs{z}^2} \dl \overline{z} \wedge \dl z + \left(\overline{z} \frac1{(1+\abs{z}^2)^2} \right) z \dl2 \overline{z} \wedge \dl z\\
&= \left(- \frac1{1+\abs{z}^2} + \overline{z} \frac1{(1+\abs{z}^2)^2}  z \right) \dl \overline{z} \wedge \dl z\\
&= \left(\frac{-1-\abs{z}^2}{(1+\abs{z}^2)} + \frac{\abs{z}^2}{(1+\abs{z}^2)^2} \right) \dl \overline{z} \wedge \dl z\\
&= \frac1{(1+\abs{z}^2)^2} \dl z \wedge \dl \overline{z}.
\end{align*}

\medskip

d) Since the matrix $\Omega$ has only one entry, its trace (i.e. $c_1$) is $\Omega$ itself, so we integrate $\frac\I{2\pi}\Omega$. Since the $z$ chart covers the whole space except a submanifold of dimension one, and hence measure zero, it suffices to integrate $\frac\I{2\pi}\Omega_z$ over $\C$, so we do:
\begin{align*}
\underline{c_1} &= \frac\I{2\pi} \iint \frac1{(1+x^2+y^2)^2} \dl3(x+\I y) \wedge \dl(x-\I y)\\
&= \frac\I{2\pi} \iint \frac1{(1+x^2+y^2)^2} \, 2 \I \dl3 x \wedge \dl y\\
&= - \frac1\pi \iint \frac1{(1+x^2+y^2)^2} \dl3 x \dl3 y.
\end{align*}

This is a nasty integral. Switch to polar coordinates:
\begin{align*}
\underline{c_1} &= - \frac1\pi \iint \frac1{(1+x^2+y^2)^2} \dl3 x \dl3 y\\
&= - \frac1\pi \int_0^{2\pi} \int_0^\infty \frac1{(1+r^2)^2} r \dl3 r \dl3 \theta\\
&= \int_0^\infty \frac {-2r}{(1+r^2)^2} \dl3 r.
\end{align*}

(There are a bunch of things that probably should be said here. The standard argument about how we're ignoring a set of null measure. Verifying that the coordinates $(r,\theta)$ are positively oriented. That's all routine and tiresome to write, so I didn't.)

Now, I have to calculate that integral. By eye, I was able to guess a primitive: $\frac1{1+r^2}$, and so
\[\underline{c_1} = \left[ \frac1{1+r^2} \right]_0^\infty = -1.\]

\medskip

e) First I need to consider a connection on the tensor product. It is natural to define, on a tensor of vector bundles $E \otimes F$,
\[\nabla_X(s_1 \otimes s_2) = (\nabla_X s_1 \otimes s_2) + s_1 \otimes \nabla_X s_2.\]

This means that if $E$ has a frame $s_1, \dots, s_e$ and $F$ has a frame $s'_1, \dots, s'_f$, the tensor has a natural frame $s_i \otimes s'_j$, to which are associated forms $\omega''_{ii'jj'}$. It is easy to tell that these can be written in terms of the connection forms on $E$ and $F$ as
\[\omega''_{ii'jj'} = \omega_{ij} + \omega'_{i'j'}.\]

What follows is easier to describe in the case where, $E$ and $F$ are line bundles, and consequently so is $E \otimes F$. In that case, there is only one $\omega'' = \omega + \omega'$, and consequently, the curvature forms on the tensor becomes
\[\Omega'' = \dl \omega'' = \Omega + \Omega'.\]

Consequently, it is trivial to check that the Chern number of a tensor of line bundles is the sum of the Chern numbers of the factors, i.e.
\[\underline{c_1}(L^{\otimes k} \otimes (L^*)^{\otimes r}) = k \underline{c_1}(L) + r \underline{c_1}(L^*).\]

We already know that $\underline{c_1}(L^*) = -1$. To compute $\underline{c_1}(L)$, let us compute the Chern numbers of duals of line bundles. In this case, if we know the cocycles $g_{12}$ of the bundle $L$, the cocycles of $L^*$ are given by $1/g_{12} = g_{21}$ (note that these are complex numbers in this case). Consequently, we know that the connection forms of $L$ transform like
\[\omega_1 = \omega_2 + g_{21} \dl2 g_{12},\]
therefore it is obvious that if we consider the forms $-\omega_i$, these transform as they should in the dual bundle. In other words, we may induce a connection on the dual by negating the connection forms. This also negates the curvature forms, and thus the Chern number. Hence, we conclude
\[\underline{c_1}(L) = -\underline{c_1}(L^*) = 1.\]

Therefore, the answer to the problem is
\[\underline{c_1}(L^{\otimes k} \otimes (L^*)^{\otimes r}) = k - r.\]

\medskip

f) First, we need to check that $\sigma$ and $\eta$ are smooth. I will only do $\sigma$.

To show smoothness, it suffices to show smoothness in the $z$ and $w$ chart because they cover the manifold. To that effect, it we compute the coordinates of $\sigma$ in the $w$ chart; we already know that they are constant equal to 1 in the $z$ chart.

So, let $\sigma = a s^*_w$. Obviously, $a = 0$ when $w=0$. For $w \neq 0$, we compute $a = \sigma(s_w)$:
\[\sigma(s_w) = \sigma(1,w) = s^*_z(1,\frac1z) = s^*_z(\frac1z s_z) = w.\]

In conclusion, $\sigma = w s^*_w$. Likewise, one can show $\eta = z s^*_z$. This is obviously smooth.

Linear independence is clear from the fact that there exist points in which one section vanishes but not the other and vice-versa.

Finally, we show that they are holomorphic. Again, I will only do $\sigma$, as $\eta$ is basically the same.

First we compute the covariant derivative in $\partial_{\overline{z}}$, as that one is very easy. Indeed, by definition, on $U_z$,
\[\nabla_{\partial_{\overline{z}}} \sigma = \omega_z(\partial_{\overline{z}}) s^*_z = -\frac{\overline{z}}{1+\abs{z}^2} \dl2z(\partial_{\overline{z}}) s^*_z.\]

Now, I don't actually know what the meaning of $\dl z (\partial_{\overline z})$ is, but from context I'm going to assume that $z$ and $\overline{z}$ can be seen as independent coordinates, and so this term is zero and consequently $\nabla_{\partial{\overline z}} \sigma = 0$, at least when $w \neq 0$. Of course, by continuity the same is true for $w = 0$.

Now we compute the derivative in the $\overline{w}$ direction, this time using the definition on $U_w$. Indeed,
\[\nabla_{\partial_{\overline{w}}} \sigma = \nabla_{\partial_{\overline{w}}} (w s^*_w) = \partial_{\overline{w}} w s^*_w + w \nabla_{\partial_{\overline{w}}} s^*_w.\]

The first term is null because $w$ does not have a $\overline{w}$ term, and so its derivative is zero. The second term is null by the same argument as previously. Finally, these calculations were done where $z\neq0$, but again by continuity they hold at $z = 0$.

This concludes the proof of holomorphicity.

\medskip

g) Let $\omega = a \sigma + b \eta$, with $a, b \in \C$. I think we also have to assume that not both are zero.

If $b = 0$, $\omega$ becomes a multiple of $\sigma$, which has only one zero at the only point where $w = 0$.

If $b \neq 0$, then $\omega$ cannot be null at $z=0$ or at $w=0$. Consequently, it suffices to examine $U_z \cap U_w$, in which we know that $s^*_z = w s^*_w$, and therefore
\[\omega = ( a w + b ) s^*_w.\]

This is null exactly when $w = -b/a$, and this is therefore the only zero of $\omega$. Proof complete.
\end{sol}

\begin{ex}

\begin{enumerate}
\item Let
\[\omega_1 = \frac{x \, \dl \conj x - \dl x \, \conj x}{2 (1+\abs x^2)}, \quad \omega_2 = \frac{y \, \dl \conj y - \dl y \, \conj y}{2 (1+\abs y^2)}.\]

Show that these define local connection forms on a complex vector bundle of rank 2, with transition function
\[g_{12}(x) = \frac x {\abs x}.\]

\item Compute the Chern number of $E$.

\item Compute the first Chern class of $E$, and the Chern number of $kE$.
\end{enumerate}
\end{ex}

\begin{sol}
a) Okay, so we check that $\omega_1 = g_{12} \omega_2 g_{21} + g_{12} \dl g_{21}$. To do so, we compute the right-hand side. Note that $g_{21} = \frac{\conj x}{\abs x}$, and $y = \frac1x$. Furthermore, since $xy = 1$, we may compute
\[0 = \dl(xy) = \dl x \, y + x \, \dl y,\]
and therefore $x \, \dl y = - \dl x \, y$. The same formula also holds for the conjugate coordinates, and with the order of both sides swapped.

In preparation for solving this exercise, we compute $g_{12} \dl g_{21}$:
\begin{multline*}
g_{12} \dl g_{21} = \frac x {\abs x} \dl(\conj x / \abs x) = \frac x {\abs x^2} \dl \conj x + \abs x \dl \frac1{\abs x}\\
= \conj y \, \dl \conj x - \frac1{\abs x} \dl \abs x = \conj y \, \dl \conj x - \frac12 \left( \conj y \, \dl \conj x + \dl x \, y \right) = \frac12 (\conj y\,  \dl \conj x - \dl x \, y).
\end{multline*}

The step where I developed $\dl \abs x$ there was kind of quick, but I used the identity $\abs x^2 = x \conj x$, took $\dl$ on both sides, expanded the left-hand side using the derivative of  the square and the right-hand side using derivative of the product. (Note: I only now looked and saw that you have the expresssion for $\dl \abs x$ in the hint. Woops.)

Proceeding with the computations:
\begin{align*}
g_{12} \omega_2 g_21 + g_{12} \dl g_{21} &= \frac x {\abs x} \frac{y \, \dl \conj y - \dl y \, \conj y}{2 (1+\abs y^2)} \frac{\conj x}{\abs x} + g_{12} \dl g_{21}\\
&= \frac1{\abs x^2} \frac{\dl \conj y \, \conj x - x \, \dl y}{2(1+\abs y^2)} + g_{12} \dl g_{21}\\
&= \frac{-\conj y \, \dl \conj x + \dl x \, y}{2(1+\abs x^2)} + g_{12} \dl g_{21}\\
&= \frac{-\conj y \, \dl \conj x + \dl x \, y}{2(1+\abs x^2)} + \frac12 (\conj y\,  \dl \conj x - \dl x \, y),
\end{align*}
and now we can add the fractions and use the fact that $-\conj y + (1+ \abs x^2) \conj y = x \conj x \, \conj y = x$ and likewise for the other term. At the end we clearly obtain $\omega_1$, which completes the proof that the $\omega_i$ are connection forms on $E$.

\medskip

b) Since $U_1$ covers all $S^4$ except one point, it suffices to integrate $c_2(E)$ over $U_1$.

Now, I'll be honest: I have no idea how this stuff works. I'm expecting a $2 \times 2$ matrix of complex-valued forms, but I got a single quaternionic matrix, and I just kind of have no clue how this works. So what you're about to see is me trying to bumble my way through this exercise. Sorry for the atrocities I might have committed.

So, uh, to compute the second Chern class, I need to see $\Omega_1$ as a $2 \times 2$ complex matrix somehow. Well, aside from the $\dl x \wedge \dl \conj x$, we have a real thing, so it makes sense that $\Omega_1$ is (kind of) the matrix
\[\Omega_1 = \begin{bmatrix}
\frac1{(1+r^2)^2}  \dl x \wedge \dl \conj x & 0\\
0 & \frac1{(1+r^2)^2}  \dl x \wedge \dl \conj x
\end{bmatrix}.\]

This matrix has determinant given by
\[- 4 \pi^2 c_2 = \frac1{(1+r^2)^4} \dl x \wedge \dl \conj x \wedge \dl x \wedge \dl \conj x,\]
so this (over $-2\pi$) is the form that I must integrate. Maybe.

Let that last part, $(\dl x \wedge \dl \conj x)^{\wedge 2}$, be a multiple $\alpha$ of the volume form on $U_1$. Then, with the help of Mathematica, I was able to compute the integral of $c_2$ as
\[\underline{c_2} = - \alpha \frac1{4\pi^2} \underbrace{\int_0^\infty \frac{r^3}{(1+r^2)^4} \dl r}_{1/12} \times 2 \times \frac\pi2 \times 2 \pi = -\frac\alpha{24}.\]

Now, I need that $\alpha$ be 24. That's a very suspect number, because it is precisely the cardinality of the group of permutations of four elements. Anyway, I figure I'll try to write $(\dl x \wedge \dl \conj x)^{\wedge 2}$ as a multiple of the unit volume form.

I will begin by writing $x = z + \J w$, where $z, w \in \C$. A remark that will be useful in the sequence: $\J$ does not commute with complex numbers, but rather satisfies the identity $\J w = \conj w \J$. Furthermore, $\conj x = \conj z - \J w$. As such, we may write
\begin{equation}\label{wedges}
\dl x \wedge \dl \conj x = (\dl z + \J \dl w) \wedge (\dl \conj z - \J \dl w) = \dl z \wedge \dl \conj z - 2 \J \, \dl \conj z \wedge \dl w - \dl w \wedge \dl \conj w.
\end{equation}

Now we take the wedge square of this \eqref{wedges}. It is a sum of a total of nine terms, but fortunately complex wedges are a lot nicer than quaternionic nonsense. In particular, while this isn't true for quaternions, for complex numbers a wedge product that contains the same term twice is immediately null. As a consequence, it is easy to show that out of the nine terms that the square of \eqref{wedges} only three of them are relevant:
\begin{align*}
\eqref{wedges}^{\wedge 2} &= - \dl z \wedge \dl \conj z \wedge \dl w \wedge \dl \conj w + (-2\J \, \dl \conj z \wedge \dl w)^{\wedge 2} - \dl w \wedge \dl \conj w \wedge \dl z \wedge \dl \conj z\\
&= - 2 \dl z \wedge \dl \conj z \wedge \dl w \wedge \dl \conj w + 4 \J \, \dl \conj z \wedge \dl w \wedge \J \, \dl \conj z \wedge \dl w\\
&= - 2 \dl z \wedge \dl \conj z \wedge \dl w \wedge \dl \conj w + 4 \J^2 \, \dl z \wedge \dl \conj w \wedge \dl \conj z \wedge \dl w\\
&= - 2 \dl z \wedge \dl \conj z \wedge \dl w \wedge \dl \conj w - 4 \dl z \wedge \dl \conj z \wedge \dl w \wedge \dl \conj w\\
&= - 6 \dl z \wedge \dl \conj z \wedge \dl w \wedge \dl \conj w.
\end{align*}

Okay, now. Let $z = z_r + \I z_i$ be a complex coordinate. It is easy to check (I think I did it in another exercise) that
\[\dl z \wedge \dl \conj z = - 2 \I  \dl2 z_r \wedge \dl2 z_i,\]
so that, all in all,
\[\eqref{wedges}^{\wedge 2} = - 6 \times 2\I \times 2\I \dl2 z_r \wedge \dl z_i \wedge \dl w_r \wedge \dl w_i,\]
which is precisely $\alpha$ times the volume form, where $\alpha = 24$. Hooray!

Sooo, yeah. The end result matches with what I wanted it to be. Umm. Yeah.

\medskip

c) Okay, well, if my matrix for $\Omega$ above is correct, we should have that
\[c_1(E) = \frac{2 \dl2 x \wedge \dl \conj x}{(1+r^2)^2}.\]

And, uh, as for $\underline{c_2}(k E)$, we begin by noting that $c(kE) = c(E)^{\wedge k}$. As a consequence, since $c(E) = 1 + c_1(E) + c_2(E)$, we obtain
\begin{align*}
c_2(kE) &= k c_2(E) + \binom k 2  c_1(E) \wedge c_1(E)\\
&= k c_2(E) + \binom k 2 \times 4 \, c_2(E)\\
&= (k + 2 k (k-1)) c_2(E) = (2k^2 - k) c_2(E),
\end{align*}
where I used the (easy to conclude (if my math is correct...)) identity
\[c_1(E) \wedge c_1(E) = 4 c_2(E).\]

So now we take the integral, and fortunately we already know the integral of $c_2(E)$, so we get
\[\underline{c_2}(kE) = (2k^2 - k) \underline{c_2}(E) = k - 2k^2.\]

And... Yeah, I think that's the full thing. Thanks for being a cool teacher, doing differential geometry was pretty nice.
\end{sol}

\end{document}