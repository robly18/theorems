\documentclass{article}

\usepackage{amsmath}
\usepackage{amssymb}
\usepackage{amsfonts}
\usepackage{mathtools}

\usepackage[thmmarks, amsmath]{ntheorem}

\usepackage{graphicx}
\usepackage{float}
\usepackage{tikz-cd}
\usepackage{adjustbox}

\usepackage{diffcoeff}
\diffdef{}{op-symbol=\mathrm{d},op-order-sep=0mu}

\usepackage{cancel}
\usepackage{interval}

\usepackage{array}

\usepackage{enumitem}

\setlist[enumerate,1]{label=(\alph*)}

\title{Algebraic Topology Homework 6}
\author{Duarte Maia}
%\date{}

\theoremstyle{plain}
\theorembodyfont{\upshape}
\theoremseparator{.}
\newtheorem{theorem}{Theorem}
\newtheorem{prop}{Prop}
\renewtheorem*{prop*}{Prop}
\newtheorem{lemma}{Lemma}
\newtheorem*{ex}{Exercise}

\theoremstyle{nonumberplain}
\theoremheaderfont{\itshape}
\theorembodyfont{\upshape}
\theoremseparator{:}
\theoremsymbol{\ensuremath{\blacksquare}}
\newtheorem{proof}{Proof}
\newtheorem{sol}{Solution}

\theoremsymbol{\text{\textit{(End proof of lemma)}}}
\newtheorem{lemmaproof}{Proof of lemma}

\newcommand{\R}{\mathbb{R}}
\newcommand{\C}{\mathbb{C}}
\newcommand{\Z}{\mathbb{Z}}
\newcommand{\Q}{\mathbb{Q}}

\newcommand{\RP}{\mathbb{RP}}

\newcommand{\kk}{\Bbbk}

\newcommand{\PP}{\mathbb{P}}
\newcommand{\FF}{\mathcal{F}}

\newcommand{\I}{\mathrm{i}}
\newcommand{\e}{\mathrm{e}}

\newcommand{\id}{\mathrm{id}}
\newcommand{\GL}{\mathrm{GL}}

\newcommand{\conj}[1]{\overline{#1}}
\newcommand{\close}[1]{\overline{#1}}

\DeclareMathOperator{\interior}{int}
\DeclareMathOperator*{\colim}{colim}
\DeclareMathOperator{\codim}{codim}
\DeclareMathOperator{\trace}{tr}
\newcommand{\grad}{\nabla}


\DeclareMathOperator{\Ext}{Ext}
\DeclareMathOperator{\Hom}{Hom}

\DeclarePairedDelimiter{\abs}{\lvert}{\rvert}
\DeclarePairedDelimiter{\norm}{\lvert}{\rvert}
\DeclarePairedDelimiter{\Norm}{\lVert}{\rVert}
\DeclarePairedDelimiter{\braket}{\langle}{\rangle}


\begin{document}
\maketitle

\begin{ex}[3.2:4]
Show that every map $f \colon \C P^n \to \C P^n$ has a fixed point if $n$ is even, and show that this also holds if $n$ is odd unless $f^*(\alpha) = -\alpha$ for $\alpha$ a generator of $H^2(\C P^n, \Z)$.
\end{ex}

\begin{sol}
The first part of the proof follows whether $n$ is even or odd.

Let $f^*(\alpha) = d \alpha$ for $d \in \Z$. Moreover, recall that (this is in Hatcher) the cohomology ring of $\C P^n$ is given by $\Z[\alpha]/\alpha^{n+1}$ and that all homologies of $\C P^n$ are free (indeed, $\Z$ in even dimensions up to $2n$), hence $h \colon H^* \to (H_*)^*$ is an isomorphism.

Now, we know then that $f^*(\alpha^k) = (f^*(\alpha))^k = d^k \alpha^k$. Moreover, using the naturality of $h$, we get that the map $f_* \colon H_k \to H_k$ (with $k = 0, \dots, 2n$) is given by multiplication by $d^k$. As such, the trace of $f_* \colon H_k \to H_k$ is itself $d^k$, and so the Lefschetz constant of $f$ is given by
\begin{equation}
\tau(f) = 1 + d + \dots + d^n.
\end{equation}

Now, the Lefschetz fixed point theorem guarantees that $f$ will have a fixed point so long as $\tau(f)$ is nonzero. If $d = 1$ this is obvious, and likewise for all $d \neq {-1,1}$ (as $d$ will not divide $\tau(f)$ but $d \mid 0$). The only remaining case is that of $d = -1$, and this is where the evenness of $n$ matters, as if $n$ is even $\tau(f)$ will be a telescopic sum totaling $\tau(f) = 1$, and only if $n$ is odd may $\tau(f)$ be zero after all, and so may $f$ have a fixed point. And again, this only happens if $f^*(\alpha) = -\alpha$, as desired.
\end{sol}

\begin{ex}[3.2:14]
I'm not doing this one. Sorry.
\end{ex}

\begin{sol}
N/A
\end{sol}

\begin{ex}[3.2:15]
Show that $p(X \times Y) = p(X) p(Y)$. Compute the Poincaré series for [a bunch of spaces].
\end{ex}

\begin{sol}
There is a missing hypothesis in the theorem, namely that the cohomologies of $X$ and $Y$ are finite-dimensional in all degrees, otherwise their Poincaré series will not make sense.

That said, since we're over a field, the cohomologies are all free, so the Künneth formula applies, and hence
\begin{equation}
H^i(X \times Y) \cong \bigoplus_{a+b=i} H^a(X) \otimes H^b(Y),
\end{equation}
whose dimension is precisely $\sum_{a+b=i} (\dim H^a(X))(\dim H^b(Y))$, which is exactly the $i$-th coefficient of $p(X) p(Y)$.

\medskip

I'm not going to compute the Poincaré series for eight different spaces. I mean, for $S^n$ it is trivially $1+x^n$, and for the projective spaces it will be a nice geometric sum (in finite dimension) or geometric series (in infinite dimension). But I'm not doing the details.
\end{sol}

\begin{ex}[3.3:3]
Show that every covering space of an orientable manifold is itself orientable.
\end{ex}

\begin{sol}
We do so by inducing an orientation on $\tilde M$ using an orientation in $M$.

Let $\tilde x \in \tilde M$, $x = p(\tilde x) \in M$. The generator of $H_n(M \mid x)$ induces a canonical generator of $H_n(U \mid x)$ by excision, where $U$ is a small enough evenly covered neighborhood of $x$. Take the preimage of $U$, and let $\tilde U$ be a subspace of $p^{-1}(U)$ which contains $\tilde x$ and such that $p|_{\tilde U}$ is a homeomorphism onto $U$. Then, $p$ induces an isomorphism $H_n(U \mid x) \cong H_n(\tilde U \mid \tilde x)$. Again, by excision, the latter is canonically isomorphic to $H_n(\tilde M \mid \tilde x)$. Thus, we have a canonical isomorphism (induced by $p$) between $H_n(\tilde M \mid \tilde x)$ and $H_n(M \mid x)$, and so picking a generator of the latter induces a natural choice of generator on the former.

Naturality properties and the fact that $p$ is a covering space will show that these choices of local orientations on $\tilde M$ are coherent and so induce an orientation on $\tilde M$.
\end{sol}

\begin{ex}[3.3:7]
Show that for any connected closed orientable $n$-manifold $M$ there is a map $f \colon M \to S^n$ such that $f_*([M]) = [S^n]$.
\end{ex}

\begin{sol}
Pick a triangulation of $M$. This triangulation corresponds to the fundamental class of $M$. Now, consider the map $f \colon M \to S^n$ which consists of collapsing all triangles except one. This takes $[M]$ to an $n$-cycle in $S^n$ such that all but one of its simplices are constant equal to some point $p$.

We claim that any such cycle is a generator of $H_n(S^n)$. This can be seen as follows. First, we know that (for $n \neq 0$, but that case is false anyway) $H_n(S^n) \cong H_n(S^n, p)$. Now, in $H_n(S^n, p)$, all the constant simplices equal to $p$ vanish in the quotient, so $f_*([M])$ is equal in homology to (the equivalence class of) the cycle with only one simplex which covers all of $S^n$. But we know that this cycle is a generator of $H_n(S^n)$, and so we have shown that our $f$ takes $[M]$ to $[S^n]$.
\end{sol}

\begin{ex}[3.3:22]
Show that $H^n_c(X \times \R; G) \cong H^{n-1}_c(X;G)$.
\end{ex}

\begin{sol}
As in example 3.34 in Hatcher, to compute compact cohomology it suffices to take the colimit as the compacts range over a directed compact cover of $X$. As such, consider the directed compact cover of $X \times \R$ given by sets of the form $K \times [-a,a]$ with $K$ compact and $a \geq 0$. Thus, we have
\begin{equation}
H^n_c(X \times \R) \cong \colim H^n(X \times \R \mid K \times [-a,a]).
\end{equation}

Moreover, we know from category theory that `colimits in two variables can be separated', so we have
\begin{multline}\label{eq:eq1}
\colim_{K \times [-a,a]} H^n(X \times \R \mid K \times [-a,a]) = \\
= \colim_K \left( \colim_{[-a,a]} H^n(X \times \R \mid K \times [-a,a]) \right).
\end{multline}

Now, we claim that $H^n(X \times \R \mid K \times [-a,a])$ is naturally isomorphic to $H^{n-1}(X \mid K)$, so \eqref{eq:eq1} simplifies to
\begin{equation}
\colim_K H^{n-1}(X \mid K) =: H^{n-1}_c(X).
\end{equation}

To prove our claim, we use the relative Künneth formula. We may do this because the relative cohomology ring $H^*(\R \mid [-a,a])$ is simply $\braket{\alpha \mid \alpha^2 = 0}$ with $\abs{\alpha} = 1$ (this can be seen by the universal coefficients theorem or by excision) so in particular all its cohomologies are free of finite rank.  With it, we get
\begin{equation}
H^*(X \times \R \mid K \times [-a,a]) = H^*(X \mid K) \otimes H^*(\R \mid [-a,a]).
\end{equation}

Now, since $H^*(\R \mid [-a,a])$ is null except on dimension 1, where it is $G$, we obtain by definition of graded tensor
\begin{equation}
H^*(X \times \R \mid K \times [-a,a]) = H^{*-1}(X \mid K),
\end{equation}
which completes the proof.
\end{sol}

\begin{ex}[3.3:25]
Show that if $M$ is a closed orientable manifold of dimension $2k$, if $H_{k-1}(M;\Z)$ is torsionfree then so is $H_k(M;\Z)$.
\end{ex}

\begin{sol}
Poincaré Duality gives $H_{k-1} = H^{k+1}$. Therefore, $H^{k+1} = \Ext(H_k, \Z) \oplus H_{k+1}^*$ is torsionfree, and thus $\Ext(H_k,\Z)$ is torsionfree. But since compact manifolds have finitely generated homologies, $H_k$ is a finitely generated abelian group, and by their classification $\Ext(H_k,\Z) = \mathrm{Tor}(H_k)$, which, by definition, is torsionfree iff it is zero, iff $H_k$ itself is torsionfree.
\end{sol}

\begin{ex}[3.3:26]
Compute the cup product structure of $H^*(S^2 \times S^8 \sharp S^4 \times S^6;\Z)$.
\end{ex}

\begin{sol}
From exercise 6a), the Künneth formula, and the universal coefficients theorem, we are able to compute that the cohomology of this space (which we will call $X$) is $\Z$ in dimensions $0, 2, 4, 6, 8, 10$, but I don't know how to compute the cup product structure. I will say what few observations I can make though.

We know that the generator of the zeroth cohomology is the identity wrt the cup product; this is a general fact when $H^0 = \Z$. Moreover, we know that the cup product is null when the dimensions add up to more than 10. Finally, if we let $g_k$ be the generator of the $k$-th cohomology, I conjecture that $g_2 \cup g_8 = g_4 \cup g_6 = g_10$, and that all other cup products are null.
\end{sol}

\end{document}