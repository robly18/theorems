\documentclass{article}

\usepackage{amsmath}
\usepackage{amssymb}
\usepackage{amsfonts,stmaryrd}
\usepackage{mathtools}

\usepackage[thmmarks, amsmath]{ntheorem}
\usepackage{fullpage}

\usepackage{graphicx}

\usepackage{diffcoeff}
\diffdef{}{op-symbol=\mathrm{d},op-order-sep=0mu}

\usepackage{cancel}
\usepackage{interval}

\usepackage{enumitem}

\setlist[enumerate,1]{label=(\alph*)}

\title{Algebra Homework 5}
\author{Duarte Maia}
%\date{}

\theorembodyfont{\upshape}
\theoremseparator{.}
\newtheorem{theorem}{Theorem}
\newtheorem{prop}{Prop}
\renewtheorem*{prop*}{Prop}
\newtheorem{lemma}{Lemma}

\newtheorem{ex}{Exercise}

\theoremstyle{nonumberplain}
\theoremheaderfont{\itshape}
\theorembodyfont{\upshape}
\theoremseparator{:}
\theoremsymbol{\ensuremath{\blacksquare}}
\newtheorem{proof}{Proof}
\newtheorem{sol}{Solution}

\newcommand{\R}{\mathbb{R}}
\newcommand{\C}{\mathbb{C}}
\newcommand{\Z}{\mathbb{Z}}
\newcommand{\N}{\mathbb{N}}
\newcommand{\Q}{\mathbb{Q}}
\newcommand{\K}{\mathbb{K}}

\newcommand{\kk}{\Bbbk}


\newcommand{\gp}{\mathfrak{p}}
\newcommand{\gq}{\mathfrak{q}}

\newcommand{\PP}{\mathbb{P}}
\newcommand{\Gr}{\mathrm{Gr}}
\newcommand{\GG}{\mathbb{G}}

\newcommand{\I}{\mathrm{i}}
\newcommand{\e}{\mathrm{e}}
\newcommand{\id}{\mathrm{id}}

\newcommand{\conj}[1]{\overline{#1}}

\newcommand{\grad}{\nabla}

\DeclareMathOperator{\sign}{sign}
\DeclareMathOperator{\image}{im}
\DeclareMathOperator{\ord}{ord}
\let\radical\relax
\DeclareMathOperator{\radical}{rad}
\DeclareMathOperator{\Ann}{Ann}
\DeclareMathOperator{\Frac}{Frac}
\DeclareMathOperator{\coker}{coker}

\newcommand{\Aff}{\mathbb{A}}

\newcommand{\HH}{\mathcal{H}}
\newcommand{\bbH}{\mathbb{H}}

\let\Im\relax
\DeclareMathOperator{\Im}{Im}
\let\Re\relax
\DeclareMathOperator{\Re}{Re}

\DeclarePairedDelimiter{\abs}{\lvert}{\rvert}
\DeclarePairedDelimiter{\norm}{\lvert}{\rvert}
\DeclarePairedDelimiter{\Norm}{\lVert}{\rVert}
\DeclarePairedDelimiter{\braket}{\langle}{\rangle}


\begin{document}
\maketitle

\begin{ex}
Show that a UFD $A$ is integrally closed in $\Frac(A)$.
\end{ex}

\begin{sol}
Given $\frac pq \in \Frac(A)$ an integral element, choose $p$ and $q$ coprime (this uses $A$ UFD). Then, by definition of integral, it must satisfy a monic equation
\begin{equation}
\frac{p^n}{q^n} + a_{n-1} \frac{p^{n-1}}{q^{n-1}} + \dots + a_0 = 0,
\end{equation}
and by clearing denominators we get
\begin{equation}
p^n = - q (a_{n-1} + \dots + a_0 q^{n-1}),
\end{equation}
hence $p^n$ is a multiple of $q$. This is impossible by coprimality unless $q = 1$, and thus $\frac pq$ was in $A$ all along.
\end{sol}

\begin{ex}
Show that the characteristic polynomial of $\beta \in L/K$ is a power of the minimal polynomial of $\beta$.
\end{ex}

\begin{sol}
Pick a basis $v_1, \dots, v_k$ of $L/K(\beta)$. Note that $k = [L:K(\beta)]$. Then, we note that $L = \bigoplus_{i = 1}^k v_i K(\beta)$, and so the linear transformation $\beta \colon L \to L$ may be represented as a `diagonal block matrix', being given by $\beta \colon K(\beta) \to K(\beta)$ on each of the $k$ components. Thus, $p_{\beta,L} = p_{\beta,K(\beta)}^k$, so it suffices to note that the characteristic polynomial of $\beta$ over $K(\beta)$ is precisely its minimal polynomial.
\end{sol}

\begin{ex}
Let $Y \subseteq \Aff^6$ be the set of matrices of rank $\leq 1$.
\begin{enumerate}
\item Show that $Y$ is algebraic.
\item Show that $Y$ is irreducible.
\item Compute the dimension of $Y$.
\end{enumerate}
\end{ex}

\begin{sol}
\leavevmode
\begin{enumerate}
\item An arbitrary $3 \times 2$ matrix has rank $\leq 1$ iff its three $2 \times 2$ minors have null determinant. Thus, $Y$ is the zero locus of an ideal generated by three polynomials.
\item Using the hint, we note that there is a surjection $\Aff^5 \to Y$, given by
\begin{equation}
(a,b,x,y,z) \mapsto \begin{bmatrix}
a \\ b \\ c
\end{bmatrix}
\begin{bmatrix}
x & y
\end{bmatrix}.
\end{equation}
\item We note that the surjection above is injective when restricted to the set $X \subseteq \Aff^5$ where $x = 1$ and $a \neq 0$. Moreover, its image is an open subset of $Y$, namely the open subset of matrices whose top left element is nonzero. Thus, we have a regular bijection between $X$ and a nonempty open subset of $Y$, and since regular bijections preserve dimension, as well as taking nonempty open subsets, we conclude that the dimension of $Y$ is the same as the dimension of $X$. Now, $X$ is obtained from $\Aff^5$ by removing the zero locus of one polynomial and intersecting with another. The first operation does not change the dimension, and the second reduces it by one, and so we conclude
\begin{equation}
\dim Y = \dim X = 4.
\end{equation}
\end{enumerate}
\end{sol}

\begin{ex}
\end{ex}

\begin{sol}
\end{sol}

\end{document}