\documentclass{article}

\usepackage{amsmath}
\usepackage{amssymb}
\usepackage{amsfonts}
\usepackage{mathtools}

\usepackage[thmmarks, amsmath]{ntheorem}

\usepackage{graphicx}

\usepackage{diffcoeff}
\diffdef{}{op-symbol=\mathrm{d},op-order-sep=0mu}

\usepackage{cancel}
\usepackage{interval}

\usepackage{enumitem}

\setlist[enumerate,1]{label=(\alph*)}

\title{Algebra Homework 1}
\author{Duarte Maia}
%\date{}

\theorembodyfont{\upshape}
\theoremseparator{.}
\newtheorem{theorem}{Theorem}
\newtheorem{prop}{Prop}
\renewtheorem*{prop*}{Prop}
\newtheorem{lemma}{Lemma}

\newtheorem{ex}{Exercise}

\theoremstyle{nonumberplain}
\theoremheaderfont{\itshape}
\theorembodyfont{\upshape}
\theoremseparator{:}
\theoremsymbol{\ensuremath{\blacksquare}}
\newtheorem{proof}{Proof}
\newtheorem{sol}{Solution}

\newcommand{\R}{\mathbb{R}}
\newcommand{\C}{\mathbb{C}}
\newcommand{\Z}{\mathbb{Z}}
\newcommand{\N}{\mathbb{N}}
\newcommand{\Q}{\mathbb{Q}}
\newcommand{\K}{\mathbb{K}}

\newcommand{\kk}{\Bbbk}

\newcommand{\PP}{\mathbb{P}}
\newcommand{\Gr}{\mathrm{Gr}}

\newcommand{\I}{\mathrm{i}}
\newcommand{\e}{\mathrm{e}}
\newcommand{\id}{\mathrm{id}}

\newcommand{\conj}[1]{\overline{#1}}

\newcommand{\grad}{\nabla}

\DeclareMathOperator{\sign}{sign}
\DeclareMathOperator{\image}{im}
\DeclareMathOperator{\ord}{ord}

\DeclareMathOperator{\Aff}{Aff}

\newcommand{\HH}{\mathcal{H}}
\newcommand{\bbH}{\mathbb{H}}

\let\Im\relax
\DeclareMathOperator{\Im}{Im}
\let\Re\relax
\DeclareMathOperator{\Re}{Re}

\DeclarePairedDelimiter{\abs}{\lvert}{\rvert}
\DeclarePairedDelimiter{\norm}{\lvert}{\rvert}
\DeclarePairedDelimiter{\Norm}{\lVert}{\rVert}
\DeclarePairedDelimiter{\braket}{\langle}{\rangle}


\begin{document}
\maketitle

\begin{ex}
Show that if $R$ is a noetherian ring, so is $R[[x]]$. Conclude that the power series ring in $n$ variables over a field is noetherian.
\end{ex}

\begin{sol}
In the following, if $p \in R[[x]]$ is nonzero, we define $L(p)$ as the leading coefficient of $p$, that is, the coefficient of the lowest-degree term of $p$.

Suppose that $R$ is noetherian, and let $I$ be an ideal of $R[[x]]$. Then, consider the following procedure. First, pick an element of $I$ of minimal order, call it $p_0$. Then, pick an element $p_1$ of $I$ of minimal order such that $L(p_1)$ is not a multiple of $L(p_0)$. Then, pick an element $p_2$ of $I$ of minimal order such that $L(p_2)$ is not in $\braket{L(p_0),L(p_1)}$, and so on until continuing is impossible. Since $R$ is noetherian, it satisfies the ascending chain condition, and so this algorithm must halt after some finite number of steps $N$. We will then show that $p_0, \dots, p_N$ generate $I$.

To this effect, we prove the following lemma. If $p \in I$ is nonzero, there exist monomials $q_i$, $i = 0, \dots, N$, which are either zero or have degree $\ord p - \ord p_i$ (if this is negative then they are forcibly zero), such that
\begin{equation}
\ord(p - \sum q_i p_i) > \ord(p),
\end{equation}
where $\ord(0)$ is zero by convention.

Once the lemma is shown, here is how the proof that $I = \braket{p_0, \dots, p_N}$ goes. Pick $p \in I$. Build $q_i^(0)$ as in the lemma, and obtain some $p^(1)$ of higher order than $p$. Repeat, obtaining $q_i^(1)$, and so on, until either infinity or $p^(n) = 0$. Then, set $Q_i = \sum_{n=0}^\infty q_i^(n)$; note that this is well-defined because all $q_i^(n)$ have different degrees as $n$ increases. Then, it is possible to verify that $p - \sum Q_i p_i = 0$, for example by looking at the image of this expression in $R[[x]]/\braket{x^r}$ for arbitrarily large $r \in \N$.\footnote{A slightly more complete sketch: The sum $\sum Q_i p_i$ has only finitely many nonzero terms in the quotient, and $p$ minus this sum is in the quotient equal to some $p^(n)$, and these are eventually zero in the quotient because their order grows to infinity.}

So, let us now show the lemma. There are two cases which are very similar.

Suppose first that the order of $p$ is greater than or equal to the order of all $p_i$. Then, $L(p) \in \braket{L(p_1), \dots, L(p_N)}$, as otherwise  the algorithm used to construct the $p_i$ could have continued for one more step. Thus, if $L(p) = \sum r_i L(p_i)$, we simply set $q_i = r_i x^{\ord(p) - \ord(p_i)}$.

Now, supose that the order of $p$ is less than the order of some $f_j$; let $j$ be minimal without loss of generality. Then, $L(p) \in \braket{L(p_1), \dots, L(p_{j-1})}$, as otherwise we obtain contradiction with the minimality of the order of $f_j$. Thus, we may again write $p = \sum_{i = 0}^{j-1} r_i L(p_i)$, for $i < j$ we set $q_i = r_i x^{\ord(p) - \ord(p_i)}$, and for $i \geq j$ we set $q_i = 0$.

Anyway, in both of these cases the $q_i$ are built precisely as to cancel out the term of degree $\ord p$ in $p$, so the sum $p - \sum q_i p_i$ has order at least $\ord p + 1$, which completes the proof of the lemma and hence the solution of the exercise.
\end{sol}

\begin{ex}
\begin{enumerate}
\item Show that the map $\Gr(2,4) \to \PP^5$ is injective.
\item Show that the image of this map consists of those $\omega \in \bigwedge^2 V \setminus \{0\}$ such that $\omega \wedge \omega = 0$.
\end{enumerate}
\end{ex}

\begin{sol}
\begin{enumerate}
\item Suppose that $W_0$ and $W_1$ are two spaces such that $W_0 \wedge W_0 = W_1 \wedge W_1$. We show that these two spaces are the same.

To this effect, let $w,w'$ be a basis of $W_0$, extended to a basis of $V$ by $v,v'$. Then, we know from multilinear algebra that $W_0$ has $w \wedge w'$ as sole generator, and that the following six vectors form a basis of $\bigwedge^2 V$:
\begin{equation}\label{eq:b}
w \wedge w', w \wedge v, w \wedge v', w' \wedge v, w' \wedge v', v \wedge v'.
\end{equation}

Now, let us look at the three cases for the possible intersection $W_0 \cap W_1$. If this intersection is two-dimensional, then $W_0 = W_1$, as desired. If this intersection is zero-dimensional, then we may as well pick $v, v'$ as a basis of $W_1$, and so $W_0 \wedge W_0$ and $W_1 \wedge W_1$ are spanned by distinct basis vectors, hence cannot be equal. Finally, if this intersection is one-dimensional, a similar argument goes if we pick $w$ as a generator of $W_0 \cap W_1$, and $v$ such that $w,v$ is a basis of $W_1$. Thus, the only admissible case is the one where $W_0 = W_1$, as desired.
\item The arguments used in the previous question show that $\bigwedge^2 W$ is always generated by an element of the form $\omega = w \wedge w'$, $w, w' \in V$. Such an element evidently satisfies
\begin{equation}
\omega \wedge \omega = w \wedge w' \wedge w \wedge w' = - w \wedge w \wedge w' \wedge w' = 0.
\end{equation}

On the other hand, suppose that $\omega \neq 0$ satisfies $\omega \wedge \omega = 0$. We pick a basis $w,w',v,v'$ of $V$, and build a basis of $\bigwedge^2 V$ as in \eqref{eq:b}. Then, if we write $\omega$ in terms of this basis
\begin{equation}
\omega = a_0 w \wedge w' + a_1 w \wedge v + a_2 w \wedge v' + a_3 w' \wedge v + a_4 w' \wedge v' + a_5 v \wedge v',
\end{equation}
we may readily calculate
\begin{equation}
\omega \wedge \omega = (a_0 a_5 - a_1 a_4 + a_2 a_3) (w \wedge w' \wedge v \wedge v'),
\end{equation}
which is null iff $a_0 a_5 - a_1 a_4 + a_2 a_3 = 0$.

Now, we show that $\omega$ is of the form $u \wedge u'$ for $u, u' \in V$ by \textit{ansatz}. The way I reached this \textit{ansatz} is laborious to write, and I don't think I'm getting graded on it, so I'm not bothering to write it down. Suppose without loss of generality (by reordering basis elements if necessary) that $a_5 \neq 0$ (this is possible because $\omega \neq 0$). Then, we may as well assume by rescaling that $a_5 = 1$, and we set
\begin{equation}
\begin{gathered}
u = a_2 w + a_4 w' + v,\\
u' = -a_1 w - a_3 w' + v'.
\end{gathered}
\end{equation}

If we do the math, we obtain
\begin{equation}
u \wedge u' = (- a_2 a_3 + a_4 a_1) w \wedge w' + a_1\,w \wedge v + a_2\,w \wedge v' + a_3\, w' \wedge v + a_4\, w' \wedge v' + v \wedge v'.
\end{equation}

Finally, the hypothesis that $\omega \wedge \omega = 0$ gives us that $a_1 a_4 - a_2 a_3 = a_0$ and thus $u \wedge u' = \omega$, as desired.

In conclusion, the elements $\omega$ satisfying $\omega \wedge \omega = 0$ are always of the form $u \wedge u'$ and vice-versa, so the image of the map is precisely the given curve.
\end{enumerate}
\end{sol}

\begin{ex}
Let $\kk$ be an algebraically closed field of characteristic different from $2$.
\begin{enumerate}
\item Produce a bijection between $\PP^1$ and $C$, which is the zero set of $X^2 + Y^2 + Z^2$ in $\PP^2$.
\item Produce a bijection between $\PP^1 \times \PP^1$ and $Q$, the zero set of $XY - ZW$ in $\PP^3$.
\end{enumerate}
\end{ex}

\begin{sol}
\begin{enumerate}
\item Consider the map $f \colon \PP^1 \to C$ given by
\begin{equation}
[X:Y] \mapsto [X^2 - Y^2 : \I(X^2 + Y^2) : 2XY].
\end{equation}

There are a few things worth mentioning. First of all, the symbol $\I$ stands for either root of the polynomial $X^2 + 1$ in $\kk$. Moreover, it is necessary to check that this map is well-defined. It is evidently homogeneous, but it remains to verify that if $(X,Y) \neq 0$ then $(X^2 - Y^2 : \I(X^2 + Y^2) : 2XY) \neq 0$. We will do this now.

Suppose that $X^2 - Y^2 = 0$. Then, $Y = \pm X$. Then, $\I (X^2 + Y^2) = 2 \I X^2$, which is a product of nonzero elements of $\kk$; this is where the fact that the characteristic of $\kk$ is not $2$ comes in.

Now, it is necessary to check that $f$ is indeed a bijection between $\PP^1$ and $C$. The verification that its image is contained in $C$ is a trivial computation, and is omitted. Thus, we verify injectivity and surjectivity.

Injectivity: Suppose that $f([X:Y]) = [A:B:C]$.\footnote{Actually, the notation here is inaccurate. What I mean is: suppose that $X^2 - Y^2 = A$, etc.} Then, note that we can recover $X$ and $Y$ up to sign, as
\begin{equation}\label{eq:xy}
X^2 = \frac{A-\I B}2, \quad Y^2 = \frac{A + \I B}2.
\end{equation}

Moreover, if a choice for $X$ is made, the choice for $Y$ is determined because $C = XY$. Thus, the pair $(X,Y)$ is determined up to sign, which is enough to define a unique element of $\PP^1$. There are some edge cases to consider when $C = 0$, but these are easily done on the side, as the only such cases are $[1:\pm \I:0]$, and from \eqref{eq:xy} we still obtain well-defined $[X:Y]$.

Surjectivity: The above argument actually gives us a way to construct $[X:Y]$ such that $f([X:Y]) = [A:B:C]$ for a chosen $[A:B:C] \in C$.\footnote{Yes, I realize I'm using $C$ to mean different things in the same expression. I apologize. But the meaning of my symbols is clear and $[A:B:D]$ just looks weird.} The hypothesis that the right-hand side is in $C$ ensures that the $X$ and $Y$ defined up to sign by \eqref{eq:xy} satisfy $X^2 Y^2 = C^2$, and thus given a choice of sign of $X$ there always exists a choice of sign of $Y$ such that $XY = C$.

\item Define the map $g \colon \PP^1 \times \PP^1 \to Q$ by
\begin{equation}
g([X:Y],[Z:W]) = [XZ:YW:YZ:XW].
\end{equation}

The verification that this map is well-defined goes as follows. First, it is trivial to verify that the right-hand side is homogeneous. Moreover, we verify that the right-hand side is never $[0:0:0:0]$ (i.e. it always makes sense). Indeed, suppose that $X, Z \neq 0$ (by symmetry, all the other three cases $Y, Z \neq 0$, etc. are analogous), then we get $XZ \neq 0$.

Finally, it is trivial to verify that the image of $g$ is contained in $Q$. So now, we show injectivity and surjectivity.

Suppose that $g([X:Y], [Z:W]) = [A:B:C:D]$. We show that we can recover $[X:Y]$ and $[Z:W]$ without ambiguity. Suppose without loss of generality that $D \neq 0$, and hence also without loss of generality $D = 1$; the cases $A \neq 0$ etc are similar by symmetry. Then, $XW = 1$, hence neither $X$ nor $W$ are null. By rescaling, we may assume $X = W = 1$. Thus, we conclude $Y = B$ and $Z = A$, thus determining $[X:Y]$ and $[Z:W]$. This proves injectivity.

On the other hand, suppose that $[A:B:C:D] \in Q$. We show that this is in the image of $g$. Again, suppose that $D = 1$ without loss of generality. Then,
\begin{equation}
g([1:B],[A:1]) = [A:B:AB:1],
\end{equation}
and since $AB = CD$ and $D = 1$ we obtain $AB = C$, and so
\begin{equation}
g([1:B],[A:1]) = [A:B:C:D],
\end{equation}
as desired. This completes the proof of surjectivity.
\end{enumerate}
\end{sol}

\end{document}