\documentclass{article}

\usepackage{amsmath}
\usepackage{amssymb}
\usepackage{amsfonts,stmaryrd}
\usepackage{mathtools}

\usepackage[thmmarks, amsmath]{ntheorem}
\usepackage{fullpage}

\usepackage{graphicx}

\usepackage{diffcoeff}
\diffdef{}{op-symbol=\mathrm{d},op-order-sep=0mu}

\usepackage{cancel}
\usepackage{interval}

\usepackage{enumitem}

\setlist[enumerate,1]{label=(\alph*)}

\title{Algebra Homework 6}
\author{Duarte Maia}
%\date{}

\theorembodyfont{\upshape}
\theoremseparator{.}
\newtheorem{theorem}{Theorem}
\newtheorem{prop}{Prop}
\renewtheorem*{prop*}{Prop}
\newtheorem{lemma}{Lemma}

\newtheorem{ex}{Exercise}

\theoremstyle{nonumberplain}
\theoremheaderfont{\itshape}
\theorembodyfont{\upshape}
\theoremseparator{:}
\theoremsymbol{\ensuremath{\blacksquare}}
\newtheorem{proof}{Proof}
\newtheorem{sol}{Solution}

\newcommand{\R}{\mathbb{R}}
\newcommand{\C}{\mathbb{C}}
\newcommand{\Z}{\mathbb{Z}}
\newcommand{\N}{\mathbb{N}}
\newcommand{\Q}{\mathbb{Q}}
\newcommand{\K}{\mathbb{K}}

\newcommand{\kk}{\Bbbk}


\newcommand{\gp}{\mathfrak{p}}
\newcommand{\gq}{\mathfrak{q}}

\newcommand{\PP}{\mathbb{P}}
\newcommand{\Gr}{\mathrm{Gr}}
\newcommand{\GG}{\mathbb{G}}

\newcommand{\I}{\mathrm{i}}
\newcommand{\e}{\mathrm{e}}
\newcommand{\id}{\mathrm{id}}

\newcommand{\conj}[1]{\overline{#1}}

\newcommand{\grad}{\nabla}

\DeclareMathOperator{\sign}{sign}
\DeclareMathOperator{\image}{im}
\DeclareMathOperator{\ord}{ord}
\let\radical\relax
\DeclareMathOperator{\radical}{rad}
\DeclareMathOperator{\Ann}{Ann}
\DeclareMathOperator{\Frac}{Frac}
\DeclareMathOperator{\coker}{coker}

\newcommand{\Aff}{\mathbb{A}}

\newcommand{\HH}{\mathcal{H}}
\newcommand{\bbH}{\mathbb{H}}

\let\Im\relax
\DeclareMathOperator{\Im}{Im}
\let\Re\relax
\DeclareMathOperator{\Re}{Re}

\DeclarePairedDelimiter{\abs}{\lvert}{\rvert}
\DeclarePairedDelimiter{\norm}{\lvert}{\rvert}
\DeclarePairedDelimiter{\Norm}{\lVert}{\rVert}
\DeclarePairedDelimiter{\braket}{\langle}{\rangle}


\begin{document}
\maketitle

\begin{ex}
\leavevmode
\begin{enumerate}
\item N/A
\item Show that the Hilbert polynomial of $\C[\vec X]/I$ is the constant $r$.
\item Show that the Hilbert polynomial of $\C[X,Y,Z]/\braket{F,G}$ is the constant $de$. Conclude the claim $\#(V(F) \cap V(G)) \leq de$.
\end{enumerate}
\end{ex}

\begin{sol}
\leavevmode
\begin{enumerate}
\item N/A
\item Consider the map of vector spaces
\begin{equation}\label{eq:themap}
\C[\vec X]_k \to \C^r
\end{equation}
given by evaluation on the $r$ points. Its kernel is precisely $I_k$, and thus, by the isomorphism theorems, if we show that this map is surjective for large enough $k$ we will have the desired conclusion.

To show surjectivity for large $k$ we show that, for any two points $p, p' \in \PP^n$, there exists a (linear!) homogeneous polynomial $s$ which separates them, in the sense that $s(p) = 0$ and $s(p') \neq 0$. Indeed, pick representatives $p,p' \in \C^{n+1}$ of unit norm, and let $v = p' - \braket{p',p}p$. Then, it is trivial to check that $\braket{v,p} = 0$, while $\braket{v,p'} = 1 - \norm{\braket{p',p}}^2 > 0$ by Cauchy-Schwarz (and the fact that $p,p'$ are not colinear). Thus, we set $s(\vec X) = v \cdot \vec X$.

This shows that the map \eqref{eq:themap} is surjective for large $k$, as follows: for any $r$ distinct points there is a homogeneous polynomial of degree $r-1$ which vanishes on all but one of them. To build it, simply multiply all the polynomials that separate the distinguished point from the rest. Thus, the map is surjective for $k=r-1$, and to show surjectivity for higher values of $k$ one may pick a chosen linear factor and take repeated powers of it.

\item Because vector spaces are nice, the dimension of $(\C[X,Y,Z]/\braket{F,G})_k$ is equal to $\dim(\C[\vec X]_k) - \dim(\braket{F,G}_k)$. The former is equal to $\binom{k+2}{2}$, and we compute the latter.

A degree $k$ homogeneous polynomial in $\braket{F,G}$ is written in the form $a F + b G$, where $a$ is homogeneous of degree $k-d$ and $b$ of degree $k-e$. However, this representation is not unique; the failure in uniqueness is given by the kernel of the map $(a,b) \mapsto aF+bG$. So, the dimension of $\braket{F,G}_k$ is given by $\binom{k-d+2}{2} + \binom{k-e+2}{2}$, minus the dimension of the aforementioned kernel.

Since the ring of polynomials is a UFD and $F$ and $G$ have no common terms, a pair $(a,b)$ is in the kernel if and only if $a = Gc$ and $b = Fc$ for some homogeneous $c$ of degree $k-d-e$. When this number is negative, the kernel is thus empty; otherwise, it has dimension $\binom{k-d-e+2}{2}$. Putting it all together, we get that, for large values of $k$, the dimension of the quotient $(\C[X,Y,Z]/\braket{F,G})_k$ is given by
\begin{equation}
\binom{k+2}{2} - \binom{k-d+2}{2} - \binom{k-e+2}{2} + \binom{k-d-e+2}{2}.
\end{equation}

This is seen to equal $de$ by expanding and simplifying.

\smallskip

We conclude the desired result, because the cardinality of $V(F)\cap V(G)$ is the (constant) Hilbert polynomial of $R = \C[X,Y,Z]/I(V(F) \cap V(G))$. Now, this is a quotient of $R_0 = \C[X,Y,Z]/\braket{F,G}$ (not necessarily equal because $\braket{F,G}$ might fail to be radical), and so the Hilbert polynomial of $R$ is bounded from above by the Hilbert polynomial of $R_0$, which as we've seen is constant equal to $de$.
\end{enumerate}
\end{sol}

\begin{ex}
Show that the following are equivalent.
\begin{enumerate}
\item There exists $f \in 1+I$ such that $fM = 0$,
\item $IM = M$.
\end{enumerate}
\end{ex}

\begin{sol}
($\rightarrow$) The inclusion $\subseteq$ is obvious; for the other, given $m \in M$, write $m = \cancel{f m} + (1-f) m \in IM$.

\smallskip

($\leftarrow$) Pick generators $m_1, \dots, m_n$ of $M$. For each $m_j$, there are $\alpha_{1j}, \dots, \alpha_{nj}$ such that
\begin{equation}
m_j = \sum \alpha_{ij} m_i.
\end{equation}

If we put the values of $\alpha_{ij}$ in a matrix $A$, and the generators $m_j$ in a column-vector $\vec m$, we have the equality
\begin{equation}
A m = m,
\end{equation}
where the left-hand side is intended as the obvious extension of matrix multiplication $M_{n \times n}(R) \times M_{n \times 1}(M) \to M_{n \times 1}(M)$.

Moreover, by Cayley-Hamilton, if $p(x) = \sum a_i x^i$ is the characteristic polynomial of $A$, we have $p(A) = 0$. Thus, we have
\begin{equation}
0 = p(A) m = \sum a_i A^i m = \sum a_i m.
\end{equation}

Thus, if we set $f = \sum a_i$, we get $f m = 0$, hence $f m_i = 0$ for all generators, and thus $f M = 0$. It suffices to note that $f \in 1 + I$, This is because the top-level coefficient of $p$ is $1$, while all the others are nontrivial combinations of the $\alpha_{ij}$, all of which are in $I$. This completes the proof.
\end{sol}

\begin{ex}
\leavevmode
\begin{enumerate}
\item What are the dimensions of the fibers of the projection $Y \to \Gr(2,4)$? Give an upper bound for the dimension of $Y$.
\item Conclude the theorem by looking at the dimensions of the domain and codomain of the projection $Y \to \C[\vec X]_d$.
\end{enumerate}
\end{ex}

\begin{sol}
\leavevmode
\begin{enumerate}
\item The fiber $f^{-1}(w)$, with $w \in \Gr(2,4)$, consists of the pairs $(w,p)$, where $p$ ranges over the degree $d$ polynomials which vanish identically on $w$. Since $w$ is fixed, to find the dimension of this fiber it suffices to compute the dimension of the set of such polynomials $p$.

By adequate translation and change of basis, which does not change the dimension of the fiber under consideration, we may suppose that $w$ is given by the plane $X_0 = X_1 = 0$. Thus, we wish to find the dimension of the space of degree $d$ homogeneous polynomials $p$ such that $p(0,0,X_2,X_3) = 0$.

Write such a polynomial as
\begin{equation}
p(X_0,X_1,X_2,X_3) = \sum_{k+\ell \leq d} p_{k\ell}(X_0,X_1) X_2^k X_3^\ell,
\end{equation}
with $p_{k\ell}(X_0,X_1)$ a homogeneous polynomial of degree $d-k-\ell$. Now, the condition of vanishing on the given plane tells us that
\begin{equation}
\sum_{k+\ell=d} p_{k\ell}(0,0) X_2^k X_3^\ell = 0,
\end{equation}
whence $p_{k\ell} = 0$ for all $k+\ell = d$ (because these are degree zero polynomials hence constant). The terms $k+\ell < d$ don't show up because all such terms are homogeneous of positive degree, and thus will always vanish on the given plane.

As a consequence, a degree $d$ polynomial is in the given conditions iff all its monomials have at least one term $X_0$ or $X_1$. Thus, the dimension of the fibers is, by elementary combinatorics,
\begin{equation}
\text{dimension of fibers } = \binom{d+3}{3} - \binom{d+1}1.
\end{equation}

\smallskip

As a consequence, the dimension of $Y$ is at most
\begin{equation}
\dim Y \leq \dim(\Gr(2,4)) + \binom{d+3}{3} - \binom{d+1}1 = \binom{d+3}{3} - d + 3.
\end{equation}

\item The solution follows by the fact that if $f \colon A \to B$ is surjective, $\dim B \leq \dim A$. Consequently, if $\dim A < \dim B$, $f$ may never be surjective. We apply this to the particular case of $A = Y$, whose dimension is at most $\binom{d+1}3 + 4$, and $B = \C[\vec X]_d$, whose dimension is $\binom{d+3}3$. Thus, whenever $\binom{d+1}3 + 4 < \binom{d+3}3$, the map fails to be surjective, and by unfolding the definitions, a polynomial not in the image is one whose zero locus contains no $2$-planes.

It suffices to check that this inequality holds for all $d \geq 4$. By unfurling the definitions, this is the same as to verify that, for $d \geq 4$, we have
\begin{equation}
(d+1)(d)(d-1) + 24 < (d+3)(d+2)(d+1),
\end{equation}
which simplifies (by subtracting one side from another and expanding) to showing that $6 d^2 + 12 d -18 = 6 (-1 + d) (3 + d)$ is positive for $d \geq 4$. This is obvious, and we are done.
\end{enumerate}
\end{sol}

\begin{ex}
Not doing this one.
\end{ex}

\begin{sol}
I hope you have a great day!
\end{sol}

\end{document}