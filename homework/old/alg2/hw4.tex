\documentclass{article}

\usepackage{amsmath}
\usepackage{amssymb}
\usepackage{amsfonts,stmaryrd}
\usepackage{mathtools}

\usepackage[thmmarks, amsmath]{ntheorem}
\usepackage{fullpage}

\usepackage{graphicx}

\usepackage{diffcoeff}
\diffdef{}{op-symbol=\mathrm{d},op-order-sep=0mu}

\usepackage{cancel}
\usepackage{interval}

\usepackage{enumitem}

\setlist[enumerate,1]{label=(\alph*)}

\title{Algebra Homework 4}
\author{Duarte Maia}
%\date{}

\theorembodyfont{\upshape}
\theoremseparator{.}
\newtheorem{theorem}{Theorem}
\newtheorem{prop}{Prop}
\renewtheorem*{prop*}{Prop}
\newtheorem{lemma}{Lemma}

\newtheorem{ex}{Exercise}

\theoremstyle{nonumberplain}
\theoremheaderfont{\itshape}
\theorembodyfont{\upshape}
\theoremseparator{:}
\theoremsymbol{\ensuremath{\blacksquare}}
\newtheorem{proof}{Proof}
\newtheorem{sol}{Solution}

\newcommand{\R}{\mathbb{R}}
\newcommand{\C}{\mathbb{C}}
\newcommand{\Z}{\mathbb{Z}}
\newcommand{\N}{\mathbb{N}}
\newcommand{\Q}{\mathbb{Q}}
\newcommand{\K}{\mathbb{K}}

\newcommand{\kk}{\Bbbk}


\newcommand{\gp}{\mathfrak{p}}
\newcommand{\gq}{\mathfrak{q}}

\newcommand{\PP}{\mathbb{P}}
\newcommand{\Gr}{\mathrm{Gr}}
\newcommand{\GG}{\mathbb{G}}

\newcommand{\I}{\mathrm{i}}
\newcommand{\e}{\mathrm{e}}
\newcommand{\id}{\mathrm{id}}

\newcommand{\conj}[1]{\overline{#1}}

\newcommand{\grad}{\nabla}

\DeclareMathOperator{\sign}{sign}
\DeclareMathOperator{\image}{im}
\DeclareMathOperator{\ord}{ord}
\let\radical\relax
\DeclareMathOperator{\radical}{rad}
\DeclareMathOperator{\Ann}{Ann}
\DeclareMathOperator{\coker}{coker}

\newcommand{\Aff}{\mathbb{A}}

\newcommand{\HH}{\mathcal{H}}
\newcommand{\bbH}{\mathbb{H}}

\let\Im\relax
\DeclareMathOperator{\Im}{Im}
\let\Re\relax
\DeclareMathOperator{\Re}{Re}

\DeclarePairedDelimiter{\abs}{\lvert}{\rvert}
\DeclarePairedDelimiter{\norm}{\lvert}{\rvert}
\DeclarePairedDelimiter{\Norm}{\lVert}{\rVert}
\DeclarePairedDelimiter{\braket}{\langle}{\rangle}


\begin{document}
\maketitle

\begin{ex}
Given rings $A \subseteq B$ and a prime ideal $\gp \subseteq A$ find a prime ideal $\gq \subseteq B$ such that $\gp = \gq \cap A$.
\end{ex}

\begin{sol}
Let $S = A \setminus \gp$. Then, by a standard Zorn lemma argument, find a maximal ideal among those that contain $\gp$ and do not intersect $S$. We claim that the resulting ideal $\gq$ is prime. Indeed, if $ab \in \gq$ and neither $a$ not $b$ are in $\gq$, we show that we may add one of them to the ideal. This uses the fact that $S$ is multiplicatively closed.

Suppose that both $\gq + \braket{a}$ and $\gq + \braket{b}$ intersect $s$. Then, we may find $q + \beta a$ and $q' + \beta' b$ in $S$. Then, their product is also in $S$, but by expanding we see that their product is in $\gq$, which contradicts the fact that $\gq$ does not intersect $S$.

Thus, we conclude that this ideal $\gq$ is prime, and since it contains $\gp$ and does not intersect $A \setminus \gp$ we obtain $\gq \cap A = \gp$.
\end{sol}

\begin{ex}
Prove that $x \in A$ is nilpotent iff $x$ belongs to every prime ideal of $A$.
\end{ex}

\begin{sol}
Evidently, if $x$ is nilpotent, for any prime ideal $\gp$ we have $x^N \in \gp$ hence $x \in \gp$, so we focus on the other implication.

Suppose that $x$ is not nilpotent. Then, the collection of nonnegative powers of $x$ is a multiplicatively closed subset of $A$ which does not contain zero; let us call it $S$. Thus, we may use the same argument as in the previous exercise to find a prime ideal $\gq$ which contains $\gp = \{0\}$, and does not intersect $S$. Thus, $\gq$ is a prime ideal which does not contain $x$, and we have shown that if $x$ is not nilpotent then there is a prime ideal which does not contain it. Equivalently, if $x$ is in the intersection of all prime ideals then $x$ is nilpotent and we are done.
\end{sol}

\begin{ex}
Let $A$ be Noetherian, let $M$ be a finitely generated $A$-module. Prove that there exists a finite ascending chain
\begin{equation}
0 = M_0 \subseteq M_1 \subseteq \dots \subseteq M_n = M
\end{equation}
where each $M_i/M_{i-1}$ is isomorphic to $A/\gp_i$ for some prime ideal $\gp_i$.
\end{ex}

\begin{sol}
Lemma: For any nonzero module $M$ over a Noetherian ring $A$, there exists a nonzero submodule $M_1$ which is isomorphic to $A/\gp$ for some prime ideal $\gp$.

Proof of lemma: Pick $m \neq 0$. If the annihilator of $m$ is prime, the span of $m$ is the $M_1$ we seek, as $\braket m \cong A/\Ann(m)$. Otherwise, there exist $a,b \in A$ such that $abm = 0$ but $am \neq 0$ and $bm \neq 0$. So now, repeat the process with $bm$, say. Notice that the annihilator of $bm$ contains the annihilator of $m$, so our ideal has not gotten smaller, and moreover contains $a$, which does not annihilate $m$, so it has in fact gotten bigger. Thus, we obtain an increasing sequence of ideals, which stops only when we find one which is prime. Since $A$ is Noetherian, it must indeed stop at some point, culminating in some multiple $a_0 m \neq 0$

Lemma: For any module $M$ over a noetherian ring $A$, and for any proper submodule $M_k$ of $M$, there exists some submodule $M_{k+1}$ which contains $M_k$ strictly, and such that $M_{k+1}/M_k$ is isomorphic to $A/\gp$ for some prime ideal $\gp$.

Proof of lemma: Apply the previous lemma to the module $M/M_k$.

Lemma: If $M$ is a finitely generated module over a Noetherian ring, any ascending chain of submodules stabilizes.

Proof of lemma: If $M_0 \subseteq M_1 \subseteq \dots$ is the chain in question, pick finitely many generators of $\bigcup M_i$. This exists because any submodule of a finitely generated module over a Noetherian ring is itself finitely generated. Now, there is some $M_i$ which contains all the generators, and at this point the chain has stabilized.

\smallskip

The solution to the exercise is obtained by repeated application of the second lemma, using the third lemma to show that the process terminates.
\end{sol}

\begin{ex}
Prove that the collection of lines contained within a fixed algebraic set is an algebraic subset of $\GG(1,n)$.
\end{ex}

\begin{sol}
Given a line $\ell$, represented in the Plücker embedding by some wedge $v \wedge w$ (modulo scaling), we write the condition that $\ell \in V(H_1, \dots, H_m)$ in an obviously algebraic way. We suppose without loss of generality that $m = 1$; the general case is obtained using the fact that intersecting algebraic sets yields algebraic sets.

The line $\ell$ is contained in $V(H)$ iff, for all $s, t \in \kk$, $H(s v + t w) = 0$. Now, $H(sv+tw)$ is a polynomial in the variables $s,v,t,w$, and we see it instead as an element of $\kk[v,w][s,t]$. In this perspective, it vanishes for all $s, t$ iff all its coefficients vanish. (This uses that the field is infinite.) However, each of its coefficients is a polynomial in $v$ and $w$, so we get the algebraic condition we sought.
\end{sol}

\begin{ex}
\leavevmode
\begin{enumerate}
\item Check that the incidence relation is algebraic.
\item Show that the set of lines which meet $X$ is algebraic.
\item Show that the union of the lines in $Y$ is algebraic.
\end{enumerate}
\end{ex}

\begin{sol}
\leavevmode
\begin{enumerate}
\item Given $\omega \in \GG(1,n)$ and $p \in \PP^n$, $p$ is in the line represented by $\omega = v \wedge w$ iff the vectors $v,w,p$ are linearly dependent, iff $v \wedge w \wedge p = 0$. Thus, the incidence relation is given by $\{(p, \omega) \mid p \wedge \omega = 0\}$, which is obviously algebraic.
\item We know that $X$ is proper because it is a closed subset of $\PP^n$. therefore, the map $X \times \GG(1,n) \to \GG(1,n)$ is closed. As such, the image of the incidence relation is closed (i.e. algebraic), and the image of this set is precisely the set of lines which intersect $X$.
\item We know that $\GG(1,n)$ is proper because it is a closed subset of $\PP^{\binom n 2}$, and $Y$ is a closed subset of it and hence is also proper. As such, the map $Y \times \PP^n \to \PP^n$ is closed, and thus the image of the incidence relation is closed. This is precisely the union of all lines in $Y$.
\end{enumerate}
\end{sol}

\begin{ex}
\leavevmode
\begin{enumerate}
\item Show that the assignment $C \mapsto C/tC$ preserves exactness of exact sequences which terminate on the right at zero where all terms are $t$-torsionfree.

\item Show that if $N,N'$ are $t$-torsionfree graded $\kk[t]$-modules, a map $N \to N'$ which descends to an isomorphism $N/tN \to N'/tN'$ is itself an isomorphism.

\item (Not solved)

\item (Not solved)
\end{enumerate}
\end{ex}

\begin{sol}
\leavevmode
\begin{enumerate}
\item A sequence which terminates on the right at zero can be interpreted as one which never terminates to the right, so we show the following lemma (which finishes the question).

Lemma: If $A,B,C,D$ are all $t$-torsionfree and $A \xrightarrow f B \xrightarrow g C \xrightarrow h D$ is exact, then $A/tA \to B/tB \to C/tC$ is exact.

Proof of lemma: Since $g \circ f = 0$ we get that $\bar g \circ \bar f = 0$, so it remains to show that if $\bar g(\bar b) = \bar 0$ then $\bar b = \bar f(\bar a)$ for some $a$.

If $\bar g(\bar b) = \bar 0$ then $g(b) \in t C$. Thus, $g(b) = tc$ for some $c$. But $t h(c) = h(g(b)) = 0$, so $h(c) = 0$ by torsionfreeness, and therefore $c = g(b')$. As such, $g(b - tb') = 0$, whence $b = f(a) + tb'$, and finally $\bar b = \bar f(\bar a)$, which concludes the proof.

\item Consider the exact diagram $0 \to \ker f \to N \to N' \to \coker f \to 0$. By the previous part, it descends to an exact diagram. But also the map $N \to N'$ descends to an isomorphism, so $\ker f / t \ker f = 0$ and $\coker f / t \coker f = 0$. But since are graded modules and $t$ has positive grading, the only way for $\ker f$ to equal $t \ker f$ is if $\ker f = 0$, and likewise for $\coker f$, which proves that both of these are null and hence $f$ is an isomorphism.
\end{enumerate}
\end{sol}

\end{document}