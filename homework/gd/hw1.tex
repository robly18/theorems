\documentclass{article}

\usepackage{amsmath}
\usepackage{amssymb}
\usepackage{amsfonts}

\usepackage[thmmarks, amsmath]{ntheorem}

\usepackage{diffcoeff}
\usepackage{cancel}

\usepackage{enumitem}

\setlist{label=\alph*)}

\title{Differential Geometry Homework 1}
\author{Duarte Maia}
\date{}

\theorembodyfont{\upshape}
\theoremseparator{.}
\newtheorem{ex}{Exercise}

\theoremstyle{nonumberplain}
\theoremheaderfont{\itshape}
\theorembodyfont{\upshape}
\theoremseparator{:}
\theoremsymbol{\ensuremath{\blacksquare}}
\newtheorem{sol}{Solution}

\newcommand{\R}{\mathbb{R}}
\newcommand{\C}{\mathbb{C}}

\newcommand{\PP}{\mathbb{P}}

\newcommand{\I}{\mathrm{i}}
\newcommand{\e}{\mathrm{e}}

\DeclareMathOperator{\inte}{int}

\begin{document}
\maketitle

\begin{ex}
Show that $U(n)$ is a smooth manifold of dimension $n^2$.
\end{ex}

\begin{sol}
Consider the function
\begin{align*}
f \colon M_n(\C) &\to S_n\\
A &\mapsto A^* A,
\end{align*}
where $S_n$ is the vector space of $n \times n$ complex matrices which are self-adjoint. It is obvious that this vector space has dimension $n^2$.

We wish to show that $I$ is a regular value of $f$. To this effect, let $f(A) = I$, and let us calculate $(\dl f)_A$.

Let $B$ be an $n \times n$ complex matrix, identified with a vector tangent to $A$. This vector is the derivative of the curve $A + tB$, so we may calculate
\begin{multline*}
(\dl f)_A(B) = \diff{}t[0] f(A + tB)\\
=\diff{}t \left(A^* A + t (A^* B + B^* A) + t^2 B^* B \right) = A^* B + B^* A.
\end{multline*}

Consequently, to show that $A$ is a regular point, it suffices to show that for all symmetric matrices $C$, there exists $B$ such that $C = A^* B + B^* A$. To this effect, we simply guess: plugging $B = \frac12 A C$ yields
\[(\dl f)_A(B) = \frac12 \cancel{A^* A} C + \frac12 C^* \cancel{A^* A} = C,\]
where we used the fact that $A^* A = I$ and $C = C^*$.

We are therefore in the hypotheses of the regular value theorem, which yields that $U(n) = f^{-1}(I)$ is an embedded submanifold of $M_n(\C)$, whose dimension is
\[\dim M_n(\C) - \dim S_n = 2 n^2 - n^2 = n^2.\]
\end{sol}

\begin{ex}
Let $X$ and $Y$ be smooth manifolds of the same dimension such that $X$ is compact and $Y$ is connected. Let $f \colon X \to Y$ be a submersion. Show that $f$ is a finite covering map, that is:

\begin{enumerate}
\item $f$ is surjective

\item For all $q \in Y$, the preimage of $q$ is finite (say $p_1, \dots, p_k$) and there exists a neighborhood of $q$, say $V$, such that $f^{-1}(V)$ is a collection disjoint neighborhoods of the $p_i$ such that $f$ is a diffeomorphism from each of these to $V$.
\end{enumerate}
\end{ex}

\begin{sol}
First, we show that $f$ is surjective. Note that since $f$ is a submersion and $X$ and $Y$ have the same dimension, $f$ is a local diffeomorphism, that is: for all $x \in X$ there exists a coordinate neighborhood of $x$ and one of $f(x)$ such that $f$ is the identity in coordinates. This is a simple consequence of the normal form theorem for submersions.

Since $f$ is a local diffeomorphism, it is trivial to show that it is also open, so $f(X)$ is open in $Y$. It is also open, because it is the image of a compact. Since $Y$ is connected, we have shown that $f(X)$ is either empty or $Y$, but the empty set is not a manifold I guess so $X \neq \emptyset$ and so $f(X) \neq \emptyset$.

Now we show the stack of pancakes property. Let $q \in Y$. Then, $f^{-1}(q)$ is a closed set in $X$, hence it is compact. Furthermore, since $f$ is a local diffeomorphism, for each $p \in f^{-1}(q)$ there is a neighborhood of $p$ which does not intersect any other element of $f^{-1}(q)$, which means that $f^{-1}(q)$ is a discrete space in the subspace topology. A discrete compact space must be finite, so $f^{-1}(q)$ contains finitely many elements, say $p_1, \dots, p_k$.

Now, for each $p_i$ there exists a neighborhood $U_i$ of $p_i$ and $V_i$ of $q$ such that $f$ is a diffeomorphism between $U_i$ and $V_i$. Taking the intersection of all $V_i$, call it $V$, and shrinking the $U_i$ accordingly, we obtain that all $U_i$ are diffeomorphic (though $f$) to $V$. It suffices to show (by possibly shrinking first) that all $U_i$ are disjoint.

To this effect, for each $i$ construct a compact neighborhood $W_i$ of $p_i$ such that $p \in \inte W_i \subseteq W_i \subseteq U_i$. Then, replace each $U_i$ by
\[U'_i = \inte W_i \setminus \bigcup_{j \neq i} W_j.\]

Clearly all the $U'_i$ are disjoint neighborhoods of $p_i$, and $f$ restricted to each is still a diffeomorphism to some neighborhood of $q$. We may therefore repeat the shrinking argument used to replace all the $V_i$ by their intersection, and the proof is complete.
\end{sol}

\begin{ex}
Show that the embedding $\PP\R^1 \hookrightarrow \PP\R^2$ cannot be obtained as a regular level set of a smooth map $\PP\R^2 \to \R$.
\end{ex}

\begin{sol}
Simply note that $\PP\R^2 \setminus \PP\R^1$ is topologically equivalent to the plane (we removed the `line at infinity' from the projective plane), and so is connected.

If there was a smooth map $f \colon \PP\R^2 \to \R$ whose preimage of, say, $0$, were $\PP\R^1$, then $f$ must take a constant sign on $\PP\R^2 \setminus \PP\R^1$, and then 0 must either be a minimum or a maximum of $f$. But regular values cannot be minima or maxima, for if $f(q) = 0$ for some regular point $q$, by definition of regular point it is easy to check that some positive and some negative values must be attained near $q$. This contradiction shows that $\PP\R^1$ is not the level set of any regular value of any function $\PP\R^2 \to \R$.
\end{sol}

\end{document}