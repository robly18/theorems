\documentclass{article}

\usepackage{amsmath}
\usepackage{amssymb}
\usepackage{amsfonts}
\usepackage{mathtools}

\usepackage{graphicx}

\usepackage[thmmarks, amsmath]{ntheorem}

\usepackage{diffcoeff}
\diffdef{}{op-symbol=\mathrm{d},op-order-sep=0mu}
\usepackage{cancel}

\usepackage{enumitem}

\setlist[enumerate]{label=\alph*)}

\title{Differential Geometry Homework 6}
\author{Duarte Maia}
\date{}

\theorembodyfont{\upshape}
\theoremseparator{.}
\newtheorem{ex}{Exercise}

\theoremstyle{nonumberplain}
\theoremheaderfont{\itshape}
\theorembodyfont{\upshape}
\theoremseparator{:}
\theoremsymbol{\ensuremath{\blacksquare}}
\newtheorem{sol}{Solution}

\newcommand{\R}{\mathbb{R}}
\newcommand{\C}{\mathbb{C}}
\newcommand{\Z}{\mathbb{Z}}

\newcommand{\PP}{\mathbb{P}}
\newcommand{\FF}{\mathcal{F}}

\newcommand{\I}{\mathrm{i}}
\newcommand{\e}{\mathrm{e}}


\DeclareMathOperator{\inte}{int}
\DeclareMathOperator{\codim}{codim}
\DeclareMathOperator{\Lie}{Lie}
\DeclareMathOperator{\Ad}{Ad}
\DeclareMathOperator{\ad}{ad}
\DeclareMathOperator{\sign}{sign}
\DeclareMathOperator{\im}{im}
\newcommand{\grad}{\nabla}
\newcommand{\into}{\mathbin{\lrcorner}}
\newcommand{\id}{\mathrm{id}}

\DeclarePairedDelimiter{\norm}{\lvert}{\rvert}
\DeclarePairedDelimiter{\abs}{\lvert}{\rvert}

\begin{document}
\maketitle

\begin{ex}
Let $0 \rightarrow F \xrightarrow{a} E \xrightarrow{f} C \rightarrow 0$ be a short exact sequence of vector bundles over a smooth manifold $M$.
\begin{enumerate}
\item Show that the sequence splits, i.e. there exists $g \colon C \to E$ such that $f \circ g = \id$.

\item Show that $E$ is isomorphic to $F \oplus C$.
\end{enumerate}
\end{ex}

\begin{sol}
a) Let $x \in M$, and pick a local frame $f_1, \dots, f_*$ for $F$. Push this frame fowards using $a$ to obtain a linearly independent collection $e_1, \dots, e_*$ on $E$. Extend this collection to a basis at $x$, and extend the vectors we added to a vector field $e_1, \dots, e_*, e'_1, \dots, e'_\bullet$ near $x$ (using, say, a local trivialization). Since linear independence is preserved under small perturbations, this collection of vector fields forms a frame near $x$. Finally, we can push this frame forwards to $C$ using $f$, and by exactness we obtain that the $e_1, \dots, e_*$ are turned to zero, but that $f(e'_1), \dots, f(e'_\bullet)$ forms a local frame for $C$.

It is with the help of this local frame that we define a local $g$, simply as $g(f(e'_i)) = e'_i$.

This $g$ is defined in a neighborhood of $x$, so now we glue several such $g$ using a partition of unity. The resulting function satisfies $f \circ g = \id$ because $f$ is linear and therefore
\[f(\sum \phi_i g_i(c)) = \sum \phi_i f(g_i(c)) = \sum \phi_i \cdot c = c.\]

\medskip

b) We have a function $a \colon F \to E$ and another $g \colon C \to E$. This induces a function $h$ from $F \oplus C$ to $E$. By the rank-nullity theorem and the short exact sequence, we know that $\dim E = \dim F + \dim C = \dim F \oplus C$, so it suffices to show that $h$ is injective.

Suppose that $h(v,w) = 0$, $v \in F$ and $w \in C$. In other words, $a(v) + g(w) = 0$. Consequently, $f(a(v) + g(w)) = 0$, but by exactness $f(a(v)) = 0$ and so we conclude
\[w = f(g(w)) = 0.\]

Therefore, $a(v) + g(w) = a(v)$, and since $a$ is injective, for this to be null we require $v = 0$. This concludes the proof that $h$ is injective and therefore that $E \cong F \oplus C$.
\end{sol}

\begin{ex}
Let $E$ be a vector bundle over a compact manifold $M$. Show that there exists a vector bundle $F$ over $M$ such that $E \oplus F$ is trivial.
\end{ex}

\begin{sol}

\end{sol}

\begin{ex}
Let $M$ be compact, oriented and connected. Show that $\chi(E \oplus F) = \chi(E) \wedge \chi(F)$.
\end{ex}

\begin{sol}
Let $U_E$ and $U_F$ be representatives of the Thom classes for $E$ and $F$. We begin by showing that, in some sense, $U_E \wedge U_F$ is a representative of the Thom class in $E \oplus F$.

To do so, we consider the pullback of $U_E$ under the projection $E \oplus F \to E$, and likewise for $F$. We call these $V_E$ and $V_F$. Now, these two are closed forms on $E \oplus F$, so it makes sense to wedge these two. Now we show that $V_E \wedge V_F$ is a representative of the Thom class in $E \oplus F$.

To this effect, we show that integration on the fibers yields one. Pick a frame for $E$ and a frame for $F$, and use these to induce coordinates $(x_1, \dots, x_*, y_1, \dots, y_\bullet)$ on the fibers. Then, we use these coordinates to integrate $V_E \wedge V_F$ on a fiber:
\[\iint (V_E \wedge V_F)(\partial_x,\partial_y) \dl3x \dl3y.\]

Now, if we expand the definition of the wedge and use the fact that the projections of $\partial_x$ on $F$ are null and vice-versa, one easily concludes that the wedge product simplifies to
\[\iint U_E(\partial_x) U_F(\partial_y) \dl3x \dl3y,\]
and the integral can be written as a product of two integrals, both of which are one by definition of the Thom class.

In conclusion: We have just shown that
\begin{equation}\label{eq1}
U_{E \oplus F} = \pi^* U_E \wedge \pi^* U_F.
\end{equation}

Now we pull both sides of \eqref{eq1} back by, say, the zero section, call it $s$. On the left side, we obtain precisely a representative of $\chi(E \oplus F)$. On the right-hand side, we get
\[s^*(\pi^* U_E \wedge \pi^* U_F) = (s^* \pi^* U_E) \wedge (s^* \pi^* U_F).\]

Finally, the last two terms can be simplified, as, for example, $s^* \pi^* U_E = (\pi \circ s)^* U_E$, where $s$ is the zero-section on $E \oplus F$ and $\pi$ is the projection on $E$, so $\pi \circ s$ is the zero-section on $E$ and consequently $s^* \pi^* U_E = \chi(E)$. 

We have thus shown that \eqref{eq1} reduces to $\chi(E \oplus F) = \chi(E) \wedge \chi(F)$, as desired.
\end{sol}

\begin{ex}
Show that if $\chi(M) = 0$ then $M$ admits a never vanishing vector field.
\end{ex}

\begin{sol}

\end{sol}

\begin{ex}
(Complicated exercise)
\end{ex}

\begin{sol}

\end{sol}

\end{document}