\documentclass{article}

\usepackage{amsmath}
\usepackage{amssymb}
\usepackage{amsfonts}
\usepackage{mathtools}

\usepackage{graphicx}

\usepackage[thmmarks, amsmath]{ntheorem}

\usepackage{diffcoeff}
\diffdef{}{op-symbol=\mathrm{d},op-order-sep=0mu}
\usepackage{cancel}

\usepackage{enumitem}

\setlist[enumerate]{label=\alph*)}

\title{Differential Geometry Homework 7}
\author{Duarte Maia}
\date{}

\theorembodyfont{\upshape}
\theoremseparator{.}
\newtheorem{ex}{Exercise}

\theoremstyle{nonumberplain}
\theoremheaderfont{\itshape}
\theorembodyfont{\upshape}
\theoremseparator{:}
\theoremsymbol{\ensuremath{\blacksquare}}
\newtheorem{sol}{Solution}

\newcommand{\R}{\mathbb{R}}
\newcommand{\C}{\mathbb{C}}
\newcommand{\Z}{\mathbb{Z}}

\newcommand{\PP}{\mathbb{P}}
\newcommand{\FF}{\mathcal{F}}

\newcommand{\I}{\mathrm{i}}
\newcommand{\e}{\mathrm{e}}


\DeclareMathOperator{\inte}{int}
\DeclareMathOperator{\codim}{codim}
\DeclareMathOperator{\Lie}{Lie}
\DeclareMathOperator{\Ad}{Ad}
\DeclareMathOperator{\ad}{ad}
\DeclareMathOperator{\sign}{sign}
\DeclareMathOperator{\im}{im}
\newcommand{\grad}{\nabla}
\newcommand{\into}{\mathbin{\lrcorner}}
\newcommand{\id}{\mathrm{id}}

\DeclarePairedDelimiter{\norm}{\lvert}{\rvert}
\DeclarePairedDelimiter{\abs}{\lvert}{\rvert}

\newcommand{\pc}{P^1_\C}

\begin{document}
\maketitle

\begin{ex}
Let $L \to \pc$ be the tautological line bundle. Let $U_z$ and $U_w$ be charts on $\pc$, both with coordinates living in $\C$, and with correspondences
\[z \mapsto [z,1], \quad w \mapsto [1,w].\]

\begin{enumerate}
\item Consider the trivializations on $U_z$ and $U_w$ induced by the (complex) frames: $s_z \colon [z,1] \mapsto (z,1)$, $s_w \colon [1,w] \mapsto (1,w)$. Determine the corresponding transition function $g_{zw}$.

\item Define the following elements of the dual bundle
\[ \omega_z = \frac{-\overline{z}}{1+\abs{z}^2} \dl z, \quad \omega_w = \frac{-\overline{w}}{1+\abs{w}^2} \dl w.\]

Show that these forms are connections forms for some connection on $L^*$.

\item Compute the curvature form on $U_z$.

\item Compute the first Chern number.
\end{enumerate}
\end{ex}

\begin{sol}
a) The transition function $g_{zw}$ satisfies: If a section is of the form $s = \alpha s_w$, then it is of the form $s = g_{zw} \alpha s_z$. Consequently, if we set $\alpha = 1$, we may write $g_{zw} s_z = s_w$, so we need only write $s_w$ in terms of $s_z$.

Well, at a point $[z,w]$, we get that
\[s_z = (\frac zw, 1), \quad s_w = (1,\frac wz) = \frac wz s_z,\]
so we conclude
\[g_{zw}[z,w] = \frac wz.\]

This notation isn't very good for the following computations, because we will only be using either the representations $[z,1]$ or $[1,w]$. Under this notation,
\[g_{zw} = w = \frac1z.\]

\medskip

b) The first step is to compute the dual cocycles $g_{zw}^*$, so that afterwards we may verify the condition $\omega_z = g^*_{zw} \omega_w g^*_{wz} + g^*_{zw} \dl g^*_{wz}$. As such, I will deduce the formula for dual cocycles.

Suppose that we have a formula for some cocycle $g_{12}$, which can be used to transform from a frame $s_1$ to a frame $s_2$ in the manner
\[(s_2)_j = \sum_i (s_1)_i (g_{12})_{ij}.\]

This allows us to compute the dual frame $s_2^*$ in terms of $s_1^*$. Let $v$ be a vector on the intersection of the domains of the two frames. Then,
\[v = \sum_j (s_2^*)_j v \, (s_2)_j = \sum_{i,j} (s_2^*)_j v \, (g_{12})_{ij} (s_1)_i.\]

As a consequence, $(s_1^*)_i v = \sum_j (s_2^*)_j v \, (g_{12})_{ij}$, i.e.
\[(s_1^*)_j = \sum_i (s_2^*)_i (g_{12})_{ji}.\]

In other words, $(g_{21}^*) = (g_{12})^T = ((g_{21})^{-1})^T$. (I think that this is surprising, because I was expecting a conjugate transpose, but the math seems to check out.) In our case, this translates to
\[g_{zw}^* = \frac1{g_{zw}} = \frac 1w; \quad g_{wz}^* = w.\]

Therefore, we compute $g^*_{zw} \omega_w g^*_{wz} + g^*_{zw} \dl g^*_{wz}$. Note that since we are in a line bundle, the first term is simply $\omega_w$, so it suffices to compute
\begin{equation*}
\omega_w + g^*_{zw} \dl g^*_{wz} = \frac{-\overline{w}}{1+\abs{w}^2} \dl2 w + z \dl2 w
\end{equation*}

To continue the computations, we need to write $\dl z$ in terms of $\dl w$. We know that (where both are defined) $zw = 1$, and therefore
\[\dl(zw) = 0\text{, i.e. } \dl z \, w + z \dl w = 0.\]

(Note: In principle, I should take care here, because I'm taking rules that are true in the real realm and applying them to complex-valued stuff. So I took my pen and paper and verified the above identity, looking at complex stuff as though they lived in $\R^2$ and doing the math on the components, and it checked out.)

Therefore, $\dl w = -\frac wz \dl z$. Note also that as a consequence of the condition $zw=1$, all instances of $w$ can be replaced by $1/z$, and so

\begin{align*}
\omega_w + g^*_{zw} \dl g^*_{wz} &=\frac{-\overline{w}}{1+\abs{w}^2} \dl2 w + z \dl2 w\\
&= \frac{-1/\overline{z}}{1+1/\abs{z}^2} \frac{-w}z \dl z - z \frac wz \dl z\\
&= \frac{1}{1+\abs{z}^2} \frac1z \dl z - \frac1z \dl2 w\\
&= \frac1z \left(\frac{1}{1+\abs{z}^2} - 1\right) \dl2 w\\
&= \frac1z \frac{-\abs{z}^2}{1+\abs{z}^2} \dl2 w\\
&= \frac{-\overline{z}}{1+\abs{z}^2} \dl2 w = \omega_z.
\end{align*}

In conclusion, the forms $\omega_w$ and $\omega_z$ transform as required, and therefore induce a connection on $\pc$.

\medskip

c) The curvature form is given by $\Omega_z = \dl \omega_z + \omega_z \wedge \omega_z$. Since we're in a line bundle, the second term vanishes, so it suffices to compute
\begin{align*}
\dl \omega_z &= - \dl \frac{\overline{z}}{1+\abs{z}^2} \wedge \dl z\\
&= - \left( \frac{\dl \overline{z}}{1+\abs{z}^2} - \overline{z} \frac{\dl(1+\abs{z}^2)}{(1+\abs{z}^2)^2} \right) \wedge \dl z\\
&= - \frac1{1+\abs{z}^2} \dl \overline{z} \wedge \dl z + \left(\overline{z} \frac{\dl(1+\abs{z}^2)}{(1+\abs{z}^2)^2} \right) \wedge \dl z\\
&= - \frac1{1+\abs{z}^2} \dl \overline{z} \wedge \dl z + \left(\overline{z} \frac{z \dl \overline{z} + \cancel{\overline{z} \dl z}}{(1+\abs{z}^2)^2} \right) \wedge \dl z\\
&= - \frac1{1+\abs{z}^2} \dl \overline{z} \wedge \dl z + \left(\overline{z} \frac1{(1+\abs{z}^2)^2} \right) z \dl2 \overline{z} \wedge \dl z\\
&= \left(- \frac1{1+\abs{z}^2} + \overline{z} \frac1{(1+\abs{z}^2)^2}  z \right) \dl \overline{z} \wedge \dl z\\
&= \left(\frac{-1-\abs{z}^2}{(1+\abs{z}^2)} + \frac{\abs{z}^2}{(1+\abs{z}^2)^2} \right) \dl \overline{z} \wedge \dl z\\
&= \frac1{(1+\abs{z}^2)^2} \dl z \wedge \dl \overline{z}.
\end{align*}

\medskip

d) Since the matrix $\Omega$ has only one entry, its trace (i.e. $c_1$) is $\Omega$ itself, so we integrate $\frac\I{2\pi}\Omega$. Since the $z$ chart covers the whole space except a submanifold of dimension one, and hence measure zero, it suffices to integrate $\frac\I{2\pi}\Omega_z$ over $\C$, so we do:
\begin{align*}
\underline{c_1} &= \frac\I{2\pi} \iint \frac1{(1+x^2+y^2)^2} \dl3(x+\I y) \wedge \dl(x-\I y)\\
&= \frac\I{2\pi} \iint \frac1{(1+x^2+y^2)^2} \, 2 \I \dl3 x \wedge \dl y\\
&= - \frac1\pi \iint \frac1{(1+x^2+y^2)^2} \dl3 x \dl3 y.
\end{align*}

This is a nasty integral. Switch to polar coordinates:
\begin{align*}
\underline{c_1} &= - \frac1\pi \iint \frac1{(1+x^2+y^2)^2} \dl3 x \dl3 y\\
&= - \frac1\pi \int_0^{2\pi} \int_0^\infty \frac1{(1+r^2)^2} r \dl3 r \dl3 \theta\\
&= \int_0^\infty \frac {-2r}{(1+r^2)^2} \dl3 r.
\end{align*}

(There are a bunch of things that probably should be said here. The standard argument about how we're ignoring a set of null measure. Verifying that the coordinates $(r,\theta)$ are positively oriented. That's all routine and tiresome to write, so I didn't.)

Now, I have to calculate that integral. By eye, I was able to guess a primitive: $\frac1{1+r^2}$, and so
\[\underline{c_1} = \left[ \frac1{1+r^2} \right]_0^\infty = -1.\]

\end{sol}

\begin{ex}

\end{ex}

\begin{sol}

\end{sol}

\end{document}