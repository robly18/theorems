\documentclass{article}

\usepackage{amsmath}
\usepackage{amssymb}
\usepackage{amsfonts}
\usepackage{mathtools}

\usepackage[thmmarks, amsmath]{ntheorem}

\usepackage{diffcoeff}
\usepackage{cancel}

\usepackage{enumitem}

\setlist{label=\alph*)}

\title{Differential Geometry Homework 2}
\author{Duarte Maia}
\date{}

\theorembodyfont{\upshape}
\theoremseparator{.}
\newtheorem{ex}{Exercise}

\theoremstyle{nonumberplain}
\theoremheaderfont{\itshape}
\theorembodyfont{\upshape}
\theoremseparator{:}
\theoremsymbol{\ensuremath{\blacksquare}}
\newtheorem{sol}{Solution}

\newcommand{\R}{\mathbb{R}}
\newcommand{\C}{\mathbb{C}}
\newcommand{\Z}{\mathbb{Z}}

\newcommand{\PP}{\mathbb{P}}
\newcommand{\FF}{\mathcal{F}}

\newcommand{\I}{\mathrm{i}}
\newcommand{\e}{\mathrm{e}}


\DeclareMathOperator{\inte}{int}
\DeclareMathOperator{\codim}{codim}
\newcommand{\grad}{\nabla}

\DeclarePairedDelimiter{\norm}{\lvert}{\rvert}

\begin{document}
\maketitle

\begin{ex}
Let $\FF_1$ and $\FF_2$ be two foliations of $M$.
\begin{enumerate}
\item Give an appropriate definition for transversality.

\item If $\FF_1 \pitchfork \FF_2$, define $\FF_1 \cap \FF_2$ and compute its codimension.
\end{enumerate}
\end{ex}

\begin{sol}
a) We will say that two foliations are transverse if its leaves are pairwise transverse.

\medskip

b) We define $\FF = \FF_1 \cap \FF_2$ as the foliation whose leaves are connected components of pairwise intersections of leaves of $\FF_1$ with leaves of $\FF_2$. We will show that this is a foliation of codimension equal to $\codim \FF_1 + \codim \FF_2$.

To be more precise, let $L^1_\alpha$ be the leaves of $\FF_1$ and $L^2_\beta$ be the leaves of $\FF_2$. Define $L_{\alpha\beta} := L^1_\alpha \cap L^2_\beta$, and decompose $L_{\alpha\beta}$ into its connected components $L_{\alpha\beta1}, L_{\alpha\beta2}, \dots$. (There may be a finite or infinite number of connected components, or none if $L_{\alpha\beta} = \emptyset$.) These $L_{\alpha\beta k}$ are the leaves of $\FF$.

It is obvious that these leaves are all disjoint and that they form a partition of $M$. We will now verify that each point is covered by a foliated chart.

The main result necessary to this effect is the following: if $\FF_1 \pitchfork \FF_2$, near every point $p$ there exist coordinates $(x,y,z)$ such that $((x,y),z)$ is a foliated chart for $\FF_1$ and $((x,z),y)$ is a foliated chart for $\FF_2$. Once this result is proven, we will be able to write foliated charts for $\FF_1 \cap \FF_2$ as $(x,(y,z))$.

To this effect, let $p \in M$ and suppose that $(x',z)$ is a foliated chart around $p$ for $\FF_1$, and that $(x'',y)$ is a foliated chart around $p$ for $\FF_2$. The first step is to show that the functions $y^i$ and $z^i$ are independent, in the sense that all the $\dl y^i$ and $\dl z^i$ are linearly independent. Once this is done, we know that we can find a coordinate chart of the form $(x,y,z)$ near $p$, and the next step is to show that this chart satisfies the desired conditions.

To show that all the $\dl y^i$ and $\dl z^i$ are linearly independent, let us consider the collection of vectors $\partial_{x'^i}$ and $\partial_{x''^i}$. These vectors span the tangent spaces to the leaves of $\FF_1$ and $\FF_2$ respectively, and the transversality hypothesis tells us that the collection of all these vectors, call it $G$, generates the tangent spaces to $M$. Consequently, a one-form is null if and only if it is null when tested on all vectors of $G$. With this in mind, to show that all the $\dl y^i$ and $\dl z^i$ are linearly independent, suppose that we are given a linear combination of these that equals zero:
\[\sum a_i \dl y^i + \sum b_j \dl z^j = 0.\]

Evaluating it on all the $\partial_{x''^*}$ yields
\[\sum a_i \dl y^i(\partial_{x''^*}) = 0,\]
because all $\dl z^j(\partial_{x''^*})$ are null. In turn, since all $\dl y^i(\partial_{x'^*})$ are null, we conclude that the form $\sum a_i \dl y^i$ is null when evaluated on all vectors of $G$, and hence
\[\sum a_i \dl y^i = 0.\]

Finally, by linear independence of the $\dl y^i$, we conclude that all the $a_i$ are null. A similar argument goes to show that all the $b_j$ are null, and the proof that the $\dl y^i$ and $\dl z^j$ are linearly independent is complete.

We are finally able to build our $(x,y,z)$ chart, so it remains to show that it is foliated in the three ways outlined above. Suppose without loss of generality that this chart is a (coordinate) rectangle so that there are no problems with connectivity. Clearly, the level sets of $z$ connected components of the leaves, because they were so in the $(x',z)$ chart and decreasing the domain of the chart cannot join different connected components. This shows that $((x,y),z)$ is a foliated chart for $\FF_1$, and the same argument shows that $((x,z),y)$ is a foliated chart for $\FF_2$. Finally, we show that $(x,(y,z))$ is a foliated chart for $\FF_1 \cap \FF_2$. Indeed, the level sets of $(y,z)$ are all contained in the same leaf of the intersection, because they are all in the same leaf $L^1_\alpha$ of $\FF_1$ and $L^2_\beta$ $\FF_2$, and since these level sets are connected they are all in the same connected component of $L_{\alpha\beta}$.

We need only show that the level sets are precisely the connected components. To this effect, consider a path $\gamma$, contained in the $(x,y,z)$ chart, contained in the same leaf $L_{\alpha\beta k}$. We will show that all points of $\gamma$ have the same $(y,z)$ coordinates. Well, this path is entirely contained in the leaf $L^1_\alpha$ and in the $(x',z)$ chart, so the $z$ coordinate must be fixed. The same argument works to show that $y$ is fixed. Consequently, any in the $(x,y,z)$ chart which never leaves the same leaf has $(y,z)$ fixed, which shows that the level sets of $(y,z)$ are indeed the connected components of the leaves. This completes the proof that $\FF_1 \cap \FF_2$ is indeed a foliation.
\end{sol}

\begin{ex}
Show that $\FF$ is simple iff $M/\FF$ has a smooth structure such that $\pi$ is a submersion.
\end{ex}

\begin{sol}
$(\leftarrow)$ If $M/\FF$ has a smooth structure such that $\pi$ is a submersion, then it is trivial to construct foliated charts such where each $y$ corresponds to exactly one leaf. Indeed, given $p \in M$, pick a coordinate chart $(x,y)$ around $p$ and another around $\pi(p)$ where $\pi$ looks in coordinates like
\[\pi(x,y) = y.\]

Shrink the neighborhood of $p$ into a rectangle so that we do not have problems with connectedness, and it suffices to show: $(x,y)$ and $(x',y')$ are in the same leaf iff $y = y'$. But both of these statements are equivalent to saying $\pi(x,y) = \pi(x',y')$.

\medskip

$(\rightarrow)$ We assume that $\FF$ is simple, and construct a smooth structure on $M/\FF$.

First, we construct coordinate charts on $M/\FF$ as follows. Given $p \in M$, consider a rectangular foliated chart $(V, (x,y))$ around $p$ such that each leaf intersects $V$ exactly once. In other words, leaves are identified by their $y$ coordinate. Therefore, it is natural to consider $U \subseteq M/\FF$ as the set of leaves which intersect $V$, and to define $\tilde y$ on $U$ as the lift of $y$. If we define such a $U$ and $\tilde y$ for each $p \in M$, the collection of all the $U$ will cover $M/\FF$, and it remains to show that 1) each $U$ is open and $\tilde y$ is a homeomorphism, and 2) the transitions between the $\tilde y$ are smooth.

To show 1), we prove that the projection map $\pi$ is open. Once we do so, $U$ is clearly open as $U = \pi(V)$. Furthermore, $\tilde y$ is continuous because given an open $W \subseteq \R^{n-k}$ (where $n$ is the dimension of $M$ and $k$ is the rank of $\FF$) we get
\[\tilde y^{-1}(W) = \{\,L \in \FF \mid  \exists_{(x,y)\in V \cap L}\, y \in W \,\} = \pi(V_1 \times (W\cap V_2)),\]
where $V = V_1 \times V_2$.\footnote{This is kind of wrong because $V$ lives in the manifold, not $\R^n$, and so this decomposition as a cartesian product makes no sense, but it should be clear what's going on here (we're doing math in $(x,y)$ coordinates) and writing it out with properly correct notation would be a gigantic mess, so I kept it like this.}
This set is clearly open. Finally, $\tilde y$ is open because, given an open $U_0 \subseteq U$, $V_0 = \pi^{-1}(U_0)$ is open, and
\[\tilde y(U_0) = y(V \cap V_0) = \pi_2((x,y)(V \cap V_0)).\]

Since $\pi_2$ is open (as a map from an open in $\R^n$ to $\R^{n-k}$) and $(x,y)$ is a homeomorphism, $\tilde y(U_0)$ is open.

We now prove that $\pi$ is open. In other words, if $V \subseteq M$ is an open set, we wish to show that $\pi^{-1} \pi(V)$ is open. To this effect, for each $p \in \pi^{-1}\pi(V)$ we construct a neighborhood of $p$ whose points are all in $\pi^{-1}\pi(V)$.

Let $L$ be the leaf containing $p$. By hypothesis, this leaf intersects $V$ at some point $q$. Consider a path of plaques $P_1, \dots, P_k$ going from $q$ to $p$. We will build a neighborhood of $p$ inductively on $k$. In order to simplify the exposition, we will only do the cases $k = 1$ and $k = 2$, but the general case is very similar.

For $k = 1$, pick the foliated chart $U$ which contains $p$ and $q$. Then, we build the neighborhood of $p$ in coordinates by considering $U \cap V$ in coordinates and `stretching it out horizontally', i.e. defining $V_0 = \pi_1(U) \times \pi_2(U \cap V)$ in coordinates. Since $p$ and $q$ have the same $y$ coordinate and $q \in U \cap V$, this is evidently a neighborhood of $p$ whose elements are all in leaves which intersect $V$.

For $k = 2$, let $p_0 \in P_1 \cap P_2$. We repeat the process above for $p_0$, obtaining a neighborhood $V_0$ of $p_0$ whose elements are all in leaves which intersect $V$. Now, we repeat the process on $p$, with $V$ as $V_0$ and $q$ as $p_0$, building a neighborhood $V_1$ of $p$. Every point in this neighborhood has its leaf intersecting $V_0$, which in turn implies that every point in $V_1$ has its leaf intersecting $V$. Consequently, $V_1$ is also a neighborhood of $p$ contained in $\pi^{-1}\pi(V)$. Again, this strategy is easily seen to work for higher $k$, and the proof that $\pi$ is open is complete.

We now turn to showing 2), that is, that the transition maps between the $\tilde y$ are smooth. To this effect, let $(x,y)$ and $(x',y')$ be two coordinate charts, and construct the respective lifts $\tilde y$ and $\tilde y'$. Consider a leaf $L$ which is in both the $\tilde y$ and the $\tilde y'$ chart. We will build a neighborhood of $\tilde y(L)$ where the function $\varphi = \tilde y' \circ \tilde y$ is a diffeomorphism.

The idea is the following. Consider a path of plaques of $L$, say $P_1, \dots, P_k$, which goes from the $(x,y)$ chart to the $(x',y')$ chart. For each foliated chart associated to each $P_i$ we may define a new $\tilde y_i$. The idea is to show that the transition between each $\tilde y_i$ to $\tilde y_{i+1}$ is smooth near $L$. Therefore, it suffices to suppose $k = 2$, as this proof will immediately generalize to any $k$.

Consequently, we suppose that $\tilde y$ and $\tilde y'$ are based on coordinate charts $(x,y)$ and $(x',y')$ which have in common a point $p \in L$. In a neighborhood $U$ of this point, the transition from $(x,y)$ to $(x',y')$ is smooth. From this we may deduce that the transition from $\tilde y$ to $\tilde y'$ is smooth in $\pi(U)$, because it can, for example, be calculated using the following sequence of smooth operations. Suppose that $p$ has local coordinates $(x_0,y_0)$.
\[\tilde y \xrightarrow{\text{Insert into $(x,y)$ chart}} (x_0,\tilde y) \xrightarrow{\text{Transition map}} (x', y') \xrightarrow{\pi} \tilde y'.\]

In other words, instead of calculating the transition map between $\tilde y$ and $\tilde y'$ by taking a value for $\tilde y$, calculating the leaf corresponding to it, and calculating $\tilde y'$ of that leaf, we are calculating a point on that leaf in $(x,y)$ coordinates (in a smooth way), followed by transitioning to $(x',y')$ coordinates and projecting on the $y'$ axis.

This completes the proof of smoothness of transitions, and we have finally finished the exercise.
\end{sol}

\begin{ex}
Consider the action of $U(1)$ on $S^3$.
\begin{enumerate}
\item Show that the orbits of the action define a 1-dimensional simple foliation $\FF$ of $S^3$.

\item Show that $S^3/U(1)$ is diffeomorphic to $S^2$.

\item Show that $\FF$ is not transverse to the Reeb foliation.
\end{enumerate}
\end{ex}

\begin{sol}
a) Since the action is smooth and proper, the quotient $S^3/U(1)$ is a smooth manifold and the quotient map is a submersion. Furthermore, this quotient is by definition the quotient $S^3/\FF$, where $\FF$ is the collection of orbits of the action. If we show that $\FF$ is a foliation, exercise 1 guarantees that it is a simple foliation.

To show that it is a foliation, we note that it is the pullback of the 0-foliation of the quotient space. In other words, we may consider $\FF_0$ as the foliation on the quotient whose leaves are just the points, and so when we pull it back via $\pi$ we obtain $\FF = \pi^*(\FF_0)$.\footnote{This actually requires checking one detail, which is that the orbits are connected, but that is just a matter of noticing that the orbits are diffeomorphic to $U(1)$, which is connected.} This is a well-defined foliation because $\pi$ is transverse to $\FF_0$, because $\pi$ is a submersion.

\medskip

b) Consider the function $p(z,w) = (\norm z^2 - \norm w^2, 2z\overline{w})$. This function goes from $S^3$ to $S^2$, where $S^2$ is given by
\[ S^2 = \{\, (h,z) \mid h \in \R, z \in \C, h^2 + \norm z^2 = 1\,\}.\]

It is clear that $p$ is a smooth function from $S^3$ to $\R^3$, and since its image is contained in $S^2$ it restricts to a smooth function $S^3 \to S^2$. Furthermore, it is obvious that it is invariant with respect to the action of $U(1)$, so $p$ lifts to some smooth function
\[\tilde p \colon S^3 / U(1) \to S^2.\]

We now show that $\tilde p$ is a diffeomorphism. We need to show three things: 1) $\tilde p$ is injective, 2) $\tilde p$ is surjective and 3) $\dl \tilde p$ is an isomorphism.

1) We will show that if $p(z,w) = p(z',w')$ then there exists a unit complex number $u$ such that $z' = uz$ and $w' = uw$.

First, we point out that, due to the definition of $S^3$, the first entry of $p(z,w)$ can be written as
\[\norm z^2 - \norm w^2 = 2 \norm z^2 - 1 = 1 - 2 \norm w^2.\]

Consequently, from $p(z,w) = p(z',w')$ we conclude that $\norm z = \norm{z'}$ and $\norm w = \norm{w'}$, and so there exist $u$ and $v$ unit complex numbers such that
\[z' = u z \text{ and } w' = v w.\]

We need now only show that we can pick $u$ and $v$ to be the same. In the case where either $z$, $w$, $z'$ or $w'$ is null, we may freely change the corresponding unit complex and the statement is proven, so we now assume that none of them is null. Then, since $p(z,w) = p(uz,vw)$, we obtain
\[2 z w = 2 u z \overline{v w} = 2 u \overline v z \overline w.\]

Since $2zw \neq 0$, we may divide out to obtain $u \overline v = 1$, and since these are unit complex numbers we conclude $u = v$. The proof of injectivity is complete.

2) To show surjectivity, let $h \in \R$ and $a \in \C$ with $h^2 + \norm a^2 = 1$, i.e. $(h,a) \in S^2$. Consider $z = \sqrt{\frac{h+1}2}$, which is a real number because $h$ is at least $-1$. Then, let
\[w = \frac{\overline a}{2z} \text{ if $z \neq 0$, and $w = 1$ otherwise.}\]

Then, we claim that $(z,w)$ is a point in $S^3$ whose image under $p$ is $(h,a)$.

First we check that $(z,w) \in S^3$. The case where $z = 0$ is trivial, so we focus on the other case:
\begin{multline*}
\norm z^2 + \norm w^2 = z^2 + \frac{\norm a^2}{4z^2} = \frac{4z^4 + \norm a^2}{4z^2}\\
= \frac{(h+1)^2 + \norm a^2}{2(h+1)} = \frac{2 h + 1 + h^2 + \norm a^2}{2(h+1)} = \frac{2h + 2}{2(h+1)} = 1.
\end{multline*}

Next, we check that $p(z,w) = (h,a)$. If $z = 0$ then $h = -1$ and $a = 0$, so clearly $p(z,w) = (h,a)$ in this case. Otherwise, similar computations to the above will show that
\begin{gather*}
\norm z^2 + \norm w^2 = \frac{2 h + 1 + h^2 - \norm a^2}{2(h+1)} = \frac{2 h + 2 h^2}{2(h+1)} = h,\\
2 z \overline w = a.
\end{gather*}

This completes the proof of surjectivity.

3) To prove that $\dl \tilde p$ is an isomorphism, we will show that $\dl p$ is surjective. This will work because if $\dl p = (\dl \pi) \circ (\dl \tilde p)$ is surjective, so is $\dl \tilde p$, and dimensional considerations imply that it is an isomorphism.

First, we calculate $\dl p$, where we see $p$ as a function from $\R^4$ to $\R^3$. In this case,
\[(\dl p)_{z,w} = 2 \cdot \begin{bmatrix}
z_r & z_i & -w_r & -w_i\\
w_r & w_i & z_r & z_i\\
- w_i & w_r & z_i & - z_r
\end{bmatrix},\]
where the subscripts denote real and imaginary part.

We want to show that this matrix is surjective from the tangent space to $S^3$ to the tangent space to $S^2$, at $(z,w)$ and $p(z,w)$ respectively. It suffices to show that this matrix has characteristic three, as this implies a one-dimensional kernel from $\R^4$ to $\R^3$, which in turn implies that its kernel from $T_{z,w}S^3$ to $T_{p(z,w)}S^2$ has dimension at most one and so the image has dimension at least two. But this matrix clearly has characteristic three, as its rows are orthonormal (this is really really easy to check). This concludes the proof of surjectivity of $\dl \tilde p$, and hence that $\tilde p$ is a diffeomorphism.

\medskip

c) We know that one of the leaves of the Reeb foliation is given by the torus $\norm z^2 = \norm w^2 = 1/2$. But this leaf is closed under the action of $U(1)$ on $S^3$, as it does not change the norm of neither $z$ nor $w$. Consequently, for any point $(z,w)$ on this torus $T_{z,w} \FF_{\text{Reeb}} + T_{z,w} \FF_{\text{Orbits}} = T_{z,w} \FF_{\text{Reeb}}$, which is two-dimensional and therefore cannot be the entire tangent space. This shows that these two foliations are not transverse.

\end{sol}

\begin{ex}
Consider the action above, restricted to $\Z_p$, which we see as a subset of $U(1)$ via the identification $[1]_{\Z_p} \approx \e^{2 \I \pi \frac qp}$.

\begin{enumerate}
\item Show that this is a properly discontinuous action.

\item Show that this action commutes with the action of $U(1)$. Show that the action of $U(1)$ on $S^3/\Z_p$ induced by the action of $U(1)$ on $S^3$ is not free.

\item Show that the action of $U(1)$ on $S^3/\Z_p$ given by
\[u \cdot [z,w] = [\sqrt[p]{u} \cdot (z,w)]\]
is free and proper. Show that the resulting quotient is isomorphic to $S^2$.
\end{enumerate}
\end{ex}

\begin{sol}
a) (This is not in the problem statement, but I think it should be:) We will first show that the action is free. This is true because if $u \cdot (z,w) = (z,w)$ then $u z = z$ and $u w = w$. It is always true that either $z$ or $w$ is nonzero, so we conclude that $u = 1 = [0]$.

Now we show that it is properly discontinuous. A free action is properly discontinuous iff the following two properties are satisfied:
\begin{enumerate}
\item For all $p$ there exists a neighborhood $U$ of $p$ which is disjoint from $g \cdot U$ for all $g \neq e$,
\item For all $p \neq q$ there exist neighborhoods $U$ of $p$ and $V$ of $q$ such that $U$ is disjoint from $g \cdot V$ for all $g$.
\end{enumerate}

Now, since the group $\Z_p$ is finite and the action is smooth (I think this should be in the problem statement too? But it is obvious) and therefore the elements act homeomorphically, property (b) is simply a consequence of the sphere being Hausdorff, as you build disjoint $U$ and $V$ for the pair $(p,g \cdot q)$ for each $g$, and intersect all the $U$ and all the $g^{-1} V$. Therefore, it remains to show property (a), but a similar argument goes, because the action is free, continuous and the space is Hausdorff. In fact, I think this whole argument shows that any free smooth action of a finite group is properly discontinuous. The more you know.

\medskip

b) Since we are seeing $\Z_p$ as a subset of $U(1)$, and acting as such, the fact that the actions of these two groups commute is a simple consequence of the commutativity of $U(1)$.

The fact that the induced action of $U(1)$ on the quotient is not free (for $p>1$!!! This should also be in the statement) is equally trivial, as any element of $\Z_p$ acts trivially on any element of the quotient, so we may consider any element of $\Z_p$ distinct from the identity as an example of a non-identity element whose action has fixed points.

\medskip

c) This new action is clearly free, as if $u \cdot [z,w] = [z,w]$ then, by definition, there exists a $p$-th root of unity $\xi$ such that
\[\sqrt[p]{u} \cdot (z,w) = \xi \cdot (z,w),\]
which implies (via freeness of the original action of $U(1)$ that $\sqrt[p]{u} = \xi$ and so that, taking the power $p$ on both sides, $u = 1$.

Now, in principle, one should also prove that this action is smooth. This is not completely obvious because the expression $\sqrt[p]u$ is either multivalued or has a branch cut. Here I'll take the multivalued approach, with the observation that the value chosen does not matter, as any two different values of this expression differ by a multiplicative factor of a $p$-th root of identity, which is irrelevant because we are taking quotients by products by such factors. Consequently, for any particular $u$ we may choose a branch of the $p$-th root function which is smooth near $u$, and obtain a local expression for $u \cdot [z,w]$ which is clearly smooth.

Finally, properness of this action is a simple consequence of the fact that all objects in play are compact. Indeed, any continuous group action of a compact Hausdorff group on a compact Hausdorff space is proper, as in this case compactness coincides with closedness, both in $G \times M$ as well as in $M \times M$,  and continuity guarantees that preimages of closed sets are also closed.

We now show that this manifold is isomorphic to $S^2$. To do so, we show that it is isomorphic to $S^3 / U(1)$.

First, consider the projection function from $S^3$ to $S^3 / U(1)$. Since this function is invariant under the action of $\Z_p$ (because we see $\Z_p$ as a subset of $U(1)$, it lifts to a smooth function $f \colon S^3 / \Z_p \to S^3 / U(1)$. In turn, this function is invariant under the action of $U(1)$ on $S^3 / \Z_p$, as
\[f(u \cdot [z,w]_{\Z_p}) = f([\sqrt[p]u z, \sqrt[p]u w]_{\Z_p}) = [\sqrt[p]u z, \sqrt[p]u w]_{U(1)} = [z,w].\]

Consequently, $f$ lifts to a smooth function $g \colon (S^3 / \Z_p) / U(1) \to S^3 / U(1)$

Similarly, the double-projection $P \colon S^3 \to (S^3 / \Z_p)/U(1)$ is invariant under the action of $U(1)$, as
\begin{align*}
P(u \cdot (z,w)) &= [[u \cdot(z,w)]_{\Z_p}]_{U(1)}\\
&= [ u^p \cdot [z,w]_{\Z_p} ]_{U(1)}\\
&= [[z,w]_{\Z_p} ]_{U(1)}\\
&= P(z,w),
\end{align*}
and so $P$ lifts to a smooth map $h \colon S^3/U(1) \to (S^3 / \Z_p)/U(1)$.

Now, it suffices to show that $g$ and $h$ are inverses, but that is trivial to see from their explicit expressions, both of which consist of erasing the brackets of one type and replacing it by the other. Consequently, they are both diffeomorphisms, and the proof is complete.
\end{sol}

\end{document}