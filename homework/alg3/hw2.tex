\documentclass{article}

\usepackage{amsmath}
\usepackage{amssymb}
\usepackage{amsfonts,stmaryrd}
\usepackage{mathtools}

\usepackage[thmmarks, amsmath]{ntheorem}
\usepackage{fullpage}

\usepackage{graphicx}

\usepackage{diffcoeff}
\diffdef{}{op-symbol=\mathrm{d},op-order-sep=0mu}

\usepackage{cancel}
\usepackage{interval}

\usepackage{enumitem}

\setlist[enumerate,1]{label=(\alph*)}

\title{Algebra Homework 2}
\author{Duarte Maia}
%\date{}

\theorembodyfont{\upshape}
\theoremseparator{.}
\newtheorem{theorem}{Theorem}
\newtheorem{prop}{Prop}
\renewtheorem*{prop*}{Prop}
\newtheorem{lemma}{Lemma}

\newtheorem{ex}{Exercise}

\theoremstyle{nonumberplain}
\theoremheaderfont{\itshape}
\theorembodyfont{\upshape}
\theoremseparator{:}
\theoremsymbol{\ensuremath{\blacksquare}}
\newtheorem{proof}{Proof}
\newtheorem{sol}{Solution}

\newcommand{\R}{\mathbb{R}}
\newcommand{\C}{\mathbb{C}}
\newcommand{\Z}{\mathbb{Z}}
\newcommand{\N}{\mathbb{N}}
\newcommand{\Q}{\mathbb{Q}}
\newcommand{\K}{\mathbb{K}}

\newcommand{\kk}{\Bbbk}


\newcommand{\gp}{\mathfrak{p}}
\newcommand{\gq}{\mathfrak{q}}

\newcommand{\PP}{\mathbb{P}}
\newcommand{\Gr}{\mathrm{Gr}}
\newcommand{\GG}{\mathbb{G}}

\newcommand{\I}{\mathrm{i}}
\newcommand{\e}{\mathrm{e}}
\newcommand{\id}{\mathrm{id}}

\newcommand{\conj}[1]{\overline{#1}}

\newcommand{\grad}{\nabla}

\DeclareMathOperator{\sign}{sign}
\DeclareMathOperator{\image}{im}
\DeclareMathOperator{\ord}{ord}
\DeclareMathOperator{\Ann}{Ann}
\DeclareMathOperator{\Frac}{Frac}
\DeclareMathOperator{\coker}{coker}


\DeclareMathOperator{\tg}{tg}

\newcommand{\Aff}{\mathbb{A}}

\newcommand{\HH}{\mathcal{H}}
\newcommand{\bbH}{\mathbb{H}}

\let\Im\relax
\DeclareMathOperator{\Im}{Im}
\let\Re\relax
\DeclareMathOperator{\Re}{Re}

\DeclarePairedDelimiter{\abs}{\lvert}{\rvert}
\DeclarePairedDelimiter{\norm}{\lvert}{\rvert}
\DeclarePairedDelimiter{\Norm}{\lVert}{\rVert}
\DeclarePairedDelimiter{\braket}{\langle}{\rangle}


\begin{document}
\maketitle

\setcounter{ex}{7}

\begin{ex}
Prove that if $x + y \sqrt2 > 1$ for $x+y \sqrt 2$ invertible, then $x$ and $y$ are both positive. Conclude that the smallest invertible element of $\Z[\sqrt2]$ which is greater than one is $1+\sqrt2$.
\end{ex}

\begin{sol}
If $x+y\sqrt2$ is invertible then we know that $(x+y\sqrt2)(x-y\sqrt2) = \pm 1$. Therefore, $\abs{x-y\sqrt2} < 1$, and so $x^2 + 2 y^2 - 1 < 2 \sqrt 2 x y$. Now, the left-hand side is strictly positive, because the only alternatives are $(x,y) = (0,0)$ (and so $x+y\sqrt2$ is not invertible) and $(x,y) = (1,0)$ (and so $x+y\sqrt2$ is not greater than $1$). As a consequence, the right-hand side must be strictly positive, and so we conclude that the sign of $x$ equals the sign of $y$, and neither is zero. Finally, note that they cannot be negative, because if they were both negative than $x+y\sqrt2$ would be negative, and so not greater than $1$. In conclusion, both $x$ and $y$ are strictly positive.

From this, it is obvious that $x+y\sqrt2$ is bigger the bigger $x$ and $y$ are, so to find the smallest number like this in these conditions, we make $x$ and $y$ be the smallest possible, so $x=y=1$.
\end{sol}

\begin{ex}
Prove that $\Z[\sqrt2]^* = \{\pm (1+\sqrt2)^n \mid n \in \Z\}$.
\end{ex}

\begin{sol}
Suppose that $u$ is an invertible element which is not in the above set. By considering the negative of $u$, and the symmetric if necessary, we may suppose without loss of generality that $u \geq 1$. Since the powers of $1+\sqrt2$ converge to $\infty$ as $n \to \infty$, we know that $u$ lies between some two powers $(1+\sqrt2)^n < u < (1+\sqrt2)^{n+1}$. But by considering instead $u (1+\sqrt2)^{-n}$, we may assume that $1<u<1+\sqrt2$. But this contradicts the previous exercise! Thus, $u$ does not exist.
\end{sol}

\begin{ex}
Prove that $\Z[\sqrt3]^* = \{\pm(2+\sqrt3)^n \mid n \in \Z\}$.
\end{ex}

\begin{sol}
Let $x + y \sqrt3$ be an invertible element of $\Z[\sqrt3]$ which is greater than one. Then, $(x+y\sqrt3)(x-y\sqrt3) = \pm 1$, whence, as above, $x^2 + 3 y^2 - 1 < 2 \sqrt3 xy$. As before, the left-hand side is strictly positive, and so $x$ and $y$ are nonzero and have the same sign, and again must be positive.

Now, note that we will have $x^2 - 3y^2 = \pm 1$, and so, mod $3$, $x^2 = \pm 1$. By checking all three possible values of $x$, we conclude that the only admissible sign is $+1$. In other words, $x^2 - 1 = 3 y^2$, and so we obtain that
\begin{equation}
6 y^2 < 2 \sqrt 3 xy.
\end{equation}
Therefore, $x > \sqrt3 y$ (because $y$ is positive), and so in particular $x > \sqrt3 > 1$. Thus, $x \geq 2$.

Finally, as before, we obtain that to minimize $x+y\sqrt3$ we may as well pick minimal $x$ and $y$, which are $2$ and $1$, and since $2+\sqrt3$ is invertible we conclude that it is the smallest invertible element which is greater than one.
\end{sol}

\begin{ex}
Show that if $y^2 = x^3 - 20$ then $y + 2 \sqrt{-5} = \alpha^3$ for some $\alpha \in \Z[\sqrt{-5}]$.
\end{ex}

\begin{sol}
We show that if some ideal $I$ is a factor in $y\pm 2 \sqrt{-5}$, then it has multiplicity a multiple of three.

If it is a factor in $y \pm 2 \sqrt{-5}$ but not in $y \mp 2 \sqrt{-5}$, then it is obviously so a multiple of three times, because if it is a factor with multiplicity $m$ then it is a factor of $(x)^3$ with multiplicity $m$. But then it is a factor of $(x)$ with multiplicity $m/3 \in \Z$, hence $m$ is a multiple of three.

Now, if it were a factor in both $y + 2 \sqrt{-5}$ and in $y - 2 \sqrt{-5}$, it would be a factor in their difference, which is $4 \sqrt{-5}$. But the only factors of this number are divisors of $2$ and $\sqrt{-5}$, and all such divisors are associated to their own conjugates. As such, since the conjugate of $y+2\sqrt{-5}$ is $y-2\sqrt{-5}$, the multiplicity $m$ of this prime is the same in both. Moreover, as before, we have that the multiplicity of this prime in $(x)$ is $2m/3$, and so $2m$ is a multiple of three, hence $m$ is a multiple of three.

In conclusion, any prime factor of $y\pm2\sqrt{-5}$ has multiplicity three, hence $(y + 2 \sqrt{-5}) = I^3$ for some ideal $I$. Since the class number is $2$, and $2$ is coprime with $3$, we conclude from the fact that $I^3$ is principal that $I$ is principal. Hence, $(y+2\sqrt{-5})^3 = (\alpha^3)$. Thus, $y+2\sqrt{-5} = \alpha^3 u$ with $u$ a unit, and since all units are cubes in $\Z[\sqrt{-5}]$ we may absorb $u$ into $\alpha$, obtaining $y+2\sqrt{-5} = \alpha^3$.
\end{sol}

\begin{ex}
Fine the solutions of $y^2 = x^3 - 20$.
\end{ex}

\begin{sol}
As above, we know that any solution satisfies $y + 2\sqrt{-5} = \alpha^3$ for some $\alpha \in \Z[\sqrt{-5}]$, let us say $\alpha = a + b \sqrt{-5}$. Then, we get the equation in $a$ and $b$:
\begin{equation}
b (3 a^2 - 5 b^2) = 2,
\end{equation}
whose solutions may be brute forced (because e.g. $b$ must be a divisor of $2$), and we get the only solutions are: $b = -1$, $a = \pm 1$.

Plugging this into $\alpha^3$, we obtain that $y = \pm14$, and hence $x = 6$.
\end{sol}

\begin{ex}
Solve $y^2 = x^3 - 54$.
\end{ex}

\begin{sol}
By a similar computation as before, we can conclude that $(y + 3 \sqrt{-6}) = I^3$ for some ideal $I$ in $\Z[\sqrt{-6}]$. Since $2$ is coprime with $3$, we obtain $I = (\alpha)^3$ for some $\alpha$, and again by absorbing units into $\alpha$ we may assume that $y + 3 \sqrt{-6} = \alpha^3$. Again we get a Diophantine equation by writing $\alpha = a + b \sqrt{-6}$:
\begin{equation}
3 b (a^2 - 2b^3) = 3,
\end{equation}
whose only solutions are again $b = -1$, $a = \pm 1$. Plugging these values into $\alpha$, we conclude $y = \pm17$, and thus $x = 7$.
\end{sol}

\begin{ex}
Solve $y^2 = x^3 - 56$.
\end{ex}

\begin{sol}
Again, we conclude that $y + 2 \sqrt{-14}$ is the cube of some ideal $I$ in $\Z[\sqrt{-14}]$. Since $4$ is coprime with $3$, we obtain that $I = (\alpha)^3$, and again absorb any units into $\alpha$ to get $y + 2 \sqrt{-14} = \alpha^3$. Write out the Diophantine equation for $\alpha = a + b \sqrt{-14}$:
\begin{equation}
b (3 a^2 - 14 b^2) = 2,
\end{equation}
whose solutions are $b = -1$, $a = \pm 2$, leading to $y = \pm 76$, and hence $x = 18$.
\end{sol}

\end{document}