\documentclass{article}

\usepackage{amsmath}
\usepackage{amssymb}
\usepackage{amsfonts,stmaryrd}
\usepackage{mathtools}

\usepackage[thmmarks, amsmath]{ntheorem}
\usepackage{fullpage}

\usepackage{graphicx}

\usepackage{diffcoeff}
\diffdef{}{op-symbol=\mathrm{d},op-order-sep=0mu}

\usepackage{cancel}
\usepackage{interval}

\usepackage{enumitem}

\setlist[enumerate,1]{label=(\alph*)}

\title{Algebra Homework 3}
\author{Duarte Maia}
%\date{}

\theorembodyfont{\upshape}
\theoremseparator{.}
\newtheorem{theorem}{Theorem}
\newtheorem{prop}{Prop}
\renewtheorem*{prop*}{Prop}
\newtheorem{lemma}{Lemma}

\newtheorem{ex}{Exercise}

\theoremstyle{nonumberplain}
\theoremheaderfont{\itshape}
\theorembodyfont{\upshape}
\theoremseparator{:}
\theoremsymbol{\ensuremath{\blacksquare}}
\newtheorem{proof}{Proof}
\newtheorem{sol}{Solution}

\newcommand{\R}{\mathbb{R}}
\newcommand{\C}{\mathbb{C}}
\newcommand{\Z}{\mathbb{Z}}
\newcommand{\N}{\mathbb{N}}
\newcommand{\Q}{\mathbb{Q}}
\newcommand{\K}{\mathbb{K}}
\newcommand{\FF}{\mathbb{F}}
\newcommand{\kk}{\Bbbk}


\newcommand{\gp}{\mathfrak{p}}
\newcommand{\gq}{\mathfrak{q}}

\newcommand{\PP}{\mathbb{P}}
\newcommand{\Gr}{\mathrm{Gr}}
\newcommand{\GG}{\mathbb{G}}

\newcommand{\I}{\mathrm{i}}
\newcommand{\e}{\mathrm{e}}
\newcommand{\id}{\mathrm{id}}

\newcommand{\conj}[1]{\overline{#1}}

\newcommand{\grad}{\nabla}

\DeclareMathOperator{\sign}{sign}
\DeclareMathOperator{\image}{im}
\DeclareMathOperator{\ord}{ord}
\DeclareMathOperator{\Ann}{Ann}
\DeclareMathOperator{\Frac}{Frac}
\DeclareMathOperator{\coker}{coker}


\DeclareMathOperator{\tg}{tg}

\newcommand{\Aff}{\mathbb{A}}

\newcommand{\HH}{\mathcal{H}}
\newcommand{\bbH}{\mathbb{H}}

\let\Im\relax
\DeclareMathOperator{\Im}{Im}
\let\Re\relax
\DeclareMathOperator{\Re}{Re}

\DeclarePairedDelimiter{\abs}{\lvert}{\rvert}
\DeclarePairedDelimiter{\norm}{\lvert}{\rvert}
\DeclarePairedDelimiter{\Norm}{\lVert}{\rVert}
\DeclarePairedDelimiter{\braket}{\langle}{\rangle}


\begin{document}
\maketitle

\setcounter{ex}{14}

\begin{ex}
Show that $\Z/p\Z$ has a square root of $-1$ iff $p \equiv 1$ mod $4$.
\end{ex}

\begin{sol}
We know that $(\Z/p\Z)^*$ is cyclic of order $p-1$, with generator $g$ say, and under this isomorphism $1$ is mapped to $g^0$ and $-1$ is mapped to $g^{\frac{p-1}2}$. Now, a square root of $-1$ would be something like $g^k$, with $2k \equiv \frac{p-1}2$ mod $p-1$. In other words, a square root of $-1$ exists iff there are integers $k$ and $N$ such that
\begin{equation}
2k - \frac{p-1}2 = (p-1)N,
\end{equation}
which is equivalent to $p-1 = 4k + 2 (p-1) N$. Now, if such $k$ and $N$ exists, get that $4 \mid p-1$, because the right-hand side is divisible by $4$ (here we use that $p-1$ is even). On the other hand, if $4 \mid p-1$ then we may simply set $k = \frac{p-1}4$ and $N = 0$, to get that such a square root exists.
\end{sol}

\begin{ex}
Let $p \equiv 1$ mod $4$. Prove that $\Z[\I] / p \Z[\I] \cong \FF_p \times \FF_p$ as rings.
\end{ex}

\begin{sol}
We use the isomorphism theorems to write
\begin{equation}
\frac{\Z[\I]}{p \Z[\I]} \cong \frac{\Z[T]/(T^2 + 1)}{(p)} \cong \frac{(\Z/p\Z)[T]}{(T^2 + 1)} \cong \frac{\FF_p[T]}{(T^2 + 1)}
\end{equation}

Now, in $\FF_p$, by the previous exercise, there is some solution $j$ to the equation $T^2 + 1$, hence this polynomial splits as $(T+j)(T-j)$. Moreover, both of these factors are degree one, hence irreducible. Thus, they are coprime (because $j \neq -j$), whence we may apply the chinese remainder theorem to get
\begin{equation}
\frac{\FF_p[T]}{(T+j)(T-j)} \cong \frac{\FF_p[T]}{(T+j)} \times \frac{\FF_p[T]}{(T-j)} \cong \FF_p \times \FF_p,
\end{equation}
as desired.
\end{sol}

\begin{ex}
Let $p \equiv 3$ mod $4$. Prove that $\Z[\I] / p \Z[\I]$ is a field. Moreover, $\Z[\I]/2 \Z[\I] \cong \FF_2[T]/(T^2)$.
\end{ex}

\begin{sol}
In any case, the computations of the previous exercise show that $\Z[\I] / p \Z[\I] \cong \FF_p[T]/(T^2 + 1)$. If $p \equiv 3$ mod $4$, the equation $T^2 + 1$ has no solution in $\FF_p$, hence the polynomial $T^2 + 1$ does not split and (because of its low degree) is irreducible. Thus, the ideal generated by it is maximal, and $\FF_p[T]/(T^2 + 1)$ is a field.

In the case $p = 2$, the polynomial $T^2 + 1$ is the same as $(T+1)^2$. As such, changing the basis to the (new) variable $S = T+1$, we obtain that $\Z[\I]/2\Z[\I] \cong \FF_2[S]/(S^2)$, as desired.
\end{sol}

\begin{ex}
Prove that $\zeta(\Z[\I], s) = \zeta(s) L(s,\chi)$.
\end{ex}

\begin{sol}
We begin by expanding $\zeta(\Z[\I],s)$. To do so, we look at the primes in $\Z[\I]$. It is known that they come in three flavors:
\begin{itemize}
\item Primes $p \in \Z$ which are congruent with $3$ mod $4$ are primes in $\Z[\I]$,
\item Primes $p \in \Z$ which are congruent with $1$ mod $4$ decompose into two distinct conjugate primes, say $p = \pi_p \times \conj{\pi_p}$, in $\Z[\I]$,
\item The prime $2 \in \Z$ decomposes into $- \I \pi_2^2$, with $\pi_2 = 1 + \I$ prime.
\end{itemize}

Now, to proceed with the computation, first we note that if $p$ is a prime in the first case, then we have
\begin{equation}
\frac1{1-\frac1{\#(\Z[\I]/p\Z[\I])^s}} = \frac1{1-\frac1{p^{2s}}} = \frac1{1-\frac1{p^s}} \times \frac1{1+\frac1{p^s}} = \frac1{1-\frac1{p^s}} \times \frac1{1-\frac{\chi(p)}{p^s}}.
\end{equation}

Now, if $p$ is a prime in the second case, we have
\begin{equation}
\frac1{1-\frac1{\#(\Z[\I]/\pi_p)^s}} \times \frac1{1-\frac1{\#(\Z[\I]/\conj{\pi_p})^s}} = \frac1{1-\frac1{p^s}} \times \frac1{1-\frac1{p^s}} = \frac1{1-\frac1{p^s}} \times \frac1{1-\frac{\chi(p)}{p^s}},
\end{equation}
which uses the fact that, by the chinese remainder theorem, $\Z[\I]/\pi_p \times \Z[\I]/\conj{\pi_p} \cong \Z[\I]/p$, which has size $p^2$, hence the size of $\Z[\I]/\pi_p$, which equals the size of $\Z[\I]/\conj{\pi_p}$, is $p$.

Finally, in the case $p = 2$ we have just the contribution of the prime $\pi_2$, which is
\begin{equation}
\frac1{1-\frac1{\#(\Z[\I]/\pi_2)^s}} = \frac1{1-\frac1{2^s}}.
\end{equation}

In conclusion, if we expand out the terms in the zeta function $\zeta(\Z[\I], s)$, we obtain a bunch of terms that come from $\zeta(s)$, and some other terms (for all $p \neq 2$, i.e. all $p$ coprime with $4$) which are precisely the ones from the $L$-function.
\end{sol}

\begin{ex}
Prove that $\zeta(\FF_q[T], s) = \frac1{1-q^{1-s}}$.
\end{ex}

\begin{sol}
To do this, we note that the size of $\FF_q[T]/f$ is precisely $q^{\deg f}$. As such, it becomes pertinent to count, for each degree $d$, how many monic polynomials of degree $d$ there are.

A little combinatorics shows that there are $q^d$.

Therefore, by expression $(*)$, we obtain
\begin{equation}
\zeta(\FF_q[T],s) = \sum_{d=0}^\infty q^d \frac1{q^{ds}} = \sum (q^{1-s})^d,
\end{equation}
which is a geometric series converging precisely to $\frac1{1-q^{1-s}}$.
\end{sol}

\begin{ex}
Prove $L_c(s,\chi) = 1$.
\end{ex}

\begin{sol}
We group the terms in $(*)$ by the degree of $f$. Indeed, suppose that we are summing over a fixed degree $d$. Then, we get
\begin{equation}
\sum_{\deg f = d} \chi(f) \, c^d \, \#(\FF_q[T]/(f))^{-s} = \sum_{\deg f = d} \chi(f) c^d q^{-ds},
\end{equation}
and now we note that we reduce to computing $\sum_{\deg f = d} \chi(f)$. Now, since $g = T$, the monic polynomials $f$ which are coprime with $g$ are all polynomials $f$ with $f(0) \in \FF_q^*$. In such cases, $\chi(f) = \xi^{f(0)}$ for some nontrivial $q$-th root of unity $\xi$. Summing over all $f$ in these conditions, we get a sum of all $q$-th roots of unity, each root repeated $q^{d-1}$ times, \emph{except} when $d = 0$. Indeed, in all other cases, $f(0)$ is `free', and will range over all elements of $\FF_q^*$, but if $d = 0$ then $f(0)$ only has one possible value, which is $1$ (because $f$ is monic). Thus, we obtain that all terms in $(*)$ cancel out with each other, except the one corresponding to $d = 0$ and $f = 1$, and so
\begin{equation}
L_c(s,\chi) = \chi(1) \, c^0 \, \#(\FF_q[T]/(1))^{-s} = 1.
\end{equation}
\end{sol}

\begin{ex}
Prove that there are infinitely many irreducible monic polynomials $h \in \FF_3[T]$ such that $h(0) = 2$.
\end{ex}

\begin{sol}
Consider $\chi(2) = -1$, and $c = 1$. If there were only finitely many $h$ in the conditions of the statement, there would be finitely many $h$ such that $\chi(h) = -1$, and for all others we have $\chi(h) = 1$. As such, expression $(*)$ for $L_c(s,\chi)$ differs by finitely many terms from expression $(*)$ for $\zeta(\FF_q[T],s)$. Therefore, $L_c(s,\chi)$ will converge iff $\zeta(\FF_q[T],s)$ converges. But if we put $s \to 1$, $L_c(s,\chi) \to 1$ but $\zeta(\FF_q[T],s) = \frac1{1-q^{1-s}} \to \infty$, which is a contradiction.
\end{sol}

\end{document}