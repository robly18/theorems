\documentclass{article}

\usepackage{amsmath}
\usepackage{amssymb}
\usepackage{amsfonts,stmaryrd}
\usepackage{mathtools}

\usepackage[thmmarks, amsmath]{ntheorem}
\usepackage{fullpage}

\usepackage{graphicx}


\usepackage{cancel}
\usepackage{interval}

\usepackage{enumitem}

\setlist[enumerate,1]{label=(\alph*)}

\title{Algebra Homework 7}
\author{Duarte Maia}
%\date{}

\theorembodyfont{\upshape}
\theoremseparator{.}
\newtheorem{theorem}{Theorem}
\newtheorem{prop}{Prop}
\renewtheorem*{prop*}{Prop}
\newtheorem{lemma}{Lemma}

\newtheorem{ex}{Exercise}

\theoremstyle{nonumberplain}
\theoremheaderfont{\itshape}
\theorembodyfont{\upshape}
\theoremseparator{:}
\theoremsymbol{\ensuremath{\blacksquare}}
\newtheorem{proof}{Proof}
\newtheorem{sol}{Solution}

\newcommand{\R}{\mathbb{R}}
\newcommand{\C}{\mathbb{C}}
\newcommand{\Z}{\mathbb{Z}}
\newcommand{\N}{\mathbb{N}}
\newcommand{\Q}{\mathbb{Q}}
\newcommand{\K}{\mathbb{K}}
\newcommand{\FF}{\mathbb{F}}
\newcommand{\kk}{\Bbbk}


\newcommand{\gp}{\mathfrak{p}}
\newcommand{\gq}{\mathfrak{q}}

\newcommand{\PP}{\mathbb{P}}
\newcommand{\Gr}{\mathrm{Gr}}
\newcommand{\GG}{\mathbb{G}}

\newcommand{\I}{\mathrm{i}}
\newcommand{\e}{\mathrm{e}}
\newcommand{\id}{\mathrm{id}}

\newcommand{\conj}[1]{\overline{#1}}

\newcommand{\grad}{\nabla}

\DeclareMathOperator{\sign}{sign}
\DeclareMathOperator{\image}{im}
\DeclareMathOperator{\ord}{ord}
\DeclareMathOperator{\Ann}{Ann}
\DeclareMathOperator{\Frac}{Frac}
\DeclareMathOperator{\coker}{coker}


\DeclareMathOperator{\tg}{tg}
\DeclareMathOperator{\Fr}{F}

\newcommand{\Aff}{\mathbb{A}}

\newcommand{\HH}{\mathcal{H}}
\newcommand{\bbH}{\mathbb{H}}

\let\Im\relax
\DeclareMathOperator{\Im}{Im}
\let\Re\relax
\DeclareMathOperator{\Re}{Re}

\DeclarePairedDelimiter{\abs}{\lvert}{\rvert}
\DeclarePairedDelimiter{\norm}{\lvert}{\rvert}
\DeclarePairedDelimiter{\Norm}{\lVert}{\rVert}
\DeclarePairedDelimiter{\braket}{\langle}{\rangle}

\newcommand{\legendre}[2]{\genfrac{(}{)}{}{}{#1}{#2}}
\newcommand{\nlegendre}[3]{\legendre{#1}{#2}_{\!\!#3}}
\newcommand{\blegendre}[2]{\genfrac{[}{]}{}{}{#1}{#2}}


\begin{document}
\maketitle

\setcounter{ex}{42}

\begin{ex}
Prove that the map $x \mapsto x^{\frac{q-1}n}$ is an isomorphism between $\FF_q^* / (\FF_q^*)^n$ and $\mu_n(\FF_q)$.
\end{ex}

\begin{sol}
Recall that $\FF_q^*$ is a cyclic group of order $q-1$. If we translate our thinking to this language, the question becomes: prove that the map `multiplication by $\frac{q-1}n$' is an isomorphism betweeen $\frac{\FF_q^*}{(\FF_q^*)^n} \cong \frac{\Z/(q-1)}{n \Z/(q-1)} = \Z/(n)$ and  the set of elements of $\Z/(q-1)$ which, when multiplied by $n$, become zero, or in other words $\frac{q-1}n \Z/(q-1)$.

This is just a group theory exercise, using the general fact that the additive group $k (\Z/(kn))$ is isomorphic to $\Z/(n)$.
\end{sol}

\begin{ex}
Prove that for irreducible distinct monic polynomials $f$ and $g$ we have
\begin{equation}
\nlegendre fgn = \nlegendre gfn \cdot (-1)^{\frac{q-1}n \deg(f) \deg(g)}.
\end{equation}
\end{ex}

\begin{sol}
By the definition of this generalized legendre symbol, it is obtained as: to compute $\nlegendre gfn$, look at $g$ mod $f$, put it to the power $\frac{q'-1}n$ (where $q'$ is the number of elements of $\FF_q[T]/(f)$, which is $q^d$ where $d = \deg f$), and obtain an element of $\mu_n(K) = \mu_n(\FF_q)$, and this is our $\nlegendre gfn$.

Now, as per the hint, the norm map of the extension $K/\FF_q$ consists of taking the taking the $r$'th power, where $r = \frac{q'-1}{q-1}$. As such, we have that $\blegendre gf$ is given by $(g \bmod f)^{\frac{q'-1}{q-1}}$ and therefore we have
\begin{equation}
\nlegendre gfn = \blegendre gf ^{\frac{q-1}n},
\end{equation}
and evidently the `reciprocity law' that we have for $\blegendre gf$ turns into the one that we are trying to prove for $\nlegendre gfn$.
\end{sol}

\begin{ex}
Prove that for an irreducible monic polynomial $f \in \FF_5[T]$ the maximal ideal $(f)$ splits completely in $\FF_5(\sqrt[4]T)$ iff $(-1)^{\deg f} f \equiv 1$ mod $T$. Decompose $T-1$ into a product of four elements.
\end{ex}

\begin{sol}
By the above definitions, we know that $f \bmod T$ is the same as $\nlegendre fT4$, because $\frac{q'-1}n = \frac{5^1 - 1}4 = 1$. As such, by the reciprocity law we have already proven,
\begin{equation}
(-1)^{\deg f} (f \bmod T) = \nlegendre Tf4.
\end{equation}

Therefore, the condition that $(-1)^{\deg f} f \equiv 1$ mod $T$ is equivalent to the statement that $\nlegendre Tf4 = 1$, which is equivalent to the statement that $T$ has a fourth root in $\FF_5[T]/(f)$, which is equivalent to the statement that the inclusion $\FF_5[T]/(f) \hookrightarrow \FF_5[\sqrt[4]T]/(f)$ is an isomorphism. I don't know where to go from here.

Anyway, to split $(T-1)$, we repeatedly apply difference of squares, and the fact that $-1$ is a square mod $5$ (namely $2^1 \equiv -1$).
\begin{equation}
(T-1) = (\sqrt{T}^2 - 1^2) = (\sqrt T - 1)(\sqrt T + 1) = (\sqrt[4]T - 1)(\sqrt[4]T + 1)(\sqrt[4]T - 2)(\sqrt[4]T + 2).
\end{equation}
\end{sol}

\end{document}