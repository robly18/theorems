\documentclass{article}

\usepackage{amsmath}
\usepackage{amssymb}
\usepackage{amsfonts}
\usepackage{mathtools}

\usepackage[thmmarks, amsmath]{ntheorem}

\usepackage{graphicx}
\usepackage{float}
\usepackage{tikz-cd}
\usepackage{adjustbox}

\usepackage{diffcoeff}
\diffdef{}{op-symbol=\mathrm{d},op-order-sep=0mu}

\usepackage{cancel}
\usepackage{interval}

\usepackage{array}

\usepackage{enumitem}

\setlist[enumerate,1]{label=(\alph*)}

\title{Differential Geometry Homework 3}
\author{Duarte Maia}
%\date{}

\theoremstyle{plain}
\theorembodyfont{\upshape}
\theoremseparator{.}
\newtheorem{theorem}{Theorem}
\newtheorem{prop}{Prop}
\renewtheorem*{prop*}{Prop}
\newtheorem{lemma}{Lemma}
\newtheorem*{ex}{Exercise}

\theoremstyle{nonumberplain}
\theoremheaderfont{\itshape}
\theorembodyfont{\upshape}
\theoremseparator{:}
\theoremsymbol{\ensuremath{\blacksquare}}
\newtheorem{proof}{Proof}
\newtheorem{sol}{Solution}

\theoremsymbol{\text{\textit{(End proof of lemma)}}}
\newtheorem{lemmaproof}{Proof of lemma}

\newcommand{\R}{\mathbb{R}}
\newcommand{\C}{\mathbb{C}}
\newcommand{\Z}{\mathbb{Z}}
\newcommand{\Q}{\mathbb{Q}}

\newcommand{\RP}{\mathbb{RP}}
\newcommand{\HH}{\mathbb{H}}

\newcommand{\kk}{\Bbbk}

\newcommand{\PP}{\mathbb{P}}
\newcommand{\FF}{\mathcal{F}}

\newcommand{\I}{\mathrm{i}}
\newcommand{\e}{\mathrm{e}}

\newcommand{\id}{\mathrm{id}}
\newcommand{\GL}{\mathrm{GL}}
\newcommand{\SO}{\mathrm{SO}}
\newcommand{\SL}{\mathrm{SL}}

\newcommand{\conj}[1]{\overline{#1}}
\newcommand{\close}[1]{\overline{#1}}

\DeclareMathOperator{\interior}{int}
\DeclareMathOperator*{\colim}{colim}
\DeclareMathOperator{\codim}{codim}
\DeclareMathOperator{\trace}{tr}
\DeclareMathOperator{\Lie}{Lie}
\newcommand{\grad}{\nabla}
\newcommand{\transp}{\top}
\DeclareMathOperator{\tg}{tg}


\let\Diff\relax
\DeclareMathOperator{\Diff}{Diff}
\DeclareMathOperator{\Ext}{Ext}
\DeclareMathOperator{\Hom}{Hom}

\DeclarePairedDelimiter{\abs}{\lvert}{\rvert}
\DeclarePairedDelimiter{\norm}{\lvert}{\rvert}
\DeclarePairedDelimiter{\Norm}{\lVert}{\rVert}
\DeclarePairedDelimiter{\braket}{\langle}{\rangle}


\begin{document}
\maketitle

\begin{ex}[\#4 from do Carmo]
\leavevmode
\begin{enumerate}
\item Show that $V(t)$ is parallel iff $\diff V t$ is perpendicular to $T_{c(t)}M$.
\item Show that great circles parametrized by arc length are parallel.
\end{enumerate}
\end{ex}

\begin{sol}
\leavevmode
\begin{enumerate}
\item We saw in class that if $\nabla^0$ is the Levi-Civita connection of $M$, which is a submanifold of $\R^n$ with LC Connection $\nabla$, we have the relation $\nabla^0_v X = \pi(\nabla_v X)$, where $\pi$ is the orthogonal projection $T_p \R^n \to T_p M$. Thus,
\begin{equation}
V(t) \text{ is parallel iff } \nabla^0_{c'(t)} V(t) = 0 \text{ iff } \pi(\nabla_{c'(t)} V(t)) = 0,
\end{equation}
and the exercise is complete because $\pi(w) = 0$ iff $w$ is orthogonal to $T_p M$, and $\nabla_{c'(t)} V(t) = \diff V t$ (because the usual Euclidean connection is trivial).
\item By rotation, we may suppose that our arc circle is the equator, parametrized by $t \mapsto (\cos(t), \sin(t), 0)$. Its derivative is $(-\sin(t), \cos(t), 0)$, and its second derivative is $(-\cos(t), -\sin(t), 0)$. That is, $c''(t) = -c(t)$, and since $T_{c(t)} S^2 = \{c(t)\}^\perp$ we get that $\pi(c''(t)) = 0$. Thus, the given curve is a geodesic.
\end{enumerate}
\end{sol}

\begin{ex}[\#8 from do Carmo]
\leavevmode
\begin{enumerate}
\item Compute the Christoffel symbols of the hyperbolic plane.
\item Parallel transport the vector $v_0 = \partial_y$ along the line $(t,1)$.
\end{enumerate}
\end{ex}

\begin{sol}
\leavevmode
\begin{enumerate}
\item Use the Koszul formula. Note that all brackets die because the coordinate frame commutes. Hence, we get (for $X$, $Y$, $Z$ vectors in the coordinate frame)
\begin{equation}
2 g(\nabla_X Y, Z) = X \cdot g(Y,Z) + Y \cdot g(X,Z) - Z \cdot g(X,Y).
\end{equation}

Now, the only way the right-hand side is non-null is if some of $X, Y, Z$ have nonconstant inner product, which is only the case if two of them are $\partial_y$. In other words, we may throw out any terms where the inner product is not between two $\partial_x$s or $\partial_x$s. Moreover, even in this case, the inner product is $y^{-2}$, which has null $x$ derivative. As such, the \emph{only} possible case in which the above expression might yield a nonzero value is if two of the vectors are the same, and the third is $\partial_y$.

To recover the Christoffel symbols, note that $\partial_x$ and $\partial_y$ are just orthogonal, not orthonormal. Thus, to recover the connection, we apply the formula
\begin{equation}
\nabla_X Y = g(\nabla_X Y, \partial_x) y^2 \partial_x + g(\nabla_X Y, \partial_y) y^2 \partial_y.
\end{equation}

As such, we immediately get
\begin{equation}
\begin{aligned}
\nabla_{\partial_x} \partial_x &= -\frac12 (\partial_y y^{-2}) y^2 \partial_y = \frac1y \partial_y,\\
\nabla_{\partial_x} \partial_y = \nabla_{\partial_y} \partial_x &= \frac12 (\partial_y y^{-2}) y^2 \partial_x = -\frac1y \partial_x,\\
\nabla_{\partial_y} \partial_y &= \frac12 (\partial_y y^{-2}) y^2 \partial_y = - \frac1y \partial_y.
\end{aligned}
\end{equation}

This gives us the desired Christoffel symbols.

\item We solve the ODE. Write $V(x) = u(x) \partial_x + v(x) \partial_y$. Then, its derivative along the curve is given by (using $y = 1$)
\begin{equation}
\begin{aligned}
\nabla_{\partial_x} V(t) &= (u'(x) \partial_x + u(x) \nabla_{\partial_x} \partial_x) + (v'(x) \partial_y + v(x) \nabla_{\partial_x} \partial_y)\\
&= (u'(x) - v(x)) \partial_x + (v'(x) + u(x)) \partial_y.
\end{aligned}
\end{equation}

Thus, we wish to solve $u'(x) = v(x)$, $v'(x) = - u(x)$. The solution to this ODE with $u(0) = 0$ and $v(0) = 1$ is known to be $u(x) = \sin(x)$, $v(x) = \cos(x)$. Thus, the parallel transport of $\partial_y$ is given by
\begin{equation}
V(x) = \sin(x) \partial_x + \cos(x) \partial_y.
\end{equation}
\end{enumerate}
\end{sol}

\begin{ex}[\#1 from do Carmo]
\leavevmode
\begin{enumerate}
\item Compute the metric.
\item Compute the geodesic equations.
\item Show that a geodesic has constant kinetic energy, and moreover satisfies Clairaut's relation.
\item Show that a geodesic of the paraboloid intersects itself infinitely many times.
\end{enumerate}
\end{ex}

\begin{sol}
\leavevmode
\begin{enumerate}
\item The surface is parametrized by $(u,v) \mapsto (f(v) \cos u, f(v) \sin u, g(v))$, and differentiating one quickly writes $\partial_u$ and $\partial_v$ in Euclidean coordinates:
\begin{equation}
\partial_u = (-f(v) \sin u, f(v) \cos u, 0),\quad \partial_v = (f'(v) \cos u, f'(v) \sin u, g'(v)).
\end{equation}

Taking the inner products of these, we easily obtain the metric
\begin{equation}
g_{11} = f(v)^2, \; g_{12} = 0, \; g_{22} = f'(v)^2 + g'(v)^2.
\end{equation}

\item First we compute the Christoffel symbols by Koszul's formula. Note that in this formula all terms with brackets die, because $u$ and $v$ are coordinates, and moreover $g(\partial_u,\partial_v) = 0$. Thus, the only surviving terms in Koszul's formula are the ones where the same vector goes into the inner product. Moreover, since the inner product in this case depends only on $v$, the only surviving terms still are the ones where the vector which is differentiating is $\partial_v$. Moreover, since our basis is orthogonal but not orthonormal, we have the formula
\begin{equation}
\nabla_X Y = g(\nabla_X Y, \partial_u) f(v)^{-2} \partial_u + g(\nabla_X Y, \partial_v) \frac1{f'^2 + g'^2} \partial_v.
\end{equation}

Thus, we easily compute
\begin{equation}
\begin{aligned}
\nabla_{\partial_u} \partial_u &= -\frac12 (2 f f') f^{-2} \partial_v = - \frac{ff'}{f'^2 + g'^2} \partial_v, \\
\nabla_{\partial_u} \partial_v &= \frac{ff'}{f^2} \partial_u,\\
\nabla_{\partial_v} \partial_v &= \frac{f' f'' + g' g''}{f'^2 + g'^2} \partial_v.
\end{aligned}
\end{equation}

Now we have the Christoffel symbols (the only nontrivial ones are $\Gamma_{11}^2$, $\Gamma_{12}^1$, and $\Gamma_{22}^2$) and we plug them into the geodesic equation $\ddot x^k + \sum \Gamma_{ij}^k \dot x^i \dot x^j = 0$ to get the equation for a geodesic $(u(t), v(t))$:
\begin{equation}
\begin{gathered}
\ddot u + 2 \frac{ff'}{f^2} \dot u \dot v = 0,\\
\ddot v - \frac{ff'}{f'^2 + g'^2} \dot u^2 + \frac{f' f'' + g' g''}{f'^2 + g'^2} \dot v^2 = 0.
\end{gathered}
\end{equation}

\item First, we differentiate the kinetic energy of the path $(u(t), v(t))$. This energy is given by $f(v)^2 \dot u^2 + (f'(v)^2 + g'(v)^2) \dot v^2$, and its derivative is given by (courtesy of Mathematica because I'm sick of computations)
\begin{equation}
2 v'(t) \left( v'' \left(f'^2+g'^2\right)+f u'^2 f'+v'(t)^2 \left(f' f''+g' g''\right)+\frac{f^2 u' u''}{v'} \right)
\end{equation}

To get rid of the pesky $u''$ we apply the first geodesic equation, and this simplifies (for a geodesic) to
\begin{equation}
2 v'(t) \left( v'' \left(f'^2+g'^2\right)-f u'^2 f'+v'(t)^2 \left(f' f''+g' g''\right) \right)
\end{equation}
and the quantity inside the parentheses is null by the second geodesic equation. Thus, we get that, assuming a path satisfies the first geodesic equation, it satisfies the second iff it has constant energy (modulo division by zero issues that happen on parallels and meridians).

\smallskip

Now let us approach Clairaut's relation. The quantity $r \cos \beta$ can be reinterpreted as follows. First of all, $r$ is the same as $f$. Moreover, $\cos \beta$ is the inner product between the (normalized) velocity $\gamma'$ and the vector $\frac1{\abs{\partial u}} \partial_u$. Thus, we differentiate the quantity $f(v) \frac1{\abs{\partial u}} \braket{\partial_u, \dot u \partial_u + \dot v \partial_v}$, which simplifies to $f(v)^2 \dot u$. Its derivative is then:
\begin{equation}
2 f(v) f'(v) \dot u \dot v + f(v)^2 \ddot u,
\end{equation}
which is null precisely if (by dividing by $f^2$)
\begin{equation}
\ddot u +  2 \frac{f f'}{f^2} \dot u \dot v = 0,
\end{equation}
which is the first geodesic equation.

\item 
\end{enumerate}
\end{sol}

\begin{ex}[\#3 from do Carmo]
\leavevmode
\begin{enumerate}
\item Prove that flows of left-invariant fields are defined for all time, and that $\varphi(t) \varphi(s) = \varphi(t+s)$.
\item Prove that if $G$ has a bi-invariant metric, then the geodesics starting at the identity coincide with one-parameter subgroups.
\end{enumerate}
\end{ex}

\begin{sol}
\leavevmode
\begin{enumerate}
\item We know that flows exist for some small time $\varepsilon$. Moreover, since the vector field $X$ is left-invariant, left-translating a solution yields another solution. Thus, considering the existing solution $\varphi(t)$ and considering $t \mapsto \varphi(\varepsilon/2) \varphi(t)$ yields a solution which goes up to time $\varepsilon + \frac12 \varepsilon$. Iterating this process one can extend the solution for all positive time, and likewise in the other direction.

The fact that left-multiplying a flow of $X$ yields another flow of $X$ also shows that $\varphi$ is a homomorphism. Indeed, differentiating $s \mapsto \varphi(t)^{-1} \varphi(t+s)$ one obtains $X$, and moreover at $s = 0$ we get the identity, so by uniqueness of solutions to ODEs we get that, for all $s$, $\varphi(s) = \varphi(t)^{-1} \varphi(t+s)$, as desired.

\item All we need to show is that $\varphi(t)$ as above is a geodesic. Its velocity is given by a left-invariant vector field $Y$, so we show that $\nabla_Y Y = 0$. Equivalently, we prove that $g(X, \nabla_Y Y) = 0$ for all left-invariant $X$.

Now, we apply the Koszul formula. Note that since $X$ and $Y$ are left-invariant, as is the metric, all terms $g(X,Y)$ and $g(Y,Y)$ are constant, and so their derivative is zero. Thus, the only remaining terms are the bracket terms, one of which contains $[Y,Y] = 0$, and so we get
\begin{equation}
g(\nabla_Y Y, X) = g(Y, [X,Y]).
\end{equation}

Now, as per the hint in do Carmo, using the bi-invariance of the metric we get
\begin{equation}
g(Y, [X,Y]) = - g(X, [Y,Y]) = 0,
\end{equation}
and thus $\nabla_Y Y = 0$. In conclusion, one-parameter subgroups are geodesics, and since there is a one-parameter subgroup for any initial velocity starting at the identity, the one-parameter subgroups are exactly the geodesics.
\end{enumerate}
\end{sol}

\begin{ex}[3]
Express the Levi-Civita connection in terms of local polar coordinates.
\end{ex}

\begin{sol}
We use the fact that the sphere parametrized by local coordinates is a surface of revolution, with $v$ taking the place of $\phi$, $u$ taking the place of $\theta$, and $f(\phi) = \cos \phi$, $g(\phi) = \sin \phi$. Thus, by exercise \#1 from do Carmo above, the LC connection is given by
\begin{equation}
\begin{aligned}
\nabla_{\partial_\theta} \partial_\theta &= \cos \phi \, \sin \phi \, \partial_\phi \\
\nabla_{\partial_\theta} \partial_\phi = \nabla_{\partial_\phi} \partial_\theta &= - \tg \phi \,  \partial_\theta\\
\nabla_{\partial_\phi} \partial_\phi &= 0.\\
\end{aligned}
\end{equation}
\end{sol}

\begin{ex}[4]
Show that $T \Sigma$ can be naturally identified with $g^* T S^2$. Show that parallel transport commutes with $g$.
\end{ex}

\begin{sol}
The identification may be performed by noting that both $\Sigma$ and $S^2$ are embedded submanifolds of $\R^3$. As such, their tangent spaces may be identified with subspaces of $\R^3$, and under this identification, by definition of $g$, $T_{g(p)} S^2 = T_p \Sigma$. Thus,
\begin{equation}
g^* TS^2 = \{ (p,v) \mid p \in \Sigma, v \in T_{g(p)} S^2 \} = \{(p,v) \mid p \in \Sigma, v \in T_p \Sigma\} = T\Sigma.
\end{equation}

Now, to prove the statement about parallel transport, we use exercise 4 from do Carmo. Indeed, $\frac{\dl.D.(V \circ g \circ \gamma)}{\dl t} = \pi\left( \diff{(V \circ g \circ \gamma)}t \right)$, where $\pi$ is the orthogonal projection on the tangent plane. If we project on $T_{\gamma(t)} \Sigma$, we obtain the derivative along $\gamma$ of $V \circ g$. On the other hand, if we project on $T_{g(\gamma(t))} S^2$, we obtain the derivative along $g \circ \gamma$ of $V$. Of course, these projections are the same. As a consequence, we conclude that the derivative $\frac{\dl.D.(V \circ g \circ \gamma)}{\dl t}$ does not depend on which of these contexts we lie, and so $V$ is parallel along $g \circ \gamma$ iff $V \circ g$ is parallel along $\gamma$, which is what we desired to prove.
\end{sol}

\begin{ex}[5]
\leavevmode
\begin{enumerate}
\item Any one-parameter subgroup of $Z(G)$ is a geodesic.
\item Show that the vertical path is a geodesic.
\item Show that geodesics are not globally length minimizing.
\item Compute the LC connection on the Heisenberg group.
\end{enumerate}
\end{ex}

\begin{sol}
\leavevmode
\begin{enumerate}
\item Apply formula from before for $g(X,\nabla_Y Y) = g(Y, [X,Y])$, and use the fact that $Y$ tangent to $Z(G)$ implies that the bracket with anything is zero. Thus, the path that follows along $Y$ is a geodesic.
\item First, note that $A_{0,0,t}$ is always in the center, and forms a one-parameter subgroup, hence is a geodesic. Left-translating by $A_{x_0, y_0, z}$ we get that $A_{x_0, y_0, z+t}$ is also a geodesic, because the metric is left-invariant.
\item The vertical path has derivative $a_{0,0,1}$ (using the notation from the previous time). Therefore, if $c = \abs{a_{0,0,1}}$ (this is constant), we have that the length of the path from $(0,0,0)$ to $(0,0,n^2)$ is equal to $c n^2$.

Now, we find a linear upper bound for the distance between $(0,0,0)$ and $(0,0,n^2)$. We note that $d(I, AB) \leq d(I, A) + d(A, AB) = d(I,A) + d(I,B)$ because the metric is left-invariant. As such, $d(I,[A^n,B^n]) \leq 2 n (d(I,A) + d(I,B))$. Finally, we note that
\begin{equation}
[A_{1,0,0}^n, A_{0,1,0}^n] = [A_{n,0,0}, A_{0,n,0}] = \text{(expand...)} = A_{0,0,n^2}.
\end{equation}

As such, $d(I, A_{0,0,n^2}) \leq 2 n (a_1 + a_2)$ as desired. Since the length of the vertical geodesic grows quadratically with $n$, we obtain that for large enough $n$ it is \emph{not} the optimal path.

\item We compute all of these terms by the Koszul formula. However, we use the following facts to make our life easier.

First of all, since $X,Y,Z$ are orthogonal, all terms in the Koszul formula which do not have brackets are zero, because they are derivatives of constants (either zero or one). Thus, the only nonzero terms are the ones with brackets, and moreover the only nonzero inner product between a bracket and a vector is the product $g([X,Y],Z) = 1$.

As a consequence, if $A$ and $B$ are two of the three orthonormal vectors, $\nabla_A B$ is going to be either plus or minus $\frac12 C$, where $C$ is the third orthonormal vector, and the sign depends on the whims of the Koszul formula. To the best of my knowledge, the correct signs are as follows:
\begin{equation}
\begin{gathered}
\nabla_Y X = - \frac12 Z,\\
\nabla_X Y = \frac12 Z,\\
\nabla_Z X = \nabla_X Z = - \frac12 Y\\
\nabla_Z Y = \nabla_Y Z = \frac12 X.
\end{gathered}
\end{equation}

\end{enumerate}
\end{sol}

\end{document}