\documentclass{article}

\usepackage{amsmath}
\usepackage{amssymb}
\usepackage{amsfonts}
\usepackage{mathtools}

\usepackage[thmmarks, amsmath]{ntheorem}

\usepackage{graphicx}
\usepackage{float}
\usepackage{tikz-cd}
\usepackage{adjustbox}

\usepackage{diffcoeff}
\diffdef{}{op-symbol=\mathrm{d},op-order-sep=0mu}

\usepackage{cancel}
\usepackage{interval}

\usepackage{array}

\usepackage{enumitem}

\setlist[enumerate,1]{label=(\alph*)}

\title{Differential Geometry Homework 1}
\author{Duarte Maia}
%\date{}

\theoremstyle{plain}
\theorembodyfont{\upshape}
\theoremseparator{.}
\newtheorem{theorem}{Theorem}
\newtheorem{prop}{Prop}
\renewtheorem*{prop*}{Prop}
\newtheorem{lemma}{Lemma}
\newtheorem*{ex}{Exercise}

\theoremstyle{nonumberplain}
\theoremheaderfont{\itshape}
\theorembodyfont{\upshape}
\theoremseparator{:}
\theoremsymbol{\ensuremath{\blacksquare}}
\newtheorem{proof}{Proof}
\newtheorem{sol}{Solution}

\theoremsymbol{\text{\textit{(End proof of lemma)}}}
\newtheorem{lemmaproof}{Proof of lemma}

\newcommand{\R}{\mathbb{R}}
\newcommand{\C}{\mathbb{C}}
\newcommand{\Z}{\mathbb{Z}}
\newcommand{\Q}{\mathbb{Q}}

\newcommand{\RP}{\mathbb{RP}}

\newcommand{\kk}{\Bbbk}

\newcommand{\PP}{\mathbb{P}}
\newcommand{\FF}{\mathcal{F}}

\newcommand{\I}{\mathrm{i}}
\newcommand{\e}{\mathrm{e}}

\newcommand{\id}{\mathrm{id}}
\newcommand{\GL}{\mathrm{GL}}

\newcommand{\conj}[1]{\overline{#1}}
\newcommand{\close}[1]{\overline{#1}}

\DeclareMathOperator{\interior}{int}
\DeclareMathOperator*{\colim}{colim}
\DeclareMathOperator{\codim}{codim}
\DeclareMathOperator{\trace}{tr}
\newcommand{\grad}{\nabla}


\let\Diff\relax
\DeclareMathOperator{\Diff}{Diff}
\DeclareMathOperator{\Ext}{Ext}
\DeclareMathOperator{\Hom}{Hom}

\DeclarePairedDelimiter{\abs}{\lvert}{\rvert}
\DeclarePairedDelimiter{\norm}{\lvert}{\rvert}
\DeclarePairedDelimiter{\Norm}{\lVert}{\rVert}
\DeclarePairedDelimiter{\braket}{\langle}{\rangle}


\begin{document}
\maketitle

\begin{ex}[do Carmo, \#1]
Prove that the antipodal mapping on $S^n$ is an isometry. Use this fact to introduce a metric on $\R \PP^n$.
\end{ex}

\begin{sol}
First, the antipodal mapping $\R^{n+1} \to \R^{n+1}$ is an isometry by linear algebra. Then, it descends to an isometry of $S^n$ because the metric there is induced from $\R^{n+1}$. Finally, the map $\pi \colon S^n \to \R \PP^n$ is a covering map, whose deck group is $\Z/2\Z$ acting by the antipodal map. Thus, $S^n \to \R\PP^n$ is a quotient covering map of a group acting by isometries, and therefore induced a well-defined metric on $\R\PP^n$ such that the projection is a local isometry.
\end{sol}

\begin{ex}[do Carmo, \#2]
Introduce a Riemannian metric on $T^n$ making the projection $\R^n \to T^n$ a local isometry.
\end{ex}

\begin{sol}
Again, $T^n$ is the quotient of $\R^n$ by the action of $\Z^n$ by translations, all of which are isometries. Thus, the quotient induces a natural metric.
\end{sol}

\begin{ex}[do Carmo, \#3]
Immerse $T^n$ into $\R^{2n}$ isometrically.
\end{ex}

\begin{sol}
See $T^n$ as $(S^1)^n$, and induce the natural map to $\R^{2n} = \C^n$ using the immersion $S^1 \to \C$.

To see that this is an isometry onto its image, it suffices to check (because $\R^n \to T^n$ is a local isometry) that the composed map $\R^n \to \R^{2n}$ is an isometry onto its image. This map is given by $(x_1, \dots, x_n) \mapsto (\e^{\I x_1}, \dots, \e^{\I x_n})$, and its (real) derivative at $\vec x$ is given by
\begin{equation}
A = 
\begin{bmatrix}
-\sin(x_1) & 0 & \cdots & 0\\
\cos(x_1) & 0 & \cdots & 0\\
0 & -\sin(x_2) & \cdots & 0\\
0 & \cos(x_2) & \cdots & 0\\
\vdots & \vdots & \vdots & \vdots\\
0 & 0 & \cdots & -\sin(x_n)\\
0 & 0 & \cdots & \cos(x_n)
\end{bmatrix}.
\end{equation}

A trivial computation shows that $(Av) \cdot (Aw) = v \cdot w$, and thus we have isometrically immersed $T^n$ in $\R^{2n}$.
\end{sol}

\begin{ex}[do Carmo, \#4]
\leavevmode
\begin{enumerate}
\item Show that the left-invariant metric of $G$ which coincides with the Euclidean metric at $e$ is given by $g_{ij} = \frac1{y^2} \delta_{ij}$.
\item Show that the Möbius transformations are isometries.
\end{enumerate}
\end{ex}

\begin{sol}
\leavevmode
\begin{enumerate}
\item By left-invariance, the metric at $(x,y)$ must be given by
\begin{equation}
g_{(x,y)}(v,w) := \braket{\dl L_{(x,y)^{-1}} v, \dl L_{(x,y)^{-1}} w}.
\end{equation}

Now, the left-action of $(x,y)$ on an element $(a,b)$ is readily seen to be
\begin{equation}
(x,y) \cdot (a,b) = (x+ay,by),
\end{equation}
whose derivative is $\dl L_{(x,y)} v = y v$. Thus, $(\dl L_{(x,y)^{-1}}) = (\dl L_{(x,y)})^{-1} = \frac1y I$. Therefore, we obtain
\begin{equation}
g_{(x,y)}(v,w) = \frac1{y^2} \braket{v,w},
\end{equation}
as desired.

\item We compute $-\frac{4 \dl3 z' \dl3 \conj{z'}}{(z'-\conj{z'})^2}$. To do so, we start by computing $\dl z'$:
\begin{equation}
\dl z' = \frac1{(cz+d)^2} ( a(cz + d) - c(az+b)) \dl2z = \frac1{(cz+d)^2} \dl2 z.
\end{equation}

Moreover, we note that $(z'-\conj{z'}) = \frac{(z-\conj z)}{(c z + d)(c \conj z + d)}$. Now we may proceed with the computations:
\begin{equation}
\begin{aligned}
-\frac{4 \dl3 z' \dl3 \conj{z'}}{(z'-\conj{z'})^2}
&= -4 \frac{\dl z}{(cz+d)^2} \frac{\dl \conj z}{(c \conj z + d)^2} + \left(\frac{(c z + d)(c \conj z + d)}{z-\conj z} \right)^2\\
&= - \frac{4 \dl3 z \dl3 \conj z}{(z - \conj z)^2}.
\end{aligned}
\end{equation}
\end{enumerate}
\end{sol}

\begin{ex}[do Carmo, \#7]
\leavevmode
\begin{enumerate}
\item Prove that a left-invariant $n$-form is right-invariant.
\item Show that there exists a left-invariant $n$-form on $G$.
\item Given a left-invariant metric $g$, show that the new metric (which I call $g'$) is bi-invariant.
\end{enumerate}
\end{ex}

\begin{sol}
\leavevmode
\begin{enumerate}
\item If $\omega = 0$ the conclusion is trivial. Otherwise, if it is non-null anywhere, it is non-null everywhere by left-invariance. Thus, we may set
\begin{equation}
R_a* \omega = f(a) \omega,
\end{equation}
with $f$ smooth. Moreover, $f(ab) \omega = R_{ab}^* \omega = R_b^* R_a^* \omega = f(b) f(a) \omega$. Thus, $f$ is an anti-homomorphism to $\R^*$, which since $\R^*$ is commutative is the same as a homomorphism. A standard argument shows that (by compacity) the image is contained in $\{1,-1\}$, and by connectedness must be exactly $1$. Thus, $R_a^* \omega = \omega$, and so $\omega$ is left-invariant.

\item Just pick any $n$-form $\omega_e$ on $T_e G$, and extend it by left-invariance to the whole space, by $\omega_a(v_1, \dots, v_n) = \omega_e((\dl L_a)^{-1} v_1, \dots, (\dl L_a)^{-1} v_n)$. This is smooth because the group operation is smooth, and it is left-invariant by direct computation.

\item Well, the given metric $g'(v,w)$ is obtained by integrating $g$ of the right-translates of $g(v,w)$. Since $g$ itself is left-invariant (and left-translation commutes with right-translation), as is the form being used for integration, $g'$ will be left-invariant. On the other hand, $g'$ is right-invariant by change of variable; this uses the fact that $\omega$ is also right-invariant.
\end{enumerate}
\end{sol}

\begin{ex}[2]
\leavevmode
\begin{enumerate}
\item Prove that $\alpha$ is invariant by deck transformations and hence induces $\hat \alpha$.
\item Prove that $H = \ker \hat \alpha$ induces a connection on the circle bundle $(T^3, \pi)$.
\item Verify that $H$ is not flat in several different ways.
\end{enumerate}
\end{ex}

\begin{sol}
\leavevmode
\begin{enumerate}
\item If $(a,b,c) \in \Z^3$, the pullback of $\alpha$ by the translation by $(a,b,c)$ is given by
\begin{equation}
\dl(z+c) - \sin(2\pi(y+b)) \dl(x+a) = \dl z - \sin(2 \pi y) \dl x = \alpha.
\end{equation}

Therefore, since $\alpha$ is invariant under the deck transformations of the covering map $\R^3 \to T^3$, it induces $\hat\alpha$ as desired.

\item We know that kernels of smooth one-forms induce smooth distributions of codimension one, so (by dimensional considerations, as the fibers have dimension one) it suffices to see that the kernel of $\hat \alpha$ never contains any vertical vector, i.e. does not contain $\partial_z$. But this is obvious, as $\hat\alpha(\partial_z) = 1$.

\item \begin{enumerate}[label=\roman*)]
\item Let $X = \partial_x + \sin(2 \pi y) \partial_z$ and $Y = \partial_y$. Both are evidently in the kernel of $\hat \alpha$. However, their bracket is easily seen to be $[X,Y] = -\cos(2 \pi y) \partial_z \neq 0$.
\item Starting at $(0,0)$, flow along $X_H = \partial_x$ in time $\Delta$, then along $Y_H = Y$ in time $\Delta$, then along $-X_H$ in time $-\Delta$ and $-Y_H$ in time $\Delta$. This is a closed loop, and it is contractible because we can make $\Delta \to 0$. However, the parallel transport along this path is the same as the flow along $X$, then $Y$, etc, which is in general (because the bracket is nonzero) not the identity. In part four we compute explicitly the transport along this curve.

\item We just compute $\hat\alpha \wedge \dl \hat \alpha = (\dl z + f \dl x) \wedge (-2\pi\cos(2\pi y) \dl y \wedge \dl x) = 2 \pi \cos(2 \pi y) \dl x \wedge \dl y \wedge \dl z \neq 0$.

This is related to i) because if $X,Y$ are in the kernel of $\hat\alpha$, we have the formula (the sign might be wrong)
\begin{equation}
(\hat\alpha \wedge \dl \hat \alpha)(Z,X,Y) = \hat\alpha(Z) \hat\alpha([X,Y]).
\end{equation}

As such, starting from $X$ and $Y$ as in i) we could prove without computation that $\hat\alpha \wedge \dl \hat\alpha \neq 0$ by plugging in $Z = [X,Y]$. On the other hand, it is also possible (but I don't want to write the details) to, starting from the fact that $\hat\alpha \wedge \dl \hat\alpha$, finding $X,Y$ in the kernel of $\alpha$ such that (for some $Z$) $\hat\alpha(Z) \hat\alpha([X,Y]) \neq 0$, and in particular $[X,Y] \neq 0$.

\item We compute it for the $\gamma$ we used in part ii). Flowing along $\pi_*(Y)$ does not (in the lift) alter the $z$ coordinate of the point we are transporting, so the change along the fiber happens only when we fow along $X$. Moreover, there is also no change in the last leg of the trip, because it happens when $y = 0$, and so $X$ has no $z$ component. Thus, the $z$ coordinate changes only in the second leg of the journey, namely by $\Delta \times \sin(2\pi\Delta)$. In other words, the parallel transport along $\gamma$ induces the map on the fiber $F = S^1 = \R/\Z$ given by $[z] \mapsto [z + \Delta \times \sin(2\pi\Delta)]$, and for most values of $\Delta$ this is a nonzero automorphism of the fiber.
\end{enumerate}
\end{enumerate}
\end{sol}

\begin{ex}[3]
\leavevmode
\begin{enumerate}
\item Show that $M_p$ is a subgroup of $\Diff(E_p)$.
\item Show that $H_p$ is a normal subgroup of $M_p$.
\item Show that for all $p,q$, $(H_p,M_p)$ is isomorphic to $(H_q,M_q)$.
\item Prove that $H$ is flat iff $H_p$ is trivial for some $p$.
\item Show that the map $\rho \colon \pi_1(B,p) \to M_p/H_p$ is well-defined.
\item Prove that if $H$ is flat and $\rho$ is the trivial homomorphism then $E$ is trivial.
\item Show that the projection $\pi \colon \widetilde B \times F \to \widetilde B$ is $\pi_1(B)$-equivariant, and hence induces a $C^\infty$ fiber bundle $\pi_\rho \colon E \to B$ with fiber $F$.
\item Let $H$ be the connection on $E$ induced by the horizontal connection on $\widetilde B \times F$. Compute $H_p$ and $M_p$. What is the monodromy representation?
\end{enumerate}
\end{ex}

\begin{sol}
\leavevmode
\begin{enumerate}
\item It is easy to show that if $\gamma$ and $\sigma$ are loops around $p$, we have $h_\gamma \circ h_\sigma = h_{\gamma \sigma}$ (where $\gamma \sigma$ is the composition of paths where first you do $\sigma$ and then you do $\gamma$). Moreover, $h_{\gamma^{-1}}$ (this means $\gamma$ backwards) is also easily seen to equal $h_\gamma^{-1}$. This proves what we want, because we have closure under composition (by composing paths) and we have inverses. Oh, and obviously we have the identity because of the constant path.

\item Using the remarks in the previous item, we have that if $c$ is contractible and $\gamma$ is arbitrary we have
\begin{equation}
h_\gamma^{-1} \circ h_c \circ h_\gamma = h_{\gamma^{-1} c \gamma},
\end{equation}
and now we just note that $\gamma^{-1} c \gamma$ is contractible (first homotope $c$ to a point, and then you get just the path $\gamma^{-1} \gamma$, which you can just pull back through $\gamma$ to homotope to a point).

\item Pick a path $\alpha$ connecting $p$ to $q$. Then, conjugation by $h_\alpha$ induces a map (which is invertible) between $M_p$ and $M_q$. Moreover, conjugating a path by $\alpha$ (by a similar reasoning to the previous item) takes contractible paths to contractible paths, so this map takes $H_p$ to $H_q$, and its inverse takes $H_q$ to $H_p$, so its restriction is a bijection between these.

\item We defined flatness as `the transport along any contractible loop is trivial'. Equivalently, with our definitions, for all $q$ we have $H_q = 0$. But by the previous item, $H_q$ are all isomorphic, so it is equivalent to $H_p = 0$ for any particular $p$.

\item We wish to show that whenever $\gamma$ and $\sigma$ are homotopic loops, $h_\gamma$ differs from $h_\sigma$ by transport by a contractible loop. But this is obvious: it differs by composition with $h_{\gamma^{-1} \sigma}$, which is contractible because $\gamma$ and $\sigma$ are homotopic.

\item If $H$ is flat then $M_p/H_p = M_p/0 = M_p$. Moreover, the statement that $\rho_H$ is trivial says that for any loop around $p$ the parallel transport is identity. Equivalently, for any two paths $p \to q$ the parallel transport is the same. Therefore, this gives us a canonical identification of any fiber with the fiber of $p$ by parallel transport along any path connecting $p$ and $q$; in other words, a trivialization.


\item We compute: $\pi(\gamma \cdot (\tilde b, f)) = \pi(\gamma \cdot \tilde b, \rho(\gamma)^{-1} f) = \gamma \cdot \tilde b = \gamma \cdot \pi(\tilde b, f)$. So this is trivial.

Therefore, passing to the quotient by the action (which one would need to check is free and other things) we get a quotient manifold, and since you are identifying some fibers with others (but never two points in the same fiber) the quotient will have a fiber bundle structure with the same fiber.

\item We start with a loop $\gamma$ around $p$. Then, transporting a point $x$ in the fiber of $p$ via $H$ is the same as transporting it in the covering $\tilde B \times F$ via the trivial connection, through the lift of $\gamma$. In other words, the point $(p,x)$ is taken to $(\gamma \cdot p, x)$. Now, to see the result in $E$ we need to pass to the quotient, and to see the result in the same fiber of $p$ we write $(\gamma \cdot p, x) \sim (p, y)$. It is easy to solve this equation for $y$, and get $y = \rho(\gamma) x$. Therefore, we conclude that $\gamma$ transports $x$ to $\rho(\gamma)x$, and hence $M_p$ is the image of $\rho$.

On the other hand, if $\gamma$ is contractible, then it is null in the fundamental group, and thus $\rho(\gamma) = 0$. Hence, $H_p = 0$.

The monodromy representation is precisely $\rho$ itself.
\end{enumerate}
\end{sol}

\begin{ex}[4]
Prove that there is no flat Ehresmann connection on $TS^2$.
\end{ex}

\begin{sol}
If there were, by item (f) of the previous question $TS^2$ would be trivial, and we know that it is not. We can apply (f) because, by hypothesis, the hypothetical connection is flat, and moreover the representation is trivial because $\pi(S^2) = 0$.
\end{sol}

\end{document}