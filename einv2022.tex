\documentclass{article}

\usepackage{amsmath}
\usepackage{amssymb}
\usepackage{amsfonts}
\usepackage{mathtools}

\usepackage[thmmarks, amsmath]{ntheorem}

\usepackage{graphicx}

\usepackage{diffcoeff}
\diffdef{}{op-symbol=\mathrm{d},op-order-sep=0mu}

\usepackage{cancel}

\usepackage{enumitem}

\setlist{label=\alph*)}

\title{Palestra para Escola de Inverno 2022\\
Códigos de Barras de Funções}
\author{Duarte Maia}
%\date{}

\theorembodyfont{\upshape}
\theoremseparator{.}
\newtheorem{theorem}{Theorem}
\newtheorem{prop}{Prop}
\renewtheorem*{prop*}{Prop}
\newtheorem{lemma}{Lemma}

\theoremstyle{nonumberplain}
\theoremheaderfont{\itshape}
\theorembodyfont{\upshape}
\theoremseparator{:}
\theoremsymbol{\ensuremath{\blacksquare}}
\newtheorem{proof}{Proof}

\newcommand{\R}{\mathbb{R}}
\newcommand{\C}{\mathbb{C}}

\newcommand{\PP}{\mathbb{P}}
\newcommand{\FF}{\mathcal{F}}

\newcommand{\I}{\mathrm{i}}
\newcommand{\e}{\mathrm{e}}


\DeclareMathOperator{\inte}{int}
\DeclareMathOperator{\codim}{codim}
\newcommand{\grad}{\nabla}

\DeclarePairedDelimiter{\norm}{\lvert}{\rvert}
\DeclarePairedDelimiter{\Norm}{\lVert}{\rVert}


\begin{document}
\maketitle

\section{Descrição}

Uma questão importante na matemática é a seguinte: Será que dois objetos geométricos dados são iguais ou diferentes? Para responder a esta questão surgiram longo dos anos vários objetos chamados invariantes. A ideia é associar a um objeto geométrico um objeto mais simples, como um número (característica de Euler), um grupo (grupo fundamental) ou um espaço vetorial (homologia sobre um corpo). Como estes objetos são muito menos complexos, são mais fáceis de distinguir, o que nos permite justificar que dois espaços são distintos.

A Topologia Algébrica, que estuda muitos destes invariantes, surge em 1895 com Poincaré, mas a ideia que surgiu originalmente aplicada à topologia acaba por ser generalizada a outros contextos, sendo agora uma ferramenta básica em geometria diferencial (cohomologia de deRham) e geometria simplética (homologia de Floer), entre outros. Nesta palestra vou falar sobre uma classe de invariantes chamados Códigos de Barras, e este conceito será explorado através de um exemplo simples mas rico: o código de barras associado a uma função real no círculo.

\section{Corpo}

\subsection{Introdução}

\subsection{Exemplo}

\subsection{Extrair Informação}

\end{document}