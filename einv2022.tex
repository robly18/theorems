\documentclass{article}

\usepackage{amsmath}
\usepackage{amssymb}
\usepackage{amsfonts}
\usepackage{mathtools}

\usepackage[thmmarks, amsmath]{ntheorem}

\usepackage{graphicx}

\usepackage{diffcoeff}
\diffdef{}{op-symbol=\mathrm{d},op-order-sep=0mu}

\usepackage{cancel}

\usepackage{enumitem}

\setlist{label=\alph*)}

\title{Palestra para Escola de Inverno 2022\\
Códigos de Barras de Funções}
\author{Duarte Maia}
%\date{}

\theorembodyfont{\upshape}
\theoremseparator{.}
\newtheorem{theorem}{Theorem}
\newtheorem{prop}{Prop}
\renewtheorem*{prop*}{Prop}
\newtheorem{lemma}{Lemma}

\theoremstyle{nonumberplain}
\theoremheaderfont{\itshape}
\theorembodyfont{\upshape}
\theoremseparator{:}
\theoremsymbol{\ensuremath{\blacksquare}}
\newtheorem{proof}{Proof}

\newcommand{\R}{\mathbb{R}}
\newcommand{\C}{\mathbb{C}}
\newcommand{\Z}{\mathbb{Z}}

\newcommand{\PP}{\mathbb{P}}
\newcommand{\FF}{\mathcal{F}}

\newcommand{\I}{\mathrm{i}}
\newcommand{\e}{\mathrm{e}}


\DeclareMathOperator{\inte}{int}
\DeclareMathOperator{\codim}{codim}
\newcommand{\grad}{\nabla}

\DeclarePairedDelimiter{\abs}{\lvert}{\rvert}
\DeclarePairedDelimiter{\norm}{\lvert}{\rvert}
\DeclarePairedDelimiter{\Norm}{\lVert}{\rVert}


\begin{document}
\maketitle

\section{Descrição}

Uma questão importante na matemática é a seguinte: Será que dois objetos geométricos dados são iguais ou diferentes? Para responder a esta questão surgiram longo dos anos vários objetos chamados invariantes. A ideia é associar a um objeto geométrico um objeto mais simples, como um número (característica de Euler), um grupo (grupo fundamental) ou um espaço vetorial (homologia sobre um corpo). Como estes objetos são muito menos complexos, são mais fáceis de distinguir, o que nos permite justificar que dois espaços são distintos.

A Topologia Algébrica, que estuda muitos destes invariantes, surge em 1895 com Poincaré, mas a ideia que surgiu originalmente aplicada à topologia acaba por ser generalizada a outros contextos, sendo agora uma ferramenta básica em geometria diferencial (cohomologia de deRham) e geometria simplética (homologia de Floer), entre outros. Nesta palestra vou falar sobre uma classe de invariantes chamados Códigos de Barras, e este conceito será explorado através de um exemplo simples mas rico: o código de barras associado a uma função real no círculo.

\section{Corpo}

\subsection{Introdução}

%begin review
A primeira coisa que têm de saber é que distinguir objetos geométricos é difícil. Considerem os seguintes objetos: Uma esfera. Uma linha. Um círculo. Um nó. Um toro. Será que são todos distintos?

A resposta a essa questão depende da noção de igualdade, mas no que se segue vamos tomar o ponto de vista que dois objetos geométricos são distintos se são existe alguma diferença intrinsínseca, ou seja, que é possível, a partir de dentro do objeto, determinar alguma distinção. Sob esta perspetiva, o nó e o círculo são o mesmo objeto, simplesmente desenhado de forma diferente.

Concluimos então que o mesmo objeto geométrico pode ser apresentado de mais de uma maneira diferente. Assim sendo, questionamo-nos se não será possível mais alguns destes objetos serem iguais uns aos outros. A resposta é que não, e a justificação passa pelos chamados invariantes geométricos.

Exemplo simples: A dimensão. É um facto não-trivial que a dimensão de um objeto geométrico não depende da sua apresentação. Consequentemente, estes dois, que têm dimensão 2, não podem ser iguais a estes, que têm dimensão 1.

Exemplo menos simples: O grupo fundamental. Aqueles que não sabem o que isto é vão dá-lo em Topologia, no terceiro ano. Isto é um objeto que pode ser associado a um espaço, e é possível calcular que o grupo fundamental da esfera é o grupo trivial e o grupo fundamental do toro é $\Z \oplus \Z$. Como estes grupos são distintos (um deles tem apenas um objeto e o outro não), estes dois espaços são necessariamente diferentes.

Uma das utilidades de invariantes geométricos é distinguir espaços, mas alguns invariantes dão-nos informação adicional. Por exemplo, existe uma invariante chamada Característica de Euler, que é calculada decompondo o espaço em triângulos e contando faces e vértices, e nos dá informação sobre campos vetoriais nesse espaço. Para aqueles que já ouviram falar do Hairy Ball Theorem, uma demonstração rápida é: A característica de Euler da esfera é 2, que é diferente de zero.

O objeto que eu vos vou apresentar hoje chama-se um Código de Barras, e é semelhante, mas não exatamente igual, aos códigos de barras que vêm nas mercearias. Códigos de barras podem ser usados para distinguir certos objetos, mas a sua maior utilidade é a informação que eles nos dão. Ou seja, não me vão ver a justificar que esta coisa e aquela são distintas porque os seus códigos de barras são distintos, mas sim a responder à questão: \emph{O que conseguimos ler de um código de barras?}

Definição: Um código de barras é uma coleção finita de intervalos. (Desenhar um exemplo) Existem vários códigos de barras associados ao mesmo objeto geométrico, tal como há vários números (como a dimensão, característica de Euler, etc.). Eu vou falar de uma construção específica, que associa um código de barras a uma função de Morse através da homologia do espaço. Não precisam de saber o que estas palavras significam.

%4 minutes

\subsection{Exemplo}

O objeto de estudo são funções reais no círculo, ou seja, a cada ponto do círculo estamos a fazer corresponder um número. Isto é inconveniente de representar, por isso vamos ver o círculo como um intervalo e depois colamos as pontas. Podemos agora representar uma função no círculo como um gráfico tipo aqueles aos quais estamos habituados, com o cuidado que a função tem que colar bem nas pontas, portanto não podemos ter descontinuidades tipo uma função linear.

Para ser um pouco mais específico, vamos estar a falar de funções de Morse, cujo significado não precisam de saber, exceto que uma função de Morse genérica parece-se assim. [Desenhar função genérica.]

Existe uma definição geral, e complicada, que associa um código de barras a uma função de Morse, mas felizmente no caso do círculo essa definição pode ser destilada numa receita simples, que vos estou prestes a apresentar.

Passo 1: Identificar o mínimo da função. Agora, vamos desenhar uma barra infinita, que vai desse valor para cima.

Passo 2: Identificar o mínimo a seguir. Agora, vamos desenhar uma barra que vai desse valor para cima, mas em vez de ir para o infinito, vamos parar assim que conseguirmos ver a barra anterior. Aqui em baixo, a nossa vista está bloqueada por estas montanhas, mas quando passamos daqui para cima conseguimos ver a barra inicial, por isso paramos aqui.

Passo 3: Repetir com os próximos mínimos. Uma propriedade essencial de funções de Morse é que existe um número finito destes.

Passo 4: Após percorrermos todos os mínimos, vamos ao máximo global e desenhamos uma barra que vai do máximo para cima. Isto termina o procedimento.

%8 minutes

%end review

\subsection{Extrair Informação}

Feito este desenho, vamos ver que informação podemos extrair. Ou seja, sabendo o código de barras de uma função, o que podemos dizer sobre ela? Claramente podemos distinguir duas funções com base no seu código de barras, mas isto seria muito trabalho para isso. O verdadeiro interesse está em reduzir funções a códigos de barras e tirar informação sobre as funções delas.

\subsubsection{Quantidades}

Primeiro que tudo, é óbvio do boneco que podemos ver qual é o máximo e o mínimo da função. Consideramos as duas barras infinitas (existem exatamente duas), e os seus extremos inferiores são o máximo e o mínimo.

Podemos também averiguar quais são os valores críticos da função, isto é, os picos e vales. Os picos são os extremos superiores das barras (exceto o máximo global, que é um caso aparte) e os vales são as partes inferiores. Ora, isto parece ser tautológico porque é assim que vos dei a receita para este caso particular, mas não seria assim tão óbvio se fossem à definição original com homologias.

A partir do código de barras é também possível calcular quantidades mais analíticas, por exemplo a variação total da função. A variação total de $f$ pode ser definida de várias formas equivalentes, mas a ideia é medir quanto é que o $f$ sobe e desce, sendo que subir 1 metro e descer 1 metro conta como andar 2 metros. Uma definição possível neste contexto é $\mathop{\mathrm{var}}(f) = \int_0^{2\pi} \abs{f'(t)} \dl t$, que é claramente uma quantidade analítica (dizer mais palavras?). No entanto, podemos ler esta quantidade do código de barras da seguinte forma: Primeiro, somamos o comprimento de todas as barras finitas a multiplicar por dois, porque estas representam uma descida e subida. Depois, somamos duas vezes o máximo menos o mínimo. (Apontar no desenho a que correspondem estas variações.) Conclusão: podemos ler a variação total de $f$ a partir do seu código de barras.

\subsubsection{Estimativas}

(Esta parte pode ser skippada se houver pouco tempo?)

Para além de ler quantidades em códigos de barras, estes também podem ser utilizados para fazer certas estimativas. Comecemos com um facto óbvio: Se $f$ e $g$ são funções distintas, os seus códigos de barras são distintos.

Ora, o interesse surge quando pensamos em como quantificar quão diferentes duas funções. Há várias formas de medir quão diferente $f$ e $g$ são (norma uniforme, norma $C^1$, normas $L^p$, e (talvez algo surpreendentemente) existe uma forma de medir quão diferentes dois códigos de barras são, chamada `bottleneck distance'. (Dar ideia da bottleneck distance?)

Uma questão óbvia que surge é a seguinte: Será que a bottleneck distance entre o código de barras de $f$ e o de $g$ está relacionado com alguma distância entre $f$ e $g$?

Infelizmente não é possível ter algo tipo igualdade, porque existem funções distintas cujos códigos de barras são iguais (e.g. transladar uma função). No entanto, é possível provar a seguinte igualdade:
\begin{equation}\label{eq1}
d(B_f,B_g) \leq \Norm{f-g}_{\text{unif}}.
\end{equation}

Este exemplo não parece muito impressionante, mas estimativas deste estilo têm utilidade teórica. Por exemplo, em geometria simplética existe um conceito chamado `distância de Hofer'. Os detalhes não são relevantes, mas o que é relevante é que isto é uma métrica num certo espaço, mas isso não é trivial de verificar. Como devem saber, uma métrica em $X$ é uma função $d \colon X \times X \to [0,\infty]$ que satisfaz: i) $d(x,x) = 0$, ii) $d(x,y) = d(y,x)$, iii) $d(x,y) + d(y,z) \geq d(x,z)$ e iv) Se $d(x,y) = 0$ então $x=y$. As primeiras três propriedades são fáceis de verificar para a distância de Hofer, mas a quarta é não-trivial, e uma possível abordagem para a demonstrar passa por uma estimativa do tipo \eqref{eq1}.

\subsubsection{Entre outros/Conclusão}

Talvez não estejam impressionados pelo poder dos códigos de barras. De facto, os exemplos dados até agora não são nada de especial. No entanto, o interesse (do meu ponto de vista) está naquilo que não viram.

Considerem o seguinte. Neste objeto que construimos estão escondidas várias quantidades importantes sobre $f$ que conhecemos. Isso levanta a questão: que propriedades é que os códigos de barras escondem que ainda não conhecemos? Considerem as seguintes quantidades: O número de barras. O tamanho da maior barra finita. O tamanho da menor barra. O maior número de barras sobrepostas. O maior número de barras contidas umas nas outras. Isto são só coisas que eu mandei agora ao calhas, mas quem sabe quais destas poderão ter alguma importância? Isso é parte daquilo que eu gosto sobre códigos de barras: são um objeto simples o suficiente para estudar, mas complicado o suficiente para ter neles escondido várias outras quantidades importantes para outros propósitos.

(Palestra over)

\end{document}